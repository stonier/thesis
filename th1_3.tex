\section{Non-Autonomous Dynamical Systems}
\label{NDSsec}

Dynamical systems theory has for the most part focused largely on
autonomous systems for which there is a group or semi-group
evolution property satisfied and attracting objects are invariant
with respect to time. Non-autonomous dynamical systems however,
typically possess more varied and complex behaviour,
exhibiting meaningful properties that are often no longer
invariant. Hence, the semi-group representation used earlier is no
longer directly valid as the initial time is now just as important
as the time elapsed.

\begin{eg}
 Consider the NDE
  \begin{equation}
  \label{NDE} % NDE - Non autonomous differential equation
  \dot{x} = f(\ta,x),
  \end{equation}
  where $\ta \in \mathbb{R}$, and $x \in E \subset \mathbb{R}^d$. Under
  suitable conditions (\cite{Mo62},\cite{Re72}) and assumptions on the state
space, $E$, and   continuity of $f$, there exists a unique solution $x =
x(\ta;t_0,x_0)$,  to the initial value problem defined by $(t_0, x_0 ) \in \mathbb{R} \times E$.
A  {\bf flow or cocycle property} (sometimes otherwise referred to as a $2$-
  parameter group property or process) is also satisfied,
  \begin{equation}
  \label{CPeg}
    x(t+\tau+t_0;t_0,x_0) = x(t+\tau+t_0;\tau+t_0,x(\tau+t_0;t_0,x_0)),
  \end{equation}
  for all $x_0 \in E$, and $t_0,t,\tau \in \mathbb{R}^+$ which is known as a
  {\bf flow or cocycle property} (sometimes otherwise referred to as a $2$-
  parameter group property or process).
\end{eg}
\begin{eg}
The NDE
\[ \dot{x} = 2\ta x, \]
has solutions for the initial value problem $(t_0, x_0)$ given by
\[ x(t+t_0,t_0,x_0) = x_0 e^{(t+t_0)^2 - t_0^2}, \]
where $t$ is the elapsed time, $\ta = t + t_0$ is the actual time, and $t_0$ is the initial time.
This then satisfies the cocycle property (\ref{CPeg}). This can be seen
by noting that
\begin{align*}
  x(t+\tau+t_0,t_0,x_0) &= x_0 e^{(t + \tau+ t_0)^2 - t_0^2} \\
                    &= [x_0 e^{(\tau+t_0)^2 - t_0^2}] e^{(t + \tau+ t_0)^2 -
                    (\tau+t_0)^2} \\
                    &= x(t+\tau+t_0,\tau+t_0,x(\tau+t_0,t_0,x_0)).
\end{align*}
\end{eg}

The cocycle property (\ref{CPeg}), is the non-autonomous counterpart
of the group or semi-group evolutionary property (\ref{SGP}), of an
autonomous dynamical system. Essentially, it is analogous to resetting the
clock.

\subsection{Skew-Product Flow and Cocycle Representation}

Several abstract formulations have evolved to serve as a
non-autonomous counterpart to the semi-group representation.

Sell's skew product flows (\cite{Se67},\cite{Se71}) on
non-autonomous differential equations (\ref{NDE}), retain a
semi-group property by replacing the dependence on the initial
time variable with a function space, effectively converting the
problem to one of an autonomous nature with an altered state
space.

To do this, Sell's Skew Product flows use the function space
$\mathcal{F}$, a set of functions $F:\mathbb{R} \times \mathbb{R}^d
\to \mathbb{R}^d$ that are continuous in both variables and Lipschitz in the second variable uniformly with respect to the first. 

We then proceed to define a group of shift operators on $\mathcal{F}$ such that $\theta_t : \mathcal{F} \to \mathcal{F}$ with $\theta_0=id$ and $\theta_{t} F(t_0, \cdot) = F (t+t_0, \cdot)$. 

Returning to the non-autonomous problem (\ref{NDE}), the product $\mathbb{R}^d \times \mathcal{F}$ of the state space with  this function space provides us with an alternate space that transforms the original non-autonomous problem to an autonomous one. Solutions may then be represented using the semi-group mappings of the previous section.

Unfortunately, an attractor for such a semi-group is then a subset of the product space $\mathbb{R}^d \times \mathcal{F}$ and its meaning in terms of the original dynamics in $\mathbb{R}^d$ is not always clear or convenient. Sell's Skew Product flows are presented formally in the following definition.

\begin{defn}[Skew-Product Flow Representation]
  Let $\{ \theta_t, t \in \mathbb{R} \}$ act as a group of shift operators
  on the function space $\mathcal{F}$ such that
  $\theta_t : \mathcal{F} \to \mathcal{F}$, with $\theta_0=id$ and
  $\theta_{t} F(t_0, \cdot) = F(t+t_0, \cdot)$. Finally let $\mathcal{X}= E
  \times \mathcal{F}$ ($E \subseteq \mathbb{R}^d$), and define $S_t:\mathcal{X} \to  
  \mathcal{X}$ by
  \[ S_t(x_0,F) = (x(t+t_0;t_0, x_0),\theta_t F). \]
  Then the family of mappings $\{S_t, t \in \mathbb{R} \}$ is a
  continuous time semi-group on the state space $\mathcal{X}$ and is known as
  a {\bf skew-product flow}.
\end{defn}

The asymptotic behaviour of semi-groups as outlined previously,
apply to this semi-group also, and can be interpreted to draw
some, though limited, information about the trajectories and
characteristics of the non-autonomous system. For example, an
attractor for such a semi-group generated by a skew product flow
representation is a subset of the product space $\mathbb{R} \times
\mathcal{F}$ and its meaning and conceptualisation in terms of the
original state space is often not always clear or convenient,
particularly if we wish to compute it numerically. However, this is
of advantage if the function space is compact.

Note that the semi-group property for the skew product flow,
\[ S_{t+\tau}(x_0,F) = S_t \circ S_{\tau}(x_0,F), \]
is also a form of the cocycle property, (\ref{CPeg}).
In fact, the skew product flow representation is a special
case of a generalised representation motivated by the
characteristics of the cocycle property.

\begin{defn}[Cocycle Representation] \label{crep}
  Let $\{\theta_{t}, t \in \TT \}$ be a group of mappings on a nonempty
parameter set $P$, that is  $\theta_{t}$ $:$ $P$ $\mapsto$ $P$
with $\theta_{0}$ $=$ $id$ and $\theta_{t} \circ \theta_{\tau}$
$=$ $\theta_{t+\tau}$ for all $t$, $\tau$ $\in$ $\TT$.

A family of mappings $\{ \Phi_{(t,p)}, t \in \TT, p \in P \}$ with
$\Phi_{(t,p)}:X \to X$ is called a {\bf cocycle} on $X$ with
respect to the group $\{\theta_t, t \in \TT \}$ of mappings on $P$
if
  \begin{align}
    (i) \quad & \Phi_{(0,p)} = id, \qquad \qquad  & \mbox{Identity Property}
                          \\
    (ii) \quad & \Phi_{(t +\tau, p)} = \Phi_{(t,\theta_{\tau}p)} \circ
         \Phi_{(\tau,p)}, & \mbox{Cocycle Property} \label{CP}
         %CP - cocycle property.
  \end{align}
for all  $t,\tau$ $\in$ ${\bf R}^{+}$ and $p$ $\in$ $P$.
\end{defn}

The set $\TT$ represents the time element for the dynamical
system, and is usually either $\mathbb{R}$ (continuous time), or
$\mathbb{Z}$ (discrete time). The state space will usually be the
set $E$, ($E$ open and $E \subset \mathbb{R}^d$). For the
following work with cocycles we will keep with the notation of
using the state space $E$, as these will pertain to ideas
presented later. However, most of these results are also
applicable to other forms of the state space or even a generalised
state space $X$, with some attention to detail. Uniqueness of the
cocycle representation for a dynamical system is assumed. When
used as the representation for a non-autonomous differential
equation, existence and uniqueness of the cocycle follows from the
usual continuity and Lipschitz conditions on $f$ in the
differential equation \cite{Mo62}.

Note: $\Phi$ will be used throughout as a cocycle or flow representation
for non-autonomous dynamical systems to distinguish it from the semi-group
representation $S_t$ introduced earlier.

\begin{eg}
  Consider again the initial value problem for Equation (\ref{NDE}),
  \[ \dot{x} = f(\ta,x). \]
  Here $\TT = \mathbb{R}$, and $x \in E \subset \mathbb{R}^d$.
  The solution $x = x(t+t_0;t_0,x_0)$ generates a cocycle
  $\{ \Phi_{(t,t_0)}; t \in \mathbb{R}^+, t_0 \in \mathbb{R} \}$ on $E$
  \[ \Phi_{(t,t_0)}(x_0)=x(t+t_0;t_0,x_0) \]
  with respect to the group $\theta_t; \mathbb{R} \mapsto
  \mathbb{R}$, where $\{\theta_t;t\in \mathbb{R}^+\}$ is defined by
  $\theta_t (t_0) = t_0 +t$.
\end{eg}

The generality involved in defining the parameter space $P$ within the
cocycle definition is to allow a certain degree of flexibility when
choosing a representation for the system. For such a generalised parameter set
we define the following metric on $P$.

\begin{defn}[Metric on $P$]
  We define the metric $|\cdot|$ on $P$ by
  \[ |p^* - p| = \{ |t^*| ; \theta_{t^*}p = p^* \}. \]
  We say $p^* > p$ if  there exists a $t^* > 0$ for which $\theta_{t^*}p = p^*$,
and $p^* < p$ if there exists a $t^* <   0$.
\end{defn}

The obvious and straight forward choice for the parameter set is
the set of all possible initial times as illustrated in the example above. That
is $P=\mathbb{R}$, with the group mapping defined by $\theta_t(t_0) = t_0 + t$.

An alternative representation for the non-autonomous dynamical system
(\ref{NDE}) may be obtained by representing $P$ as a function space,
$\mathcal{F}$, defined by the set of functions $\{f \in \mathcal{F}; f:
\mathbb{R} \times E \mapsto E \}$, which are continuous in both variables,
and Lipschitz in the second variable uniformly with respect to the first
(these are the typical continuity conditions constraining the function
which defines the non-autonomous differential equation (\ref{NDE})). The
group mapping is defined as $\theta_t(f(\cdot,\cdot)) =
f(\cdot+t,\cdot)$. For initial values $x_0$, and
$f(t_0,x_0)$,  we then have $\{\Phi_{(t,f)};t \in
\mathbb{R}^+, f \in \mathcal{F}\}$ defined by $\Phi_{(t,f)}(x_0) =
x(t+t_0;t_0, x_0)$, being a cocycle on $E$. This construction is identical to that used 
for Sell's Skew Product flows except that it retains the original state space and
preserves the non-autonomous aspect of the problem. While it appears less
natural to represent solutions this way it is advantageous in the fact that the function
space $\mathcal{F}$ can often be chosen to be a compact metric space.

The cocycle formalism introduced here provides a natural generalisation of semi-groups to non-autonomous systems with the advantage of being able to retain the original state space (this is in contrast with representations such as Sell's Skew Product flows). This is of particular advantage since attracting and stable objects may be meaningfully represented on the original state space.

Cocycles have been instrumental in developing numerical and random dynamical systems theory (\cite{Ar98}, \cite{FlSc96}, \cite{CrFl94}, \cite{Sc92}) and many of the results concerning asymptotic behaviour and attractors can be usefully transferred and reworked in the context of deterministic non-autonomous dynamical systems.

\subsection{Asymptotic Behaviour: Stability}

As mentioned earlier, the classical concepts of stability, and
asymptotic stability are easily transferred to non-autonomous
systems. These represent the basis for a classical analysis of
stability. However, they represent a limited view of stability
within a non-autonomous environment, and also  do not characterise
attracting structures that may themselves be time-varying. These
issues will be discussed in greater detail in Chapter
\ref{cocyclechapter}.

Using the cocycle representation for a non-autonomous system, a
{\em stable set}, and an {\em asymptotically stable set} are
defined as follows.

\begin{defn}[Stability]\label{Stable}
   A nonempty compact subset $A_0 \subset E$, ($E$ open and $E
   \subset \mathbb{R}^d$) is {\bf stable} under the cocycle mapping
   $\{ \Phi_{(t,p)} : t \in \mathbb{R}^+, p \in P\}$ on $E$, if for any
   $\epsilon > 0$, and any $p \in P$, there exists a $\delta(p,
   \epsilon)>0$ such that
   \begin{equation}\label{SS}
   H^*(\Phi_{(t,p)}(\mathcal{N}_{\delta(p)}(A_0)),A_0) < \epsilon,
            \qquad \forall t \geq 0.
   \end{equation}
\end{defn}

If the $\delta$ is independent of $p$, then the stability is
referred to as uniform with respect to $p$. $A_0$ is said to be
{\bf uniformly stable}.

\begin{defn}[Asymptotically Stability]
   A nonempty compact subset $A_0 \subset E$, ($E$ open and $E \subset
   \mathbb{R}^d$) is {\bf asymptotically stable} under the cocycle mapping
   $\{ \Phi_{(t,p)} : t \in \mathbb{R}^+, p \in P\}$ on $E$, if it is
   both {\em stable}, and for any $p \in P$, there exists a $\delta(p)>0$
   so that for each $x_0 \in \mathcal{N}_{\delta(p)}(A_0)$,
\begin{equation}\label{ASS1}
   H^*(\Phi_{(t,p)}(x_0),A_0) \rightarrow 0,
            \quad \text{as} \quad t \rightarrow \infty.
   \end{equation}
\end{defn}
Alternatively, the attraction property (\ref{ASS1}) may be
restated slightly differently.  For each $p \in P$, there exists a
$\delta(p) > 0$, so that for each $\epsilon > 0$, and $x_0 \in
\mathcal{N}_{\delta(p)}(A_0)$ there is a $T(x_0, p, \epsilon)$
such that
\begin{equation}\label{ASS2}
   \dist(\Phi_{(t,p)}(x_0),A_0) < \epsilon, \qquad \forall t > T.
\end{equation}

Similarly, if $A_0$ is uniformly stable, $\delta$ is
independent of $p$, and $T=T(\epsilon)$ only, then $A_0$ is
said to be {\bf uniformly asymptotically stable} under the cocycle mapping
$\Phi$. This is the case in autonomous systems when $P$ is a singleton
set, and hence any stable objects are automatically uniform with respect
to $p$. The above definition then reduces to the one given in
(\ref{AASS}).

If in the above definition, $T$ is independent of $x_0$, that is
$T = T(p,\e)$ only, then $A_0$ is said to be \textbf{
equi-asymptotically stable}.

We proceed with a few examples to illustrate these concepts.

\begin{eg}
Consider the non-autonomous dynamical system generated by
\begin{equation}\label{asseg1eq}
 \dot{x} = -x + e^{-\ta},
\end{equation}
where $x \in \mathbb{R}$, and $\ta \in \mathbb{R}$. The resulting solutions
may be expressed as a cocycle (note $t$ is the elapsed time)
\[ \Phi_{(t,t_0)}(x_0) = (t e^{-t_0} + x_0) e^{-t}, \]
where the group shift mapping $\theta_t$ acts on the parameter space $P =
\mathbb{R}$, and is defined by $\theta_t (t_0) = t +t_0$. Refer to Figure
\ref{asseg1} for a graph of solutions for various values of $x_0$ at $t_0 =
0$. It is noted that for different values of $t_0$, the families of
solutions follow a similar pattern as that depicted in Figure \ref{asseg1}.
In particular, the only major difference is in the initial point for which the
solutions are strictly monotonically decreasing for all $t$. This occurs
for all $x_0 > e^{-t_0}$.

\begin{figure}[htb]
\begin{center}
%\framebox[6.0cm][c]{
\leavevmode
\hbox{
\epsfxsize=8.5cm
\epsffile{eps/asseg1.eps}  }%}
\protect\caption{Asymptotic Convergence without Stability}
      \protect\label{asseg1}
\end{center}
\end{figure}

Obviously, the set $A_0 = \{0\}$ is asymptotically approached.
However it is not asymptotically stable as it lacks the required
property for stability (all solutions in a neighbourhood of $A_0$
are guaranteed to travel outside a small enough epsilon
neighbourhood. Take for example, $\epsilon = 0.1$.

Note, however that it is `eventually stable'. This is often referred to as {\bf
eventual asymptotic stability}, see \cite{RoHaLa77,Yo66}.
\end{eg}

\begin{eg}
Consider the non-autonomous dynamical system generated by the ordinary
differential equation (where conditions for uniqueness of solutions are
assumed to be satisfied)
\[ \dot{x} = f(p,x), \]
where the state space $X=\mathbb{R}$. It is also known to possess an
asymptotically stable set $A_0$.

Since $A_0$ is asymptotically stable, there exists a $T(x_0,p,\epsilon)$ for
every $p \in P$ guaranteeing attraction of any $x_0 \in
\mathcal{N}_{\delta(p)}(A_0)$ to within an $\epsilon$-neighbourhood of
$A_0$ in finite time. However, since the resulting solutions are
unique, the only dependence $T$ has on the initial state $x_0 \in
\mathcal{N}_{\delta(p)}$, is at the boundaries of the
neighbourhood. For $\mathcal{N}_{\delta(p)}(A_0) \subset \mathbb{R}$,
this consists of only two points. Hence for arbitrary $x_0$,
\[ T(x_0,p,\epsilon) = \max \{T(x_1,p,\epsilon),T(x_2,p,\epsilon)\}, \]
where $x_1,x_2$ are the boundary points. As $x_1,x_2$ depend on
$\delta(p)$, we can conclude $T = T(p, \epsilon)$ only. As a result, any
asymptotically stable set on $\mathbb{R}$ is automatically {\bf
equi-asymptotically stable}. Note that this is not the case with higher
dimensional state spaces as then the neighbourhood boundary consists of an
infinite number of points, and an upper bound for the maximum may not exist.
\end{eg}

\subsection{Lyapunov Functions and Stability}
\label{NLFsec}

The concepts and theorems of Section \ref{ALF1} and \ref{ALF2} may
be extended to dynamical systems of non-autonomous differential
equations (\ref{NDE}). As the system is now dependent on initial
times, the Lyapunov function is required to be a function of both
state and time. As we are generalising to take into account more general parameter 
fields $P$, the Lyapunov function will be of the form $V=V(p,x)$. Additionally, if the function $f(p,x)$ is periodic in $p$, then a Lyapunov function may be
chosen (or found to exist) which is also periodic in $p$ (refer to
\cite{RoHaLa77}).

To investigate the rate of change of $V$ we use the upper right hand \textbf{Dini Derivative} of $V(p,x)$ calculated with respect to time
\[ \overline{D_t}^{+} V(p,x) = \overline{\lim_{h\to 0^+}} \left[
    \frac{V( \theta_h p, \Phi_{(h, p)}(x)) - V(p,x)}{h} \right], \]
where $\Phi_{(t, p)}(x)$ is a solution to the differential equation (\ref{NDE}).

%As with autonomous systems, a simplified and equivalent representation exists. We define this by

%\[ D^{+}_{(\ref{NDE})} V(p,x) = \overline{\lim_{h\to 0^+}} \left[
%    \frac{V( \theta_h p, x+hf(p,x)) - V(p,x)}{h} \right], \]

%Again, since both ${D}^+_{(\ref{ADE})} V(x)$ and the Dini Derivative are equivalent, they may
%always be interchanged where convenient. We will typically use the notation 
%${D}^+_{(\ref{NDE})} V(p, x)$ when considering the decrescence of $V$ as it quickly and accurately
% displays the system (\ref{NDE}) for which solutions $\Phi_{(t,p)}(x)$ belong. 

The following theorems apply equally to autonomous systems
and in fact reduce to those presented in Section \ref{ALF2} where
$V(p,x)$ need only be a function of $x$.


\subsubsection{Stability Theorems}

\begin{therm}[Lyapunov Stability - Theorem]
\label{LStherm} \hfill \\
  A nonempty compact subset $A_0$ of $E$ is {\em stable}
  if there exists a Lyapunov function $V:P \times
  \mathcal{N}_R(A_0), \mapsto \mathbb{R}^+$, for some $R>0$ that satisfies
  the following properties for some $a \in \mathcal{K}$ and for every
  $(p,x) \in P \times \mathcal{N}_R(A_0)$:
  \begin{enumerate}
    \item $x \in A_0$, $\qquad V(p,x)=0$,
    \item $a({\rm dist}(x,A_0)) \leq V(p,x)$,
    \item $V(p,x)$ is locally Lipschitz in $x$, uniformly with $p$.
    \item $\overline{D_t}^{+} V(p,x) \leq 0$.
  \end{enumerate}
  If in addition there is a $b \in \mathcal{K}$ so that we have $V(p,x)
  \leq b({\rm dist}(x,A_0))$, then $A_0$ is {\em locally uniformly stable}.
\end{therm}

\begin{therm}[Lyapunov Uniform Asymptotic Stability - Theorem]
\label{LAStherm} \hfill \\
  A nonempty compact subset $A_0$ of $E$ is {\em uniformly
  asymptotically stable} if there exists a Lyapunov function
  $V:P \times \mathcal{N}_R(A_0), \mapsto \mathbb{R}^+$, for
  some $R>0$ that satisfies the following properties for some $a,b$ and $c
  \in \mathcal{K}$ and for every $(p,x) \in P \times
  \mathcal{N}_R(A_0)$:
  \begin{enumerate}
    \item $a({\rm dist}(x,A_0)) \leq V(p,x) \leq b({\rm dist}(x,A_0))$,
    \item $V(p,x)$ is locally Lipschitz in $x$, uniformly with $p$.
    \item $\overline{D_t}^{+} V(p,x) \leq -c({\rm dist}(x,A_0))$.
  \end{enumerate}
\end{therm}

{\bf Remark 1.} If the above two theorems are satisfied with
$\mathcal{N}_R(A_0) = E$ and
\[ a(r) \to \infty \quad \text{as} \quad r \to \infty, \]
then the stability of $A_0$ is {\em global} in each case.

{\bf Remark 2.} The existence of a Lyapunov function satisfying
the above three conditions, with the exception that the first is satisfied
only so that
\begin{enumerate}
  \item $a({\rm dist}(x,A_0)) \leq V(p,x), \quad V(p,0) = 0$,
\end{enumerate}
that is, there is no upper bounding $b \in \mathcal{K}$ for $V$, then there
can be no guarantee that $V$ implies uniform asymptotic stability, nor even
asymptotic stability.

{\bf Remark 3.} The last condition may be replaced by
$\overline{D_t}^{+} V(p,x) \leq -c V(p,x)$ and the result is still valid.

\subsubsection{Converse Theorems}

Variations of a converse theorem for Lyapunov's Theorem for
uniform asymptotic stability above, exist in several forms. In
many cases it is useful to know only that given an asymptotically
stable set, a Lyapunov function does exist, and exhibits certain
characteristics, even if the actual function is not known. We will
use a converse theorem in developing various results analysing
perturbations of existing systems in later chapters. For
reference, it is given below.

\begin{therm}[Lyapunov Uniform Asymptotic Stability - Converse Theorem]
  \label{LUASCtherm} \hfill \\
  Given a uniformly asymptotically stable subset $A_0$ of $E$
  there exists a Lyapunov function $V:P \times
  \mathcal{N}_R(A_0), \mapsto \mathbb{R}^+$, for some $R>0$ that satisfies
  the following properties for some $a \in \mathcal{K}$, and some
  constant $c > 0$, and for every $(p,x) \in P \times
  \mathcal{N}_R(A_0)$:
\begin{enumerate}
    \item $V(p,x) \geq a({\rm dist}(x,A_0))$, $\qquad V(p,0)=0$,
    \item $|V(p,x)- V(p,x')| \leq h(p) ||x-x'||$,
    \item $\overline{D}^{+}_t V(p,x) \leq -c V(p,x)$,
  \end{enumerate}
  where $h(p)$ is a continuous scalar function.
\end{therm}

{\bf Remark 1.} If the system is autonomous, or if the original
function $f(p,x)$ in (\ref{NDE}) satisfies a Lipschitz condition for
any compact set $K$ on $E$, then the function $h(p)$ may be simply
represented by a constant.

\endinput
