
\section{Absorbing Neighbourhoods}
\label{ANsec}

It is possible to extend the concept of the absorbing set introduced in
Definition \ref{abs} for autonomous dynamical systems to a broader
classification of an absorbing neighbourhood for application within
non-autonomous dynamical systems.

However, the applicability of using such absorbing neighbourhoods
in a non-autonomous environment (compare with the Limit Set
Theorem (\ref{attset}) for autonomous systems) is relevant only to
objects possessing pullback asymptotically stable characteristics.
In this situation, Theorem \ref{attset} may be extended with
application to pullback attracting structures. A similar extension
is not possible for non-autonomous systems with forward
asymptotically stable properties. This is considered in more
detail in Section \ref{secFAN}.

It is also important to note that the use of absorbing neighbourhoods
is identical to that of the original absorbing sets when applied to autonomous
systems.

\subsection{Pullback Absorbing Neighbourhoods}

For generality we consider families of sets dependent on $p \in P$ that possess
pullback absorbing properties. This allows us to conveniently describe
motions around pullback attractors that may vary with $p \in P$. Thus we
will refer to them as pullback absorbing neighbourhoods.

A set $A$ is said to {\bf pullback absorb} another set
$B$ at $p \in P$ if there exists a $T = T(p,B)$ such that
\[ \Phi_{(t, \theta_{-t}(p))}(B) \subset A \qquad \forall t > T. \]
Similarly, a family of sets $\hat{A} = \{A(p):p \in P \}$ is
said to {\bf pullback absorb} another family of sets $\hat{D} = \{D_p:p \in
P\}$ at $p \in P$, if there exists a $T=T(p,\hat{D}) >0$ such that,
\[ \Phi_{(t,\theta_{-t}p)}(D_{\theta_{-t}(p)}) \subset A(p)) \qquad
             \forall t > T. \]
A {\em pullback absorbing neighbourhood} is then defined as

\begin{defn}[Pullback Absorbing Neighbourhoods] \label{PANdef}
   A family $\hat{B}=\{B(p);p \in P\}$ of uniformly bounded compact subsets
   of $E$, is called a {\bf Pullback Absorbing Neighbourhood} for a
   cocycle $\{\Phi_{(t,p)}; t \in \mathbb{R}^{+},p \in P\}$ on $E$
   if it pullback absorbs a uniformly bounded $\delta$ -
   neighbourhood system of $\hat{B}$. That is, there exists an open
   $\delta$-neighbourhood system
   $\hat{\mathcal{N}}_{\hat{\delta},\hat{B}}$ defined by a delta set
   $\hat{\delta} = \{\delta_p \in \mathbb{R}^+; p\in P\}$ so that for each
   $p \in P$, there exists a $T_p>0$ such that
   \begin{equation}
     \Phi_{(t,\theta_{-t}p)}(\mathcal{N}_{\delta_p}(B(\theta_{-t}(p)))
     \subset B(p) \qquad \forall t > T_p.
   \end{equation}
\end{defn}

In many situations a single set $B$, for a dynamical system may be found
which satisfies the requirements for a pullback absorbing neighbourhood for
all $p$, that is $\hat{B} = B$. Indeed when this is the case many of the
results associated with it become greatly simplified. The situation arising
in Example \ref{per2deg} is such a case.

In certain cases it may be possible to find a pullback absorbing
neighbourhood that is positively invariant, however it is usually easier
and less restrictive to consider neighbourhoods which satisfy only the
above definition. Note that a pullback absorbing neighbourhood
automatically satisfies the following property.

\begin{lemma} \label{panepilem}
If $\hat{B}$ is a pullback absorbing neighbourhood, then for each $p \in P$
there exists a $T_p > 0$ such that
\[ \Phi_{(t,\theta_{-t}(p))}(B(\theta_{-t}(p)))
     \subset B(p) \qquad \forall t > T_p. \]
\end{lemma}

\vspace{0.5cm}

\begin{eg}
For this illustration we refer back to Example \ref{introeg}. A diagram
illustrating the system's dynamics is repeated here for convenience (see
Figure \ref{pb2eg}).
\begin{figure}[htb]
\begin{center}
%\framebox[6.0cm][c]{
\leavevmode
\hbox{
\epsfxsize=9.5cm
\epsffile{eps/pbeg.eps}  }%}
\protect\caption{Introductory Example}\protect\label{pb2eg}
\end{center}
\end{figure}

Finding a pullback absorbing neighbourhood that is
positively invariant typically requires a detailed knowledge of the system's
dynamics. Here however, a single constant set such as $B =
[-2,2]$, that is merely pullback absorbing for any $p \in P$, may be used
to verify the existence of a pullback attractor. This will be shown shortly
using Theorem \ref{abspat}, and is indeed a much simpler approach than finding a
positively invariant family.
\end{eg}

Pullback absorbing neighbourhoods provide a means with which to
identify pullback attractors in a similar fashion to that of absorbing sets
for semi-group attractors.

Again, it is often much easier to find a pullback absorbing neighbourhood system
rather than the pullback attractor itself. Indeed, in many systems the attractor
cannot be found and written explicitly and an estimate based on the progression
of an absorbing neighbourhood must be made.

The following theorem is an extension of Theorem \ref{attset}
applied to pullback absorbing neighbourhoods for cocycles.

A similar theorem has been presented by B. Schmalfuss \cite{Sc99},
however the construction for the pullback attractor differs
slightly as was mentioned in Section \ref{ANSsec}. The key
difference lies in the assumption of uniformity present in a neighbourhood of
attraction for the attractor which (as has been shown) may not neccessarily 
exist.  Positive invariance of the absorbing
neighbourhood is also assumed \cite{Sc99}. Although technically more
complicated, neither uniformity nor positive invariance are required here.

\begin{therm}
  \label{abspat}
  Let $\{\Phi_{(t,p)};t \in \mathbb{R}^+, p \in P \}$ be a cocycle of
  continuous mappings on $E$ with a Pullback Absorbing Neighbourhood
  $\hat{B}$. Then there exists a pullback attractor $\hat{A} = \{ A(p); p
  \in P \}$ uniquely determined for each $p \in P$ by
  \begin{equation}
  \label{patfromabs}
  A(p) = \bigcap_{\tau \geq 0} \overline{\bigcup_{t \geq \tau}
            \Phi_{(t,\theta_{-t}p)}(B(\theta_{-t}p))}.
  \end{equation}
\end{therm}

\begin{prf}
  \hspace{3mm} \\
  \hspace*{3mm} Proposing as our attractor the family of sets $\hat{A} =
  \{ A(p); p \in P \}$ defined above we have
  \[ A(p) = \bigcap_{\tau \geq 0} \overline{\bigcup_{t \geq \tau}
            \Phi_{(t,\theta_{-t}p)}(B(\theta_{-t}p))} \]
  Since $A(p)$ is an infinite intersection of nested compact sets, it must
  contain at least one accumulation point. Hence we may conclude $A(p)$ is
  non-empty. To show that this is indeed a pullback attractor, it must meet the
  requirements in Definition \ref{PAdef}. This will be done in three steps.

  \hspace*{3mm} i) {\em Uniform Boundedness and Compactness} : From
  (\ref{patfromabs}) and using Lemma \ref{panepilem}
  \begin{align*}
  A(p) &= \bigcap_{\tau \geq 0} \overline{\bigcup_{t \geq \tau}
            \Phi_{(t,\theta_{-t}p)}(B(\theta_{-t}p))}, \\
  &\subseteq \bigcap_{\tau \geq T_p} \overline{\bigcup_{t \geq \tau}
            \Phi_{(t,\theta_{-t}p)}(B(\theta_{-t}p))}, \\
  &\subseteq \bigcap_{\tau \geq T_p} \overline{\bigcup_{t \geq \tau}
            B(p)}, \\
  &= \overline{B(p)}.
  \end{align*}
  Hence $\hat{A}$ is uniformly bounded since $\hat{B}$ is uniformly bounded
  with respect to $p$. The attractor set is also compact as it is an
  intersection of compact sets.

  \hspace*{3mm} ii) {\em Pullback Property} : We need to find a pullback
  convergent neighbourhood system as in (\ref{PAPC}). First we shall show
  that $\hat{A}$ pullback attracts $\hat{B}$ at every $p \in P$. That is,
  for each $p \in P$
  \begin{equation}
  \label{abscatp2}
    \lim_{t \to \infty} H^* (\Phi_{(t, \theta_{-t}p)} (B(\theta_{-t}p)),
           A(p)) = 0,
  \end{equation}
  where $A(p)$ is defined as above.

  Assume this is not the case. Then for some $\epsilon > 0$ there exists
  sequences $t_j \to \infty$ and $x_j \in \Phi_{(t_j,
  \theta_{-t_j}p)} (B(\theta_{-t_j}p))$ such that
  \begin{equation}
  \label{abscatp3}
    \dist(x_j,A(p)) > \epsilon \qquad \forall j.
  \end{equation}
  For large enough $j$, $x_j \in B(p)$ (Lemma \ref{panepilem}). Now since
  $B(p)$ is compact, there exists a subsequence $t_{j'} \rightarrow
  \infty$ and an associated convergent subsequence, $x_{j'} \to x_0$ with
  $x_0 \in B(p)$. For any $\tau > 0$, the $x_{j'}$ satisfy
  \[ x_{j'} \in \bigcup_{t \geq \tau} \Phi_{(t,\theta_{-t}p)}
          (B(\theta_{-t}p)), \hspace{2cm} \quad t_{j'}>\tau. \]
  As $x_0$ is the limit, we have for any $\tau \geq 0$,
  \[ x_0 \in \overline{\bigcup_{t \geq \tau} \Phi_{(t,\theta_{-t}p)}
         (B(\theta_{-t}p))}, \]
  and hence
  \[ x_0 \in \bigcap_{\tau \geq 0} \overline{\bigcup_{t \geq \tau}
            \Phi_{(t,\theta_{-t}p)}(B(\theta_{-t}p))}. \]
That is, $x_0 \in A(p)$, which contradicts (\ref{abscatp3}) and this implies
$\dist(x_0,A(p)) \geq \e$. As the choice of $p$ was arbitrary, (\ref{abscatp2})
holds true for all $p \in   P$.

  From (\ref{abscatp2}), for any $\epsilon > 0$, and each $p \in P$, there
  exists a $t_1 = t_1(\epsilon,p)$ such that
  \[ H^*(\Phi_{(t_1,\theta_{-t_1}p)}
           (B(\theta_{-t_1}p),A(p)) < \epsilon. \]
  Let us take $\hat{\mathcal{N}}_{\hat{\delta}^*,\hat{A}}$ as our
  neighbourhood system for the attractor, $\hat{A}$, defined by
  $\hat{\delta}^* = \{ \delta^*_p; \delta^*_p =
  \delta_{\theta_{-t_1}(p)}, \delta_p \in \hat{\delta} \}$
  where $\hat{\delta}$ defines $\hat{\mathcal{N}}_{\hat{\delta}, \hat{B}}$,
  the associated $\hat{\delta}$-neighbourhood of $\hat{B}$. We need to show
  that $\hat{A}$ pullback attracts the system
  $\hat{\mathcal{N}}_{\hat{\delta}^*,\hat{A}}$.

  Using the cocycle property, and because $B(p)$ is pullback absorbing, we
  can formulate attraction for elements of
  $\hat{\mathcal{N}}_{\hat{\delta}^*,\hat{A}}$
  \begin{equation*}
  \begin{split}
  &H^*(\Phi_{(t,
  \theta_{-t}p)}(\mathcal{N}_{\delta^*_p}(A(\theta_{-t}p)),A(p)) \\
  & \qquad \leq H^*(\Phi_{(t,
  \theta_{-t}p)}(\mathcal{N}_{\delta^*_p}(B(\theta_{-t}p)),A(p)) \\
  &\qquad = H^*(\Phi_{(t_1, \theta_{-t_1}p)}
          \circ \Phi_{(t-t_1,
          \theta_{-t}p)}
          (\mathcal{N}_{\delta^*_p}(B(\theta_{-t}p)),A(p)) \\
  &\qquad \leq H^*(\Phi_{(t_1, \theta_{-t_1}p)}
          (B(\theta_{-t_1}p),A(p)) \\
  &\qquad \leq \epsilon
  \end{split}
  \end{equation*}
  for all $t > t_{(\theta_{-t_1}p)} +t_1$, where
  $t_{(\theta_{-t_1}p)}$ is the finite absorption time described in the definition of  
  the pullback absorbing neighbourhood.
  Hence $\hat{A}$ satisfies the pullback property for each $p \in P$, that is,
  \[ \lim_{t \to \infty} H^* (\Phi_{(t, \theta_{-t}p)}
            (\mathcal{N}_{\delta^*_p}(A(\theta_{-t}p)),A(p)) = 0. \]

  \hspace*{3mm} iii) {\em $\Phi$-Invariance} : We are required to prove
  that the family of sets constituting $\hat{A}$ is equivariant as defined
  in (\ref{PAInv}). Consider an element of the family $A(\theta_{-t^*}p)
  \in \hat{A}$ for arbitrary $p$, and any $t^*>0$
  \begin{equation}\label{shiftedatt}
     A(\theta_{-t^*}p) = \bigcap_{\tau \geq 0} \overline{ \bigcup_{t \geq
             \tau} \Phi_{(t, \theta_{-(t + t^*)}p)} (B(\theta_{-(t +
             t^*)}p))}.
  \end{equation}

  First, we need to show
  \begin{equation}\label{inclusion}
  \begin{split}
  & \Phi_{(t^*, \theta_{-t^*}p)}\left(\bigcap_{\tau \geq 0} \overline{
\bigcup_{t         \geq \tau} \Phi_{(t, \theta_{-(t + t^*)}p)} (B(\theta_{-(t +
        t^*)}p))}\right), \\
  & \qquad = \bigcap_{\tau \geq 0} \Phi_{(t^*, \theta_{-t^*}p)}\left( \overline{
     \bigcup_{t \geq \tau} \Phi_{(t, \theta_{-(t + t^*)}p)} (B(\theta_{-(t +
        t^*)}p))}\right). \\
  \end{split}
  \end{equation}
  The inclusion "$\subset$" is trivial. To prove "$\supset$", let $x$ be an element
  of the right hand side,
  \[ x \in \bigcap_{\tau \geq 0} \Phi_{(t^*, \theta_{-t^*}p)}\left( \overline{
     \bigcup_{t \geq \tau} \Phi_{(t, \theta_{-(t + t^*)}p)} (B(\theta_{-(t +
        t^*)}p))})\right). \]
  Then for each $\tau > 0$, there exists
  \[ x_{\tau} \in \overline{\bigcup_{t \geq \tau} \Phi_{(t, \theta_{-(t +
        t^*)}p)} (B(\theta_{-(t + t^*)}p))}, \]
  such that $x = \Phi_{(t^*, \theta_{-t^*}p)}(x_{\tau})$. For $\tau$ large
  enough, $x_{\tau} \in B(\theta_{-t^*}p)$ due to Lemma \ref{panepilem}.
  $B(\theta_{-t^*}p)$ is compact and hence there exists a subsequence
  $\tau' \to \infty$, and an associated convergent subsequence
  $x_{\tau'} \to x^*$ with $x^* \in B(\theta_{-t^*}p)$.

  Now, given $\tau > 0$, we have
  \[ x_{\tau'} \in \overline{\bigcup_{t \geq \tau} \Phi_{(t, \theta_{-(t +
        t^*)}p)} (B(\theta_{-(t + t^*)}p))} \qquad \forall \tau' > \tau. \]
  Further because of closure,
  \[ x^* \in \overline{\bigcup_{t \geq \tau} \Phi_{(t, \theta_{-(t +
        t^*)}p)} (B(\theta_{-(t + t^*)}p))}, \]
  for any $\tau > 0$. Hence
  \[ x^* \in \bigcap_{\tau \geq 0} \overline{\bigcup_{t \geq \tau} \Phi_{(t,
        \theta_{-(t + t^*)}p)} (B(\theta_{-(t + t^*)}p))}. \]
  Finally, the limit and continuity of the cocycle implies that
  \[ \Phi_{(t^*,\theta_{-t^*}p)}(x^*) = x, \]
  and so $x$ also belongs to
  the left hand side. This verifies (\ref{inclusion}).

  Returning to (\ref{shiftedatt}), using the result above and also making
  use of the cocycle property, we have
  \begin{align*}
  \Phi_{(t^*, \theta_{-t^*}p)} &(A(\theta_{-t^*}(p))) \\
  &=\Phi_{(t^*, \theta_{-t^*}p)}\left(\bigcap_{\tau \geq 0}
       \overline{ \bigcup_{t
       \geq \tau} \Phi_{(t, \theta_{-(t + t^*)}p)} (B(\theta_{-(t +
       t^*)}p))}\right), \\
  &= \bigcap_{\tau \geq 0} \Phi_{(t^*, \theta_{-t^*}p)}\left(
     \overline{\bigcup_{t \geq \tau} \Phi_{(t, \theta_{-(t + t^*)}p)}
     (B(\theta_{-(t + t^*)}p))}\right), \\
  &= \bigcap_{\tau \geq 0} \overline{\bigcup_{t \geq \tau} \Phi_{(t^*,
     \theta_{-t^*}p)} \Phi_{(t, \theta_{-(t + t^*)}p)} (B(\theta_{-(t +
     t^*)}p))}, \\
  &= \bigcap_{\tau \geq 0} \overline{\bigcup_{t \geq \tau} \Phi_{(t+t^*,
     \theta_{-(t+t^*)}p)} (B(\theta_{-(t + t^*)}p))}, \\
  &= \bigcap_{\tau \geq t^*} \overline{\bigcup_{s \geq \tau} \Phi_{(s,
     \theta_{-s}p)} (B(\theta_{-s}p))},
  \end{align*}
  where we have made the substitution $s = t + t^*$. Now, for all $\tau <
  t^*$,
  \[ \overline{\bigcup_{s \geq \tau} \Phi_{(s, \theta_{-s}p)}
     (B(\theta_{-s}p))} \supseteq \overline{\bigcup_{s \geq t^*} \Phi_{(s,
     \theta_{-s}p)} (B(\theta_{-s}p))}. \]
  Hence
  \begin{align*}
  \Phi_{(t^*, \theta_{-t^*}p)} &(A(\theta_{-t^*}p)) \\
  &= \bigcap_{\tau \geq 0} \overline{\bigcup_{s \geq \tau} \Phi_{(s,
     \theta_{-s}p)} (B(\theta_{-s}p))} \\
  &= A(p). \\
  \end{align*}
  Thus the conditions for $\Phi$-Invariance are satisfied.
\end{prf}

 \begin{eg}[ {\em Attractor for a Perturbed Limit Cycle} ] \hfill \\ \label{per2deg}
  Consider the two dimensional autonomous dynamical system
  \begin{align*}
    \dot{x} &= y + x - x(x^{2} + y^{2}), \\
    \dot{y} &= -x +y - y(x^{2} + y^{2}). \\
  \end{align*}
  This system possesses an attractor $A_0$, being the stable limit
  cycle centred on the unit circle. Now, suppose the original system is
  perturbed slightly with a small non-autonomous perturbation resulting in
  the dynamical system defined by
  \begin{align*}
    \dot{x} &= y + x - x(x^{2} + y^{2}) + \epsilon \cos(2\ta), \\
    \dot{y} &= -x +y - y(x^{2} + y^{2}) + \epsilon \cos(2\ta). \\
  \end{align*}
  A transformation to polar co-ordinates yields
  \begin{align*}
    \dot{r} &= r(1-r^2) + (\cos \theta + \sin \theta) \epsilon \cos(2\ta), \\
    \dot{\theta} &= -1 + \frac{1}{r^2} (\cos \theta - \sin \theta) \epsilon
                             \cos(2\ta). \\
  \end{align*}
  If the perturbation parameter $\epsilon$, is kept small, it could be
  expected that a pullback attractor may exist in the vicinity of the original
  attractor $A_0$. To begin our search for such a pullback attractor, let us
  start with the dynamics.

  The $\theta$ dynamics remain rotating in the same
  direction if $\epsilon$ is small enough, so we need only be concerned with
  the radial variable. Similarly if we take $\epsilon$ small enough, let us
  say $\epsilon < 0.5$, then we have $\dot{r}$ positive for $r \leq 0.5$, and
  $\dot{r}$ negative for $r \geq 1.5$. Knowing this, we construct a
  toroidal shaped pullback absorbing neighbourhood for the non-autonomously perturbed
  system
  \[ B = \{ (r,\theta) ; 0.5 \leq r \leq 1.5 \}. \]
  In this case, our pullback neighbourhood needs only a single set, and
  satisfies all the conditions required for a pullback absorbing
  neighbourhood.

    \begin{figure}[htb]
  \begin{center}
  %\framebox[6.0cm][c]{
  \leavevmode
  \hbox{
  \epsfxsize=9.5cm
  \epsffile{eps/per2dc.eps}  }%}
  \protect\caption{Pullback Attraction to a Non-Autonomously
                          Perturbed Limit Cycle}
        \protect\label{per2dpic}
  \end{center}
  \end{figure}

  By Theorem \ref{abspat} we verify the
  existence of a pullback attractor, $\hat{A}$ contained within $B$. In fact,
  by numerically pullback integrating $B$ to arrive at an
  estimation for (\ref{patfromabs}), we obtain a good approximation for
  $\hat{A}$. Using this approach, initial states from the inner and
  outer boundaries of $B$ were mapped (using a Runge-Kutta method)
  from a pulled back initial time towards a fixed final time. This process
  was repeated for various values of the final time between $\ta = 40$ and $\ta
  = 50$, and the final approximations for $A(\ta)$ were plotted, see Figure
  \ref{per2dpic}. As can be seen the original fixed limit cycle has
  evolved with the non-autonomous perturbation to become a structure with
  similar characteristics, but now moving periodically (due to the
  nature of the $\cos(2t)$ perturbation) around the old limit cycle, $A_0$.
\end{eg}

A converse result for the above theorem also holds under
restricted conditions on the neighbourhood. If a non-autonomous
dynamical system possesses a pullback attractor for which its
pullback attracting neighbourhood system is known to be
independent of $p$ (that is, $\hat{\delta} = \delta$), then it can
be shown that there exists an associated pullback absorbing
neighbourhood. This condition effectively places a uniformity
requirement on the pullback attractivity of the pullback
attractor. This result is proved in \cite{Kl98}, and reiterated in a
similar fashion here for completeness.

\begin{therm}
  \label{catabs}
  Let $\{\Phi_{(t,p)};t \in \mathbb{R}^+, p \in P \}$ be a cocycle of
  continuous mappings on $E$ with a pullback attractor $\hat{A} = \{ A(p); p
  \in P \}$ that pullback attracts a $\delta$-neighbourhood system
  $\hat{\mathcal{N}}_{\delta,\hat{A}}$ for some $\delta > 0$. Then there
  exists a Pullback Absorbing Neighbourhood $\hat{B}$, associated with
  $\hat{A}$.
\end{therm}
\begin{prf} \hfill \\
  By Theorem \ref{papasthm}, the pullback attractor $\hat{A}$ is {\em
  pullback asymptotically stable}, and hence also {\em pullback stable}. If
  we take $\epsilon = \delta$ (where $\delta$ is such that $\hat{A}$
  pullback attracts $\hat{\mathcal{N}}_{\delta,\hat{A}}$), then there
  exists a corresponding $\hat{\delta}^* = \{ \delta_p^* \in \mathbb{R}^+ ;
  p \in P \}$ as in the definition for pullback stability.

  We construct a $\delta$-neighbourhood system
  $\hat{\mathcal{N}}_{(\hat{\delta}^*, \hat{A})}$ of $\hat{A}$ defined by
  $\hat{\delta}^*$ (note that this system will also serve as a pullback
  attracting neighbourhood system for the pullback attractor $\hat{A}$),
  and propose as our Pullback Absorbing Neighbourhood $\hat{B} = \{ B(p); p
  \in P \}$ where
  \[ B(p) = \overline{\bigcup_{t \geq 0} \Phi_{(t, \theta_{-t}(p))}
        (\mathcal{N}_{\delta_p^*}(A(\theta_{-t}(p)))}, \]
  for each $p \in P$. To see that it is indeed a pullback absorbing
  neighbourhood for the pullback attractor $\hat{A}$ we must show it
  satisfies the conditions in Definition \ref{PANdef}. This is accomplished
  in two stages.

  \hspace*{3mm} i) {\em Boundedness and Compactness}: Given the
  pullback stability of $\hat{A}$ and the method of construction of $B(p)$,
  it is easy to see that $B(p) \subseteq \mathcal{N}_{\delta}(A(p))$ for
  every $p \in P$. Hence $\hat{B} \subseteq \mathcal{N}_{(\delta,
  \hat{A})}$ which is uniformly bounded and so $\hat{B}$ is also uniformly
  bounded. It is also compact due to the closure.

  \hspace*{3mm} ii) {\em Pullback Absorption} :
  Note that $\hat{\mathcal{N}}_{\delta, \hat{A}}$ also qualifies as a
  $\delta$ - neighbourhood system for $\hat{B}$. We will proceed to show
  that $\hat{B}$ pullback absorbs this $\delta$-neighbourhood system.

  $\hat{A}$ pullback attracts $\hat{\mathcal{N}}_{\delta, \hat{A}}$,
  and hence for each $p \in P$, there exists a $T=T(\delta^*_p, p)$ such
  that
  \begin{align*}
  \Phi_{(t,\theta_{-t}(p))}(\mathcal{N}_{\delta}(A(\theta_{-t}(p)))
  &\subseteq \mathcal{N}_{\delta^*_p}(A(p))\\
  &\subseteq B(p), \\
  \end{align*}

  for all $t > T$. Thus $\hat{B}$ pullback absorbs the
  $\delta$-neighbourhood system $\hat{\mathcal{N}}_{\delta, \hat{A}}$.
\end{prf}

\subsection{Forward Absorbing Neighbourhoods}
\label{secFAN}

A forward absorbing neighbourhood may be constructed in a similar
manner to that above. However, analysis of a limiting object as in Theorem
\ref{abspat}  cannot be undertaken in a similar fashion.

To see this, consider an appropriately defined
neighbourhood with forward absorbing properties.

The set $A$ is said to {\bf forward absorb} another set
$B$ from $p \in P$ if there exists a $T = T(p,B)$ such that
\[ \Phi_{(t, (p))}(B) \subset A \qquad \forall t > T. \]
Similarly, a family of sets $\hat{A} = \{A(p):p \in P \}$ is
said to {\bf forward absorb} another family of sets $\hat{D} = \{D_p:p \in
P\}$ at $p \in P$, if there exists a $T=T(p,\hat{D}) >0$ such that,
\[ \Phi_{(t, p)}(D(p)) \subset A( \theta_{t}p)) \qquad
             \forall t > T \]
A {\em forward absorbing neighbourhood} is then defined as

\begin{defn}[Forward Absorbing Neighbourhood] \label{FANdef}
   A family $\hat{B}=\{B(p);p \in P\}$ of uniformly bounded compact subsets
   of $E$, is called a {\bf Forward Absorbing Neighbourhood} for a
   cocycle $\{\Phi_{(t,p)}; t \in \mathbb{R}^{+},p \in P\}$ on $E$
   if it forward absorbs a uniformly bounded $\delta$-neighbourhood
   system of $\hat{B}$. That is, there exists an open
   $\delta$-neighbourhood system
   $\hat{\mathcal{N}}_{\hat{\delta},\hat{B}}$ defined by a delta set
   $\hat{\delta} = \{\delta_p \in \mathbb{R}^+; p\in P\}$ so that for each
   $p \in P$, there exists a $T_p>0$ such that
   \begin{equation}
     \Phi_{(t, p)}(\mathcal{N}_{\delta_p}(B(p))
     \subset B(\theta_t p) \qquad \forall t > T_p.
   \end{equation}
\end{defn}

\begin{eg}
Consider again Example \ref{introeg}. In this dynamical system
the constant set $B=[-2,2]$ satisfies the requirements for a forward absorbing
neighbourhood. However, if the limiting set is calculated in a similar fashion
as \ref{patfromabs}, we find
\begin{align*}
A &= \bigcap_{\tau \geq 0} \overline{\bigcup_{t \geq \tau} \Phi_{(t,p)}(B)}, \\
  &= [ -1/ \sqrt{2}, 1/ \sqrt{2}]. \\
\end{align*}
$A$ is not invariant and hence does not satisfy the conditions for
a forward attractor.
\end{eg}

It is not even possible to draw any conclusions regarding the existence of a
forward attractor within a forward absorbing neighbourhood. This is illustrated
in the following counter-example which possesses a forward absorbing
neighbourhood that does not contain a forward attractor.

\begin{eg}
Consider the non-autonomous dynamical system arising from the ODE
\[ \dot{x} = \frac{10e^{-\ta}}{1+10e^{-\ta}}( - x + \tanh(\ta/2) ) +
                \frac{2e^{-\ta}}{(1+e^{-\ta})^2}. \]
Its behaviour is similar to those examples presented in Section
\ref{stabatt} and is shown in Figure \ref{acasegfig}.

We propose the set $B = [-1.5,1.5]$ as an absorbing neighbourhood. By
considering bounds on the derivative, it can be seen that $\dot{x}$ is always
negative (and positive respectively) on the upper (and lower respectively)
boundaries of the neighbourhood. Hence $B$ is positively invariant
and is both a pullback and forward absorbing neighbourhood. By Theorem
\ref{abspat}, we can conclude there exists a pullback attractor contained
within $B$. Determined analytically, the pullback attractor takes the form
$A(\ta) = \tanh(\ta/2)$.

\begin{figure}[htb]
\begin{center}
%\framebox[6.0cm][c]{
\leavevmode
\hbox{
\epsfxsize=9.5cm
\epsffile{eps/acaseg.eps}  }%}
\protect\caption{$\hat{A}$ is not a Forward Attractor}
\protect\label{acasegfig} \end{center}
\end{figure}

However, as seen in Figure \ref{acasegfig}, it is clear that $B$ does not
forward converge to a distinctly defined forward attractor, nor even to the
pullback attractor $\hat{A}$.

Analytically this is clear upon investigation of the solutions expressed using
a cocycle representation,
\[ \Phi_{(t,t_0)}(x_0) = \tanh((t+t_0)/2) +\frac{1+10e^{-(t+t_0)}}
         {1+10e^{-t_0}}(x_0 - \tanh(t_0/2)). \]
Solutions originating at any $(t_0,x_0)$ do not approach any closer than a
distance of $(x_0 - \tanh(t_0/2))/(1+10e^{-t_0})$ from the pullback attractor.
\end{eg}

The above result is in contrast with both pullback results and analogous results
for autonomous systems. The reasons permitting attractor structures to exist
within pullback absorbing neighbourhoods and not necessarily for forwards
absorbing neighbourhoods, or of a method to guarantee existence of a forward
attractor are open topics for further research.

\endinput
