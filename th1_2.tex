\section{Autonomous Dynamical Systems}
\label{ADSsec}

The predominant characteristic of an autonomous dynamical system
is its dependence solely on the time elapsed and not on the
current value of the time itself. As a result the behaviour of the
system is simplified in comparison to that of a non-autonomous
system, and any attracting objects or stability properties it may
possess are generally invariant with respect to a time parameter.

\subsection{Semi-group Representation}

The definition of a semi-group when applied to a collection of
mappings can be used to describe the evolution of autonomous dynamical
systems on a state space $E$ as outlined below.

\begin{defn}[Semi-group Representation]
  A family of mappings $\{S_t, t \in \TT \}$ with $S_t$ $:$
  $E$ $\mapsto$ $E$ for each $t$ $\in$ $\TT$ is called a {\bf
  semi-group\/} on $E$ if
  \begin{align}
    (i) \quad & S_0 = id, \qquad \qquad  & \mbox{Identity Property}
                          \\
    (ii) \quad & S_{t + \tau} = S_t \circ S_{\tau},
         & \mbox{Semi-group Property} \label{SGP} %SGP - semi group prop.
  \end{align}
  for  all $t,\tau$ $\in$ $\TT$, where $id$ is the identity
  mapping.
\end{defn}

Elements of $\TT$ are representative of the time set used. Usually
$\TT = \mathbb{R}^+$ (continuous dynamical systems), or $\TT =
\mathbb{Z}$ (discrete systems).

One can consider the evolution of an initial state by following the mapping
as it traces the trajectory followed by the state with increasing time.
For continuous systems this is often called the {\bf flow} of the
dynamical system for that initial state. It is also useful to consider the
flow of nonempty subsets of the state space for fixed periods of time. In
this way, the dynamical system as a whole can be more easily observed,
noting features such as basins of attraction, cycles, etc.

\begin{eg}
  Autonomous differential equations generate dynamical systems which
  may be represented by a semi-group mapping. Consider the ODE
  \begin{equation}
  \label{ADEeg}
  \dot{x} = -x,
  \end{equation}
  where $x \in \mathbb{R}$. The ODE
  generates an autonomous dynamical system with solution
  \[x(\ta) = x_0 e^{-(\ta-t_0)}, \]
  to the initial value problem defined by $x(t_0)=x_0$. Solutions are
  dependent only on the time elapsed $t=(\ta-t_0)$, and not the initial time.
  In fact any ODE with
  $\dot{x} = f(x)$ under assumptions guaranteeing uniqueness and
  extendability (notably continuity and Lipschitz continuity of $f$ with
  respect to $x$ - refer to any well versed book on ordinary differential
  equations, \cite{Mo62},\cite{Re72}), generates an autonomous dynamical
  system.

  The solution may also be written in the form $\{S_t,t \in \mathbb{R}^+ \}$
  using the semi-group representation above. In this case the state space is
  simply the Euclidean Space, $\mathbb{R}^d$, and the mapping is defined by
  \begin{align} \label{SG} %SG - Semi-Group
    S_0(x_0) &= x_0, \\
    S_t(x_0) &= x(t + t_0), \notag \\
             &= x_0 e^{-t}, \notag
  \end{align}
  for each $t \in \mathbb{R}^+$ and $x_0 \in \mathbb{R}^d$. (Note that $t$
  in this representation indicates the time elapsed and not the actual
  time).
\end{eg}

Being autonomous by definition, the behaviour of a semi-group
generating dynamical system is determined largely by any
properties or structures it may possess as the elapsed time is
allowed to become infinite, its \emph{asymptotic} behaviour. There
is a host of literature on autonomous dynamical systems and their
asymptotic behaviour. The presence of various structures such as
limit cycles, attractors, stable and asymptotically stable sets
often clearly describe the asymptotic nature of the overall
system. Elements of Lyapunov theory (\cite{Yo66}) can also
predetermine the existence and location of some of these features.
For completeness, a few of the definitions for such structures and
Lyapunov results for autonomous dynamical systems using the
semi-group representation follow.

\subsection{Asymptotic Behaviour: Stability}

Firstly, we will briefly introduce a few distance terminologies that will be
used extensively in the following definitions and examples.

For consistency, lower case letters will be used to denote point elements of
the space, (e.g. $x,y \in E$), and upper case letters for nonempty
subsets of the space (e.g. $A, A_0 \subset E$).

$||\cdot||$ represents the usual metric on the space, $E$, representing
the distance between two points.

The distance of a point $x \in E$ from a compact set $A$ is defined as
\[ \dist(x,A) = \min_{a \in A}||x-a||. \]

The $\delta$-neighbourhood of a set $A$ is given by
\[ \mathcal{N}_{\delta}(A) = \{x;\dist(x,A) <\delta, \delta > 0 \}. \]

The {\bf Hausdorff Separation} $H^*(A,B)$ of nonempty compact subsets $A,B
\subset E$, is defined as
\[ H^*(A,B) = \max_{a \in A} \dist (a,B) = \max_{a \in A} \min_{b \in
                        B}||a-b||, \]
The quantity $H(A,B) = \max \{ H^*(A,B), H^*(B,A) \}$ then satisfies the properties 
for a metric and is called the Hausdorff metric on the space
$\mathcal{H}(\mathbb{R}^d)$ of nonempty compact subsets of
$\mathbb{R}^d$. $H^*(A,B)$ is often called the Hausdorff
semi-metric on $\mathcal{H}(\mathbb{R}^d)$. It is a measure of the difference
between two sets $A$ and $B$ (refer to Figure \ref{hdorffpic}).

\begin{figure}[htb]
\begin{center}
  \framebox[5.5cm][c]{ \leavevmode \hbox{
  \epsfxsize=5.0cm
  \epsffile{eps/hdorff.eps}  } }
  \protect\caption{Hausdorff Semi-Metric}
      \protect\label{hdorffpic}
\end{center}
\end{figure}

The following concepts of stability, asymptotic stability and
attraction assist in understanding asymptotic behaviour within
autonomous dynamical systems. They are extendable to
non-autonomous systems for which they characterise stability only
in part (refer to Section \ref{NDSsec}).  A
comprehensive analysis of stability for non-autonomous dynamical
systems is covered in Chapter \ref{cocyclechapter}. Utilising the
semi-group notation introduced previously, {\em stable} and {\em
asymptotically stable} sets are defined below.

\begin{defn}[Stable Set]\label{UStable}
   A nonempty compact subset $A_0 \subset E$, ($E$ open and $E
   \subset \mathbb{R}^d$) is {\bf stable} under the semi-group
   mapping $\{ S_t : t \in \mathbb{R}^+\}$, if for every $\epsilon > 0$
   there exists a $\delta(\epsilon)>0$ such that
   \begin{equation}\label{ASS}
    H^*(S_t(\mathcal{N}_{\delta}(A_0)),A_0) < \epsilon, \qquad \forall t
          \geq 0.
   \end{equation}
\end{defn}

\begin{defn}[Asymptotically Stable Set]
   A nonempty compact subset $A_0 \subset E$, ($E$ open and $E \subset
   \mathbb{R}^d$) is {\bf asymptotically stable} if it is
   {\em stable}, and in addition, if there exists a $\delta>0$, so that for
   each $\epsilon > 0$, there is a $T=T(\epsilon)>0$ such that for every $x_0
   \in \mathcal{N}_{\delta}(A_0)$
   \begin{equation}
     \label{AASS}
     H^{*} \left(S_t(x_0),A_0 \right) < \epsilon,
                           \qquad \forall t \geq T(\epsilon).
   \end{equation}
\end{defn}

If any $\epsilon$-neighbourhood of $A_0$ can be reached in finite time
by every bounded subset of $\mathbb{R}^d$, then $A_0$ is said to be {\bf
globally asymptotically stable}.

A nonempty set $B$ is said to be {\bf positively invariant} if
\[ S_t(B) \subseteq B, \qquad \forall t > 0. \]
Note that stable and asymptotically stable sets are positively invariant (see
\cite{St94}). Additionally, a non-empty set $B$ is said to be {\bf S-invariant}
if
\[ S_t(B) = B, \qquad \forall t > 0. \]

\begin{figure}[htb]
\begin{center}
  \framebox[9.5cm][c]{ \leavevmode \hbox{ \epsfxsize=9.0cm
  \epsffile{eps/stab.eps}  } } \protect\caption{Concepts of
  Stability}
      \protect\label{stabpic}
\end{center}
\end{figure}


\subsection{Asymptotic Behaviour: Attractors}

The asymptotic behaviour of a semi-group generating dynamical
system is predominately determined by its limit sets and their
attracting properties. Attractivity within the system is
characterised by the approach of a point or set within a finite
time to a neighbourhood of the attracting set. Formally, a set $A$
is said to {\bf attract} another set $B$ if for every
$\epsilon$-neighbourhood of $A$, there exists a $T(\epsilon,A,B$)
such that $S_t(B) \subset \mathcal{N}_{\epsilon}(A)$ for all
$t>T$. For example, an asymptotically stable set attracts an open
neighbourhood of itself. As a result, it is often referred to as
an {\em attracting set}. The definition of an attractor extends
the idea of an asymptotically stable set by requiring it to be the
minimal and invariant attracting set.

\begin{defn}[Semi-Group Attractor]\label{att} \hfill \\
   A nonempty compact and bounded subset $A_0 \subset E$, ($E$ open and $E
   \subset \mathbb{R}^d$) is called an {\bf attractor} of a semi-group
   $\{S_t; t \in \mathbb{R}^{+}\}$ on $E$ if 
  \begin{align}
   \label{AI}
     & S_t \left( A_0 \right)  =  A_0, \qquad \text{for each} \quad t \in
                   \mathbb{R}^{+}, & &
                   \text{(Invariance Property)} \\
\intertext{and if there exists a $\delta>0$ such that}
   \label{AA}
     & \lim_{t \to \infty} H^{*} \left(S_t(\mathcal{N}_{\delta}(A_0)),A_0
                   \right) = 0. & & \text{(Attraction Property)}
   \end{align}
\end{defn}

If the attractor attracts every bounded set of $\mathbb{R}^d$ as well as
a neighbourhood of itself, it is said to be a {\bf global attractor}.

\begin{therm}\label{attassthm}
A semi-group attractor is asymptotically stable.
\end{therm}
\begin{prf}
By definition, it is automatic that an attractor attracts a
neighbourhood of itself. Therefore it is only required
to show stability. \cite{St94} provides a complete proof.
\end{prf}

On the other hand, an asymptotically stable set need not necessarily be an
attractor. This is illustrated with the simple example given
below.

\begin{eg}
\label{egexp}
  Consider again the dynamical system arising from the differential equation
  \[ \dot{x}=-x, \]
  Solutions are given by
  \[ S_t(x) = x_0 e^{-t}. \]
  Clearly the origin is an attractor, and also an
  asymptotically stable set as it attracts every open neighbourhood of
  itself. Now, consider the set $B=[-b,b]$ for some bounded $b>0$. It is
  nonempty, compact, and attracts an open neighbourhood of itself, and
  so is asymptotically stable. However $S_t(B) = [-b
  e^{-t},b e^{-t}]$ is a strict subset of $B$, and thus fails to comply with
  the property of invariance.
\end{eg}

The property of invariance ensures that attractors are
approached asymptotically, whereas asymptotically stable sets may
be penetrated in finite time. The relationship between an
asymptotically stable set and an attractor can be formally
connected using the concept of limit sets.

\begin{defn}[$\omega$-limit Sets]
   Given a dynamical system on the state space $E$, $E \subset \mathbb{R}^d$,
   with semi-group representation $\{S_t; t \in \mathbb{R}^{+}\}$, the {\bf
   $\omega$-limit set} of a set $B \subset E$ is defined as
   \begin{equation}
    \omega(B) = \{ x \in E \, ; \, \exists (t_{i}, x_{i}) \in \mathbb{R}^+ \times B, 
        t_{i} \rightarrow \infty, S_{t_{i}}(x_{i}) \to x \hspace{2mm} as \hspace {2mm} 
        i \to \infty \}
   \end{equation}
\end{defn}
It is worth noting that in general
\[ \omega(B) = \bigcup_{b \in B} \omega(b), \]
does not always hold, as can be seen in the following example.

\begin{eg}
  Consider the Bernoulli equation
  \[ \dot{x} - x + x^{3} = 0. \]
  Solutions using a semi-group representation are given by
  \[ S_t(x) = \frac{x}{[ x^{2} + (1 - x^{2})\exp^{-2t}]^{\frac{1}{2}}}. \]
  \begin{figure}[htb]
  \begin{center}
  \leavevmode
  \hbox{
  \epsfxsize=9.0cm
  \epsffile{eps/wlimits.eps}  }
  \protect\caption{$\omega$-Limit Sets}
      \protect\label{wlimitpic}
  \end{center}
  \end{figure}
  Refer to Figure (\ref{wlimitpic}).
  To contrast the difference between limit points and limit sets,
  consider any bounded interval $B=[-b,b]$, with $b>0$. We have $S_t(B) =
  [-S_t(b),S_t(b)]$. Hence, $\omega(B) = [-1,1]$. In particular, note that the set 
  $[-1,1]$ is invariant under the semi-group mapping (that is, $S_t([-1,1]) =
  [-1,1]$ for any $t > 0$). 

  In contrast however, if we consider the limit of any individual state 
  $b \in B$, we find that it approaches one of three distinct limits - $\{-1\}, \{0\}$ 
  or $\{1\}$. Thus if we consider as a limit for $B$ the set consisting of the union of
  limit points generated by individual states within $B$, we find that it is the point 
  set $\{-1,0,1\}$. 

  Note that true to the definition of a limit set it is
  always possible to find sequences $\{x_k;x_k\in [-1,1]\},
  \{t_k;t_k\to\infty\}$ for any $x \in [-1,1]$ so that $S_{t_k}(x_k) \to
  x$.
\end{eg}

\begin{therm}
\label{uastoatt}
  The omega (or positive limit set) set of an asymptotically
  stable set is an attractor.
\end{therm}
\begin{prf}
It can be easily shown that $\omega$-limit sets are $S$-invariant,
that is, $S_t(\omega(B))= \omega(B)$ for all $t>0$. Refer to
\cite{ArPl90}. So the proof for the above theorem involves showing
that the $\omega$-limit set attracts a neighbourhood of itself.
This follows upon consideration of the neighbourhood defined by
the asymptotic property of the original set, and showing that it
is attracted by the $\omega$-limit set. A detailed outline of the
proof is given in Theorem 6.4 of \cite{St94}.
\end{prf}

Using $\omega$-limit sets, the attraction
property of an attractor, (\ref{AA}) may be equivalently written
\[ H^{*} \left(\omega(\mathcal{N}_{\delta}(A_0)),A_0 \right) = 0. \]

The existence of an attractor however, is often difficult to determine from
its neighbourhoods, but can be more easily found with the identification of
{\em absorbing sets}.

\begin{defn}[Absorbing Sets]
\label{abs}
  A nonempty compact and bounded subset $B \subset E$, is called an {\bf
  absorbing set} for a semi-group $\{S_t,t \in \mathbb{R}^+ \}$ on $E$ if
  there exists a $\delta >0$, and a $T=T(\delta) > 0$ such
  that

  \begin{equation}
  S_t(\mathcal{N}_{\delta}(B)) \subseteq B, \qquad \forall t \geq
                         T(\delta).
  \end{equation}
\end{defn}

By definition, it is immediate that an absorbing set is
asymptotically stable. Absorbing sets are also often referred to
as {\em attracting sets}.  It is often simpler to find an
absorbing set within a dynamical system, than to find the actual
attractor. Once the absorbing set $B$ is found, Theorem \ref{uastoatt} can be utilised
to construct the attractor $A_0$ as the $\omega$-limit set of the absorbing
structure.
\begin{equation}
  A_0 = \omega(B).
\end{equation}
An alternative, yet equivalent construction for the attractor
using absorbing sets is given by the well known result below.

\begin{therm}
  \label{attset}
  \begin{equation}
    A_0 = \bigcap_{\tau \geq 0} \overline{\bigcup_{t \geq \tau}S_{(t)}(B)}
  \end{equation}
\end{therm}

\begin{prf}
  \hspace{3mm} \\
  \hspace*{3mm} i) {\em $\omega(B) \subseteq \bigcap_{\tau \geq 0}
  \overline{\bigcup_{t \geq \tau} S_{(t)}(B)}$:} Let $y \in \omega(B)$ Then
  there exists a sequence $\{x_n,t_n\}$ with $x_n \in B$ such that $S_{t_n}(x_n) \to
  y$. Now for  every value of $\tau$, $\exists N = \min \{n \in \mathbb{Z}:t_n > \tau
  \}$  so that $S_{t_n}(x_n) \in \overline{\bigcup_{t \geq \tau} S_{(t)}(B)}$
  $\forall n > N$. As this set is closed and bounded, and thus compact, the
  limit $y$   is also contained therein for every $\tau$. Hence $y \in
  \bigcap_{\tau   \geq 0} \overline{\bigcup_{t \geq \tau} S_{(t)}(B)}$.

  \vspace{2mm}

  \hspace*{3mm} ii) {\em $\bigcap_{\tau \geq 0} \overline{\bigcup_{t \geq
  \tau} S_{(t)}(B)} \subseteq \omega(B)$:} Consider $y \in \bigcap_{\tau
  \geq 0} \overline{\bigcup_{t \geq \tau} S_{(t)}(B)}$. For all values
  of $\tau$, we have $y \in \overline{\bigcup_{t \geq \tau} S_{(t)}(B)}$.
  Now for any given $\epsilon > 0$ we can find a $z \in \bigcup_{t \geq
  \tau} S_{(t)}(B)$, such that $\dist(z,y)<\epsilon$. Then there exists a
  $x \in B$ and $t > \tau$ such that $S_t(x)=z$. Now consider the sequence
  $\epsilon_n \to 0$ and set $\tau_n = \max(t_{n-1},n)$ so that $\tau_n \to
  \infty$. Then for each $\epsilon_n$, there exists a $x_n \in B$ and $t_n
  > \tau_n$ such that $S_{t_n}(x_n) \to y$, with $t_n \to \infty$. Hence $y
  \in \omega(B)$.
\end{prf}

\subsection{Lyapunov Functions - Concepts and Terminology} \label{ALF1}

Often it is not convenient or even possible to construct
explicitly the stable and asymptotically stable sets that
characterise the stability of a dynamical system. However, several
methods are available for determining stability without requiring
direct calculation of these sets. One such method is the
calculation of limit sets. Further, the analysis of boundedness of
solutions \cite{Yo66}, or their prolongations under certain
circumstances, can also provide useful information regarding
system stability.

For dynamical systems generated by ordinary differential
equations, auxiliary functions such as those of Lyapunov type can
also provide a convenient way to characterise the stability of an
arbitrarily shaped set $A_0$ without requiring explicit knowledge
of the solutions of the differential equation. Yoshizawa
\cite{Yo66}, and Rouche/Habets/Laloy \cite{RoHaLa77} detail a
fairly comprehensive summary of the various necessary and
sufficient conditions involving Lyapunov functions for a compact
set $A_0$ to possess some form of stability for both ADE and
NDE's. For reference we will briefly list a few of the relevant
definitions and stability theorems which involve the use of an
auxiliary, Lyapunov type, function for ADE's that will be
used in later chapters. The non-autonomous counterparts will be
presented in a subsequent section. These definitions will be used 
later in the thesis.

The following theorems and definitions pertain to autonomous dynamical
systems arising from ordinary differential equations of the form
\begin{equation} \label{ADE} %ADE - autonomous differential equation
  \dot{x} = f(x),
\end{equation}
where $f$ is required to be continuous and is assumed to be locally
Lipschitz on the state space $E$, ($E$ open and $E
\subset \mathbb{R}^d$).

\begin{defn}[The Lyapunov Function]
The simplest Lyapunov functions used are $C^1$ functions of the type
\[ V(x): E \to \mathbb{R}, \qquad \text{$E \subset \mathbb{R}^d.$} \]
\end{defn}
A Lyapunov function will be assumed to be locally Lipschitz in $x$ on $E$,
that is, there exists a neighbourhood of $x$, $\mathcal{N} \subset
E$, and a constant $L > 0$ such that
\[ |V(x) - V(x')| \leq L \|x-x'\|, \quad \forall x, x' \in \mathcal{N}. \]
In many cases the function $V$ is differentiable
in which case the Lipschitz condition is
automatically satisfied and the rate of change of $V$ may be calculated through the
usual rules of differentiation. If this is not the situation, then the upper right hand Dini 
Derivative form can be employed to characterise the function's rate of change with respect to time. 

\begin{defn}[Upper Right Hand Dini Derivitive]\label{DDdef}
The Upper Dini Right Hand Derivative of $V$ with respect to time is defined by
\begin{align}
  \overline{D}^+_t V[x(t)] &= \overline{\lim_{h\to 0+}} \left[
    \frac{V (x(t+h)) - V (x(t))}{h} \right], \nonumber \\
    &= \overline{D}^+ V(x) f(x), \\ \nonumber
\end{align}
where $x(t)$ is the solution to the differential equation (\ref{ADE}). 
\end{defn}
When $V$ is differentiable, then the Dini derivative is equivalent to the usual time derivative. 

A simplified and equivalent representation for the Dini Derivative is used throughout 
the work by Yoshizawa \cite{Yo66} (where equivalence is also shown) and Kloeden  \cite{Kl98,PkSt97}  to analyse rates of change for Lyapunov functions in autonomous systems. We shall use this representation to investigate Lyapunov stability of perturbed autonomous systems in Chapter
\ref{pertautchapter}. We define this function by
\begin{equation}
  \label{Dini} 
  {D}^+_{(\ref{ADE})} V(x) = \overline{\lim_{h\to 0+}} \left[
    \frac{V(x+hf(x)) - V (x)}{h} \right]. \\
\end{equation}
The subscript (\ref{ADE}) refers to the dynamical system from which the trajectories $x(t)$ are calculated.

It is important to note that since both ${D}^+_{(\ref{ADE})} V(x)$ and the Dini Derivative are equivalent, they may always be exchanged where convenient. We will typically use the notation 
${D}^+_{(\ref{ADE})} V(x)$ when considering the decrescence of $V$ as it quickly and accurately displays the system (\ref{ADE}) for which solutions $x(t)$ belong. This avoids confusion when considering a perturbative analysis. It is also the same notation as that adopted by Yoshizawa and Kloeden. 

Finally, we define a class of monotonically increasing functions $\mathcal{K}$.

\begin{defn}[Class $\mathcal{K}$]
We define a function $a$ to be of class $\mathcal{K}$, that is $a \in
\mathcal{K}$, if $a: \mathbb{R}^+ \to \mathbb{R}^+$ is a continuous,
monotonically increasing function with $a(0)=0$.
\end{defn}

Note that these functions are defined such that they are not necessarily 
strictly monotone.

\subsection{Lyapunov Functions - Stability Theorems} \label{ALF2}

The following theorems apply to dynamical systems generated by (\ref{ADE}),
and are extensions of those referenced in \cite{RoHaLa77}, and \cite{Yo66}.

\begin{therm}[Lyapunov Stability]
  \label{LADES} \hfill \\
  A nonempty compact subset $A_0$ of $E$ is {\em locally stable}
  if and only if there exists a Lyapunov function $V:\mathcal{N}_R(A_0),
  \mapsto \mathbb{R}^+$, for some $R>0$ that satisfies the following
  properties for every $x \in \mathcal{N}_R(A_0)$:
 \begin{enumerate}
  \item $V(x) = \begin{cases}
                      0 & \text{if $x \in A_0$}  \\
                      >0 & \text{if $x \notin A_0$}
                 \end{cases}$,
  \item $V(x)$ is locally Lipschitz in $x$,
  \item ${D}^+_{(\ref{ADE})} V(x) \leq 0$,  
               \qquad \text{$\forall x \in \mathcal{N}_R(A_0)$.}
\end{enumerate}
\end{therm}

The conditions for asymptotic stability are similar, except that
the Dini Derivative must be restricted further so that it only
ever vanishes when the state is within $A_0$ itself.

\begin{therm}[Lyapunov Asymptotic Stability] \label{LADEAS} \hfill \\
  A nonempty compact subset $A_0$ of $E$ is {\em locally asymptotically
  stable} if and only if there exists a Lyapunov function
  $V:\mathcal{N}_R(A_0), \mapsto \mathbb{R}^+$, for some $R>0$ that
  satisfies the following properties for some $c \in \mathcal{K}$ and
  every $x \in \mathcal{N}_R(A_0)$:
  \begin{enumerate}
  \item $V(x) = \begin{cases}
                      0 & \text{if $x \in A_0$}  \\
                      >0 & \text{if $x \notin A_0$}
                 \end{cases}$,
  \item $V(x)$ is locally Lipschitz in $x$,
  \item ${D}^+_{(\ref{ADE})} V(x)
                    \leq - c({\rm dist}(x,A_0))$.
  \end{enumerate}
\end{therm}

If the first condition in both theorems is replaced by an extra
radial unboundedness condition, i.e.
\[ V(x) \geq a({\rm dist}(x,A_0)), \]
for some $a \in \mathcal{K}$, with $a(r) \to \infty$ as $r \to \infty$, and
the neighbourhood $\mathcal{N}_R(A_0)$ increased to include the entire
state space $E$, then the stability in both cases is {\em global}.

The motivation for the radial unboundedness condition is to ensure that the
contour curves (or surfaces) $V(x) = V_{\alpha}$ correspond to closed
curves. If the curves are not closed, it is possible for the state
trajectories to drift away from the equilibrium point, even though the
state keeps passing through contour curves corresponding to smaller and
smaller $V_{\alpha}$'s.

\begin{eg}
Suppose we wish to analyse stability at the origin (i.e. $A_0 = \{
0 \}$, $A_0 \subset \mathbb{R}^2$) of a dynamical system with a
Lyapunov function of the form
\[ V = [ x^2/(1+x^2)] + y^2. \]
The curves $V(x) = V_{\alpha}$ for $V_{\alpha}>1$ are open. Refer
to Figure \ref{radunbpic}. Note that although it may satisfy the
conditions for local stability with a neighbourhood being the
entire state space $E$, it can be seen that divergence of the state can
occur whilst moving through lower and lower $V_{\alpha}$ curves.
Hence using this Lyapunov function, global stability cannot be
assured.
\end{eg}

  \begin{figure}[htb]
  \begin{center}
  \leavevmode \framebox[8.5cm][c] {
  \hbox{
  \epsfxsize=8.0cm
  \epsffile{eps/radunb.eps}  } }
  \protect\caption{Radial Unboundedness Condition}
      \protect\label{radunbpic}
  \end{center}
  \end{figure}

\endinput
