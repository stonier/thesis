\section{Loci Dynamics for $\hat{A}$}\label{seclocidynhatA}

\subsection{Construction of $\mathcal{D}_l$}

Earlier it was noted that the generation of the system of loci
$\mathcal{D}_l$ for constant pullback attractors/pullback asymptotically stable
families $A$ could be extended to time-varying families $\hat{A}$.

The difficulty lies in tracing a single loci analysing attraction to $p \in P$ for an initial state $x_0$. Using the previous method, each point on the loci is generated by
the image of the solution at $p$ from the initial state $x_0$ pulled back in
time. However, to obtain meaningful results regarding the attractive properties
of the dynamical system requires that $x_0$ remains within the local
neighbourhood of attraction as it is pulled back. This is not always
possible for a time-varying attractor with a time-varying neighbourhood of
attraction.

Such a point might also traverse repeatedly across the attractor as it is
pulled back, destroying the illusion of a forward attracting object in
$\mathcal{D}_l$.

\begin{figure}[htb]
\begin{center}
\input{eps/loci6.pstex_t} \caption{Loci Analysis of $\hat{A}$ - I}
\protect\label{loci6}
\end{center}
\end{figure}

\begin{figure}[htb]
\begin{center}
\input{eps/loci7.pstex_t} \caption{Loci Analysis of $\hat{A}$ - II}
\protect\label{loci7}
\end{center}
\end{figure}

This is illustrated in Figures \ref{loci6} and \ref{loci7} for a sinusoidally
varying pullback attractor which pullback attracts solutions exponentially.
In the first diagram, analysis of $x_a$ by pulling it straight back in time is
clearly not possible as it may fall outside of the local neighbourhood of
pullback attraction. In addition, applying the same process to a point on the
attractor, $x_b$ generates a loci that repeatedly traverses across $A(p)$ in
$\mathcal{D}_l$. As a result, with this approach, $A(p)$ is not invariant, nor
even stable in $\mathcal{D}_l$.

An alternative means for generating the loci may be achieved by
using a sequence of uniformly bounded initial states contained within the
neighbourhood system of pullback attraction that are associated with each $x_0
\in \mathcal{N}_{\delta_p}(A(p))$. These sequences are denoted by $\hat{x}(x_0)
= \{x_0(\theta_{-t}p) \in \mathcal{N}_{\delta_p}(A(\theta_{-t}p)); t \geq 0 \}$
for each $x_0 \in \mathcal{N}_{\delta_p}(A(p))$ and defined in the same manner
as in the definition for pullback asymptotic stability (Definition
\ref{PASdef}).

This process is illustrated in Figures \ref{loci8} and
\ref{loci9} for the sinusoidally varying attractor with exponential pullback
attraction used to generate Figures \ref{loci6} and \ref{loci7}. Here $A(p)$ is
both invariant and forward asymptotically stable in $\mathcal{D}_l$.

To ensure continuity and uniqueness of the loci, (essential in requiring that
$\mathcal{D}_l$ is an appropriately defined dynamical system) the initial
sequences $\hat{x}(x_0)$, must also possess continuity and uniqueness. To
guarantee uniqueness, the initial sequences must be constructed so that for all
$t \geq 0$, each element $x_0(\theta_{-t}p)$ is unique to one and only one
initial value $x_0 \in \mathcal{N}_{\delta_p}(A(p))$. That is, if there exists
two sequences $\hat{x}(x_0), \hat{x}(x_1)$ with $x_0(\theta_{-t}p) =
x_1(\theta_{-t}p)$ for some $t \geq 0$, then $x_0 = x_1$, and $\hat{x}(x_0) =
\hat{x}(x_1)$.

\begin{figure}[htb]
\begin{center}
\input{eps/loci8.pstex_t} \caption{Time Varying Loci Analysis of $\hat{A}$ - I}
\protect\label{loci8}
\end{center}
\end{figure}

\begin{figure}[htb]
\begin{center}
\input{eps/loci9.pstex_t} \caption{Time Varying Loci Analysis of $\hat{A}$ - II}
\protect\label{loci9}
\end{center}
\end{figure}

Finally, the loci mapping may be generated as before, where
each loci is associated with an initial sequence $\hat{x}(x_0) \in
\mathcal{N}_{\delta_p, \hat{A}}$ with $x_0 \in \mathcal{N}_{\delta_p}(A(p))$.
Each point on the loci is represented by the couple $(\theta_{-t}p, \Phi_{(t,
\theta_{-t}p)}(x_0(\theta_{-t}p))$ parameterised for any $t \geq 0$.

A cocycle representation of the form
\begin{equation}\label{eqtvloci}
  \phi_{(t, 0)}(x_0) = \Phi_{(t, \theta_{-t}p)}(x_0(\theta_{-t}p)).
\end{equation}
is used to define loci trajectories associated with the initial value $(0,
\hat{x}(x_0))$.

\begin{lemma}[The Loci Dynamical System $\mathcal{D}_l$]
The loci mappings defined by (\ref{eqtvloci}) form a dynamical
system $\mathcal{D}_l$.
\end{lemma}
\begin{proof}
  Utilising the above construction, the proof follows similarly to that for
Lemma \ref{lemDlconst}.
\end{proof}

{\bf Example 1: } Initial sequences may be formulated by introducing a
co-ordinate system on the local neighbourhood system of $\hat{A}$. A two
dimensional system for example, could utilise a neighbourhood co-ordinate
represented by the couple $(\alpha, \varphi)$, based on the
co-ordinate distance ($\alpha$) of a point in the neighbourhood away from an
invariant solution passing through $\hat{A}$, and an angle of rotation
($\varphi$)  centred on the invariant solution. Each initial sequence
$\hat{x}(x_0)$ may then be uniquely defined by the co-ordinate value of $x_0$,
and the subsequent loci dynamical system generated based on these values.

{\bf Example 2: } If the elements of $\hat{A}$ constitute a single point in
$\mathbb{R}^d$ for each $p \in P$, then the transformation defined by
$x_0(\theta_{-t}p) = A(\theta_{-t}p) + (x_0 - A(p))$ provides a suitable
construction for defining continuous and unique initial sequences, and the
generation of the loci dynamical system $\mathcal{D}_l$.

{\bf Remark:} The manner of generation of the initial sequences that determine
a pullback analysis is abstractly arbitrary. In fact, what reasoning determines
that single states are pulled straight back in time for a pullback analysis of
systems with a constant attracting object $A$? The only restrictions placed on
the sequences defined above is that of uniform boundedness and containment
within a local neighbourhood. This is to ensure initial sequences cannot be
chosen so that the rate of pullback attraction isn't nullified by a quickly
diverging sequence, and as a consequence, any confirmation of pullback
asymptotic behaviour erroneously missed.

Much of the pullback asymptotic theory given so far concerns a pullback
analysis on $\delta$-neighbourhoods (ideal for  a local analysis), and this
automatically restricts the construction of initial sequences required if
pullback analysis of single states, or bounded sets is to be undertaken. Since
the use of $\delta$-neighbourhoods as a tool to correctly capture the effects of
pullback attraction has proven useful, any subsequent conditions we will place
on the construction of the initial sequences to correctly determine a pullback
analysis, will endeavour to reflect $\delta$-neighbourhood pullback asymptotic
theory and avoid any conflicts that may arise.

\subsection{Eventual Forward Asymptotic Stability in $\mathcal{D}_l$}

As mentioned, the actual composition of the uniformly bounded initialising
sequences as defined above is arbitrary, and although it provides the
means to properly establish the loci dynamical system $\mathcal{D}_l$, it does
not guarantee that properties of invariance or stability are transferred to the
loci dynamical system.

Nevertheless, arbitrarily chosen, they still ensure $A(p)$ retains properties of
eventual asymptotic stability in $\mathcal{D}_l$. This is analogous to the
result given in Lemma \ref{lempastoefas} for constant sets $A$.

\begin{lemma}[Eventual Asymptotic Stability in $\mathcal{D}_l$]
\label{lempastoefashatA}
If $\hat{A}$ is globally pullback asymptotically stable, then for each $p \in
P$, $A(p)$ is eventually globally forward asymptotically stable in
$\mathcal{D}_l$.
\end{lemma}
\begin{prf}
The proof follows similarly to the proof for Lemma \ref{lempastoefas} with a
pullback analysis based upon pulling back the initialising sequences rather than
a fixed point.
\end{prf}

\subsection{Pullback Analysis of Bounded Sets}

Before proceeding to extend the notion of Loci Stability for time varying sets
$\hat{A}$, a means to analyse pullback properties of bounded and compact
sets (compare with the pullback analysis of the bounded set $B$ in
Definition \ref{defnlocistab}) is required.

Consider any bounded and compact set $B$ in a suitable neighbourhood of
$\hat{A}$ at $\theta_{-t^*}p$ for some $t^*>0$.

An essential requirement for pullback analysis of $B$ is that the
initialising sequences originating from $B$ correspondingly define a family of
bounded, compact and \textit{connected} sets associated with $B$. We establish
this as a condition required of the initialising sequences.

\textbf{A1 - } For each $p \in P$, and any
bounded, compact and connected set $B_0$ at $p$, the initial
sequences defined by $\hat{B}(B_0)$ generate a family of bounded, compact and
connected sets defined by $\hat{B}(B_0) = \{ B_0(\theta_{-t}p) ; t \geq 0 \}$
where
\[ B_0(\theta_{-t}p) = \bigcup_{x_0 \in B_0} x_0(\theta_{-t}p). \]

This concept extends also to allow generation of a family of bounded, compact
and connected sets associated with some $B$ at $t^*$ by tracing the initial
sequences passing through $B$. This family is denoted by $\hat{B}(B_{t^*}) =
\{B_{t^*}(\theta_{- \tau} p); \tau \geq t^* \}$ where
\[ B_{t^*}(\theta_{- \tau} p) = \bigcup_{x \in B} x_0(\theta_{- \tau}p). \]
where each $x \in B$ is the $(\theta_{-t^*}p)$'th element of some initial
sequence $\hat{x}(x_0)$.

Note that for a pullback analysis of $B$ from $t^*$ only elements
further back in time are considered.

\begin{figure}[htb]
\begin{center}
\input{eps/loci10.pstex_t} \caption{Pullback Analysis of Bounded Sets}
\protect\label{loci10}
\end{center}
\end{figure}

Essentially this procedure is automatically satisfied for the usual
pullback analysis of a $\delta$-neighbourhood of $\hat{A}$, although some extra
care is needed for general sets $B$ since the appropriate construction
isn't neatly defined by a $\delta$ value.

An actual physical illustration of what is required is simpler to grasp, and
should be immediately obvious. An example of this is presented in Figure
\ref{loci10} with the additional marking of $B$'s image at $p$ which will be
discussed shortly in connection with loci stability.

\subsection{Loci Stability}

If the initial sequences for a pullback analysis are generated in compliance
with \textbf{A1}, then loci stability for time varying $\hat{A}$ may be defined
as follows.

\begin{defn}[Loci Stability - $\hat{A}$]\label{defnlocistabhatA}
A uniformly bounded family of compact sets $\hat{A} = \{ A(p); p \in P \}$ is
said to be {\bf Loci Stable} if a pullback analysis in compliance with
\textbf{A1} can be made so that for any $\e > 0$, there exists a
$\hat{\delta} = \{ \delta_p \in \mathbb{R}^+ ; p \in P \}$ that satisfies the
following condition for any bounded and compact set $B \subset \mathbb{R}^d$.
If $B$ satisfies the property
\[ \Phi_{(t^*, \theta_{-t^*}p)}(B) \subseteq
                     \mathcal{N}_{\delta_p}(A(p)), \]
for some $t^* > 0$ and $p \in P$, then
\[ H^*(\Phi_{(t, p)}(B_{t^*}(\theta_{-t}p)), A) < \e \hspace{1cm} \forall t \geq
             t^*. \]
\end{defn}

{\bf Remark : } Note that not all choices of initial sequences in
compliance with \textbf{A1} for a pullback analysis of loci
stability will be valid. It is only required that a single choice
is available. A finite pullback asymptotically stable set $A$ for
example, may be loci stable if solutions are pulled straight back,
yet it is easy to construct sequences that cause the images (or
loci) to criss cross $A$ in $\mathcal{D}_l$. Nevertheless, the
fact that a particular choice exists is enough to characterise the
properties of loci stability. That is, once the image is close, it
stays relatively close.

\subsection{Forward Asymptotic Stability in $\mathcal{D}_l$}

Pullback asymptotic stability of $\hat{A}$ together with loci stability as
defined above ensure attraction \textit{and} stability of the loci around $A(p)$
in $\mathcal{D}_l$ in a similar manner to the construction in Section
\ref{ssecLoci2}.

If the initial sequences are constructed as above,  the proof for the theorem
below then follows similarly to the proof for Theorem \ref{thmpastofas}.

\begin{therm}[Forward Asymptotic Stability in $\mathcal{D}_l$]
\label{thmpastofashatA}
If $\hat{A}$ is globally pullback attracting and loci stable, then for
each $p \in P$, $A(p)$ is globally forward asymptotically stable in
$\mathcal{D}_l$.
\end{therm}

\subsection{Forward Attractors in $\mathcal{D}_l$}

One further condition is placed on the construction of the initial sequences
for the loci mappings if $\hat{A}$ is a pullback attractor:

\textbf{A2 - } For every $x_0 \in A(p)$, the associated initial sequence is the
invariant solution passing through $\hat{A}$ backwards in time. That is,
$\hat{x}(x_0) = \{x_0(\theta_{-t}p) \in A(\theta_{-t}p); \Phi_{(t, \theta_{-t}p)
}(x_0(\theta_{-t}p)) = x_0, t \geq 0 \}$.

This condition ensures invariance of the forward attractor $A(p)$ in
$\mathcal{D}_l$.

\begin{lemma}[Forward Attractors in $\mathcal{D}_l$]
Let $\hat{A}$ be a pullback attractor for the non-autonomous dynamical system
(\ref{NDEeq}). Then for each $p \in P$, if the loci are formulated in
compliance with \textbf{A2}, $A(p)$ is a forward attractor in $\mathcal{D}_l$.
\end{lemma}

This concludes an analysis of the dynamical system $\mathcal{D}_l$. The
following section proceeds with an investigation on the effects of discretising
the original dynamical system as observed in $\mathcal{D}_l$.

\endinput
