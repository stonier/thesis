\section{Forward Lyapunov Theory}
\label{FLyapsec}

At present there are several results concerning forwards stability of time
varying families of sets (refer to Yoshizawa, \cite{Yo66}), though these results
are infrequently used due to lack of information concerning attraction to the
family of sets anywhere except approaching infinity. The results for
Forward Lyapunov Theory for families of sets are similar, in general terms, to those of  Yoshizawa. They are extended here to incorporate an analysis for local
neighbourhoods, and to allow for general parameter sets $P$ (as opposed to the
construction used in \cite{Yo66} where $P=\mathbb{R}^+$).

\subsection{Sufficiency Theorems}

\begin{therm}[Forward Stability] \hfill \\ \label{fsthm}
  Given a family of uniformly bounded compact sets $\hat{A} =
  \{A(p);p   \in P \}$, suppose there exists a Lyapunov function $V: P \times
  \hat{\mathcal{N}}_{\e, \hat{A}} \to \mathbb{R}$ for some $\e > 0$ which
  satisfies the following conditions:
\begin{itemize}
  \item[a)] $V(p, x) = 0$ for each $p \in P$ and $x \in A(p),$
  \item[b)] $a(dist( x, A(p))) \leq V(p, x),$
    where $a \in \mathcal{K}$,
  \item[c)] $\overline{D_t}^+ V(p, x) \leq 0$.
  \item[d)] $V(p, x)$ is continuous in both variables and locally Lipschitz in
    $x$.
\end{itemize}
Then $\hat{A}$ is \textbf{forward stable}.
\end{therm}
\begin{prf} \hfill \\
Let $p \in P$ be arbitrarily chosen. For any $\e > 0$ we may choose a
$\delta_p(\e) > 0$ such that
\[ \dist(x_0, A(p)) < \delta_p \implies V(p, x_0) < a( \e), \]
because of the continuity of $V$ in the state variable. Now suppose that
some solution $\Phi_{(t^*, p)}(x_0)$ with $\dist(x_0, A(p)) < \delta_p$ satisfies
$\dist( \Phi_{(t^*, p)}(x_0), A(\theta_{t^*}p)) = \e$ at some $t^* > 0$.
We have by property {\em b)},
\begin{align*}
  a( \e ) &\leq V(\theta_{t^*}p, \Phi_{(t^*, p)}(x_0) ), \\
  &\leq V(p, x_0), \\
  &< a( \e).
\end{align*}
This is a contradiction. Hence all trajectories with initial values
$x_0 \in \mathcal{N}_{\delta_p}(A(p))$ must remain within an
epsilon neighbourhood of $\hat{A}$. Thus $\hat{A}$ is forward stable.
\end{prf}

\begin{therm}[Uniform Forward Stability] \hfill \\
\label{ufsthm}
  Given a family of uniformly bounded compact sets $\hat{A} = \{A(p);p
  \in P \}$, suppose there exists a Lyapunov function $V: P \times
  \hat{\mathcal{N}}_{\e, \hat{A}} \to \mathbb{R}$ for some $\e > 0$ which
  satisfies the following conditions:
\begin{itemize}
  \item[a)] $V(p, x) = 0$ for each $p \in P$ and $x \in A(p),$
  \item[b)] $a(dist( x, A(p))) \leq V(p, x) \leq b( dist( x, A(p)))$
    where $a, b \in \mathcal{K}$,
  \item[c)] $\overline{D_t}^+ V(p, x) \leq 0$,
  \item[d)] $V(p, x)$ is continuous in both variables and locally Lipschitz in
    $x$.
\end{itemize}
Then $\hat{A}$ is \textbf{uniformly forward stable}.
\end{therm}
\begin{prf} \hfill \\
  Choose $\delta > 0$, $\delta = \delta(\e)$ so that $b(\delta) < a(\e)$. Note
  that this is possible since $a, b$ are both continuous and class
  $\mathcal{K}$.

  We conclude that solutions originating from within
  $\mathcal{N}_{\delta}(A(p))$, for arbitrary $p \in P$, must remain within the
  $\e$ - neighbourhood of $\hat{A}$ for all future times.

To see this, assume otherwise. That is, there exists some $p
\in   P$, $x \in   \mathcal{N}_{\delta}(A(p))$, and a time $t^*>0$ such that
  \[ dist( \Phi_{(t^*, p)}(x), A(\theta_{t^*}p) ) = \e. \]
  Then we have,
  \[ a(\e) \leq V(\theta_{t^*}p, \Phi_{(t^*, p)}(x)) \leq V(p, x) \leq b(\delta)
               < a(\e), \]
  which is a contradiction. Hence the original assertion must be valid, and
  $\hat{A}$ is uniformly forward stable.
\end{prf}

\begin{therm}[Uniform Forward Asymptotic Stability] \hfill \\
\label{fasthm}
  Given a family of uniformly bounded compact sets $\hat{A} = \{A(p);p
  \in P \}$, suppose there exists a Lyapunov function $V: P \times
  \hat{\mathcal{N}}_{\e_o, \hat{A}} \to \mathbb{R}$ for some $\e_0 > 0$ which
  satisfies the following conditions:
\begin{itemize}
  \item[a)] $V(p, x) = 0$ for each $p \in P$ and $x \in A(p),$
  \item[b)] $a(dist( x, A(p))) \leq V(p, x) \leq b( dist( x, A(p)))$
    where $a, b \in \mathcal{K}$,
  \item[c)] $\overline{D_t}^+ V(p, x) \leq -cV(p, x)$ for some constant $c >
    0$,
  \item[d)] $V(p, x)$ is continuous in both variables and locally lipschitz in
    $x$.
\end{itemize}
Then $\hat{A}$ is \textbf{uniformly forward asymptotically stable}.
\end{therm}
\begin{prf} \hfill \\
  By Theorem \ref{fsthm}, $\hat{A}$ is uniformly forward stable. Thus there
  exists some $\delta_0 = \delta_0(\e_0)$ such that for all $p \in P$, $x \in
  \mathcal{N}_{\delta_0}(A(p))$, solutions are guaranteed to exist
  and remain within the domain of definition for $V(p, x)$. That is, for all
  $t > 0$,
  \[ dist( \Phi_{(t,p)}(x), A(\theta_{t}p) < \e_0. \]
  Consider some $\e > 0$ with $\e \leq \delta_0$. Then there exists
  a $\delta = \delta(\e)$ as defined for uniform forward stability. It will be
  shown that every solution from $x \in \mathcal{N}_{\delta_0}(A(p))$,
  satisfies

  \[ \dist(\Phi_{(t^*, p)}(x),A(\theta_{t^*}p)) < \delta(\e), \]
  at some time $t^* > 0$.

  Assume that this is not the case. Then there exists some
  $x \in \mathcal{N}_{\delta_0}(A(p))$ such that
  \[ \dist(\Phi_{(t, p)}(x),A(\theta_{t}p)) > \delta(\e), \hspace{1cm} \forall
   t > 0. \]
  Now by c),
  \[ V( \theta_t p, \Phi_{(t, p)}(x)) \leq e^{-ct}V(p, x), \]
  for all $t>0$. Let $\displaystyle T = - \ln [
  a(\delta)/b(\delta_0) ] /c  $. Then
  \[ a(\delta) \leq V( \theta_t p, \Phi_{(t, p)}(x)) \leq e^{-ct}V(p, x) <
    a(\delta), \]
  for all $t > T$. Consequently,  we have a contradiction. Thus there exists
  some $t^* \leq T$   such that $\dist(\Phi_{(t^*, p)}(x),A(\theta_{t^*}p)) <
  \delta(\e)$, and  hence for all $t > T$ we have (by definition of
   $\delta(\e)$)
  \[ \dist(\Phi_{(t, p)}(x),A(\theta_{t}p)) < \e. \]
  Since this argument holds for each $p \in P$ and all $x \in
  \mathcal{N}_{\delta_0}(A(p))$ we have the required result. Note that $T$
  depends on $\e$ only (through $\delta(\e)$), as needed for uniformity.
\end{prf}

{\bf Remark:} The required asymptotic attraction can be achieved
with alternative (usually slightly weaker) conditions on the Dini
Derivative (for example, with $\overline{D_t}^+ V(p, x) < 0)$) .
The manipulation of the proof follows in a similar manner.

\subsection{Converse Theorems}

\begin{therm}[Uniform Forward Asymptotic Stability] \hfill \\
\label{confuasthm}
Suppose the dynamical system (\ref{NDEeq}) possesses a family $\hat{A} =
\{A(p) ; p \in P \}$ that is uniformly forward asymptotically stable. Then
there exists a Lyapunov function $V:P \times \mathcal{N}_{K,\hat{A}}$ defined
on a neighbourhood of $\hat{A}$, $\mathcal{N}_{K,\hat{A}}$ which satisfies:
\begin{itemize}
  \item[a)] $V(p, x) = 0$ for each $p \in P$ and $x \in A(p),$
  \item[b)] $a(dist( x, A(p))) \leq V(p, x) \leq b( dist( x, A(p)))$
    where $a, b \in \mathcal{K}$,
  \item[c)] $\overline{D_t}^+ V(p, x) \leq -cV(p, x)$ for some constant $c >
    0$,
  \item[d)] $V(p, x)$ is continuous in both variables and locally Lipschitz in
    $x$.
\end{itemize}
\end{therm}
\begin{prf} \hfill \\
  Since $\hat{A}$ is a uniformly asymptotically stable family of sets, given any
$\e^* > 0$, then there exists a $\delta(\e^*) > 0$ that satisfies the properties
of Lemma \ref{intro1lem}.

For this $\e^*>0$, we define:
\begin{align*}
  L(p) &= L(p, \e^*), \\
\intertext{as the Lipschitz constant for $f$ on a $\e^*$ bounded region of
the state space;}
 T &= T( \e ), \\
\intertext{as defined in property ii) of Lemma \ref{intro1lem};}
  F(p, \e)  &= 1 + \max |f(\theta_t p,x)|, \\
\intertext{where the maximum is taken over all $-T(\e) \leq t \leq T(\e)$ and $x
\in \mathcal{N}_{\e^*}A(\theta_t p)$;}
  \mathcal{A}(p, \e) &= e^{cT(\e)}2 F(p, \e) \exp \left( \int_0^{T(\e)}
        L(\theta_s p)ds \right), \\
\intertext{defined for any arbitrarily chosen $c> 0$.}
\end{align*}

  Utilising a slightly modified form of the result by J. Massera, detailed
  in \cite{Yo66}, there exists functions $l, g$ satisfying $l(p) > 0,$ $0 < g(\e)
  \leq 1$ for $\e > 0$ and $g(0) = 0$, such that
\begin{equation} \label{eqmass}
   g(\e)\mathcal{A}(p,\e) \leq l(p).
\end{equation}
  Finally we begin composing the Lyapunov function for the required purpose. For
$n = 1, 2 \dots$, we define $V_n(p, x)$ for each $p \in P$   and $x \in
\mathcal{N}_{\delta}(A(p))$ by:
\begin{equation*}
  V_n(p, x) = g(1/n) \sup \left\{ D_n( dist( \Phi_{(\tau, p)}(x),
  A(\theta_{\tau} p) ) ) e^{c \tau } ; \tau \geq 0 \right\}.
  \end{equation*}
  Here the function $D_n( r )$ is defined as
  \begin{equation*}
  D_n( r ) = \begin{cases}
  r - 1/n & ( r \geq 1/n ), \\
  0 & ( 0 \leq r \leq 1/n ).
  \end{cases}
  \end{equation*}

  {\em i)}
  From the definition of $V_n(p, x)$ and the invariance of $A(p)$ it is clear
  that for each $p \in P$,
  \begin{equation}
  V_n(p, x) = 0 \hspace{1cm} \forall x \in A(p).
  \end{equation}

  {\em ii) Lower Bound -} Define $a_n(r)$ by
  \[ a_n(r) = g(1/n) A_n(r), \]
  where $A_n(r)$ is given by
  \begin{equation*}
  A_n( r ) = \begin{cases}
  1/\left(n(n-1)\right) & (r \geq 1/(n-1) ), \\
  r - 1/n & ( 1/n \leq r \leq 1/(n-1) ), \\
  0 & ( 0 \leq r \leq 1/n ).
  \end{cases}
  \end{equation*}
  Note that $A_n(r) \leq D_n(r)$, and that $a_n$ is a non-negative
  monotonically increasing function and is continuous with respect to $r$.
  Then if we set $r = dist( x, A(p) )$, we have
  \begin{align*}
  V_n(p, x) &\geq g(1/n) D_n(dist( x, A(p) ) ), \\
  &\geq a_n(r). \\
  \end{align*}

  {\em iii) Upper Bound -} Note that
  \[ D_n( dist( \Phi_{(\tau, p)}(x), A(\theta_{\tau} p) ) ) = 0 \]
  for all $\tau > T(1/n)$. Using this property together with that of
  uniform forward stability for $\hat{A}$ we have,
  \begin{align*}
  V_n(p, x) &= g(1/n) \sup \{ D_n( dist(
  \Phi_{(\tau, p)}(x), A(\theta_{\tau} p) ) ) e^{c \tau }; \tau \geq 0 \}, \\
   &\leq g(1/n) \sup \{ D_n( \e^* )
   e^{c \tau } ; 0 \leq \tau \leq T(1/n) \}, \\
   &\leq g(1/n) e^{cT(1/n)} \e^*( dist(x, A(p)) ), \\
   &\leq g(1/n) \mathcal{A}(p^*, 1/n) \e^*( dist(x, A(p)) ), \\
   &\leq l(p^*) \e^*( dist(x, A(p)) ),
  \end{align*}
  for any arbitrarily chosen $p^* \in P$.
  Here $\e^*( dist(x, A(p)) )$ refers to the inverse function associated with
the function $\delta =   \delta( \e^* )$ corresponding to the uniform forward
stability of $\hat{A}$. We have also used the fact that $e^{cT(1/n)} <
\mathcal{A}(p^*, 1/n)$ for arbitrarily chosen $p^* \in P$.

  {\em iv) Decrescence -} Let $h > 0$ be some constant. Now for any
  $x \in \mathcal{N}_{\delta}(A(p))$, let $x^*$ denote the state at some time
  $h$ later. That is, $x^* = \Phi_{(h, p)}(x)$. Then
  \begin{align*}
  V_n( \theta_h p, x^* ) &= g(1/n) \sup \{ D_n(
  dist( \Phi_{(\tau, \theta_h p)}(x^*), A(\theta_{\tau+h} p) ) )
  e^{c \tau } ; \tau \geq 0 \} \\
  &= g(1/n) \sup \{ D_n(
  dist( \Phi_{(\tau + h, p)}(x), A(\theta_{\tau+h} p) ) )
  e^{c \tau } ; \tau \geq 0 \} \\
  &= g(1/n) \sup \{ D_n(
  dist( \Phi_{(\tau, p)}(x), A(\theta_{\tau} p) ) ) e^{c \tau }
  e^{-ch} ; \tau \geq h \} \\
  &\leq e^{-ch} V_n(p, x). \\
  \end{align*}
  Taking the Dini derivative of $V_n(p, x)$ for solutions at $p, x$ we obtain,
  \begin{align*}
  \overline{D_t}^+ V_n(p, x) &= \overline{\lim_{h \to 0^+}} \frac{V_n(
  \theta_h p, x^* ) - V_n(p, x)}{h}, \\
  &\leq \overline{\lim_{h \to 0^+}} \frac{(e^{-ch} - 1) V_n(p, x)}{h}, \\
  &= - c V_n(p, x).
  \end{align*}

  {\em v) Continuity -} Let $p, p' \in P$, such that $\theta_{t^*}p' = p$ for
  some $t^* > 0$, and $x \in \mathcal{N}_{\delta}(A(p)), x' \in
  \mathcal{N}_{\delta}(A(p'))$. Then
  \begin{align*}
  |V_n(p, x) - &V_n( p',x')| \leq g(1/n) | \sup \{
    e^{c\tau} D_n(\Phi_{(\tau, p)}(x)) ; \tau \geq 0 \}\\
  &\hspace{1cm}  - \sup \{
    e^{c\tau} D_n(\Phi_{(\tau, p')}(x')) ; \tau \geq 0 \} |, \\
  &\leq g(1/n) \sup \{ |\Phi_{(\tau, p)}(x) - \Phi_{(\tau,
    p')}(x')| e^{c\tau} ; 0 \leq \tau \leq T \}, \\
  &\leq g(1/n) e^{cT(1/n)} \sup \{ |\Phi_{(\tau, p)}(x) - \Phi_{(\tau,
    p)}(X)|  \\
  & \hspace{2cm} + |\Phi_{(\tau+t^*, p')}(x') - \Phi_{(\tau, p')}(x')| ; 0 \leq
    \tau \leq T \}, \\
  \end{align*}
  where $X = \Phi_{(t^*, p')}(x')$.

  Set
  \[ \mathcal{L}(p, 1/n) = \exp \left( \int_0^T L(\theta_{-s} p) ds \right), \]
  for ease of notation, noting that $\mathcal{L}(p, 1/n) > 1$ for all $p, n$.
  Now from Lemma \ref{intro2lem} we have
  \begin{align}
  \label{ctsbnd}
  |V_n(p, x) - V_n( p',x')| &\leq g(1/n) e^{cT(1/n)} \left\{ F(p, 1/n) |p-p'|
    \right. \notag \\
  &\hspace{1cm} \left. + \mathcal{L}(p, 1/n) |X - x| \right\}.
  \end{align}
  Let $t^*$ be small enough so that $t^* < T(1/n)$, then
  \begin{align*}
  |X - x| &\leq (|X - x'| + |x' - x|), \\
  &\leq (F(p, 1/n)<p, p'> + |x' - x|. \\
  \end{align*}
  Substituting back into (\ref{ctsbnd}), we have
  \begin{align*}
  |V_n(p, x) & - V_n( p',x')| \\
   & \leq g(1/n) e^{cT(1/n)} \left\{ F(p, 1/n) \right.
             (1 + \mathcal{L}(p, 1/n))<p, p'> \\
   & \hspace{1cm} \left. + \mathcal{L} (p, 1/n)|x - x'| \right\}, \\
  &\leq g(1/n) e^{cT(1/n)} F(p, 1/n) 2 \mathcal{L}(p, 1/n)) \\
  &\hspace{1cm} (<p, p'> + |x - x'|), \\
  &\leq g(1/n) \mathcal{A}(p, 1/n) (<p, p'> + |x - x'|), \\
  &\leq l(p) (<p, p'> + |x - x'|), \\
  \end{align*}
  for all $p'$ close enough to $p$.  Hence each $V_n(p, x)$ is continuous and
  locally Lipschitz with respect to both $p$, and $x$.

  Finally we define the Lyapunov function $V$ by
  \[ V(p, x) = \sum_{n=1}^{\infty} \frac{1}{2^n} V_n(p, x). \]
  Note that convergence of this is automatically ensured as a consequence of
  iii). Properties a) - d) will be verified sequentially.

  {\em a)} Obviously from i) we have for each $p \in P$ and all $x \in A(p)$
  \[ V(p, x) = 0. \]

  {\em b-i) Lower Bound - } From ii), if we set
  \[ a(r) = \sum_{n=1}^{\infty} \frac{1}{2^n} a_n(r), \]
  we have $a(r) \in \mathcal{K}$. Clearly $a(0)=0$. Also
  $a(0) > 0$ for $r>0$ since for any $r$ there exists an $n$ such that $(1/n) <
  r$ and hence $a(r) > a_n(r) = g(1/n) (r - 1/n)$. Since each
  $a_n$ is upper bounded by $1/n(n-1)$, and $g(\cdot) < 1$, the
  Weierstrass M-test can be used to conclude that the infinite
  series is continuous. Finally, to see that it is a
  lower bound for $V(p,x)$,
  \begin{align*}
  V(p, x) &= \sum_{n=1}^{\infty} \frac{1}{2^n} V_n(p, x), \\
    &\geq \sum_{n=1}^{\infty} \frac{1}{2^n} a_n(\dist(x, A(p)), \\
    &\geq a(\dist(x,A(p)). \\
  \end{align*}

  {\em b-ii) Upper Bound - } From iii) we have
  \begin{align*}
  V(p, x) &= \sum_{n=1}^{\infty} \frac{1}{2^n} V_n(p, x), \\
    &\leq \sum_{n=1}^{\infty} \frac{1}{2^n} l(p^*) \e^*(\dist(x,
            A(p))), \\
    &\leq l(p^*) \e^*(\dist(x, A(p))), \\
    &\leq b(\dist(x, A(p))). \\
  \end{align*}
  where $b(r) = l(p^*) \e^*(r)$. Note that the function $b \in \mathcal{K}$
since   $\e^*(\delta)$ is continuous and monotonically increasing from zero.

  {\em c) Decrescence - } From iv),
  \begin{align*}
  V( \theta_h p, \Phi_{(h, p)}(x) ) &= \sum_{n=1}^{\infty} \frac{1}{2^n}
       V_n( \theta_h p, \Phi_{(h, p)}(x) ), \\
    &\leq e^{-ch} \sum_{n=1}^{\infty} \frac{1}{2^n} V_n(p,x), \\
    &\leq e^{-ch} V(p, x). \\
  \end{align*}

  Again, taking the Dini derivative for $V(p, x)$, we arrive at the required
  result.

  {\em d) Continuity and Lipschitz Properties of V -} This follows
  directly from the continuity and Lipschitzness of each $V_n$ in v).
\end{prf}

\begin{therm}[Forward Equi-Asymptotic Stability] \hfill \\
\label{confasthm} Suppose the dynamical system (\ref{NDEeq})
possesses a family $\hat{A} = \{A(p) ; p \in P \}$ that is forward
equi - asymptotically stable. Then there exists a Lyapunov
function $V$ defined on a neighbourhood of $\hat{A}$, $\mathcal{N}_{K,\hat{A}}$,
satisfying
\begin{itemize}
  \item[a)] $V(p, x) = 0$ for each $p \in P$ and $x \in A(p),$
  \item[b)] $a(\dist( x, A(p))) \leq V(p, x)$
    where $a\in \mathcal{K}$,
  \item[c)] $\overline{D_t}^+ V(p, x) \leq -cV(p, x)$ for some constant $c >
    0$,
  \item[d)] $V$ is continuous in both variables and locally Lipschitz in
    $x$.
\end{itemize}
\end{therm}
\begin{prf} \hfill \\
The proof is similar to that of Theorem \ref{confuasthm}
where the rate of attraction is now dependent on $T=T(p, \e)$ and
the neighbourhoods $\delta = \delta(p, \e)$. The resulting
properties follow identically with the exception that the upper
bounding function $b \in \mathcal{K}$, does not hold.
\end{prf}

\subsection{Complete Lyapunov Theory}
\label{CLiapssec}

Recalling that uniformity guarantees complete stability/asymptotic
stability (Lemmas \ref{upscslem}, \ref{ufscslem},
\ref{upascaslem}, \ref{ufascaslem}) it suffices to show using
Lyapunov functions that if a family of sets $\hat{A} = \{A(p); p
\in P\}$ is uniformly forward stable/asymptotically stable, then
its stability is complete. That is, it is also uniformly pullback
stable/asymptotically stable. Similarly, the converse theorems
will also hold for completely stable/asymptotically stable
families of sets that are uniform.

The current development of \textit{pullback attractors} (\cite{CrFl94},
\cite{FlSc96}, \cite{FS95}, \cite{Kl98}) generally involve structures that are
uniform in nature. For these, use of forward stability theory and Lyapunov
theory for such structures is perfectly applicable.

For completeness, the theorems may be applied for uniform complete
stability/asymptotic stability and are referenced below.

\begin{therm}[Sufficiency]
Given a family of uniformly bounded compact sets $\hat{A} = \{A(p);p \in P\}$,
and a Lyapunov function $V(p, x)$ satisfying the conditions in either Theorem
\ref{fsthm} or \ref{fasthm} respectively, implies uniform complete stability or
uniform complete asymptotic stability of $\hat{A}$ respectively.
\end{therm}

\begin{therm}[Converse]
If a family of uniformly bounded compact sets $\hat{A} = \{A(p);p \in P\}$ is
uniformly completely asymptotically stable, then there exists a function
$V(p,x)$ satisfying conditions a)-d) in Theorem \ref{confuasthm}.
\end{therm}

\endinput
