\section{Discrete Stability and Absorbing Neighbourhoods}

The following definitions and theorems for discrete systems are analogous to those presented
in Section \ref{ANsec} and are listed below for completeness and as a reference.
They are equally valid for representations of difference equations as well as
for discretised continuous systems.

In the following we use the usual notation (for example, $\hat{A}$) to
represent a family of sets. However to accurately distinguish the discrete
family appropriate for discretisations of continuous systems in future we will
use $\hat{A}^h, \hat{A}^{\bf h}$ for constant and variable time-step
discretisations respectively.

\begin{defn}[Discrete Forward Stability] \label{DFSdef}
   A discrete family $\hat{A} = \{A(p);p \in P\}$ of
   uniformly bounded compact subsets of $E$, is {\bf
   forward stable} for the discrete cocycle
   $\{\Phi_{(n, p)}; n \in \mathbb{Z}^{+}, p \in P\}$ on $E$
   if for any $\e > 0$ there exists a
   $\delta$-neighbourhood system
   $\hat{\mathcal{N}}_{(\hat{\delta} ,\hat{A})} =
   \{\hat{\mathcal{N}}_{\delta_p, \hat{A}}; \delta_p \in \hat{\delta}, p \in P
   \}$ defined by a delta set $\hat{\delta} = \{\delta_p \in \mathbb{R}^+; p
   \in P\}$, so that for each $p \in P$
   \[ H^* \left(\Phi_{(n, p)}
           (\mathcal{N}_{\delta_p}(A(p))), A(\theta_n p) \right) < \e \qquad
           \forall n \geq 0. \]
\end{defn}

\begin{defn}[Discrete Pullback Stability] \label{DPSdef}
   A discrete family $\hat{A} = \{A(p);p \in P\}$ of
   uniformly bounded compact subsets of $E$, is {\bf
   pullback stable} for the discrete cocycle
   $\{\Phi_{(n, p)}; n \in \mathbb{Z}^{+}, p \in P\}$ on $E$
   if for any $\e > 0$ there exists a
   $\delta$-neighbourhood system
   $\hat{\mathcal{N}}_{(\hat{\delta} ,\hat{A})} =
   \{\hat{\mathcal{N}}_{\delta_p, \hat{A}}; \delta_p \in \hat{\delta}, p \in P
   \}$ defined by a delta set $\hat{\delta} = \{\delta_p \in \mathbb{R}^+; p
   \in P\}$, so that for each $p \in P$,
   \[ H^* \left(\Phi_{(n, \theta_{-n}p)}
           (\mathcal{N}_{\delta_p}(A(\theta_{-n} p))), A(p) \right) < \e \qquad
           \forall n \geq 0. \]
\end{defn}


\begin{defn}[Discrete Forward Asymptotic Stability] \label{DFASdef}
   A discrete family $\hat{A} = \{A(p);p \in P\}$ of
   uniformly bounded compact subsets of $E$, is {\bf
   forward asymptotically stable} for the discrete cocycle
   $\{\Phi_{(n, p)}; n \in \mathbb{Z}^{+}, p \in P\}$ on $E$ if it is forward
   stable and if there exists a $\delta$-neighbourhood system
   $\hat{\mathcal{N}}_{(\hat{\delta} ,\hat{A})} =
   \{\hat{\mathcal{N}}_{\delta_p, \hat{A}}; \delta_p \in \hat{\delta}, p \in P
   \}$ defined by a delta set $\hat{\delta} = \{\delta_p \in \mathbb{R}^+; p
   \in P\}$, so that for each $p \in P$ and any initial value
   $x_0 \in \mathcal{N}_{\delta_p}(A(p))$,
   \begin{align}
      & \lim_{n \to \infty} \dist \left(\Phi_{(n, p)}
           (x_0), A(\theta_n p) \right) = 0.
   \end{align}
\end{defn}

\begin{defn}[Discrete Pullback Asymptotic Stability] \label{DPASdef}
   A discrete family $\hat{A} = \{A(p);p \in P\}$ of
   uniformly bounded compact subsets of $E$, is {\bf
   pullback asymptotically stable} for the discrete cocycle
   $\{\Phi_{(n, p)}; n \in \mathbb{Z}^{+}, p \in P\}$ on $E$ if it is pullback
   stable and if there exists a $\delta$-neighbourhood system
   $\hat{\mathcal{N}}_{(\hat{\delta} ,\hat{A})} =
   \{\hat{\mathcal{N}}_{\delta_p, \hat{A}}; \delta_p \in \hat{\delta}, p \in P
   \}$ defined by a delta set $\hat{\delta} = \{\delta_p \in \mathbb{R}^+; p
   \in P\}$, so that for each $p \in P$ and any sequence of initial values
   $\hat{x}_p \in \mathcal{N}_{\delta_p, \hat{A}}$,
   \begin{align}
      & \lim_{n \to \infty} \dist \left(\Phi_{(n, \theta_{-n}p)}
           (x(\theta_{-n}p)), A(p) \right) = 0.
   \end{align}
\end{defn}

\begin{defn}[Discrete Forward Attractor] \label{DFAdef}
   A discrete family $\hat{A} = \{A(p);p \in P\}$ of
   uniformly bounded compact subsets of $E$, is said to be a {\bf
   discrete forward attractor} for the cocycle $\{\Phi_{(n, p)}; n \in
   \mathbb{Z}^{+}, p \in P\}$ on $E$ if there exists a
   $\delta$-neighbourhood system $\hat{\mathcal{N}}_{(\hat{\delta}
   ,\hat{A})} = \{\hat{\mathcal{N}}_{\delta_p, \hat{A}}; \delta_p \in
   \hat{\delta}, p \in P \}$ defined by a delta set $\hat{\delta} =
   \{\delta_p \in \mathbb{R}^+; p \in P\}$, so that for each $p
   \in P$ the forward attractor $\hat{A}$ satisfies two
   properties,
   \begin{align}
      & \Phi_{(n,p)} \left( A(p) \right) = A(\theta_n p) \qquad
                \text{for each} \quad n \in \mathbb{Z}^{+}. \\
      & \lim_{n \to \infty} H^{*} \left(\Phi_{(n, p)}
           (\mathcal{N}_{\delta_{p}}(A(p)), A(\theta_{n}p)
           \right) = 0.
   \end{align}
\end{defn}

\begin{defn}[Discrete Pullback Attractor] \label{DPAdef}
   A discrete family $\hat{A} = \{A(p);p \in P\}$ of
   uniformly bounded compact subsets of $E$, is said to be a {\bf
   discrete pullback attractor} for the cocycle $\{\Phi_{(n, p)}; n \in
   \mathbb{Z}^{+}, p \in P\}$ on $E$ if there exists a
   $\delta$-neighbourhood system $\hat{\mathcal{N}}_{(\hat{\delta}
   ,\hat{A})} = \{\hat{\mathcal{N}}_{\delta_p, \hat{A}}; \delta_p \in
   \hat{\delta}, p \in P \}$ defined by a delta set $\hat{\delta} =
   \{\delta_p \in \mathbb{R}^+; p \in P\}$, so that for each $p
   \in P$ the pullback attractor $\hat{A}$ satisfies two
   properties,
   \begin{align}
      & \Phi_{(n,p)} \left( A(p) \right) = A(\theta_n p) \qquad
                \text{for each} \quad n \in \mathbb{Z}^{+}. \\
      & \lim_{n \to \infty} H^{*} \left(\Phi_{(n, \theta_{-n}p)}
           (\mathcal{N}_{\delta_{p}}(A(\theta_{-n}p)), A(p)
           \right) = 0.
   \end{align}
\end{defn}

The same properties of {\em uniformity}, {\em equi-asymptotic stability} and
{\em completeness} hold for the discrete definitions above as for continuous
stability theory.

We may also construct a pullback absorbing neighbourhood to analyse pullback
attractors as was done in Section \ref{ANsec} for continuous systems.

\begin{defn}[Discrete Pullback Absorbing Neighbourhoods] \label{dPANdef}
   A family $\hat{B}=\{B(p);p \in P\}$ of uniformly bounded compact subsets
   of $E$, is said to be a {\bf Discrete Pullback Absorbing Neighbourhood} for a
   discrete cocycle $\{\Phi^h_{(n,p)}; n \in \mathbb{Z}^{+},p \in P\}$ on $E$
   if it pullback absorbs a uniformly bounded $\delta$ -
   neighbourhood system of $\hat{B}$. That is, there exists an open
   $\delta$-neighbourhood system
   $\hat{\mathcal{N}}_{(\hat{\delta},\hat{B})}$ defined by a delta set
   $\hat{\delta} = \{\delta_p \in \mathbb{R}^+; p\in P\}$ so that for each
   $p \in P$, there exists a $N_p>0$ such that
  \begin{equation}
     \Phi^h_{(n,\theta_{-n}(p))}(\mathcal{N}_{\delta_p}(B(\theta_{-n}(p)))
     \subset B(p) \qquad \forall n > N_p,
   \end{equation}
\end{defn}

The discrete pullback absorbing neighbourhood automatically satisfies the
following property.

\begin{lemma} \label{dpanepilem}
If $\hat{B}$ is a pullback absorbing neighbourhood, then for each $p \in P$
there exists a $N_p > 0$ such that
\[ \Phi_{(n,\theta_{-n}(p))}(B(\theta_{-n}(p)))
     \subset B(p) \qquad \forall n > N_p. \]
\end{lemma}

The following theorem is an extension of Theorem \ref{abspat} to discrete
dynamical systems. The proof follows in a similar manner to its
continuous counterpart and is given for completeness.

\begin{therm}
  \label{dabspatthm}
  Let $\{\Phi_{(n,p)};n \in \mathbb{Z}^+, p \in P \}$ be a discrete cocycle
  on $E$ with a Discrete Pullback Absorbing Neighbourhood $\hat{B}$. Then
  there exists a Discrete Pullback Attractor $\hat{A} = \{ A(p); p \in P \}$
  uniquely determined for each $p \in P$ by
  \begin{equation}
  \label{dpatfromabs}
  A(p) = \bigcap_{\eta \geq 0} \overline{\bigcup_{n \geq \eta}
            \Phi_{(n,\theta_{-n}(p))}(B(\theta_{-n}(p)))}.
  \end{equation}
\end{therm}

\begin{prf}
  \hspace{3mm} \\
  To show that $\hat{A}$ is indeed a discrete pullback attractor, it must meet
  the requirements given in Definition \ref{DPAdef}.

  \hspace*{3mm} i) {\em Uniform Boundedness and Compactness} : From
  (\ref{dpatfromabs}) and using Lemma \ref{dpanepilem}
  \begin{align*}
  A(p) &= \bigcap_{\eta \geq 0} \overline{\bigcup_{n \geq \eta}
            \Phi_{(n,\theta_{-n}(p))}(B(\theta_{-n}(p)))}, \\
  &\subseteq \bigcap_{\eta \geq N_p} \overline{\bigcup_{n \geq \eta}
            \Phi_{(n,\theta_{-n}(p))}(B(\theta_{-n}(p)))}, \\
  &\subseteq \bigcap_{\eta \geq N_p} \overline{\bigcup_{n \geq \eta}
            B(p)}, \\
  &= \overline{B(p)}.
  \end{align*}
  Hence $\hat{A}$ is uniformly bounded since $\hat{B}$ is uniformly bounded
  with respect to $p$. The attractor set is also compact as it is an
  intersection of compact sets.

  \hspace*{3mm} ii) {\em Pullback Property} : We need to find a
  discrete pullback convergent neighbourhood system as in (\ref{DPAdef}). We
  will firstly show that $\hat{A}$ pullback attracts $\hat{B}$. That is, for
  each $p \in P$
  \begin{equation}
  \label{dabspat1eq}
    \lim_{n \to \infty} H^* (\Phi_{(n, \theta_{-n}(p))} (B(\theta_{-n}(p))),
           A(p)) = 0.
  \end{equation}
  where $A(p)$ is defined as above.

  Assume that this is not the case. Then for some $\epsilon > 0$ there exists
  sequences $n_j \to \infty$ and $x_j \in \Phi_{(n_j,
  \theta_{-n_j}(p))} (B(\theta_{-n_j}(p)))$ such that
  \begin{equation}
  \label{dabspat2eq}
    H^*(x_j,A(p)) > \epsilon \qquad \forall j.
  \end{equation}
  For large enough $j$, $x_j \in B(p)$ (Lemma \ref{dpanepilem}). Now Since
  $B(p)$ is compact, there exists a subsequence $n_{j'} \rightarrow
  \infty$ and an associated convergent subsequence, $x_{j'} \to x_0$ with
  $x_0 \in B(p)$. Furthermore $H^*(x_0, A(p)) \geq \e$. The $x_{j'}$ satisfy
  \[ x_{j'} \in \bigcup_{n \geq \eta} \Phi_{(n,\theta_{-n}(p))}
          (B(\theta_{-n}(p))), \qquad \forall \eta>0 \quad \text{and}
          \quad n_{j'}>\eta. \]
  As $x_0$ is the limit, we also have for any $\eta \geq 0$,
  \[ x_0 \in \overline{\bigcup_{n \geq \eta}
         \Phi_{(n,\theta_{-n}(p))}(B(\theta_{-n}(p)))}, \]
  and hence
  \[ x_0 \in \bigcap_{\eta \geq 0} \overline{\bigcup_{n \geq \eta}
            \Phi_{(n,\theta_{-n}(p))}(B(\theta_{-n}(p)))}. \]
  That is $x_0 \in A(p)$, which contradicts (\ref{dabspat2eq}). As the
  choice of $p$ was arbitrary, (\ref{dabspat1eq}) holds true for all $p \in
  P$.

  From (\ref{dabspat1eq}), for any $\epsilon > 0$, and each $p \in P$, there
  exists a $n_1 = n_1(\epsilon,p)$ such that
  \[ H^*(\Phi_{(n_1,\theta_{-n_1}(p))}
           (B(\theta_{-n_1}(p)),A(p)) < \epsilon. \]
  Let us take $\hat{\mathcal{N}}_{\hat{\delta}^*,\hat{B}}$ as our
  neighbourhood system for the attractor, $\hat{A}$, defined by
  $\hat{\delta}^* = \{ \delta^*_p; \delta^*_p =
  \delta_{\theta_{-n_1}(p)}, \delta_p \in \hat{\delta} \}$
  where $\hat{\delta}$ defines $\hat{\mathcal{N}}_{\hat{\delta}, \hat{B}}$,
  the associated $\hat{\delta}$-neighbourhood of $\hat{B}$. We need to show
  that $\hat{A}$ pullback attracts the system
  $\hat{\mathcal{N}}_{\hat{\delta}^*,\hat{B}}$.

  Using the cocycle property, and because $B(p)$ is pullback absorbing, we
  can formulate attraction for elements of
  $\hat{\mathcal{N}}_{\hat{\delta}^*,\hat{B}}$
  \begin{equation*}
  \begin{split}
  &H^*(\Phi_{(n,
  \theta_{-n}(p))}(\mathcal{N}_{\delta^*_p}(A(\theta_{-n}(p))),A(p)) \\
  &\qquad \leq H^*(\Phi_{(n,
  \theta_{-n}(p))}(\mathcal{N}_{\delta^*_p}(B(\theta_{-n}(p))),A(p)) \\
  &\qquad = H^*(\Phi_{(n_1, \theta_{-n_1}(p))}
          \circ \Phi_{(n-n_1,
          \theta_{-n}(p))}
          (\mathcal{N}_{\delta^*_p}(B(\theta_{-n}(p))),A(p)) \\
  &\qquad \leq H^*(\Phi_{(n_1, \theta_{-n_1}(p))}
          (B(\theta_{-n_1}(p)),A(p)) \\
  &\qquad \leq \epsilon,
  \end{split}
  \end{equation*}
  for all $n > n_{(\theta_{-n_1}(p))} +n_1$, where
  $n_{(\theta_{-n_1}(p))}$ is as defined in the definition for
  a pullback absorbing neighbourhood (the finite absorption time).
  Hence $\hat{A}$ satisfies the pullback property for each $p \in P$, i.e.
  \[ \lim_{n \to \infty} H^* (\Phi_{(n, \theta_{-n}(p))}
            (\mathcal{N}_{\delta^*_p}(A(\theta_{-n}(p))),A(p)) = 0. \]

  \hspace*{3mm} iii) {\em $\Phi$-Invariance} : \hfill \\
  Consider pullback evolution of  $A(\theta_{-n^*}(p))
  \in \hat{A}$ for arbitrary $p$ and any $t^* > 0$
    \begin{align*}
  \Phi_{(n^*, \theta_{-n^*}(p))} &(A(\theta_{-n^*}(p))) \\
  &=\Phi_{(n^*, \theta_{-n^*}p)}\left(\bigcap_{\eta \geq 0}
       \overline{ \bigcup_{n
       \geq \eta} \Phi_{(n, \theta_{-n - n^*}(p)} (B(\theta_{-n -
       n^*}(p)))}\right), \\
  &= \bigcap_{\eta \geq 0} \Phi_{(n^*, \theta_{-n^*}p)}\left(
     \overline{\bigcup_{n \geq \eta} \Phi_{(n, \theta_{-n - n^*}(p)}
     (B(\theta_{-n - n^*}(p)))}\right), \\
  &= \bigcap_{\eta \geq 0} \overline{\bigcup_{n \geq \eta} \Phi_{(n^*,
     \theta_{-n^*}p)} \Phi_{(n, \theta_{-n - n^*}(p)} (B(\theta_{-n -
     n^*}(p)))}, \\
  &= \bigcap_{\eta \geq 0} \overline{\bigcup_{n \geq \eta} \Phi_{(n+n^*,
     \theta_{-n-n^*}p)} (B(\theta_{-n - n^*}(p)))}, \\
  &= \bigcap_{\eta \geq n^*} \overline{\bigcup_{u \geq \eta} \Phi_{(u,
     \theta_{-u}p)} (B(\theta_{-u}(p)))},
  \end{align*}
  where we have made the substitution $u = n + n^*$. Now, for all $\eta <
  n^*$,
  \[ \overline{\bigcup_{u \geq \eta} \Phi_{(u, \theta_{-u}p)}
     (B(\theta_{-u}(p)))} \supseteq \overline{\bigcup_{u \geq n^*} \Phi_{(u,
     \theta_{-u}p)} (B(\theta_{-u}(p)))}. \]
  Hence
  \begin{align*}
  \Phi_{(n^*, \theta_{-n^*}(p))} &(A(\theta_{-n^*}(p))) \\
  &= \bigcap_{\eta \geq 0} \overline{\bigcup_{u \geq \eta} \Phi_{(u,
     \theta_{-u}p)} (B(\theta_{-u}(p)))} \\
  &= A(p). \\
  \end{align*}
  The conditions for $\Phi$-Invariance are satisfied.
\end{prf}

\endinput
