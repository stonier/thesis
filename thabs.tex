\nonumchapter{Introduction}
\setcounter{page}{1}
\pagestyle{plain}

The development and use of cocycles for analysis of
non-autonomous behaviour is a technique that has been known for several
years. Initially developed as an extension to semi-group theory for
studying non-autonomous behaviour, it was extensively used in analysing
random dynamical systems \cite{Ar98,CrFlDe97,CrFl94,FlSc96}.

Many of the results regarding asymptotic behaviour developed for random
dynamical systems, including the concept of cocycle attractors were
successfully transferred and reinterpreted for deterministic non-autonomous
systems primarily by P. Kloeden and B. Schmalfuss
\cite{KlSc95,KlSc96,Sc92,Sc99}. The theory concerning cocycle attractors was
later developed in various contexts specific to particular classes of dynamical
systems \cite{ChKlSc98,ChKlSc00,FS95}, although a comprehensive understanding
of cocycle attractors (redefined as pullback attractors within this thesis) and
their role in the stability of non-autonomous dynamical systems was still at
this stage incomplete.

It was this purpose that motivated Chapters 1-3 to define and formalise the
concept of stability within non-autonomous dynamical systems.  The approach
taken incorporates the elements of classical asymptotic theory, and
refines the notion of pullback attraction with further development
towards a study of pullback stability and pullback asymptotic stability.
In a comprehensive manner, it clearly establishes both pullback and forward
(classical) stability theory as fundamentally unique and essential components of
non-autonomous stability. Many of the introductory theorems and examples
highlight the key properties and differences between pullback and forward
stability. The theory also cohesively retains all the properties of
classical asymptotic stability theory in an autonomous environment. These
chapters are intended as a fundamental framework from which further research in
the various fields of non-autonomous dynamical systems may be extended.

A preliminary version of a Lyapunov-like theory that characterises pullback
attraction is created as a tool for examining non-autonomous behaviour in
Chapter 5. The nature of its usefulness however is at this stage restricted
to the converse theorem of asymptotic stability.

Chapter 7 introduces the theory of Loci Dynamics. A transformation
is made to an alternative dynamical system where forward
asymptotic (classical asymptotic) behaviour characterises pullback
attraction to a particular point in the original dynamical system.
This has the advantage in that certain conventional techniques for
a forward analysis may be applied.

The remainder of the thesis, Chapters 4, 6 and Section 7.3, investigates the
effects of perturbations and discretisations on non-autonomous dynamical systems
known to possess structures that exhibit some form of stability or attraction.
Chapter 4 investigates autonomous systems with semi-group attractors, that have
been non-autonomously perturbed, whilst Chapter 6 observes the effects of
discretisation on non-autonomous dynamical systems that exhibit properties of
forward asymptotic stability. Chapter 7 explores the same problem of
discretisation, but for pullback asymptotically stable systems. The
theory of Loci Dynamics is used to analyse the nature of the discretisation,
but establishment of results directly analogous to those discovered in
Chapter 6 is shown to be unachievable. Instead a case by case analysis is
provided for specific classes of dynamical systems, for which the results
generate a numerical approximation of the pullback attraction in the original
continuous dynamical system.

The nature of the results regarding discretisation provide a non-autonomous
extension to the work initiated by A. Stuart and J. Humphries \cite{St94,StHu96}
for the numerical approximation of semi-group attractors within autonomous
systems. Of particular importance is the effect on the system's asymptotic
behaviour over non-finite intervals of discretisation.

\endinput
