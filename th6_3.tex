\chapter[Pullback Numerical Approximation Theory]
         {Discretisation of Pullback Asymptotic Behaviour}
\label{chpbackdisc}

The results of this chapter pertain to non-autonomous
differential equations of the form $\dot{x} = f(p, x)$, that are
known to possess a pullback equi-asymptotically stable family
$\hat{A}$. The situation for pullback behaviour is inherently more
complicated than that of forward behaviour as will be illustrated.

The focus is to understand the effects of
discretisation on pullback asymptotically stable families, however
finding an approach to solving a theorem analogous to Theorem
\ref{numeasthm1} for pullback equi-asymptotically stable families
is not as clear, and a completely analogous result is yet open to
further research.

We begin by investigating several key
aspects that clarify the nature of pullback behaviour.

\section{Duality of Attraction}

\subsection{Introduction}

Here we consider one particular aspect of pullback behaviour, which will be
referred to as the duality of attraction. It involves a procedure which fixes a
$p \in P$, and transforms the pullback problem into one in which the pullback
behaviour is characterised by forward attraction. This has the advantage in that
theory for forward asymptotically stable systems may then be applied. As far
as the author is aware, no other research has currently been published in
this direction.

We then consider various cases for pullback attraction and their
discretisations, and form several results under various
restrictions on the rate of pullback attraction and the structure of the
function $f$.

\subsection{Loci Dynamics}
\label{ssecLoci1}
Pullback asymptotic stability to a fixed point $p \in P$ can be
modelled equivalently with a forward asymptotically stable system
by considering the dynamical system resulting from an analysis of
the system's sensitivity to initial times. To illustrate the
fundamental concept, we begin with the example below.

\begin{eg}
Consider the dynamical system arising from the NDE
\[ \dot{x} = 2tx. \]
Solutions for this system may be expressed in cocycle form as
\[ \Phi_{(t,t_0)}(x_0) = x_0 e^{(t+ t_0)^2 - t_0^2}, \]
where the parameter space $P = \mathbb{R}$ and $t_0 \in \mathbb{R}$.
Alternatively, by considering pullback attraction to the point $t_0$, we may
express solutions by
\[ \Phi_{(t, t_0-t)}(x_0) = x_0 e^{t_0^2 - (t_0-t)^2}. \]
It is easily proven that the origin for this system is a pullback
attractor without forward convergence properties. This can be seen
graphically in Figure \ref{loci0} where the initial state is set
at $x_0 = 1$ and pullback attraction to $t_0 = 1$ is considered.

\begin{figure}[htb]
\begin{center}
%\framebox[6.0cm][c]{
\leavevmode
\hbox{
\epsfxsize=9.8cm
\epsffile{eps/loci0.eps}  }%}
\protect\caption{Locus of Transformation}
      \protect\label{loci0}
\end{center}
\end{figure}

The solutions at $t_0$ define a mapping with image $\Phi_{(t,t_0-t)}(x_0)$
associated with each $t \geq 0$. If the locus of points defined by
$(t_0 - t, \Phi_{(t,t_0-t)}(x_0))$ is plotted as illustrated,
the locus forms a continuous trajectory in reverse that asymptotes to the
origin as $t$ is increased. Similar loci may be plotted for each $x_0$,
generating a system resembling a dynamical system with forward asymptotic
properties.

The behaviour of the loci may in fact be calculated by determining the rate of
change of the image ($\Phi_{(t,t_0-t)}(x_0)$) with respect to $t$. Hence we
have
\begin{align*}
\frac{d}{dt}\left(\Phi_{(t,t_0-t)}(x_0)\right) &= \frac{d}{dt} \left( x_0
                e^{t_0^2 - (t_0-t)^2)} \right), \\
&= 2(t_0 - t) \Phi_{(t,t_0-t)}(x_0).
\end{align*}
As $x_0$ is arbitrary, the dynamics for the loci is then simply governed by the
equation $d\Phi / dt = 2(t_0 - t) \Phi$, a dynamical system for which the
origin is forward asymptotically stable.
\end{eg}

In fact pullback attraction to any point $p$ of a pullback
attractor $A$ is equivalent to a dynamical
system possessing the set $A$ as a forward attractor, generated
in the same fashion as the example above. This is formalised as follows.

The dynamical system that will be referred to in the Lemmas to
follow is generated by the non-autonomous differential equation
$\dot{x} = f(p, x)$, and is assumed to possess a pullback
attractor, $A$. This will be later extended to account for varying
pullback attractors, $\hat{A}$.

Since $A$ is constant, analysing the properties of pullback
attraction of a single state $x_0$ is valid for all $t \geq 0$. Each locus
corresponds to a single initial state $x_0 \in \mathcal{N}_{\delta_p}(A)$, and
the mapping defining the locus is determined by the couple $(\theta_{-t}p,
\Phi_{(t,\theta_{-t}p)}(x_0))$ for all $t \geq 0$.

\begin{lemma}
Given any $p \in P$, then for each $x_0 \in \mathcal{N}_{\delta_p}(A)$, the
loci defined by $(\theta_{-t}p, \Phi_{(t, \theta{-t}p)}(x_0))$ for
$t \geq 0$ are continuous and unique.
\end{lemma}
\begin{prf}
  {\em Continuity:}

Let $\e > 0$, and $x_0 \in \mathcal{N}_{\delta_p}(A)$ be arbitrary.
Then
\[ |\Phi_{(t + h, \theta_{-(t+h)}p)}(x_0) - \Phi_{(t,\theta_{-t}p)}(x_0)| =
         |\Phi_{(t,\theta_{-t}p)}(x_0^*) - \Phi_{(t,\theta_{-t}p)}(x_0)|, \]
where $x_0^* = \Phi_{(h, \theta_{-(t+h)}p)}(x_0)$. Using
the same principle as in Lemma \ref{intro2lem},
\begin{align*}
  |\Phi_{(t + h,\theta_{-(t+h)}p)}(x_0) - \Phi_{(t, \theta_{-t}p)}(x_0)| &\leq
         |x_0^*-x_0| \exp\left( L(\theta_{-t} p)h\right), \\
  &\leq h F(t) \exp\left( L(\theta_{-t} p)t \right). \\
\end{align*}
Here $F(t) = \sup \{ f(\theta_{-\tau}p, x) ; t \leq \tau \leq t+h
\}$ and $L(\theta_{-t}p)$ is the maximum local Lipschitz bound for
$f$ over the interval $(\theta_{-t}p, p)$.

Choose $h(t, \e) \leq \e / ( F(t) \exp\left( L(\theta_{-t} p)t
\right)$. Then
\[ |\Phi_{(t + h, \theta_{-(t+h)}p)}(x_0) - \Phi_{(t,\theta_{-t}p)}(x_0)| \leq
  h F(t) \exp\left( L(\theta_{-t} p) \right) \leq \e, \]
as required.

{\em Uniqueness: }

Assume the loci are not unique. That is, two loci cross paths at
some $t^*>0$. Since each loci is generated from unique initial
states, there exists $x_0, x_1 \in \mathcal{N}_{\delta_p}(A)$ with
$x_0 \neq x_1$ such that $\Phi_{(t^*,\theta_{-t^*}p)}(x_0)) =
\Phi_{(t^*, \theta_{-t^*}p)}(x_1))$. However, this contradicts the
uniqueness of solutions for the original dynamical system. Hence
each locus is necessarily unique.
\end{prf}

Points along each locus may be defined by the mapping $\{
\phi_{(t, t_0)}; t, t_0 \in \mathbb{R}^+ \}$, with $\phi_{(t,
t_0)}: E \to E$ and
\[ \phi_{(t , t_0)}(\phi_0) = \Phi_{(t_0 + t, \theta_{-(t_0+t)}p)}
   (\Phi_{(-t_0, p)}(\phi_0)). \]
$E$ is the state space for the original system.

In terms of the original dynamical system, each point on the loci
fixed by $t_0, \phi_0$ is simply a collection of images at $p$
resulting from pulling back an initial value $x_0$ associated with
$\phi_0$. That is, $x_0 = \Phi_{(-t_0, p)}(\phi_0)$.

\begin{lemma}
The loci form a continuous dynamical system, $\mathcal{D}_l$, for which the
group of mappings $\{ \phi_{(t,t_0)}; t, t_0 \in \mathbb{R}^+ \}$ with
$\phi_{(t,t_0)}: E \to E$ forms a cocycle on $E$ with respect to the group
$\{\theta_t, t \in \mathbb{R}^+ \}$, where $\theta_{t}t_0 = t_0 + t$.
\end{lemma}
\begin{prf}
{\em Identity:}
\begin{align*}
  \phi_{(0,t_0)}(\phi_0) &=  \Phi_{(t_0, \theta_{-t_0}p)}((\Phi_{(-t_0, p)}
              (\phi_0)), \\
  &= \phi_0. \\
\end{align*}
{\em Cocycle Property:}
\begin{align*}
  \phi_{(t_1+t_2, t_0)}(\phi_0) &= \Phi_{(t_0+t_1+t_2,
        \theta_{-(t_0+t_1+t_2)}p)}(\Phi_{(-t_0, p)}(\phi_0)), \\
  &= \Phi_{(t_0+t_1+t_2,
        \theta_{-(t_0+t_1+t_2)}p)} \\
  & \hspace{1cm} \left(\Phi_{(-(t_0+t_1), p)}(
        \Phi_{(t_0+t_1, \theta_{-(t_0+t_1)}p)}( \Phi_{(-t_0, p)}(\phi_0) )
        )\right), \\
  &= \Phi_{(t_0+t_1+t_2,
        \theta_{-(t_0+t_1+t_2)}p)}\left(\Phi_{(-(t_0+t_1), p)}(
        \phi_{(t_1, t_0)}(\phi_0)) \right), \\
  &= \phi_{(t_2,t_0+t_1)}(\phi_{(t_1,t_0)}(\phi_0)). \\
\end{align*}
Hence the cocycle property for $\phi$ is satisfied.
\end{prf}

In some special cases the loci dynamics may be formulated (see Example
\ref{egsepscld}), and if $P = \mathbb{R}$ a general expression for the ordinary
differential equation which determines the loci dynamics may be formed.

\begin{lemma}[Loci Dynamics]
If $f$ and its partial derivatives are continuous, and the parameter set $P
= \mathbb{R}$, then the loci dynamics at each $t_0 \in \mathbb{R}$ are modelled
by the ordinary differential equation with initial value $(0, \phi_0)$,
\begin{equation} \label{eqloci}
 \frac{d\phi}{dt} = \int_{t_0 - t}^{t_0}\frac{\partial f^*}{\partial t}(\tau,
   t_0, t, \phi_0) d\tau + f( t_0 - t, \phi_0),
\end{equation}
where $f^*(\tau, t_0, t, \phi_0) = f( \tau, \Phi_{(\tau
- (t_0 - t),  t_0 - t)}(\phi_0))$.
\end{lemma}
\begin{prf}
Note that
\begin{align*}
  \phi_{(t, 0)}(\phi_0) &= \Phi_{(t, t_0 - t)}(\phi_0), \\
       &= \phi_0 + \int_{t_0 - t}^{t_0} f(\tau, \Phi_{(\tau - (t_0-t), t_0 -
                   t)}(\phi_0))d\tau.
\\ \end{align*}
If $f$ is continuous, and its partial derivative with respect to $t$ exists and
is continuous also, then application of the derivative arrives at the required
result.
\end{prf}

\subsection{$\mathcal{D}_l$ - Asymptotic Behaviour}
\label{ssecLoci2}

\textit{We now pursue the question of forward asymptotic attraction within
the dynamical system $\mathcal{D}_l$.}

If $A$ is locally pullback asymptotically stable, then each point on an
individual loci must lie within the set $\Phi_{t, \theta_{-t}p}
(\mathcal{N}_{\delta_p} (A))$ for any given $t > 0$.

We shall consider initial states for $\mathcal{D}_l$ at $(t_0, \phi_0)$
where $\phi_0 \in \Phi_{t_0, \theta_{-t_0}p} (\mathcal{N}_{\delta_p}
(A))$.

By continuity and uniqueness of the trajectories in the original
system, there exists $x_0 \in \mathcal{N}_{\delta_p}(A)$
associated with the loci that $\phi_0$ lies on such that
$\Phi_{(t_0, \theta_{-t_0}p)}(x_0) = \phi_0$, or
equivalently $x_0 = \Phi_{(-t_0, p)}(\phi_0)$. See Figure \ref{loci3}.

\begin{figure}[htb]
\begin{center}
\input{eps/loci3.pstex_t} \caption{Loci in $\mathcal{D}_l$}
\protect\label{loci3}
\end{center}
\end{figure}

Then we have
\begin{align*}
  \phi_{(t, t_0)}(\phi_0) &= \Phi_{(t +t_0,\theta_{-(t+t_0)}p)}(\Phi_{(-t_0,p)}
           (\phi_0) ), \\
  &= \Phi_{(t +t_0,\theta_{-(t+t_0)}p)}(x_0). \\
\end{align*}
Since $x_0 \in \mathcal{N}_{\delta_p}(A)$, then $A$ pullback attracts $x_0$ and
\begin{align*}
  \lim_{t \to \infty} \dist(\phi_{(t, t_0)}(\phi_0),A) &= \lim_{t \to \infty}
             \dist(\Phi_{(t +t_0,\theta_{-(t+t_0)}p)}(x_0),A), \\
  &= 0.  \\
\end{align*}
Hence for each $t_0$, $A$ forward attracts any $\phi_0 \in \Phi_{(t_0,
\theta_{-t_0}p)}(\mathcal{N}_{\delta_p}(A))$.

Finally, for $A$ to be forward asymptotically stable, $A$ must
forward attract a $\delta$-neighbourhood system
$\mathcal{N}_{\hat{\delta}, A}$. There are two possible cases here.

\textbf{i) Asymptotic Attraction - } If
\[ A \subset \Phi_{(t, \theta_{-t}p)}(\mathcal{N}_{\delta_p}(A)), \]
for all $t \geq 0$, then such a neighbourhood system may be easily found.
Indeed if $A$ is a pullback attractor, then this is automatically the case as
solutions asymptote to the attractor. Refer to Figure \ref{loci1}. In this
diagram $A$ is a pullback attractor that pullback attracts solutions
asymptotically, that is in infinite time.

\begin{figure}[htb]
\begin{center}
\input{eps/loci1.pstex_t} \caption{Region of Loci Generated by Asymptotic Attraction}
\protect\label{loci1}
\end{center}
\end{figure}

We have now verified the following result.
\begin{lemma}
If $A$ is a pullback attractor for (\ref{NDEeq}) then $A$ is a forward attractor
for $\mathcal{D}_l$.
\end{lemma}
\begin{prf}
Attraction and existence of an attracting neighbourhood have already been
verified. Invariance follows from the invariance of $A$ as a pullback attractor
in the original system.
\end{prf}

\textbf{Remark:} The neighbourhood system for forward attraction is
not constant as it is bounded by pullback attraction of solutions
from $\mathcal{N}_{\delta_p}(A)$ in the original system. As
\[ \lim_{t \to \infty} \dist (\Phi_{(t, \theta_{-t}p)}(
      \mathcal{N}_{\delta_p}(A)), A) \to 0, \]
then the neighbourhood system for $\mathcal{D}_l$ vanishes.

\textbf{ii) Finite Attraction - } $A$ pullback attracts $\mathcal{N}_{\delta_p}$
in finite time $t^* > 0$. In this case,
\[ \Phi_{(t, \theta_{-t}p)}(\mathcal{N}_{\delta_p}(A)) \subseteq A, \]
for all $t > t*$.

As a result, all initial values $(t_0. \phi_0)$ chosen so
that $t_0 > t^*$ where $\phi_0 \in \Phi_{(t_0,
\theta_{-t_0}p)}(\mathcal{N}_{\delta_p}(A))$ are actually contained in $A$.
Consequently there is no guarantee that $A$ forward attracts points in a
neighbourhood of itself. This is seen in Figure \ref{loci2} where $A$ is a
forward asymptotically stable set that pullback attracts solutions in the
original system in finite time.

\begin{figure}[htb]
\begin{center}
\input{eps/loci2.pstex_t} \caption{Region of Loci Generated by Finite
Attraction} \protect\label{loci2}
\end{center}
\end{figure}

The proof for the following lemma follows easily, and allows for
pullback asymptotically stable sets which attract a local
neighbourhood in finite time.

\begin{lemma}
If $A$ is globally pullback asymptotically stable, then $A$ is a
global forward asymptotically stable set for the system.
$\mathcal{D}_l$.
\end{lemma}

{\bf Remark:} Understanding the effect of attraction to a single
point $p \in P$ by analysis of the dynamical system of loci
$\mathcal{D}_l$ does not take into account the pullback attraction
to $A$ for all $p \in P$. This may be a limiting factor in
observing completely the behaviour of a non-autonomous system
possessing a pullback attractor, and in particular understanding
the effects of discretisation.

\begin{eg}[Special Case - Loci Dynamics]\label{egsepscld}
Consider the NDE
\[ \dot{x} = f(t)g(x). \]
We assume $f, 1/g$ are continuous and bounded over the
interval of consideration, and hence integrable.
Let $F, G$ denote the primitives of $f$, and
$1/g$ respectively. Separating variables, and integrating:
\[ G(x(t;t_0,x_0)) = G(x_0) + F(t) - F(t_0). \]
Upon analysing solutions in a pullback sense, we have
\[ G(\Phi_{(t, t_0-t)}(x_0)) = G(x_0)  + F(t_0) - F(t_0 - t). \]
Finally, differentiating with respect to $t$, and rewriting solutions in terms
of the cocycle mapping $\phi$ on $\mathcal{D}_l$, the loci dynamics are
determined by
\begin{equation}
\label{scaseeq}
 \frac{d \phi}{dt} = f( t_0 - t) g(\phi).
\end{equation}
\end{eg}


\subsection{Extension to Time Varying Families $\hat{A}$}

\subsubsection{Construction of $\mathcal{D}_l$}

Earlier it was noted that the generation of the system of loci
$\mathcal{D}_l$ for constant pullback attractors/pullback asymptotically stable
families $A$ could be easily extended to time varying families $\hat{A}$.

The difficulty lies in tracing a single loci analysing attraction to $p \in P$ for an initial
state $x_0$. Using the previous method, each point on the loci is generated by
the image of the solution at $p$ from the initial state $x_0$ pulled back in
time. However, to obtain meaningful results regarding the attractive properties
of the dynamical system requires that $x_0$ remains within the local
neighbourhood of attraction as it is pulled back. This is not always possible
for a time-varying attractor.

********Insert figure or should be well reasoned by now?**************

An alternative means for generating the loci can be achieved by
using a sequence of initial states associated with each $x_0 \in
\mathcal{N}_{\delta}(A(p))$. These are denoted by $\hat{x}(x_0) =
\{x_0(\theta_{-t}p) \in \mathcal{N}_{\delta_p}(A(\theta_{-t}p)); x_0 \in
\mathcal{N}_{\delta_p}(A(p)), t \geq 0 \}$ for each $x_0 \in
\mathcal{N}_{\delta_p}(A(p))$ and defined in the same manner as used in the
definition for pullback asymptotic stability (Definition \ref{PASdef}).

To ensure continuity and uniqueness of the loci, the initial
sequences $\hat{x}(x_0)$, must also possess continuity and
uniqueness. Uniqueness of initial sequences is such that for any
$t \geq 0$, each element $x_0(\theta_{-t}p)$ is unique to one and
only one initial value $x_0 \in \mathcal{N}_{\delta_p}(A(p))$.
That is, if there exists two sequences $\hat{x}(x_0),
\hat{x}(x_1)$ such that for some $t$,  $x_0(\theta_{-t}p) =
x_1(\theta_{-t}p)$, then $x_0 = x_1$, and $\hat{x}(x_0)
= \hat{x}(x_1)$.

Finally, the loci mapping may be generated as follows. For each
$\hat{\phi_0}$ with $\phi_0 \in \mathcal{N}_{\delta_p}(A(p))$,
\begin{equation}\label{eqtvloci}
  \phi_{(t, 0)}(\phi_0) = \Phi_{(t, \theta_{-t}p)}(\phi_0(\theta_{-t}p)).
\end{equation}

The following results then follow in a similar manner to those presented in
Sections \ref{ssecLoci1} and \ref{ssecLoci2}

\begin{lemma} The loci mappings defined in (\ref{eqtvloci}) form a dynamical
system $\mathcal{D}_l$.
\end{lemma}

\subsubsection{Asymptotic Behaviour of $\hat{A}$ in $\mathcal{D}_l$}

\begin{lemma}
If $\hat{A}$ is a global pullback asymptotically stable family for the
non-autonomous dynamical system (\ref{NDEeq}), then for each $p \in P$,
$A(p)$ is the forward attractor for the dynamical system $\mathcal{D}_l$
generated by the loci mappings (\ref{eqtvloci}.
\end{lemma}

One further condition is placed on the construction of the initial sequences
for the loci mappings if $\hat{A}$ is a pullback attractor:

\textbf{A1 - } For every $x_0 \in A(p)$, the associated initial sequence is the
invariant solution passing through $\hat{A}$ backwards in time. That is,
$\hat{x}(x_0) = \{x_0(\theta_{-t}p) \in A(\theta_{-t}p); \Phi_{(t, \theta_{-t}p)
}(x_0(\theta_{-t}p)) = x_0, t \geq 0 \}$.

This condition ensures invariance of the forward attractor $A(p)$ in
$\mathcal{D}_l$.

\begin{lemma}
Let $\hat{A}$ be a pullback attractor for (\ref{NDEeq}). If the loci
mappings (\ref{eqtvloci} are formulated in compliance with \textbf{A1},
then for each $p \in P$, $A(p)$ is a forward attractor on the dynamical system
$\mathcal{D}_l$.
\end{lemma}

{\bf Remark: } Initial sequences may be formulated by introducing a
co-ordinate system on the local neighbourhood system of $\hat{A}$. A two
dimensional system for example, could utilise a neighbourhood co-ordinate
represented by the couple $(\alpha, \varphi)$, based on the
co-ordinate distance ($\alpha$) of a point in the neighbourhood away from an
invariant solution passing through $\hat{A}$, and an angle of rotation
($\varphi$)  centred on the invariant solution. Each initial sequence
$\hat{x}(x_0)$ may then be uniquely defined by the co-ordinate value of $x_0$.

Similarly, if $\hat{A}$ is a singleton set for each $p \in P$, then the
transformation defined by $x_0(\theta_{-t}p) = A(\theta_{-t}p) + (x_0 - A(p))$
provides a suitable construction for generation of continuous and unique initial
sequences.


\subsection{Discretisation of $\mathcal{D}_l$}

\begin{eg}[Special Case]\label{egsepsc}
Consider again the NDE in Example \ref{egsepscld}
\[ \dot{x} = f(t)g(x). \]
We assume the existence of a globally pullback equi-asymptotically
stable set $A$. Since $f, g$ are continuous and bounded over a
finite interval, and since $g(x) = 0$ only on $A$, the system is
integrable.

Consequently, the loci dynamics for $\mathcal{D}_l$ are determined by
\[  \frac{d \phi}{dt} = f( t_0 - t) g(\phi). \]
Thus $A$ is a global forward equi-asymptotically stable set for
$\mathcal{D}_l$ and discretising this system with a variable
time step scheme by Theorem \ref{numeasthm1} will generate a
discrete forward equi-asymptotically stable set that approximates
$A$.
\end{eg}

{\bf Remark:} Generating a discrete equi-asymptotically stable set
that approximates $A$ by discretising the forward
equi-asymptotically stable system $\mathcal{D}_l$ does not
however, give any indication of the effect of discretisation on
the original system.

In principle, we need to consider how discretising the original
system affects the evolution of solutions in $\mathcal{D}_l$. The
statement of the problem proceeds as follows.

As in Section \ref{disfassec}, we will be considering a numerical
scheme (possibly using a variable time-step construction) applied
to a non-autonomous dynamical system defined by $\dot{x} = f(p,
x)$ possessing a global pullback equi-asymptotically stable set
$A$. The discretisation will be used to approximate pullback
attraction to some $p_0 \in P$.

The numerical scheme generates a discrete cocycle $\{ \Phi^{{\bf
h}}_{(n,(p, {\bf h}))}, n \in \mathbb{Z}^+, p \in P, {\bf h} \in
\mathcal{H}^{\rho} \}$ as used in Section \ref{disfassec}. To
analyse pullback attraction of an initial state $x_0 \in
\mathcal{N}_{\delta_p}(A)$ to $p_0$, a discrete sequence $\{ x_n
\}$ can be generated as follows
\begin{equation}
  \label{disseqeq}
  x_n = \Phi^{{\bf h}}_{(n, (\theta_{-n}p,\psi_{-n}{\bf h}))}(x_0).
\end{equation}

\begin{figure}[htb]
\begin{center}
\input{eps/pdisc1.pstex_t} \caption{Pullback Discrete Sequence}
\protect\label{pdisc1}
\end{center}
\end{figure}

%\begin{figure}[htb]
%\begin{center}
%\leavevmode
%\hbox{
%\epsfxsize=9.8cm
%\epsffile{eps/pdisc1.eps}  }
%\protect\caption{Pullback Discrete Sequence}
 %     \protect\label{pdisc1}
%\end{center}
%\end{figure}

The sequence $\{ x_n \}$ represents a series of approximated
images at $p_0$, or equivalently, a discretised approximation on
the loci generating the forward equi-asymptotically stable system
$\mathcal{D}_l$ (refer to Figure \ref{pdisc1}).

The error arising from $x_n$ between the continuous evolution
$\phi_{(h(t_n), t_n)}(x_n)$ and the discretised solution $x_{n +
1}$ over one step form a local truncation error for pullback
analysis on $\mathcal{D}_l$ (refer to Figure \ref{pdisc2}).

\begin{figure}[htb]
\begin{center}
%\framebox[6.0cm][c]{
\leavevmode \hbox{ \epsfxsize=12.8cm
\epsffile{eps/pdisc2.eps}  }%}
\protect\caption{Pullback Truncation Error}
      \protect\label{pdisc2}
\end{center}
\end{figure}

If the local truncation error is $O(h^2)$ or higher, Theorem
\ref{numeasthm1} may be applied to the forward equi-asymptotically
stable system $\mathcal{D}_l$ (note that a local truncation error
of order $O(h)$ will not guarantee positive invariance of the
attracting neighbourhood in Theorem \ref{numeasthm1}). In general
however, this is not the case.

\subsubsection{Local Truncation Error for $\mathcal{D}_l$}

The dynamics for the system of loci $\mathcal{D}_l$ is governed by
an ODE. For this we may calculate
\begin{align*}
  \phi_{(t, 0)}(x_0) &= \Phi_{(t, \theta_{-t}p)}(x_0), \\
       &= x_0 + \int_{0}^t f(\theta_{(\tau-t)}p, \Phi_{(\tau, \theta_{-t}p)}(x_0))d\tau, \\
       &= x_0 + F(\tau, t)|^{\tau = t}_{\tau = 0}, \\
\end{align*}
where $F(\tau, t)$ is the primitive of the function
$f(\theta_{(\tau-t)}p, \Phi_{(\tau, \theta_{-t}p)}(x_0))$ with
respect to $\tau$. Hence,
\begin{align}
  \label{Dlodeeq}
  \frac{d\phi}{dt} &= \left( \frac{\partial}{\partial \tau}F(\tau, t)
               \right)\left|^{\tau = t}
               + \left( \frac{\partial}{\partial t}F(\tau, t)\right)\right|^{\tau = t}_{\tau =
               0}, \nonumber \\
          &= f(p, \Phi_{(t, \theta_{-t}p)}(x_0)) \left.
               + \left( \frac{\partial}{\partial t}F(\tau, t)\right)\right|^{\tau = t}_{\tau =
               0}, \nonumber \\
          &= f(p, \phi) + g(t, \phi),
\end{align}
where $g(t, \phi)$ is comprised of terms from $(\delta/\delta t)
F(\tau, t)|^{\tau = t}_{\tau = 0}$.

belonging to the latter half of the above expression.

For a numerical scheme of any order, it can be shown that the
numerical algorithm generating the discrete sequence of points
$\{x_n\}$  illustrated in Figure \ref{pdisc2} is of the form
\begin{equation}
\label{pddiffeq} x_{n+1} = x_n + C(t_n)h(t_n) + O(h(t_n)^2), \\
\end{equation}
where $C(t_n)$ is dependent on $f$ and its derivatives, and is
similar to an accumulation error. The long and tedious
calculations generating (\ref{pddiffeq}) have been omitted for
brevity.

Expressing $\phi_{(h(t_n, t_n)}(x_n)$ as a Taylor series around
$t_n$ (using (\ref{Dlodeeq})),
\[ \phi_{(h(t_n), t_n)}(x_n) = x_n + h(t_n) (f(p, x_n)
                  + g( t_0, x_n)) + O(h(t_n)^2). \]
The local truncation error, $|\phi_{(h(t_n), t_n)}(x_n) -
x_{n+1}|$ is of order $h(t_n)$. This is due to the fact that in
general, the linear terms for $x_{n+1}$ and $\phi_{(h(t_n,
t_n)}(x_n)$ do not cancel. Hence Theorem \ref{numeasthm1}, in
general, does not hold for $\mathcal{D}_l$.

The following example is an exception, and in fact generates a
local truncation error of order $O(h^2)$.

\begin{eg}
Consider the NDE generated by
\[ \dot{x} = f(t)x. \]
The set $A = \{0\}$ is assumed to be a pullback attractor. The
resulting dynamics for the system of loci $\mathcal{D}_l$ is
correspondingly determined by (\ref{scaseeq})
\[ \dot{\phi} = f(t_0-t)\phi. \]
Utilising a Euler scheme for the numerical method, the discrete
sequence of points $\{x_n\}$ generated in the pullback sense can
be explicitly calculated as follows

\begin{align*}
  x_{n+1} &= x^n_{n+1}( 1 + h(t_0)f(t_0 - t_{1})), \\
          &= x_0 \prod^{n+1}_{i=1} (1+h(t_{i-1})f(t_0 - t_{i})).  \\
\end{align*}
Here $x^n_{n+1}$ represents the $n$'th step in the discrete
sequence generating the image $x_{n+1}$ at $p$. Also expressing
$\phi$ as a power series around $(t_n,x_n)$,
\begin{align*}
\phi_{(h(t_{n}), t_n)}(x_n) &= x_n + h(t_n)(f(t_0 - t_n)x_n +
                    O(h(t_n)^2), \\
          &= x_0 \left(\prod_{i=1}^{n} (1+h(t_{i-1})f(t_0 - t_i))
                 \right) \\
          & \hspace{1cm} (1+h(t_n)f(t_0-t_n))+O(h(t_n)^2). \\
\end{align*}
The local truncation error can then be formed
\begin{align*}
|\phi_{(h(t_{n}), t_n)}&(x_n) - x_{n+1}| = |x_0
              \left(\prod_{i=1}^{n} (1+h(t_{i-1})f(t_0 - t_i))
              \right) \\
     & \hspace{1cm} (1+h(t_n)f(t_0-t_n) - 1 - h(t_n)f(t_0 -
              t_{n+1}))| + O(h(t_n)^2), \\
     &= |x_0 \left(\prod_{i=1}^{n} (1+h(t_{i-1})f(t_0 - t_i))
              \right) \\
     & \hspace{1cm} h(t_n) (f(t_0-t_n) - f(t_0-t_{n+1}))|
              + O(h(t_n)^2), \\
     &= |x_0 \left(\prod_{i=1}^{n} (1+h(t_{i-1})f(t_0 - t_i))
              \right) h(t_n)^2f'(t^*)| + O(h(t_n)^2), \\
     &\leq C(t_n)h(t_n)^2,
\end{align*}
for some $t_0 - t_{n+1} \leq t^* \leq t_0 - t_n$ as determined
by application of the mean value theorem.
$C(t_n) > 0$ is appropriately defined upon consideration of the bounds of the
derivative over the finite interval $t_0 - t_{n+1} \to t_0 - t_n$. Hence the
local truncation error is of $O(h(t_n)^2)$, and Theorem \ref{numeasthm2} may be
applied to guarantee the existence of a discrete equi-asymptotically family
approximating $A$ in $\mathcal{D}_l$. The limiting set of this family can then
be used as an approximation for $A$ at $t_0$ in the original dynamical system.
\end{eg}

\subsection{Discretisation by Lyapunov Method}

Assume the dynamical system defined by $\dot{x} = f(p, x)$
possesses a pullback equi-asymptotically stable family $\hat{A}$.







\endinput
