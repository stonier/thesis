\section{Pullback Lyapunov Theory}
\label{PLiapsec}

The difficulties in characterising the pullback behaviour of a dynamical system
lie in the fact that pullback asymptotic behaviour is not determined by the
behaviour along a trajectory, but characterised instead by the sensitivity of
the function to changes in the initial time.

As a result, determining the decrescent nature of such a function isn't easily
ascertained as in the forward case and finding a suitable function to determine
the system's pullback properties will be difficult to realise. Consequently we
will only deal with the converse theorem for pullback equi-asymptotic stability,
a result which is useful in understanding the numerics or perturbations of
such systems (recall the approach used for the simpler, perturbed autonomous
problems of Chapter \ref{pertautchapter}).

A further complication with pullback systems is that the rate of
attraction at some $p_0 \in P$ is independent of the rate of
attraction at a uniquely different point $p_1 \in P$. As a result,
the generated Lyapunov function will necessarily have the form
$V=V(p, t, x)$. This is in contrast to the forward case, where
the rate of attraction is able to be determined with respect to the current
state and no memory of the initial state is needed.

A Lyapunov function for cocycle (pullback) attractors is generated by P.
Kloeden in \cite{Kl98}. However it does not possess a true decrescence
property to characterise the rate of pullback attraction, and hence is expected
to be of limited value for an in depth pullback analysis of non-autonomous
dynamical systems. Although the function is used in \cite{KlKo01}, it is
essentially a forward analysis of a uniform object, and it is the
characterisation of the function along trajectories (forward behaviour) that
guarantees the result.

Kloeden's result in \cite{KlKo01} was also proved for this thesis
(Theorem \ref{numuasthm}) both concurrently and independently
using existing Lyapunov theory for uniform forward asymptotic
stability, since forward Lyapunov functions characterise behaviour
along trajectories in the same way as the Lyapunov function in
\cite{Kl98}.

An alternative Lyapunov-like function for pullback analysis is developed here
that most importantly possesses the essential \textit{decrescence property}
characterising the rate of pullback attraction at any point in time. Alone
it does not constitute a comprehensive Lyapunov-like pullback analysis,
however it does fulfill the following points.
\begin{itemize}
  \item [1)] Explores the difficulty in establishing a Lyapunov theory
for pullback dynamics.
  \item [2)] Provides a useful tool for numerical purposes.
  \item [3)] Forms the basis for further development of a Lyapunov-like
theory for a pullback analysis of dynamical systems.
\end{itemize}

Before proceeding, a method of evaluating the rate of attraction as the
initial state is pulled further back is needed. The Dini
Derivative provided a means of identifying the rate of change of
the Lyapunov function for non-smooth functions in the forward
case, so we will make use of a slightly modified construction for
the pullback case. For this we will use the notation
$D^{p}_{(\ref{NDEeq})}V(p, t, x)$ and define it as
\[ D^p_{(\ref{NDEeq})}V(p, t, x) = \overline{\lim_{h \to 0}} \frac{V(p,
  t+h, x) - V(p, t, x)}{h}, \]
where the superscript $p$ distinguishes it it as the rate of change function used for \textit{pullback} systems. The theorem is finally presented as follows.

\begin{therm}[Pullback Equi-Asymptotic Stability] \hfill \\
\label{conpasthm}
Suppose the dynamical system (\ref{NDEeq}) possesses a family $\hat{A} =
 \{A(p) ; p \in P \}$ that is pullback equi-asymptotically stable.
Then there exists a Lyapunov function $V:P \times \mathbb{R}^+ \times
\mathcal{N}_{K, \hat{A}}$ defined on a neighbourhood of $\hat{A}$,
$\mathcal{N}_{K, \hat{A}}$, which satisfies for each $p \in P$, $t \geq 0$
\begin{itemize}
  \item[a)] $V(p, t, x) = 0$ for each $x \in A(p),$
  \item[b)] $a(dist( \Phi_{(t, \theta_{-t} p)}(x), A(p))) \leq V(p, t, x)$
    where $a \in \mathcal{K}$,
  \item[c)] $D^{p}_{(\ref{NDEeq})}V(p, t, x) \leq -cV(p, t, x)$ for some
    constant $c > 0$,
  \item[d)] $V(p, \cdot, \cdot)$ is continuous in $t$ and locally Lipschitz
  in $x$.
\end{itemize}
\end{therm}
\begin{prf} \hfill \\
Let $p \in P$ be arbitrary. Then by the same principles as in Lemma
\ref{intro1lem}, there exists a $\delta > 0$ such that for any $\e > 0$, and
some $\e^*(\delta) > 0$, and $T = T(p, \e)$,
\begin{align*}
  dist(\Phi_{(t, \theta_{-t}p)}(x), A(p)) &< \e^*& \text{for all} \qquad & t >
    0, \\
  dist(\Phi_{(t, \theta_{-t}p)}(x), A(p)) &< \e& \text{for all} \qquad & t >
    T(p, \e). \\
\end{align*}
First we define some preliminary constants before proceeding with the
construction of a valid Lyapunov function.

Let $F= F(p, \e)$ where,
\[ F(p, \e)= 1 + \max_{t,x} \{ |f(\theta_{-t} p, x)| ; 0 < t < T( p, \e), x \in
  \mathcal{N}_{\delta}(A(\theta_{-t}p)), \} \]
and for some constant $c > 0$, define the function $\mathcal{A} =
\mathcal{A}(p, \e)$ by
\[\mathcal{A}(p, \e) = e^{cT(p, \e)} \exp \left(  \int_0^{T(p,
                   \e)}  L(\theta_{-s} p)ds \right)  F(p, \e),  \]
where $L(\cdot)$ is the Lipschitz constant for the function $f(p, x)$. By
J. Massera (\cite{Yo66}), there exists functions $l, g$ satisfying $l(p) > 0, 0
< g(\e) \leq 1$ for $\e > 0$ and $g(0) = 0$, such that
\[ g(\e)\mathcal{A}(p,\e) \leq l(p). \]
We need to analyse the behaviour of initial states as they are pulled back
in time. However, as $\hat{A}$ may be varying with $p$ we  make use of the
notation introduced earlier whereby we consider sequences of initial states
based on an initial state $x \in \mathcal{N}_{\delta}(A(p))$ defined by
$\hat{x} = \{ x_{\tau} ; \tau \geq 0, \dist(x_{\tau}, A(\theta_{-t}p) \leq
\dist(x, A(p)) \}$. The set of all sequences $\hat{x}$ thus defined
will be denoted by $X_x$.

Now, for $n=1,2,\dots,$ we define $V_n(p, t, x)$ for each $p \in P$, $t
> 0$, and $x \in \mathcal{N}_{\delta}(A(p))$ by:
\begin{align*}
  V_n(p, t, x) = g(1/n) \sup &\{ \sup_{\hat{x} \in X_x}
        D_n(dist(\Phi_{(\tau+t, \theta_{-(\tau+t)}
        p)}(x_{\tau+t}),A(p))) \\
  & e^{c\tau} ; \tau \geq 0 \},
\end{align*}
where $D_n( r )$ is a real valued function such that
\begin{equation*}
  D_n( r ) = \begin{cases}
  r - 1/n & ( r \geq 1/n ), \\
  0 & ( 0 \leq r \leq 1/n ).
  \end{cases}
\end{equation*}

{\em i) Invariance - } By invariance of the cocycle on $\hat{A}$
it is immediate that for each $p \in P$, and all $t > 0$, $x \in
\mathcal{N}_{\delta} (A(p))$,
\begin{equation*}
  V_n(p, t, x) = 0.
\end{equation*}

  {\em ii) Lower Bound -} Define $A_n(r)$ by
  \begin{equation*}
  A_n( r ) = \begin{cases}
  1/n(n-1) & (r \geq 1/(n-1) ), \\
  r - 1/n & ( 1/n \leq r \leq 1/(n-1) ), \\
  0 & ( 0 \leq r \leq 1/n ),
  \end{cases}
  \end{equation*}
  and set
  \[ a_n(r) = g(1/n) A_n(r). \]
  Here $A_n(r) \leq D_n(r)$, and that $a_n$ is a non-negative
  monotonically increasing function and is continuous with respect to $r$.
  Denote $r = dist( \Phi_{(t,
  \theta_{-t}p)}(x_t), A(p) )$ where $x_t \in
  \hat{x}$, then for any $\hat{x} \in X_x$ we have
  \begin{align*}
  V_n(p, t, x) &\geq g(1/n) \sup_{\hat{x} \in X_x} D_n( dist( \Phi_{(t,
    \theta_{-t}p)}(x_t), A(p) ) ), \\
  &\geq g(1/n) A_n(r), \\
  &\geq a_n(r). \\
  \end{align*}

  {\em iii) Upper Bound -} $\hat{A}$ is pullback stable, hence for each $x
  \in \mathcal{N}_{\delta}(A(p))$ and all $\hat{x} \in X_x$,
  \begin{align*}
    V_n&(p, t, x) \\
     & =g(1/n) \sup_{\tau} \{ \sup_{\hat{x} \in X_x}
        D_n(dist(\Phi_{(\tau+t, \theta_{-(\tau+t)} p)}(x_{\tau+t}),A(p)))
        e^{c\tau} \},  \\
    & \leq g(1/n) e^{cT(p, 1/n)} D_n( \e^*), \\
    &\leq l(p) \e^*,
  \end{align*}
  where $\e^*$ is as defined earlier for pullback stability. Note that this
  upper bound is dependent on $p$, as opposed to the upper bound generated for
  uniform equi-asymptotic stability.

  {\em iv) Decrescence -} Let $h > 0$ be some small constant.  Then
  \begin{align*}
  V_n&(p, t + h, x) \\
  &= g(1/n) \sup_{\tau \geq 0} \sup_{\hat{x} \in X_x}
        D_n(dist(\Phi_{(\tau+t+h, \theta_{-(\tau+t+h)} p)}(x_{\tau + t + h}),
        A(p))) e^{c\tau}, \\
  &= g(1/n) \sup_{\tau \geq h} \sup_{\hat{x} \in X_x} D_n(dist(\Phi_{(\tau+t,
        \theta_{-(\tau+t)} p)}(x_{\tau + t}), A(p))) e^{c\tau} e^{-ch}, \\
  &= e^{-ch} V_n(p, t, x).
  \end{align*}
  Taking $D^{p}_{(\ref{NDEeq})}V(p, t, x)$,
  \begin{align*}
  D^{p}_{(\ref{NDEeq})}V_n(p, t, x) &= \overline{\lim_{h \to 0^+}} \frac{V_n(
     p, t+h, x ) - V_n(p, t, x)}{h}, \\
  &\leq \overline{\lim_{h \to 0^+}} \frac{(e^{-ch} - 1) V_n(p, t, x)}{h}, \\
  &\leq - c V_n(p, t, x).
  \end{align*}

  {\em v) Local Lipschitz condition in $x$-} Let $x,  x' \in
  \mathcal{N}_{\delta}(A(p))$. Without loss in generality we will
  assume that
  \[ \dist(x, A(p)) \geq \dist(x', A(p)). \]
  Now for any $t$ so that
  $t > T(1/n)$, $V_n(p, t, x) = V_n(p, t, x') = 0$, for which the fulfillment
  of Lipschitzness is trivial, hence we will only consider the situation for
  which $0 \leq t \leq T(1/n)$.
  \begin{align*}
  |V_n(p, &t, x) - V_n( p, t, x')| \\
    & = g(1/n) | \sup_{\tau} \{ \sup_{\hat{x} \in
      X_x} e^{c\tau} D_n(dist(\Phi_{(\tau+t, \theta_{-(\tau+t)} p)}(x_{\tau+t}),
      A(p)))\} \\
    & \hspace{1cm} - \sup_{\tau} \{ \sup_{\hat{x}' \in X_{x'}} e^{c\tau}
      D_n(dist(\Phi_{(\tau+t, \theta_{-(\tau+t)} p)}(x'_{\tau+t}), A(p))) \} |, \\
    &\leq g(1/n) \sup_{0 \leq \tau \leq T - t} \{ e^{c\tau} |  \sup_{\hat{x} \in
       X_x} \dist(\Phi_{(\tau+t, \theta_{-(\tau+t)} p)}(x_{\tau+t}),A(p)) \\
       & \hspace{1cm} - \sup_{\hat{x}' \in X_{x'}} \dist(\Phi_{(\tau+t,
      \theta_{-(\tau+t)} p)}(x'_{\tau+t}),A(p)) |  \}. \\
  \end{align*}
  Since $dist(\Phi_{(\tau+t, \theta_{-(\tau+t)} p)}(x),
  A(p)) < 1/n $  for all $\tau > (T - t)$, the supremum may
  be taken over the bounded interval indicated. If the supremum is obtained for
  both expressions by an element within $X_{x'}$, then the argument is trivial.
  Hence we will assume that the first expression has a supremum for some
  $\hat{x}^* \in X_x$ but for which $\dist(x^*_{\tau + t}, A(\theta_{-(\tau +
  t)} p)) > dist( x', A(p))$. Also, consider some $\hat{x}^{'*} \in X_{x'}$
  chosen so that $|x^{'*}_{\tau + t} - x^*_{\tau + t}|$ is a minimum. Then
  \begin{align*}
  |V_n(p, t, x) &- V_n( p, t, x')| \\
    &\leq g(1/n) \sup_{0 \leq \tau \leq T-t} \{ e^{c\tau} |
      \dist(\Phi_{(\tau+t, \theta_{-(\tau+t)} p)}(x^*_{\tau+t}),A(p)) \\
    & \hspace{1cm} - \dist(\Phi_{(\tau+t, \theta_{-(\tau+t)} p)}
      (x^{'*}_{\tau+t}),A(p)) | \}, \\
    &\leq g(1/n) \sup_{0 \leq \tau \leq T-t} \{ e^{c\tau} | \Phi_{(\tau+t, \theta_{-(\tau+t)}
      p)}(x^*_{\tau+t}) \\
    &\hspace{1cm} - \Phi_{(\tau+t, \theta_{-(\tau+t)} p)}
      (x^{'*}_{\tau+t}) |  \}. \\
  \end{align*}
  Setting
  \[ \mathcal{L}(p, \e) = \exp \left( \int_0^{T(p, \e)} L(\theta_{-s} p) ds
    \right), \]
  for ease of notation, then by Lemma \ref{intro2lem},
  \begin{align*}
  |V_n(p, t, x) - V_n( p, t, x')| &\leq g(p, 1/n) e^{c(T(1/n)-t)}
    \mathcal{L}(p, 1/n) |x^*_{\tau + t} - x^{'*}_{\tau + t}|, \\
    &\leq e^{-ct} l(p) ((\dist(x, A(p)) - \dist(x', A(p))), \\
    &\leq e^{-ct} l(p) |x - x'|. \\
  \end{align*}
  Hence each $V_n(p, t, x)$ is Lipschitz with respect to $x$.

  {\em vi) Continuity in $t$-} Without loss of generality, let $0 < t' < t$,
  and $x \in \mathcal{N}_{\delta}(A(p))$. Then
  \begin{align*}
  |V_n&(p, t, x) - V_n( p, t', x)| \\
    &= g(1/n) \left| \sup \{
    e^{c\tau} \sup_{\hat{x} \in X_x} D_n(dist(\Phi_{(\tau+t,
    \theta_{-(\tau+t)} p)}(x_{\tau + t}), A(p))) ; \tau \geq 0 \} \right. \\
  & \hspace{5mm} - \left. \sup \{ e^{c\tau} \sup_{\hat{x} \in X_x}
    D_n(dist(\Phi_{(\tau+t', \theta_{-(\tau+t')} p)}
    (x_{\tau + t'}), A(p))) ; \tau \geq 0 \} \right|, \\
  &\leq g(1/n) \sup \sup_{\hat{x} \in X_x} \{ e^{c\tau}
    \left |D_n(dist(\Phi_{(\tau+t, \theta_{-(\tau+t)}
    p)}(x_{\tau + t}),A(p))) \right. \\
  & \hspace{5mm} \left. - D_n(dist(\Phi_{(\tau+t', \theta_{-(\tau+t')} p)}
    (x_{\tau + t'}), A(p))) \right| ; 0 \leq \tau \leq T - t' \}, \\
  &\leq g(1/n) \sup \sup_{\hat{x} \in X_x} \{ e^{c\tau} | \Phi_{(\tau+t,
    \theta_{-(\tau + t)} p)}(x_{\tau + t}) \\
  & \hspace{2cm} - \Phi_{(\tau+t',\theta_{-(\tau+t')} p)}(x_{\tau + t'})| ; 0
    \leq \tau \leq T - t' \},  \\
  &\leq g(1/n) \sup \sup_{\hat{x} \in X_x} \{ e^{c\tau} | \Phi_{(\tau+t',
    \theta_{-(\tau+t')} p)}(X_{\tau + t'}) \\
  & \hspace{2cm} - \Phi_{(\tau+t',\theta_{-(\tau+t')}
    p)}(x_{\tau + t'}) | ; 0 \leq \tau \leq T - t' \},  \\
  \end{align*}
  where $X_{\tau + t'} = \Phi_{(t-t', \theta_{-(\tau + t)}p)}(x_{\tau + t})$.
  Recalling $\mathcal{L}(p, \e)$ defined earlier, then by Lemma \ref{intro2lem},
  and considering Lemma \ref{intro3lem} on the bounded interval $[0, T(p,
  1/n)]$ we have,
\begin{align}
  |V_n(p, &t, x) - V_n( p, t', x)| \\
  & \leq g(1/n) \sup_{\tau} \sup_{\hat{x} \in X_x} \{
    e^{c\tau} \mathcal{L}(p, 1/n) |X_{\tau + t'} - x_{\tau + t'}|
    ; 0 \leq \tau \leq T - t' \}, \notag \\
  &\leq g(1/n) e^{c T((p, 1/n)} e^{-ct'} \mathcal{L}(p, 1/n) F(p, 1/n) |t - t'|,
    \notag \\
  &= e^{-ct'} |t - t'|. \notag \\
  \end{align}
  Note for all $t, t' > T(p, 1/n)$, the function $V_n =0$, so we need only
  consider the difference on this set.

  As a result each $V_n$ is continuous and in fact Lipschitz with
  respect to $t$.


  Finally, we define the Lyapunov function $V$ by
  \[ V(p, t, x) = \sum_{n=1}^{\infty} \frac{1}{2^n} V_n(p, t, x). \]
  Note that this series converges as a consequence of iii). Properties
  a) - d) will be verified sequentially.

  {\em a)} Clearly from i) we have for each $p \in P$, $t \geq 0$, and all $x
  \in A(\theta_{-t}p)$
  \[ V(p, x) = 0. \]

  {\em b-i) Lower Bound - } From ii), if we set
  \[ a(r) = \sum_{n=1}^{\infty} \frac{1}{2^n} a_n(r), \]
  we have $a \in \mathcal{K}$. Clearly $a(0)=0$. By the Weierstrass M-test
  it is a continuous function. Also $a(r) > 0$   for $r>0$ since for any $r$
  there exists an $n$ such that $(1/n) <   r$ and hence $a(r) > a_n(r) = (r -
  1/n)$. It is a lower   bound for $V(p, t, x)$, where $r = dist( \Phi_{(t,
  \theta_{-t}p)}(x), A(p))$,   as shown below.
  \begin{align*}
  V(p, t, x) &= \sum_{n=1}^{\infty} \frac{1}{2^n} V_n(p, t, x), \\
    &\geq \sum_{n=1}^{\infty} \frac{1}{2^n} a_n(r), \\
    &\geq a(r). \\
  \end{align*}

    {\em c) Decrescence - } From iv),
  \begin{align*}
  V( p, t+h, x ) &= \sum_{n=1}^{\infty} \frac{1}{2^n} V_n( p, t+h, x ), \\
    &\leq e^{-ch} \sum_{n=1}^{\infty} \frac{1}{2^n} V_n(p, t, x), \\
    &\leq e^{-ch} V(p, t, x). \\
  \end{align*}

  Again, taking the Dini derivative for $V$, we arrive at the required
  result.

  {\em d) Continuity and Lipschitz Properties of V -} This follows
  directly from the continuity and Lipschitzness of each $V_n$ in v).
\end{prf}


\endinput
