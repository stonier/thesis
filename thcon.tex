\nonumchapter{Conclusions}

The nature of stability and numerical approximation over non-finite intervals
for non-autonomous dynamical systems has been a topic left relatively unexplored
until only recently.

Research primarily by P. Kloeden and B.Schmalfuss initiated an investigation
into attractive structures within a non-autonomous context, and this has been
an initiative taken up by several other authors since (notably D.Cheban,
P.Flandoli and V.Kozyakin among others). The results generated however,
primarily reflect the use of pullback attractors in their field of interest,
and a comprehensive analysis of non-autonomous stability was still incomplete.

The initial chapters of this thesis were written to provide as comprehensively
as possible, the fundamentals of non-autonomous stability.  In scope, it
introduces several new concepts, but also incorporates the work on pullback
attractors by P.Kloeden et. al, as well as retaining existing classical
asymptotic stability theory as an integral component. It also introduces a
preliminary Lyapunov theory for pullback stability, although its usefulness as a
tool may well be restricted to the Theorems of the converse nature.

A comprehensive stability theory for non-autonomous dynamical
systems as composed here should  provide the essential basis for which further
research in control, chaos theory and numerical approximation of non-autonomous
dynamical systems will benefit.


The latter half of the thesis devotes its attention to the numerical
approximation of non-autonomous dynamical systems over non-finite intervals, a
topic initially explored by A. Stuart for autonomous dynamical systems with
relevance to understanding computer models that approximated real time dynamics.
Obtaining equivalent results for non-autonomous dynamical systems however is
increasingly difficult as one diverges from properties of uniformity. Results
for uniform and non-uniform attraction were found for forward asymptotic theory,
however the case for pullback asymptotic approximations are shown to be much
more complicated.  Developing the loci theory as an aid for understanding
pullback asymptotic behaviour proved to be useful in determing
numerically approximated behaviour of such systems on a case by case
basis. As such, this leaves  it as a basis for application to further
problems.


\endinput
