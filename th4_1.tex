
\chapter[Non-Autonomously Perturbed Systems]{Perturbed Autonomous Dynamical Systems}
        \label{pertautchapter}

In this chapter we consider autonomous differential equations (both
continuous and discrete) which are known to possess a semi-group attractor.
In practice however, the use of an autonomous model will almost
always be subject to small perturbations, and it is often
desirable to determine under what conditions the perturbed system
will retain an attracting object with similar characteristics to
that of the original attractor. A Lyapunov approach involving the
use of absorbing sets is applied here to guarantee the conditions
needed to ensure the existence of a pullback attractor in the
perturbed system with similar characteristics to that of the
original semi-group attractor.

\section{Continuous Autonomous Systems}\label{Cpertautsec}

The results that follow regarding continuously perturbed
autonomous systems are a collection of those  published by
P.Kloedon and D.Stonier in \cite{PkSt97}.

\subsection{The Non-Autonomous Perturbation}

We consider autonomous dynamical systems generated by the ordinary
differential equation
\begin{equation} \label{aodeeq}
  \dot{x} = f(x),
\end{equation}
where $x \in E$, $E \subset \mathbb{R}^d$. We assume that the following conditions 
hold for the autonomous system:

\textbf{P1)} $f$ is continuous and Lipschitz with respect to $x$.

\textbf{P2)} The autonomous system possesses a semi-group attractor $A_0$, as
introduced in Definition \ref{att}.

We then subject the original ordinary differential equation to a
non - autonomous perturbation $g:\mathbb{R}
\times E \to \mathbb{R}^d$, to obtain a non-autonomous ordinary
differential equation on $E$,
\begin{equation} \label{pertnodeeq}
  \dot{x} = f(x) + \epsilon g(\ta,x),
\end{equation}
where $\epsilon > 0$ is a small parameter.

\textbf{P3)} The perturbing function is uniformly bounded. That is
\begin{equation}\label{eqpertbd}
  \sup_{(\ta,x) \in \mathbb{R} \times \mathbb{R}^d} ||g(\ta,x)|| \leq
       K < \infty.
\end{equation}

The non-autonomous system (\ref{pertnodeeq}) generates a cocycle
$\{\Phi^{\epsilon}_{(t,t_0)} ; t \in \mathbb{R}^+, t_0 \in \mathbb{R} \}$ over
the parameter set $P = \mathbb{R}$ with shift mapping $\theta_t(t_0) = t_0 + t$.
The main theorem shows that the non-autonomously perturbed system possesses a
pullback attractor $\hat{A}_{\epsilon} = \{ A^{\e}(\ta) ; \ta \in \mathbb{R}
\}$.

\subsection{Main Theorem}

Before proceeding to the Main Theorem the following lemma is presented. It is an integral part of the proof for the main theorem that follows.

%$A = [1+M^{-1}]e^{M|t_1 - t_0|}$???

\begin{lemma} \label{lem1npertthm}Let $A = e^{M|t_1 - t_0|}$ where $M$
is the uniform Lipschitz constant of $G(\ta,x) = f(x) + \epsilon g(\ta,x)$ in $x$
uniformly in $\ta \in \mathbb{R}$. That is, for any $x_1, x_2 \in E$, 
\[ ||G(\ta, x_1) - G(\ta, x_2)|| \leq M ||x_1-x_2||, \]
for all $\ta \in \mathbb{R}$. Then
  \[ A^{-1}||x-y|| \leq ||\Phi^{\epsilon}_{(t_1 - t_0,t_0)}(x) -
        \Phi^{\epsilon}_{(t_1 - t_0,t_0)}(y)|| \leq A||x-y|| \]
\end{lemma}
\begin{prf}
  Write $\Delta \Phi^{\epsilon}_{(t,t_0)} := \Phi^{\epsilon}_{(t,t_0)}(x) -
  \Phi^{\epsilon}_{(t,t_0)}(y)$ for $0 \leq t \leq t_1 - t_0$. Then
  \begin{align*}
  \Delta \Phi^{\epsilon}_{(t,t_0)} &= (x-y) +  \\
        & \int^{t+t_0}_{t_0} [G(s, \Phi^{\epsilon}_{(s,t_0)}(x)) - G(s,
        \Phi^{\epsilon}_{(s,t_0)}(y))]ds. \\
  \end{align*}
  Using the Lipschitz condition, we have
  \[ ||\Delta \Phi^{\epsilon}_{(t,t_0)}|| \leq ||x-y|| + M
        \int^{t+t_0}_{t_0} ||\Delta \Phi^{\epsilon}_{(s,t_0)}||ds \]
and the right hand inequality then follows by application of the Gronwall
inequality (Lemma \ref{intro2lem}).

To obtain the left hand inequality, we repeat the argument
starting at $t_1$   and integrate backwards to $t_0$ from the previously
obtained endpoints at   $t_1$.
\end{prf}

\begin{therm}[Non-Autonomous Perturbation]\label{npertthm}
For the perturbed autonomous dynamical system (\ref{pertnodeeq}), suppose
conditions \textbf{P1 - P3} hold.

Then there exists a pullback attractor $\hat{A}_{\epsilon} = \{
A^{\e}(\ta) ; \ta \in \mathbb{R} \}$ such that
\[ \lim_{\epsilon \to 0^+} H^*(A^{\e}(t), A_0) = 0 \]
for all $\ta \in \mathbb{R}$. In addition, the component sets
$A^{\e}(\ta)$ are continuous in $\ta$, that is,
\[ \lim_{\ta \to t_0} H(A^{\e}(\ta),A^{\e}(t_0)) = 0 \qquad
        \forall t_0 \in \mathbb{R}, \]
and have constant Hausdorff dimension
\[ \text{dim}_{H} A^{\e}(t_0) = \text{dim}_{H} A^{\e}(t_1)
                \qquad \forall t_0,t_1 \in \mathbb{R}. \]
\end{therm}
\begin{prf}
{\em 1. Existence of a Pullback Absorbing Set:} \hfill \\ 
The semi-group attractor
$A_0$ is a uniformly asymptotically stable set, and by Theorem
\ref{LUASCtherm}, there exists a Lyapunov function $V$ on some
neighbourhood, $\mathcal{N}_R(A_0)$, of the attractor which characterises
this uniform asymptotic stability. This Lyapunov function will be used to
construct a pullback absorbing set for the non-autonomous system.

We will consider the effect of the Lyapunov function on a subset of this
neighbourhood, defined by $\mathcal{N}(A_0) = \{ x \in \mathcal{N}_R(A_0);
V(x) < a(R) \}$, where the function $a(\cdot)$ is the lower bounding
class $\mathcal{K}$ function in Theorem \ref{LUASCtherm} corresponding to
the Lyapunov function $V(x)$. This neighbourhood is chosen to ensure that
solutions pulled back from within this set remain within
$\mathcal{N}_R(A_0)$ as will be seen later.

We now proceed to determine the rate of change of $V$ for solutions in the perturbed system (\ref{pertnodeeq}). For any $\ta \in \mathbb{R}$ and $x \in \mathcal{N}_R(A_0)$,

\begin{align}\label{lineq}
 D^{+}_{(\ref{pertnodeeq})}V(x) &= \overline{\lim_{h \to 0^+}} \left\{
        \frac{V(x+h(f(x) + \epsilon g(\ta,x))) - V(x)}{h} \right\}
        \nonumber \\
 &= \overline{\lim_{h \to 0^+}} \left\{\frac{V(x+h(f(x) + \epsilon g(\ta,x)))
        - V(x + hf(x))}{h} \right. \nonumber \\
 & \hspace{2cm} \left. + \frac{V(x + hf(x)) - V(x)}{h} \right\} \nonumber \\
 & \leq \overline{\lim_{h \to 0^+}} \left\{\frac{Lh\epsilon ||g(\ta,x)||}{h}
        + \frac{V(x + hf(x)) - V(x)}{h} \right\} \nonumber  \\
 & \leq L \epsilon ||g(\ta,x)|| + \overline{\lim_{h \to 0^+}} \frac{V(x +
        hf(x)) - V(x)}{h} \nonumber \\
 & \leq L K \epsilon + D^{+}_{(\ref{aodeeq})}V(x) \nonumber  \\
 & \leq L K \epsilon - c V(x), \nonumber \\
\end{align}

where $L$ is the Lipschitz constant of $V(x)$. The
result of Theorem \ref{LUASCtherm} has been used on the last line.

Now for all $x \not \in B^{\e}$, where
\[ B^{\e} =  \{ x \in \mathcal{N}_{R}(A_0); V(x) \leq 2LK
                \epsilon/ c \}, \]
we have from (\ref{lineq})
\begin{equation}\label{lyineq}
  D^{+}_{(\ref{pertnodeeq})}V(x) \leq - LK \epsilon.
\end{equation}
If $\epsilon$ is small enough so that
\begin{equation}\label{pertsize}
  \epsilon < c a(R)/2LK,
\end{equation}
then $B^{\e}$ is strictly a subset of $\mathcal{N}(A_0)$.

To see that the set $B^{\e}$ is a pullback absorbing neighbourhood
for the perturbed system, consider the Lyapunov function on solutions $x_0
\in \mathcal{N}(A_0) \backslash B^{\e}$, pulled
back from time $t_0$. Using (\ref{lyineq}), we have
\[ V(\Phi_{(\tau,t_0-t)}(x_0)) \leq V(x_0) - LK\epsilon \tau, \]
for all $0 \leq \tau \leq t$ such that $\Phi_{(\tau,t_0 - t)}(x_0) \not \in
B^{\e}$. Hence there exists a $T(x_0) > 0$ such that $\Phi_{(\tau,t_0
- t)}(x_0) \not \in B^{\e}$ for all $\tau < T(x_0)$, but
$\Phi_{(T,t_0 - t)}(x_0) \in B^{\e}$.

Using similar reasoning it can be seen that $B^{\e}$ is actually
positively invariant for solutions of (\ref{pertnodeeq}), and so for all
$x_0 \in \mathcal{N}(A_0)$ we have
\[ \Phi_{(t,t_0-t)}(x_0) \in B^{\e} \qquad \forall t > T(x_0). \]
An upper bound $T^*$ for $T(x_0)$ exists for all $x_0 \in \mathcal{N}(A_0)$
since the Lyapunov function is bounded above on $\mathcal{N}(A_0)$ by
$a(R)$. Hence
\[ \Phi_{(t,t_0-t)}(\mathcal{N}(A_0)) \in B^{\e} \qquad \forall t > T^*. \]
Thus $B^{\e}$ pullback absorbs a neighbourhood of itself, and is a
pullback absorbing neighbourhood for the cocycle $\{ \Phi_{(t,t_0)}; t \in
\mathbb{R}^+, t_0 \in \mathbb{R} \}$ generated by the non-autonomous
differential equation (\ref{pertnodeeq}).

We then apply Theorem \ref{abspat} to this cocycle and pullback
absorbing neighbourhood to verify the existence of a pullback attractor,
$\hat{A}_{\epsilon}$ contained within $B^{\e}$.

{\em 2. Approximation Property:} For $x \in B^{\e}$ we have
\[ a({\rm dist}(x,A_0))  \leq V(x) \leq 2LK \epsilon / c, \]
so ${\rm dist}(x,A_0) \leq a^{-1}(2LK \epsilon /c)$. Hence
\[ H^*(B^{\e}, A_0) \leq a^{-1}(2LK \epsilon /c), \]
and since $A^{\e}(t_0) \subset B^{\e}$, we have
\begin{align*}
  H^*(A^{\e}(t_0),A_0) &\leq a^{-1}(2LK \epsilon /c). \\
\intertext{Consequently,}
  H^*(A^{\e}(t_0),A_0) &\rightarrow 0^+ \qquad \text{as} \qquad
             \epsilon \rightarrow 0^+. \\
\end{align*}
{\em 3. Continuity:} The pullback attractor family is
$\Phi^{\epsilon}$-Invariant. Further the continuity of the
cocycle in all of its variables implies the continuity of
$\Phi^{\epsilon}_{(\cdot,t_0)}(\cdot): \mathbb{R}^+ \times
\mathcal{H}(E) \to \mathcal{H}(E)$. Hence
\begin{align*}
  H(A^{\e}(t_0 + t),A^{\e}(t_0)) &= H(\Phi^{\epsilon}
       (A^{\e}(t_0)), A^{\e}(t_0)). \\
  \intertext{Consequently}
  H(A^{\e}(t_0 + t),A^{\e}(t_0))  & \rightarrow 0^+ \qquad \text{as}
\qquad t \rightarrow 0. \\
\end{align*}

{\em 4. Hausdorff Dimension:} Note that
  $\Phi^{\epsilon}_{(t_1-t_0,t_0)}(\cdot)$ for any $t_0 \leq t_1 \in
  \mathbb{R}$ is a bi-Lipschitz mapping by Lemma \ref{lem1npertthm}. So by
  Corollary   2.4, page 30 of \cite{Fa90} the sets $A^{\e}(t_0)$ and
  $A^{\e}(t_1)$ have the same Hausdorff dimension.

This completes the proof of Theorem \ref{npertthm}.
\end{prf}

{\bf Remark 1.} To guarantee the existence of a pullback attractor, the
size of the perturbation is required to be restricted as given by inequality
(\ref{pertsize}). To actually calculate the bound for $\epsilon$ is, in some
cases not possible, as the form of the Lyapunov function may not be known,
only that it exists, in which case the form of the function $a(\cdot)$ is
also unknown.

Note that in the case that the original attractor $A_0$ is global, then
restricting the size of the scalar value $\epsilon$ is unnecessary.

\subsection{Time Periodic Perturbations}

For time periodic perturbations, the cocycle and the pullback attractor
components are also periodic with the same period as the perturbations and
a forwards convergence property also holds. Hence the pullback attractor,
is in fact a complete attractor.

\begin{cor}[Periodic Non-Autonomous Perturbation] \label{pnpertcor} \hfill
                        \\
If in addition, the perturbations $g(\cdot,x)$ are periodic with period
$T$, then the cocycle is $T$- periodic, that is
\[ \Phi^{\epsilon}_{(t,t_0)}(x_0) = \Phi^{\epsilon}_{(t,t_0+T)} \qquad
        \forall t \in \mathbb{R}^+, t_0 \in \mathbb{R}, x_0 \in E, \]
and the components of the pullback attractor $\hat{A}_{\epsilon}$ are also
periodic with period $T$. Additionally, $\hat{A}$ is in fact a complete
attractor.
\end{cor}
\begin{prf}
{\em 1. Periodicity:} Let $g$ be periodic in $\ta$ with period $T$. Then
$G(\ta,x) = f(x) + g(\ta,x)$ is also $T$-periodic in $\ta$, specifically
$G(\ta-T,x) = G(\ta,x)$ for all $\ta \in \mathbb{R}$ and $x \in E$.

Let $x$ be the solution of the differential equation
(\ref{pertnodeeq}) with initial value $x(t_0) = x_0$. Consider
the shifted solution $X$ defined by $X(\ta) = x(\ta-T)$ with initial value
$X(t_0 + T) = x_0$. It can be seen that it satisfies
(\ref{pertnodeeq}), since
\begin{align*}
  \frac{dX}{d \ta}(\ta) &= \frac{dx}{d \ta}(\ta-T) \\
  &= \frac{dx}{d \tau}(\tau) \hspace{1.5cm} \text{where} \qquad \tau =
                \ta-T  \\
  &= G(\tau,x(\tau)) \\
  &= G(\ta-T,X(\ta)) = G(\ta,X(\ta)), \qquad \text{by T-periodicity of G}, \\
\end{align*}
and so is a solution of the differential equation. Hence by uniqueness we
must have $\Phi_{(t,t_0)} = \Phi_{(t,t_0+T)}$, that is, $T$-periodicity of
the cocycle.

To see that the pullback attractor $\hat{A}_{\epsilon}$ is also
$T$-periodic, we replace the $t_0$ above with $t_0-t$ where $t \geq 0$. We
then have $\Phi^{\epsilon}_{(t,t_0-t)}(x_0) = \Phi^{\epsilon}_{(t,t_0 - t +
T )} (x_0)$, and hence by (\ref{patfromabs}),
\[ A^{\e}(t_0) = \bigcap_{\tau \geq 0} \overline{\bigcup_{t \geq \tau}
            \Phi^{\epsilon}_{(t,t_0-t)}(B^{\e})} = \bigcap_{\tau
            \geq 0} \overline{\bigcup_{t \geq \tau}
            \Phi^{\epsilon}_{(t,t_0 + T -t)}(B^{\e})} =
            A^{\e}(t_0 + T). \]
Thus $\hat{A}_{\epsilon}$ is $T$-periodic.

{\em 2. Forwards Convergence:} The rate of attraction governed by
(\ref{pertnodeeq}) is completely determined on the compact interval
$[t_0, t_0 + T]$. In fact the parameter space may be compactified
so that $P = t_0$ mod $T$. As a result the rate of attraction will
be uniform and since the attracting neighbourhood is also
independent of the initial time, $\hat{A}^{\e}$ a uniform attractor.
Hence by Lemma \ref{uattlem}, $\hat{A}^{\e}$ is a complete
attractor.
\end{prf}

\subsection{Asymptotically Vanishing Perturbations}

For asymptotically vanishing perturbations, the cocycle and the pullback
attractor components converge to the autonomous semi-group and its
attractor, respectively.

\begin{cor}[Vanishing Perturbations]
Consider the dynamical system (\ref{pertnodeeq}) and suppose conditions
\textbf{P1 - P3} hold. In addition, the perturbations satisfy the condition
\begin{equation}\label{vaneq}
 \sup_{x \in E} ||g(\ta,x)|| \rightarrow 0 \qquad \text{as} \qquad \ta
                        \rightarrow \infty.
\end{equation}
Then the cocycle $\{\Phi^{\epsilon}_{(t,t_0)} ; t \in \mathbb{R}^+, t_0
\in \mathbb{R} \}$ converges uniformly for each $t \geq 0$ as $t_0
\rightarrow \infty$ to the semi-group $\{S_t ; t \in \mathbb{R}^+ \}$ of
the autonomous system (\ref{aodeeq}). That is
\[ \sup_{x_0 \in E} || \Phi^{\epsilon}_{(t,t_0)}(x_0) - S_t(x_0)||
        \rightarrow 0 \qquad \text{as} \qquad t_0 \rightarrow \infty, \]
for each $t \in \mathbb{R}^+$. Also the pullback attractor components
satisfy
\[ \lim_{\tau \rightarrow \infty} H^* (A^{\e}(\tau),A_0) = 0, \]
where $A_0$ is the semi-group attractor.
\end{cor}
\begin{prf}
{\em 1. Asymptotic Convergence of the Cocycle to the Semi-Group:}
  Due to the vanishing of the perturbations (\ref{vaneq}), given any
  $\gamma > 0$ there exists a $T(\gamma) \in \mathbb{R}^+$ such that
  \[ \sup_{x \in E} ||g(\ta,x)|| \leq \gamma  \qquad \forall \ta \geq
                        T(\gamma), \]
  where $T(\gamma) \rightarrow \infty$ as $\gamma \rightarrow 0^+$.
  Let $S_t(x_0)$ be the solution of the autonomous equation (\ref{aodeeq}),
  and $\Phi^{\epsilon}_{(t,t_0)}(x_0)$ be the corresponding solution of the
  non-autonomous equation (\ref{pertnodeeq}), and define
  \[ \Delta(t,t_0,x_0) = ||\Phi^{\epsilon}_{(t,t_0)}(x_0) - S_t(x_0)||. \]
  Writing the solutions to the differential equations in integral form 
  we obtain for any $t_0 \geq T(\gamma)$, and $t \geq 0$,
  \begin{align*}
    \Delta(t,t_0,x_0) & \leq || \int^t_0 \left(
                f(\Phi_{(s,t_0)}(x_0)) - f(S_s(x_0)) \right) ds || \\
    & \hspace{2cm} + \epsilon || \int^t_0 g(t_0 + s, x(s;t_0;x_0)) ds || \\
    & \leq \int^t_0 || f(\Phi_{(s,t_0)}(x_0)) - f(S_s(x_0)) || ds  \\
    & \hspace{2cm} + \epsilon \int^t_0 || g(t_0 + s, \Phi_{(s,t_0)}(x_0)) || ds \\
    & \leq M \int^t_0 \Delta(s,t_0,x_0) ds + \epsilon \gamma t, \\
  \end{align*}
  where M is the Lipschitz constant of $f$. The Gronwall inequality
  (Lemma \ref{intro2lem}) then gives
  \[ \Delta(t,t_0,x_0) \leq \frac{\epsilon \gamma}{M} t \exp (Mt). \]
  In terms of the cocycle and the semi-group
  \[ ||\Phi^{\epsilon}_{(t,t_0)}(x_0) - S_t(x_0)|| \leq \frac{\epsilon
        \gamma}{M} t \exp (Mt), \]
  for $t_0 \geq T(\gamma)$, any $t \geq 0$ and any $x_0 \in E$. The
  asymptotic convergence as $t_0 \rightarrow \infty$ for fixed $t \geq 0$
  (in fact, uniformly in $t$ in bounded intervals) follows since for any
  $c > 0$ there exists a $\gamma$ and hence a $T(\gamma)$ such that for all
  $t_0 > T(\gamma)$, we have $||\Phi^{\epsilon}_{(t,t_0)}(x_0) - S_t(x_0)||
  < c$.

{\em 2. Asymptotic Convergence of the Pullback Attractor:}
With $\gamma$ and $T(\gamma)$ as above, and from the Lyapunov inequality
(\ref{lineq}), we obtain
\begin{align*}
  D^{+}_{(\ref{pertnodeeq})}V(x) &\leq L \e ||g(\ta,x)|| +
        D^{+}_{(\ref{aodeeq})}V(x) \\
  & \leq L \e ||g(\ta,x)|| - c V(x) \\
  & \leq L \e \gamma - c V(x) \leq - L \e \gamma, \\
\end{align*}
for all $x \not \in B^{\e, \gamma} = \{ x \in \mathcal{N}_{\delta}(A_0) ;
V(x) \leq 2L \e \gamma / c \}$ and $\ta \geq T(\gamma)$.

From this it can be seen that $B^{\e, \gamma}$ is positively invariant with
respect to the cocycle $\Phi_{(t,t_0)}$ for all $t_0 \geq T(\gamma)$ and
$t \geq 0$.

Also trajectories starting outside of $B^{\e, \gamma}$ enter it after a
finite time. In particular, there exists a $T(\e, \gamma) \geq 0$
such that
\[ a(\dist(\Phi_{(t,t_0)}(x_0), A_0) \leq V(\Phi_{(t,t_0)}(x_0)) \leq 2L
        \e \gamma /c, \]
for all $x_0 \in B^{\e}$, $t_0 \geq T(\gamma)$ and $t \geq T(\e,
\gamma)$, from which we have
\[ \dist(\Phi_{(t,t_0)}(x_0), A_0) \leq a^{-1}(2L \e \gamma /c). \]
As $A^{\e}(t_0) \subset B^{\e}$, and since $\hat{A}_{\e}$ is
$\Phi$-Invariant we have
\[ H^*(A^{\e}(t_0 + t),A_0) \leq a^{-1}(2L \e \gamma /c), \]
for $t_0 \geq T(\gamma)$ and $t \geq T(\e, \gamma)$. This gives the
desired asymptotic limit since $T(\gamma) \rightarrow \infty$ as $\gamma
\rightarrow 0^+$.

\end{prf}

\endinput
