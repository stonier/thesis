\chapter[Numerical Approximation - II]
         {Discretisation of Pullback Asymptotic Behaviour}
\label{chpbackdisc}

The focus of this chapter is to understand the effects of discretisation on
pullback asymptotically stable families. An analysis of the pullback
numerics however, is inherently more complicated than that of the forward
numerics analysed in Chapter \ref{Discsec}. Finding an approach to verify
results analogous to Theorem \ref{numeasthm1} for pullback equi-asymptotically
stable families is not as clear, and a completely analogous result is yet
open to further research.

Before the problem is discretised, an investigation into the
pullback asymptotic behaviour for a single $p \in P$ is undertaken
by means of a transformation applied to the original dynamical
system. Pullback asymptotic behaviour in the original system is
characterised by forward asymptotic behaviour in the transformed
problem. The transformation has the benefit of allowing
conventional techniques to be applied to the problem of
discretisation, as well as assisting in visualising the pullback
attraction in the original dynamical system.

The theory is first introduced, followed by application of numerical methods
that utilise the transformed problem to establish a numerical approximation of
the original pullback behaviour.

\section{Duality of Attraction}\label{secduality}

\subsection{Introduction}

We consider one particular aspect of pullback behaviour,
and we refer to this as the \textit{duality of attraction}. The procedure
fixes a single $p \in P$, and transforms the dynamical system
into one  in which, as previously mentioned, the pullback behaviour at $p$ is
characterised by forward attraction. As far as the author
is aware, no other research has currently been published in this direction.

\subsection{Loci Dynamics for $A$}
\label{ssecLoci1}
Pullback asymptotic stability to a fixed point $p \in P$ can be
modelled by considering the dynamical system resulting from an
analysis of the system's sensitivity to initial times. To illustrate the
fundamental concept, we begin with the example below.

\begin{eg}
Consider the dynamical system arising from the NDE
\[ \dot{x} = 2\ta x. \]
Solutions for this system may be expressed in cocycle form as (noting that
in cocycle form $t$ represents the time elapsed rather than actual time)
\[ \Phi_{(t,t_0)}(x_0) = x_0 e^{(t+ t_0)^2 - t_0^2}, \]
where the parameter space $P = \mathbb{R}$ and $t_0 \in \mathbb{R}$.
Alternatively, by considering pullback attraction to the point $t_0$, we may
express solutions by
\[ \Phi_{(t, t_0-t)}(x_0) = x_0 e^{t_0^2 - (t_0-t)^2}. \]
It is easily proven that the origin for this system is a pullback
attractor without forward convergence properties. This can be seen
graphically in Figure \ref{loci0} where the initial state is set
at $x_0 = 1$ and pullback attraction to $t_0 = 1/2$ is considered.

\begin{figure}[htb]
\begin{center}
\input{eps/loci0.pstex_t} \caption{Locus of Transformation}
\protect\label{loci0}
\end{center}
\end{figure}

The solutions at $t_0$ define a mapping with image $\Phi_{(t,t_0-t)}(x_0)$
associated with each $t \geq 0$. If the locus of points defined by
$(t_0 - t, \Phi_{(t,t_0-t)}(x_0))$ is plotted as illustrated,
the locus forms a continuous trajectory in reverse that asymptotes to the
origin as $t$ is increased. Similar loci may be plotted for each $x_0$,
generating a system resembling a dynamical system with forward asymptotic
properties.

The behaviour of the loci may in fact be calculated by determining the rate of
change of the image ($\Phi_{(t,t_0-t)}(x_0)$) with respect to $t$. Hence we
have
\begin{align*}
\frac{d}{dt}\left(\Phi_{(t,t_0-t)}(x_0)\right) &= \frac{d}{dt} \left( x_0
                e^{t_0^2 - (t_0-t)^2)} \right), \\
&= 2(t_0 - t) \Phi_{(t,t_0-t)}(x_0).
\end{align*}
As $x_0$ is arbitrary, the dynamics for the loci is then simply governed by the
equation $d\Phi / dt = 2(t_0 - t) \Phi$, a dynamical system for which the
origin is forward asymptotically stable.
\end{eg}

The dynamical system that will be referred to in the ideas that
follow is generated by the non-autonomous differential equation
$\dot{x} = f(p, x)$, and is assumed to possess a constant set $A$ that is either
pullback asymptotically stable or is a pullback attractor.  These ideas are then
extended to time-varying families, $\hat{A}$ that possess pullback
behaviour in Section \ref{seclocidynhatA}.

Since the pullback attracting object $A$ is for the moment assumed to be a
constant set, analysing the properties of pullback attraction by pulling back a
single state $x_0$ is valid ($x_0$ is necessarily contained in the neighbourhood
of attraction regardless of how far in time it is pulled back). Each locus then
corresponds to a single initial state $x_0 \in \mathcal{N}_{\delta_p}(A)$, and
the elements defining the locus are determined by the couple $(\theta_{-t}p,
\Phi_{(t,\theta_{-t}p)}(x_0))$ for all $t \geq 0$.

\begin{lemma}
Given any $p \in P$, then for each $x_0 \in \mathcal{N}_{\delta_p}(A)$, the
loci defined by $(\theta_{-t}p, \Phi_{(t, \theta{-t}p)}(x_0))$ for
$t \geq 0$ are continuous and unique.
\end{lemma}
\begin{prf}
  {\em Continuity:}

Let $\e > 0$, and $x_0 \in \mathcal{N}_{\delta_p}(A)$ be arbitrary.
Then
\[ |\Phi_{(t + h, \theta_{-(t+h)}p)}(x_0) - \Phi_{(t,\theta_{-t}p)}(x_0)| =
         |\Phi_{(t,\theta_{-t}p)}(x_0^*) - \Phi_{(t,\theta_{-t}p)}(x_0)|, \]
where $x_0^* = \Phi_{(h, \theta_{-(t+h)}p)}(x_0)$. Using
the same principle as in Lemma \ref{intro2lem},
\begin{align*}
  |\Phi_{(t + h,\theta_{-(t+h)}p)}(x_0) - \Phi_{(t, \theta_{-t}p)}(x_0)| &\leq
         |x_0^*-x_0| \exp\left( L(\theta_{-t} p)h\right), \\
  &\leq h F(t) \exp\left( L(\theta_{-t} p)t \right). \\
\end{align*}
Here $F(t) = \sup \{ f(\theta_{-\tau}p, x) ; t \leq \tau \leq t+h
\}$ and $L(\theta_{-t}p)$ is the maximum local Lipschitz bound for
$f$ over the interval $(\theta_{-t}p, p)$.

Choose $h(t, \e) \leq \e / ( F(t) \exp\left( L(\theta_{-t} p)t
\right)$. Then
\[ |\Phi_{(t + h, \theta_{-(t+h)}p)}(x_0) - \Phi_{(t,\theta_{-t}p)}(x_0)| \leq
  h F(t) \exp\left( L(\theta_{-t} p) \right) \leq \e, \]
as required.

{\em Uniqueness: }

Assume the loci are not unique. That is, two loci cross paths at
some $t^*>0$. Since each loci is generated from distinct initial
states, there exists $x_0, x_1 \in \mathcal{N}_{\delta_p}(A)$ with
$x_0 \neq x_1$ such that $\Phi_{(t^*,\theta_{-t^*}p)}(x_0)) =
\Phi_{(t^*, \theta_{-t^*}p)}(x_1))$. However, this contradicts the
uniqueness of solutions for the original dynamical system. Hence
each locus is necessarily unique.
\end{prf}

Points along each locus may be defined by the mapping $\{
\phi_{(t, t_0)}; t_0 \in \mathbb{R}^+, t \in [-t_0, \infty) \}$, with $\phi_{(t,
t_0)}: E \to E$ and
\[ \phi_{(t , t_0)}(\phi_0) = \Phi_{(t_0 + t, \theta_{-(t_0+t)}p)}
   (\Phi_{(-t_0, p)}(\phi_0)). \]
$E$ is the state space for the original system.

In terms of the original dynamical system, the loci for which
$(t_0, \phi_0)$ is an element is simply a collection of images at $p$
resulting from pulling back an initial value $x_0$ associated with
$\phi_0$. This association is determined by
$x_0 = \Phi_{(-t_0, p)}(\phi_0)$.  This can be seen graphically in Figure
\ref{loci3}.

\begin{figure}[htb]
\begin{center}
\input{eps/loci3.pstex_t} \caption{Loci in $\mathcal{D}_l$}
\protect\label{loci3}
\end{center}
\end{figure}

\begin{lemma}\label{lemDlconst}
The loci form a continuous dynamical system, $\mathcal{D}_l$, for which the
group of mappings $\{ \phi_{(t,t_0)}; t_0 \in \mathbb{R}^+, t \in [-t_0, \infty)
\}$ with $\phi_{(t,t_0)}: E \to E$ forms a cocycle on $E$ with respect to the
group $\{\theta_t, t \in \mathbb{R}^+ \}$, where $\theta_{t}t_0 = t_0 + t$.
\end{lemma}
\begin{prf}
{\em Identity:}
\begin{align*}
  \phi_{(0,t_0)}(\phi_0) &=  \Phi_{(t_0, \theta_{-t_0}p)}((\Phi_{(-t_0, p)}
              (\phi_0)), \\
  &= \phi_0. \\
\end{align*}
{\em Cocycle Property:}
\begin{align*}
  \phi_{(t_1+t_2, t_0)}(\phi_0) &= \Phi_{(t_0+t_1+t_2,
        \theta_{-(t_0+t_1+t_2)}p)}(\Phi_{(-t_0, p)}(\phi_0)), \\
  &= \Phi_{(t_0+t_1+t_2,
        \theta_{-(t_0+t_1+t_2)}p)} \\
  & \hspace{1cm} \left(\Phi_{(-(t_0+t_1), p)}(
        \Phi_{(t_0+t_1, \theta_{-(t_0+t_1)}p)}( \Phi_{(-t_0, p)}(\phi_0) )
        )\right), \\
  &= \Phi_{(t_0+t_1+t_2,
        \theta_{-(t_0+t_1+t_2)}p)}\left(\Phi_{(-(t_0+t_1), p)}(
        \phi_{(t_1, t_0)}(\phi_0)) \right), \\
  &= \phi_{(t_2,t_0+t_1)}(\phi_{(t_1,t_0)}(\phi_0)). \\
\end{align*}
Hence the cocycle property for $\phi$ is satisfied.
\end{prf}

In some special cases the loci dynamics may be formulated (see Example
\ref{egsepscld}), and if $P = \mathbb{R}$ a general expression for the ordinary
differential equation which determines the loci dynamics may be formed.

\begin{lemma}[Loci Dynamics]
If $f$ and its partial derivatives are continuous, and the parameter set $P
= \mathbb{R}$, then the loci dynamics at each $t_0 \in \mathbb{R}$ are modelled
by the ordinary differential equation with initial value $(0, \phi_0)$,
\begin{equation} \label{eqloci}
 \frac{d\phi}{dt} = \int_{t_0 - t}^{t_0}\frac{\partial f^*}{\partial t}(\tau,
   t_0, t, \phi_0) d\tau + f( t_0 - t, \phi_0),
\end{equation}
where $f^*(\tau, t_0, t, \phi_0) = f( \tau, \Phi_{(\tau
- (t_0 - t),  t_0 - t)}(\phi_0))$.
\end{lemma}
\begin{prf}
Note that
\begin{align*}
  \phi_{(t, 0)}(\phi_0) &= \Phi_{(t, t_0 - t)}(\phi_0), \\
       &= \phi_0 + \int_{t_0 - t}^{t_0} f(\tau, \Phi_{(\tau - (t_0-t), t_0 -
                   t)}(\phi_0))d\tau.
\\ \end{align*}
If $f$ is continuous, and its partial derivative with respect to $t$ exists and
is continuous also, then application of the derivative arrives at the required
result.
\end{prf}

\begin{eg}[SDS - Loci Dynamics]\label{egsepscld}
Consider the NDE
\[ \dot{x} = f(\ta)g(x), \]
for which we take $P = \mathbb{R}$, and
assume $f, 1/g$ are continuous and bounded over the
interval of consideration, and hence integrable.

We consider pullback attraction to an arbitrary choice of $t_0 \in \mathbb{R}$.

Equilibrium points, identified by $g(x) = 0$, are
invariant in the original system, and hence are invariant in the loci
dynamical system.

Elsewhere, let $F, G$ denote the primitives of $f$, and
$1/g$ respectively. Separating variables, and integrating from $t_0 - t \to t_0$:
\[ G(\Phi_{(t, t_0-t)}(x_0)) = G(x_0)  + F(t_0) - F(t_0 - t). \]
Finally, differentiating with respect to $t$, and rewriting solutions in terms
of the cocycle mapping $\phi$ on $\mathcal{D}_l$, the loci dynamics are
determined by
\begin{equation}
\label{scaseeq}
 \frac{d \phi}{dt} = f( t_0 - t ) g(\phi).
\end{equation}
\end{eg}

\subsection{$\mathcal{D}_l$ - Asymptotic Behaviour}
\label{ssecLoci2}

\textit{We now pursue the question of forward asymptotic attraction and
stability within the dynamical system $\mathcal{D}_l$.}

If $A$ is locally pullback asymptotically stable, and $x_0$ lies in the
local neighbourhood of pullback attraction at $p$ (that is, $x_0 \in
\mathcal{N}_{\delta_p}(A)$), then each point on the loci generated by $x_0$ must
lie within the set $\Phi_{(t, \theta_{-t}p)} (\mathcal{N}_{\delta_p} (A))$ for
any given $t > 0$.

On the basis of this observation, we shall consider initial states for
the dynamical system $\mathcal{D}_l$ at $(t_0, \phi_0)$ where $\phi_0 \in
\Phi_{(t_0, \theta_{-t_0}p)} (\mathcal{N}_{\delta_p} (A))$. This is to ensure
that the loci that passes through the point $(t_0, \phi_0)$ will reflect the
original system's pullback asymptotic behaviour.

By continuity and uniqueness of the trajectories in the original
system, there exists $x_0 \in \mathcal{N}_{\delta_p}(A)$
associated with any loci passing through $\phi_0$ chosen in the fashion given
above, so that
\[ \Phi_{(t_0, \theta_{-t_0}p)}(x_0) = \phi_0, \]
or equivalently $x_0 = \Phi_{(-t_0, p)}(\phi_0)$. This was previously
illustrated in Figure \ref{loci3}.

Then
\begin{align*}
  \phi_{(t, t_0)}(\phi_0) &= \Phi_{(t +t_0,\theta_{-(t+t_0)}p)}(\Phi_{(-t_0,p)}
           (\phi_0) ), \\
  &= \Phi_{(t +t_0,\theta_{-(t+t_0)}p)}(x_0). \\
\end{align*}
Since $x_0 \in \mathcal{N}_{\delta_p}(A)$, then $A$ pullback attracts $x_0$ and
\begin{align*}
  \lim_{t \to \infty} \dist(\phi_{(t, t_0)}(\phi_0),A) &= \lim_{t \to \infty}
             \dist(\Phi_{(t +t_0,\theta_{-(t+t_0)}p)}(x_0),A), \\
  &= 0.  \\
\end{align*}
Hence for each $t_0$, $A$ forward attracts any $\phi_0 \in \Phi_{(t_0,
\theta_{-t_0}p)}(\mathcal{N}_{\delta_p}(A))$.

However, for $A$ to be forward asymptotically stable, $A$ must
forward attract a local $\delta$-neighbourhood system
$\mathcal{N}_{\hat{\delta}, A}$. The situation developed here allows for two
possible scenarios:

\textbf{i) Asymptotic Attraction - } If
\[ A \subset {\rm int} (\Phi_{(t, \theta_{-t}p)}(\mathcal{N}_{\delta_p}(A))), \]
for all $t \geq 0$, then there always exists a $\delta$-neighbourhood of $A$ in
$\mathcal{D}_l$ defined by  $\hat{\delta} = \{ \delta_t ; t
\in \mathbb{R}^+ \}$ where
\[ \mathcal{N}_{\delta_t}(A) \subset  \Phi_{(t,
\theta_{-t}p)}(\mathcal{N}_{\delta_p}(A)). \]


Indeed if $A$ is a pullback attractor, then this is automatically the case as
solutions asymptote to the attractor.

Asymptotic pullback attraction is illustrated in  Figure
\ref{loci1} where $A$ is a pullback attractor, and solutions are pullback
attracted to $A$ in infinite time.

\begin{figure}[htb]
\begin{center}
\input{eps/loci1.pstex_t} \caption{Region of Loci Generated by Asymptotic Attraction}
\protect\label{loci1}
\end{center}
\end{figure}

The following result is an automatic consequence of the asymptotic attraction
of a forward attractor.
\begin{lemma}[Forward Attractors in $\mathcal{D}_l$]
\label{lemlociatt}
If $A$ is a pullback attractor then $A$ is a forward attractor in
$\mathcal{D}_l$.
\end{lemma}
\begin{prf}
Attraction and existence of an attracting neighbourhood have already been
verified. Invariance follows from the invariance of $A$ as a pullback attractor
in the original system.
\end{prf}

\textbf{Remark:} The neighbourhood system for forward attraction is
not constant as it is bounded by pullback attraction of solutions
from $\mathcal{N}_{\delta_p}(A)$ in the original system. As
\[ \lim_{t \to \infty} \dist (\Phi_{(t, \theta_{-t}p)}(
      \mathcal{N}_{\delta_p}(A)), A) \to 0, \]
then the neighbourhood system for $A$ in $\mathcal{D}_l$ also vanishes.

\textbf{ii) Finite Attraction - } $A$ pullback absorbs some, or
all states lying in a neighbourhood of $A$, that is $\mathcal{N}_{\delta_p}(A)$.
In this case,
\[ A \backslash \Phi_{(t, \theta_{-t}p)}(\mathcal{N}_{\delta_p}(A)) \neq \{0\},
\]
for some $t > 0$.

As a result, all initial values $(t_0, \phi_0)$ in $\mathcal{D}_l$ chosen so
that $\phi_0 \in \Phi_{(t_0, \theta_{-t_0}p)}(\mathcal{N}_{\delta_p}(A))$
may for some values of $t_0$ include only points that are contained in $A$.
Consequently, forward attraction of points close to $A$ in $\mathcal{D}_l$ at
$t_0$ is unable to be determined without knowledge of the pullback
attraction in the original system beyond a local neighbourhood of $A$.

This is illustrated in Figure \ref{loci2} where the loci
neighbourhood is displayed for a set $A$ that pullback attracts solutions in the
original dynamical system in finite time, $t^*$.

The behaviour of points outside this loci neighbourhood, in
particular for any $(t_0, \phi_0)$ with $t_0 > t^*$, cannot be
determined from the loci dynamics generated from the neighbourhood
$\mathcal{N}_{\delta_p}(A)$. Hence there is no guarantee that $A$
is a forward asymptotically stable set in $\mathcal{D}_l$.

\begin{figure}[htb]
\begin{center}
\input{eps/loci2.pstex_t} \caption{Region of Loci Generated by Finite
Attraction} \protect\label{loci2}
\end{center}
\end{figure}

If however $A$ is globally pullback asymptotically stable (and possibly finite
pullback attracting) then an attracting neighbourhood of $A$ in $\mathcal{D}_l$
may always be determined by a pullback analysis of all trajectories in the
original dynamical system.

\begin{lemma}[Eventual Forward Asympotic Stability in $\mathcal{D}_l$]
\label{lempastoefas}
If $A$ is globally pullback asymptotically stable, then  for each $p \in P$, $A$
is eventually globally forward asymptotically stable in $\mathcal{D}_l$.
\end{lemma}
\begin{prf}
Let $p \in P$ be arbitrarily chosen and consider any initial value in
$\mathcal{D}_l$ defined by $(t_0, \phi_0)$.

Given $(t_0, \phi_0)$, there exists an $x_0$ such that in the original dynamical
system
\[ \Phi_{(-t_0, \theta_{-t_0}p)}(x_0) = \phi_0. \]
Since $A$ is globally pullback attracting in the original system, it pullback
attracts $x_0$. Hence there exists a loci originating at
$x_0$ that is forward attracted by $A$ in $\mathcal{D}_l$. Since $\phi_0$
lies on this loci it is concluded that $A$ forward attracts $\phi_0$, and
since the initial value $(t_0, \phi_0)$ was chosen arbitrarily, $A$ must be
globally forward attracting in $\mathcal{D}_l$.
\end{prf}

Recall the reference to `eventual asymptotic stability' in Section
\ref{NDSsec}. Forward eventual asymptotic stability implies
forward attraction of solutions, but not necessarily forward stability.

\subsection{Loci Stability}

To ensure forward stability of the loci in $\mathcal{D}_l$, stronger
requirements on the pullback behaviour in the original dynamical are required.

Since equi-asymptotic attraction and invariance are sufficient
requirements for stability (refer to \cite{BhSz67}), the forward
stability of $A$ in $\mathcal{D}_l$ for pullback attractors is
automatically guaranteed. As a result, Lemma \ref{lemlociatt}
holds independently of any further requirements on the pullback
stability of $A$ in the original dynamical system.

Forward stability of $A$ in $\mathcal{D}_l$ for globally pullback asymptotically
stable sets however, is not necessarily true. The following example illustrates
a case for which a globally pullback asymptotically stable set $A$ is globally
forward attracting (equivalently globally eventually forward asymptotically
stable) but not forward stable in $\mathcal{D}_l$.

\begin{eg}\label{egpstabprob}
Consider the NDE
\[ \dot{x} = -(1/2 + 2 \cos (3\ta) ) x. \]
Set $P = \mathbb{R}$. Clearly the origin is a pullback attractor, however we
will consider here the stability of $A = [-1, 1]$ in $\mathcal{D}_l$.

Solutions with initial value $(t_0, x_0)$ are given by
\[ x(\ta) = x_0  \exp \left( \frac{1}{2}( t_0 - \ta) + \frac{2}{3} (\sin(3t_0) -
            \sin(3\ta)) \right), \]
or using a cocycle representation analysing an initial state $x_0$
pulled back from $t_0$ by time $t$,
\[ \Phi_{(t, t_0 - t)}(x_0) = x_0 \exp \left( - \frac{1}{2} t + \frac{2}{3}
             (\sin(3(t_0 - t)) - \sin(3 t_0)) \right). \]
We shall investigate pullback attraction to $t_0 =
0$. From the above equation
\[ \Phi_{(t, -t)}(x_0) = x_0 \exp \left( -\frac{1}{2}t - \frac{2}{3} \sin(3t)
                  \right). \]
Hence the image at $t_0 = 0$ is monotonically decreasing as $( -\frac{1}{2}t -
\frac{2}{3} \sin(3t)) < 0$ for all $t \geq 0$. As a result, $A$ is pullback
equi-asymptotically stable (and more importantly pullback stable) at $t_0 = 0$.

The loci trajectories in the associated loci dynamical system $\mathcal{D}_l$
for $t_0 = 0$ are of the form
\begin{align*}
  \phi_{(t, 0)}(x_0) &= \Phi_{(t, -t)}(x_0), \\
    &= x_0 \exp \left( -\frac{1}{2}t - \frac{2}{3} \sin(3t) \right). \\
\end{align*}
Figure \ref{loci4} illustrates the loci trajectory associated with initial state
$x_0 = 2$.  As the motion is symmetric around the time axis, the
diagram is restricted to illustrate the behaviour of the loci for $x > 0$ only.

\begin{figure}[htb]
\begin{center}
\input{eps/loci4.pstex_t} \caption{Non-stable Loci in $\mathcal{D}_l$}
\protect\label{loci4}
\end{center}
\end{figure}

Here $A$ forward attracts the initial state shown, but it is not forward
stable. For example, let $\e = 0.5$, and consider any $\delta$-neighbourhood
chosen whilst the loci is initially within $A$. Regardless of the choice of
$\delta$, points on this loci will emerge and travel beyond the
$\e$-neighbourhood. Hence $A$ is not forward stable in $\mathcal{D}_l$.
\end{eg}

The critical difficulty with Example \ref{egpstabprob} is that the definition of
pullback stability is not as \textit{strong} as forward stability in terms of
the behaviour characterised by the definitions.

For example, suppose $A$ is forward stable. Then the property of forward
stability for $A$ implies two defining features that are a natural consequence
of its definition.

\textbf{Properties of Stability}

\begin{description}
  \item[S1 - ] The trajectory of any solution that begins in close proximity to
$A$ remains in close proximity to $A$.
  \item[S2 - ] Any trajectory that is at some time caught in
close proximity to $A$ remains trapped and close to $A$ thereafter.
\end{description}

Switching to a pullback analysis at some $p \in P$, the focus is on the
behaviour of the images at $p$, or equivalently, the loci, as an initial state
is pulled back rather than the trajectories themselves.

The definition of pullback stability automatically possesses the first property.
That is, the image of any point close to $A$ remains close to $A$ as it
is pulled back for all $t \geq 0$. However, it lacks the second property. As an
initial state is pulled back, if at any time the image is caught within a small
enough neighbourhood of $A$ there is no guarantee that it remains trapped.
Consequently, any loci that approach $A$ in
$\mathcal{D}_l$ are not guaranteed to remain close to $A$.

This reasoning leads to the conclusion that a stronger definition of
pullback stability may be required. At the time of writing it is not
immediately clear that it is essential, although redefining the structure of
non-autonomous dynamical systems with an alternative definition for pullback
stability that characterises the second property would be an interesting
exercise, and is open for further work.

For the present task however, a stronger version of pullback
stability  is defined that will be an essential requirement for the ensuing
problems involving discretisation.

\begin{defn}[Loci Stability - $A$]\label{defnlocistab}
A compact set $A$ is said to be \textbf{Loci Stable} with
respect to the cocycle $\{ \Phi_{(t, p)} ; t \in \mathbb{R}^+, p \in P \}$ on
$E$ if for any $\e > 0$, there exists a $\hat{\delta} = \{ \delta_p \in
\mathbb{R}^+ ; p \in P \}$ so that for any bounded and
compact set $B$ with the property
\[ \Phi_{(t^*, \theta_{-t^*}p)}(B) \subseteq
                     \mathcal{N}_{\delta_p}(A(p)), \]
for some $t^* > 0$, and $p \in P$, then
\[ H^*(\Phi_{(t, p)}(B), A) < \e \hspace{1cm} \forall t \geq t^*. \]
\end{defn}

\textbf{Remark 1:} If $A$ is loci stable, then it is
pullback stable by the original definition. This can be seen by
letting $B = \mathcal{N}_{\delta_p}(A(p))$, with $t^* = 0$ in the above
definition.

\textbf{Remark 2:} The above definition characterises \textit{both} properties
\textbf{S1} and \textbf{S2}. In particular, if the image of a solution as it is
pulled back (or equivalently its loci in $\mathcal{D}_l$) comes in close enough
proximity to $A$, then the image as it is pulled further back ( or the remainder
of the loci's trajectory) remains trapped by an $\e$-neighbourhood of $A$
thereafter.

Making use of this definition allows us to extend the result of Lemma
\ref{lempastoefas} to guarantee the forward asymptotic stability of $A$ in
$\mathcal{D}_l$.

\begin{therm}[Forward Asympotic Stability in $\mathcal{D}_l$]
\label{thmpastofas}
If $A$ is globally pullback attracting and loci stable, then
$A$ is globally forward asymptotically stable in $\mathcal{D}_l$.
\end{therm}
\begin{prf}
Forward attraction of solutions was shown in Lemma \ref{lempastoefas}.
Thus it is only required to show forward stability.

Let $\e >0$, and consider the loci dynamical system generated at
some $p \in P$. Finally, let $\delta_p = \delta_p(\e)$ be chosen so that it
satisfies the conditions for loci stability of $A$ at $p$.

\begin{figure}[htb]
\begin{center}
\input{eps/loci5.pstex_t} \caption{Forward Stability of Loci in $\mathcal{D}_l$}
\protect\label{loci5}
\end{center}
\end{figure}

Consider any $(t^*, \phi^*)$ with $\phi^* \in \mathcal{N}_{\delta_p}(A)$.
$\phi^*$ consequently lies on a loci trajectory in $\mathcal{D}_l$ that travels
within a $\delta_p$-neighbourhood of $A$ at time $t^*$. Under an analysis of the
original system, $\phi^* \in \mathcal{N}_{\delta_p}(A)$ at $p$, and is the image
of some point  $x^*$ that has been pulled back in time $t^*$. Refer to Figure
\ref{loci5}

Since $A$ is loci stable in the original dynamical system, then
\begin{align*}
 \dist(\phi_{(t, t^*)}(\phi^*), A) &=  \dist( \Phi_{(t + t^*,
                  \theta_{-(t+t^*)}p)}(x^*), A), \\
                  &< \e, \\
\end{align*}
for all $t > 0$, confirming that $A$ is forward stable in $\mathcal{D}_l$.
\end{prf}

{\bf Remark:} Understanding the effect of attraction to a single
point $p \in P$ by analysis of the loci dynamical system
$\mathcal{D}_l$ does not take into account the pullback attraction
to $A$ for all $p \in P$. This may be a limiting factor in
observing completely the behaviour of a non-autonomous system
possessing a pullback attractor, and in particular understanding
the effects of discretisation.

\endinput
