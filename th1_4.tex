\section{Further Asymptotic Characteristics}

Although the stability of many non-autonomous systems may be
examined with the use of the definitions and theorems in the
previous section, a wide class of non-autonomous systems possess
structures (particularly time varying structures) with attracting
characteristics that go beyond the scope of these definitions and theorems to
describe and characterise. As a result several new concepts for convergence and
stability, and attracting structures are needed.

We begin with an introductory example to illustrate these issues and introduce
the concept of pullback attraction as an additional method to form a more
comprehensive analysis of the system's stability.

\subsection{Non-Autonomous Asymptotic Behaviour}

\begin{eg}\label{introeg}
{\rm
Consider the autonomous system on $\mathbb{R}$
\[ \dot{x} = -x, \]
which has global attractor $A_0$ $=$ $\{0\}$, and the non-autonomously
perturbed system,
\[ \dot{x} = - x + \sin \ta. \]
Solutions are defined explicitly for $t$ $\geq$ $0$ ($t$ is elapsed time) and are
expressed using a cocycle representation by \vspace{2mm}
\begin{equation}\label{attobj}
\Phi_{(t,t_0)}(x_0) =
       \frac{1}{2} \left(\sin (t+t_0) - \cos (t+t_0)\right) +
       \left( x(t_0) - \frac{1}{2}
       \left(\sin t_0 - \cos t_0 \right)\right) \, e^{-t},
\end{equation}
where the group shift mapping $\theta_t:\mathbb{R} \rightarrow \mathbb{R}$
is defined by $\theta_t (t_0) = t_0 +t$. }
\end{eg}

We wish to analyse the system's asymptotic behaviour, and more importantly
identify any objects or structures which have attracting properties.

The usual, and most obvious way to formulate asymptotic behaviour is to
consider the limit set of the forwards trajectory. For any $t_0$ the
$\omega$-limit set for the cocycle (\ref{attobj}) here is
\[ \omega_(t_0,\mathbb{R}) = [-1/\sqrt{2},1/\sqrt{2}]. \]
However, this set is not $\Phi$-invariant,
($\Phi_{(t,t_0)}([-1/\sqrt{2},1/\sqrt{2}])$ $\neq$ $[-1/\sqrt{2}, 1/\sqrt{2}]$), 
which is a fundamental property of any attracting object. Limit sets in non-autonomous
systems have been investigated in \cite{FS95,Vi92}, however they
have the disadvantage that the resulting $\omega$-limit sets are
generally not invariant (as in this case) with respect to $\Phi$.
It may be too restrictive searching for a single constant set
which is invariant under $\Phi$. This directs us to search for a
family of sets that may be $\Phi$-invariant.

\begin{defn}[$\Phi$-Invariance] \hfill \\
The family of nonempty sets $\hat{A} = \{ A(\tau) ; \tau \in \mathbb{R}
\}$ is {\bf invariant} under $\Phi$ (or equivalently {\bf
$\Phi$-invariant})  if
\[ \Phi_{(t,t_0)}(A(t_0)) = A(t+t_0), \hspace{2cm} \forall t \in
                \mathbb{R}^+, \quad t_0 \in \mathbb{R}. \]
\end{defn}

An `intuitive' look at the long term behaviour of solutions $\Phi$
as $t \rightarrow \infty$ elucidates a transient component which
vanishes exponentially, and a steady state component that is time
varying. Thus solutions are attracted exponentially to the steady
state solution. To characterise this, let us define a family of
sets $\hat{A} = \{A(\tau); \tau \in \mathbb{R} \}$ as described
above, where
\begin{equation}\label{ssobj}
  A(\tau) = \frac{1}{2} \left(\sin \tau - \cos \tau \right).
\end{equation}
$\hat{A}$ is obviously $\Phi$-invariant, and also possesses the
usual characteristics of an attracting set as trajectories of
$\Phi$ approach it as the elapsed time is increased. This can be
seen by taking
\[ \lim_{t \rightarrow \infty} H^*(\Phi_{(t,t_0)}(x_0),A(t+t_0))=0. \]
However, does this attraction hold only for large $t$, or can we consider
attraction to an element $A(t) \in \hat{A}$ for any arbitrary time $t$?

The answer is yes, however it requires a subtle addition to the usual
concepts of convergence. The typical forwards convergence analysis
entails increasing the final time whilst the initial time remains fixed.
However, to consider convergence to a set $A(t^*)$ of this attracting
family, for a particular $t^*$, it would be reasonable to fix the final
time, and start progressively earlier (pulling back the initial time) in
order to finish at $t^*$.  By taking this limit for (\ref{attobj}), and any
$x_0 \in \mathbb{R}$, we find
\[ \lim_{t \rightarrow \infty} H^*(\Phi_{(t,t^*-t)}(x_0),A(t^*))=0. \]
This is referred to as {\bf pullback attraction} (or pullback
convergence), and  has the advantage in that attraction to a single element of
the attracting structure may be mathematically characterised with an approach
involving the use of limit sets.

Pullback convergence for this particular problem is illustrated in
the diagram below, showing pullback convergence of the interval
$[-2, 2]$ to $\hat{A}$ at $t = 8$.

\begin{figure}[htb]
\begin{center}
%\framebox[6.0cm][c]{
\leavevmode
\hbox{
\epsfxsize=9.5cm
\epsffile{eps/pbeg.eps}  }%}
\protect\caption{Pullback Convergence}\protect\label{pbeg}
\end{center}
\end{figure}

\endinput
