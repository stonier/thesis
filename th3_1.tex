
\chapter[Discrete Dynamical Systems]{Discrete Dynamics of Non-Autonomous Systems}
  \label{discretechapter}

Many of the results pertaining to continuous dynamical systems that have
been presented in previous chapters are also relevant to discrete and
numerically approximated dynamical systems. The theory of cocycles, pullback
absorbing sets and stability translate under a slightly different notation for
discretised systems.

\section{Difference Equations}\label{diffeqns}

Discrete non-autonomous dynamical systems are often represented by
difference equations of the form
\begin{equation} \label{diffeq}
x_{n+1} = f_n(x_n),
\end{equation}
where each $f_n$ is a Lipschitz continuous mapping on the state
space $E \subset \mathbb{R}^d$ for all $n \in \mathbb{Z}^+$.

\subsection{Constant Time-Step Discretisations}

We will be interested in approximating continuous dynamical
systems generated by non-autonomous ordinary differential equations,
\begin{equation}\label{NDE2}
  \dot{x} = f(p,x),
\end{equation}
which are known to possess unique solutions $\Phi_{(t,p_0)}(x_0)$
for the initial value problem $(p_0, x_0)$ as introduced in Section
\ref{NDSsec}.

Such dynamical systems are often approximated using a numerical scheme as
in \cite{St73}, \cite{St94}. This is often a Taylor Series or Runge-Kutta method,
and the discretised system can then be represented with a difference
equation in a similar form to that of (\ref{diffeq}).  A one-step
numerical scheme for a non-autonomous differential equation can be
expressed in the form
\begin{align}\label{numeq}
   x_{n+1} &= x_{n} + F_h(p_n,x_{n}), \\
   &= \tilde{F}_h(p_n, x_n).
\end{align}
Here $p_n = \theta_{nh}p_0$, and $F_h$ is the increment function for the
one-step method used.

It is important to note here that both the initial time and the
step size are necessary in order to evaluate the solution because
of the problem's non-autonomous nature. This will be of particular
importance when considering attraction of the discretised system
later. If the system is autonomous, then the initial time is not
of concern and the problem reduces to that of the ordinary
difference equation (\ref{diffeq}).

\subsection{Variable Time-Step Discretisations} \label{vtssec}

In place of using a constant time step we may discretise the dynamical
system (\ref{NDE2}) with variable time steps $h_n > 0$. Such a
discretisation is then expressed in the form
\begin{align}\label{vtsnumeq}
   x_{n+1} &= x_n + F_{h_n}(p_n, x_n), \\
   &= \tilde{F}_{h_n}(p_n, x_n).
\end{align}
In order to analyse the system's stability we will utilise
the structure defined below as a basis for formulating the variable time-step
system.

Let $\rho > 0$, and $(H^{\rho},
d_{H^{\rho}})$ denote the compact metric space of all bi-infinite real
sequences ${\bf h} = \{h_n\}_{n \in \mathbb{Z}}$, where $0
\leq h_n \leq \rho$ for all $n \in \mathbb{Z}$ with the metric
\[ d_{H^{\rho}} ({\bf h}_{(1)}, {\bf h}_{(2)}) = \sum_{n = -
                \infty}^{\infty} 2^{-|n|}|h_n^{(1)} - h_n^{(2)}|. \]
The sequence ${\bf h}$ then represents the set of variable step sizes for a
particular discretised system. A further property needed by the sequence {\bf
h} is reachability. That is any future time $\theta_tp$ is reachable by taking
a summation of steps in {\bf h}.  This may be expressed mathematically by
the condition
\[ \sum_{n=n_0}^{\infty} h_n = \infty, \]
for any $n_0 \in \mathbb{Z}$. Associating an initial value $p_0$ with such a
sequence as a couple, written $(p_0, {\bf h}) \in P \times H^{\rho}$, we then
define
\begin{align*}
  p_1 &= \theta_{h_0}p_0, \\
  p_2 &= \theta_{h_1}p_1, \\
  & \quad \cdot \cdot \cdot \\
  p_n &= \theta_{h_{n-1}}p_{n-1}.
\end{align*}
The couple $(p_0, {\bf h})$ completely defines the sequence of
discrete values that determines the discrete dynamical system.
Thus the cross product space $P \times H^{\rho}$ serves as a parameter set
for the cocycle representation of a variable time step discretisation.


\subsection{Local Truncation Error}
\label{truncerrorssec}

Errors between the numerical approximation and the actual state
using a one-step numerical scheme are analysed through the use of a
bound on the local error produced by the method over one
iteration. This is known as the truncation error and is outlined
below. (Refer to \cite{JoRi82})

A one-step numerical scheme for a non-autonomous differential equation
is known as $r$-th order scheme if its local (one-step) discretisation
error satisfies a bound of the form,
\begin{equation}\label{trunceq}
  ||x_{n+1} - \Phi_{(h,p_n)}(x_n)|| \leq C_{r}h^{r+1}.
\end{equation}
Similarly, if the numerical scheme uses variable time steps,
\begin{equation}\label{truncvstepeq}
  ||x_{n+1} - \Phi_{(h_n, (p_n, {\bf h}))}(x_n)|| \leq C_{r}h_n^{r+1}
       \leq C_{r}\rho^{r+1}.
\end{equation}
The order $r$ for most commonly used one-step methods is generally
dependent on the smoothness of the function $f$ in
(\ref{NDE2}) and the order of the Taylor series used. The
truncation constant, $C_r$ is dependent on the bounds of $f$
and its derivatives (up to the order of  the numerical scheme)
over a finite interval of discretisation.

A note concerning the truncation constant needs to be mentioned
here. In non-autonomous systems, it may become impossible to place bounds on $f$ and its derivatives, even when the solutions are expected to
remain in a bounded (possibly compact) region of the state space.  This becomes a
factor as we analyse asymptotic behaviour and consider discretisation on
non-finite intervals of time. In these cases caution is required and more details will be given later. For autonomous systems this issue is not a concern
if the state is guaranteed to remain within a bounded region.

For systems where a single truncation constant may not be
appropriately defined, we alternatively define the truncation
error with a variable bound that is modified from step to step.
This issue is revisited later in this thesis.

\endinput
