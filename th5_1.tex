
\chapter[Lyapunov Theory]{Lyapunov Theory for Non-Autonomous
         Dynamical Systems} \label{Lyapchapter}

\section{Introduction}

Lyapunov functions provide an effective practical and theoretical tool in
assisting in the analysis of a dynamical's system stability, either in
verification of its stability or as a method of determining controls to ensure
its stability. They have also been used theoretically to
characterise a particular stability property, which has been useful in
approximating asymptotically stable sets in perturbed autonomous systems (of
which numerical approximations are included) \cite{St94}.

Lyapunov functions were introduced earlier (Section \ref{ALF1}, \ref{NLFsec})
to illustrate their use and importance in asserting the forward stability of
sets that remain constant over time. In this chapter we will concern ourselves
with \textit{time-varying} families of sets, and Lyapunov functions associated with
forward, complete and pullback stability results respectively and the
difficulties arising with using such techniques for pullback behaviour.

To clearly state the problem we will initially define a few general assumptions
concerning the dynamical system under investigation and the character of
Lyapunov functions.

We consider the non-autonomous differential equation
\begin{equation}
\label{NDEeq}
\dot{x} = f(p, x ),
\end{equation}
and assume the following properties hold.

{\bf F1:} The function $f(p, x)$ is continuous in both $p$ and $x$.

{\bf F2:} $f(p, \cdot)$ satisfies locally a Lipschitz condition with respect to
$x$. That is, for any given $\delta > 0$, and $x'$ such that $|x - x'| <
\delta$, there exists a constant $L(p, \delta)$ satisfying
\[ |f(p, x) - f(p, x')| < L(p, \delta)|x - x'|. \]

{\bf F3:} There is a group of mappings $\{\theta_t, t \in
\mathbb{R}^+ \}$ with $\theta_t : P \to P$, continuous on $P$ and satisfying
$\theta_{t} \circ \theta_{\tau}$ $=$ $\theta_{t+\tau}$ for all $t$, $\tau$
$\in$ $T$.

A Lyapunov function $V(p, x)$ will be assumed to be a scalar continuous
function that satisfies locally a Lipschitz condition with respect to $x$.
The Dini Derivative is used to measure the rate of change of $V$ for trajectories 
in (\ref{NDEeq}). It is repeated here for ease of reference.
\begin{equation}
\label{Dinieq}
  \overline{D_t}^+ V(p, x) = \overline{\lim_{h \to 0}} \frac{V(\theta_{h}p,
\Phi_{(h, p)}(x)) - V(p, x)}{h}.
\end{equation}

\subsection{Lemmas}
The subsequent lemmas will be utilised throughout the rest of this
chapter in the various Lyapunov proofs. The first ascertains a
single neighbourhood for a uniformly forward asymptotically stable
family, for which solutions from any state within the
neighbourhood are guaranteed to remain close for all time, and to
also attract asymptotically toward the sets in consideration.

\begin{lemma}
\label{intro1lem}
If $\hat{A} = \{A(p); p \in P\}$ is uniformly forward asymptotically stable,
then given any $\e^* > 0$ there exists a $\delta = \delta(\e^*)$ such that for
any $\e > 0$, and each $p \in P$, $x \in \mathcal{N}_{\delta} (A(p))$,
\begin{align*}
  i) \quad & \dist( \Phi_{(t, p)}(x), A(\theta_{t}p)) < \e^* \hspace{1cm} &
  \text{for all} \quad t > 0, \\
  ii) \quad & \dist( \Phi_{(t, p)}(x), A(\theta_{t}p)) < \e \hspace{1cm} &
  \text{for all} \quad t > T(\e),
\end{align*}
where $T=T(\e)$ is as defined for uniform forward asymptotic stability.
The same parameters then also hold for uniform pullback asymptotic stability.
\end{lemma}
\begin{prf}
  Let $\delta = \min \{ \delta_1, \delta_2  \}$ where $\delta_1 =
  \delta_1(\e^*)$ as defined for stability, and $\delta_2$ as defined for
  asymptotic attraction. The results i), ii) follow immediately.

  Due to uniformity, the equivalence of forward and pullback stability
  guarantees the same parameters will satisfy the conditions for pullback
  asymptotic stability.
\end{prf}

The following two lemmas (the first being Gronwall's Lemma which can be found in
one version or another in various texts, see for example \cite{Pe91}) provide
bounds on differences in solutions beginning from initial values close to one
another.

\begin{lemma}[Gronwall's Lemma]
\label{intro2lem}
Given $x_0, x_1$ such that $|\Phi_{(t, p)}(x_0) - \Phi_{(t, p)}(x_1))| <
\e$, for some $\e > 0$, and any $t > 0$, then
\[ | \Phi_{(t, p)}(x_0) - \Phi_{(t, p)}(x_{1} ) | \leq |x_{0} - x_{1}|
\exp\left( \int_{0}^{t} L(\theta_{s} p, \e) ds\right). \]
\end{lemma}
\begin{prf}
We have
\begin{align*}
| \Phi_{(t, p)}(x_0) - &\Phi_{(t, p)}(x_{1} ) |  \\
  &\leq |x_0 - x_1| + \int_0^t
       |f(s, \Phi_{(s, p)}(x_0)) - f(s, \Phi_{(s, p)}(x_1))| ds, \\
  &\leq |x_0 - x_1| + \int_0^t L(\theta_{s} p, \e)|\Phi_{(s, p)}(x_0) -
            \Phi_{(s, p)}(x_1))| ds. \\
\end{align*}
Taking the derivative with respect to $t$,
\[  \frac{d|\Phi_{(t, p)}(x_0) - \Phi_{(t, p)}(x_{1} ) |}{dt} \leq
      L(\theta_{t}p,\e)|\Phi_{(t, p)}(x_0)  - \Phi_{(t, p)}(x_1))|, \]
or
\[  \frac{d}{dt} \left(|\Phi_{(t, p)}(x_0) - \Phi_{(t, p)}(x_{1} ) | \exp\left(
   - \int_{0}^{t} L(\theta_{s} p, \e) ds\right) \right) \leq 0, \]
Hence we arrive at the required result,
\[ | \Phi_{(t, p)}(x_0) - \Phi_{(t, p)}(x_{1} ) | \leq |x_{0} - x_{1}|
           \exp\left( \int_{0}^{t} L(\theta_{s} p, \e) ds\right). \]
\end{prf}

The following lemma is an elementary result that provides a bound for
differences of solutions that lie on the same trajectory. It is also
feasible to consider the maximum over larger intervals, as will be assumed
in Theorem \ref{confasthm}.

\begin{lemma}
\label{intro3lem}
For each $p, x$ such that any solutions in future time exist and are unique,
and any $t' < t$,
\[  | \Phi_{(t, p)}(x) - \Phi_{(t', p)}(x) | \leq \max_s \{ |f(\theta_s p, x)|
  \} |t - t'|, \]
where the maximum is taken over the interval $t' < s < t$.
\end{lemma}


\endinput
