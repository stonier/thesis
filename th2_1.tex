
\chapter[Non-Autonomous Dynamical Systems]{Stability of Non-Autonomous Dynamical Systems}
\label{cocyclechapter}

\section{Stability and Attraction}\label{stabatt}
\subsection{Introduction}

As illustrated in the preceding example, the original concepts of
stability, asymptotic stability, and the development of attracting
objects that are simply a single and constant set, are restrictive
and do not always provide a conclusive analysis of the behaviour
and stability of a non-autonomous dynamical system. Frequently,
the generic `attracting' object for a cocycle will often consist
of a family of sets rather than just a single set. Yoshizawa (see
\cite{Yo66}) introduced briefly the concept of using a family of
sets for an attracting object, however, the theory only considers
forward asymptotic convergence to the attracting object. This has
the disadvantage of being unable to determine attractivity except
of an `eventual' nature, and also lacks the use of a limit set
theorem analogous to Theorem \ref{attset} for autonomous systems.


As well as redeveloping the notion of an attracting object as a family of
sets, it should also be feasible to consider stability and asymptotic
stability to a family of sets in both the forward \textit{and} pullback sense.
Analysing forward convergence within a system is certainly nothing new,
and the notion of pullback convergence that was introduced has also been
around for some time, though within the context of dynamical systems it is
a relatively new approach. It was first used in analysing random dynamical
systems, and has its value in that it does allow an extension of the limit set
theorem (Theorem \ref{attset}) for time-varying pullback attractors in non-autonomous
systems (pullback attractors are discussed in more detail in Section
\ref{ANSsec}). This proved in part the motivation to formalise the concepts
of pullback stability and asymptotic stability in order to facilitate a
more complete pullback analysis. The definitions for pullback stability and
pullback asymptotic stability are newly developed here, with particular emphasis
placed on a local analysis of stability. The ensuing development of attractors
is also made with regards to a local analysis. This is in contrast with the
global analysis for pullback attractors used by Kloeden et. al.
(\cite{KlSc95,KlSc96}.

To begin with however, we formalise our definition and notation for a
family of sets within a non-autonomous system and the concept of a
neighbourhood of a family of sets. These are essential in implicitly
describing the nature of structures that evolve within non-autonomous
systems, and have been constructed axiomatically from those for autonomous
systems.

\begin{notn}[Family of Sets, ${\bf \hat{A}}$]
  We will say a collection of nonempty subsets over the parameter
  space $P$, represents a {\em family of sets} on $E$ and denote it by
  \[  \hat{A} = \{ A(p) ; p \in P \} \]
\end{notn}

The concept of a neighbourhood for a stable or attracting object is
frequently used, and to extend it to a family of sets we define and
use the following notations throughout the remainder of this thesis.

\begin{notn}[${\bf \delta}$-Neighbourhood of ${\bf \hat{A}}$]
A $\delta$-neighbourhood, $\hat{\mathcal{N}}_{\delta,\hat{A}}$ of a family
of sets $\hat{A}$ is defined as the family
\[\hat{\mathcal{N}}_{\delta,\hat{A}} = \{ \mathcal{N}_{\delta}(A(p)) ; p
                \in P \}, \]
for some $\delta > 0$. It will be referred to as a $\delta$-neighbourhood
of $\hat{A}$.
\end{notn}

To illustrate, consider a family of sets $\hat{A}$ that uniform attracts solutions within a local neighbourhood. Uniform attraction implies that the $\delta - neighbourhood$ of $\hat{A}$ may be chosen such that $\delta$ is constant with respect to time. This does not mean that the neighbourhood itself is fixed with respect to time! Indeed, if $\hat{A}$ is moving, then its neighbourhood must also move with it.

Rather than considering the local neighbourhood of each individual set in $\hat{A}$, we have defined above the whole family as a single entity - the $\delta$-neighbourhood of $\hat{A}$ By using this terminology, we may conveniently discuss the properties of solutions with initial value $(x_0, p_0)$ lying anywhere (with respect to initial state and time) within the local neighbourhood. Note that if $\hat{A}$ does not vary with time, the above terminology becomes equivalent with the usual concept of a neighbourhood for an analysis of a dynamical system's stability.

As a practical illustration, recall the attracting object $\hat{A}$
defined by (\ref{ssobj}) discussed in Example \ref{introeg}. A $\delta$-neighbourhood with $\delta = 1.25$ (shaded region) uniformly attracts any solution with initial value lying in this neighbourhood. Refer to Figure \ref{dnhoodpic}.

\begin{figure}[htb]
\begin{center}
%\framebox[6.0cm][c]{
\leavevmode
\hbox{
\epsfxsize=6.2cm
\epsffile{eps/dnhood.eps}  }%}
\protect\caption{$\hat{\mathcal{N}}_{\delta,\hat{A}}$
               - $\delta$-neighbourhood of the family $\hat{A}$}
        \protect\label{dnhoodpic}
\end{center}
\end{figure}

The motivation for the following definition may not be so clear.

Uniform attraction involves an analysis of local neighbourhoods. For fixed sets $A$, a fixed local neighourhood may be represented by $\mathcal{N}_{\delta}(A)$. For a family of sets $\hat{A}$, a fixed local neighbourhood can be represented by $\hat{\mathcal{N}}_{\delta,\hat{A}}$ as discussed above.

An analysis of asymptotic stability (not necessarily uniform) requires an added level of complexity. For each $p \in P$ we must consider a different local neighbourhood.
For fixed sets $A$ this is typically represented by $\mathcal{N}_{\delta_p}(A)$ for each $p \in P$. For a family of sets $\hat{A}$, this corresponds to a local neighbourhood family $\hat{\mathcal{N}}_{\delta_p, \hat{A}}$ for each $p \in P$.

We collectively group the neighbourhood families defined over $P$ as a system of families and will refer to them as follows.

\begin{notn}[${\bf \delta}$-Neighbourhood System of ${\bf \hat{A}}$]
A $\delta$-neighbourhood system, $\hat{\mathcal{N}}_{\hat{\delta},\hat{A}}$
of a family $\hat{A}$ is defined as
\[ \hat{\mathcal{N}}_{\hat{\delta},\hat{A}}= \{
       \hat{\mathcal{N}}_{\delta_p, \hat{A}} ; \delta_p \in
       \hat{\delta}, p \in P \}, \]
for some set of uniformly bounded $\delta$ values $\hat{\delta} = \{\delta_p
;\delta_p>0, p \in P \}$.
\end{notn}

At an initial glance, this might not appear as intuitive or as simple as defining a varying local neighbourhood family by 
\[ \hat{\mathcal{N}}_{\hat{\delta},\hat{A}}= \{
       \mathcal{N}_{\delta_p}(A(p)) ; \delta_p \in
       \hat{\delta}, p \in P \}. \]
For a forward analysis, this definition suffices - however it does not provide a suitable construction of a neighbourhood for a pullback analysis of asymptotic stability. The reasons for this will become clear later as we investigate pullback asymptotic stability in more depth.

For convenience, any $\delta$ and $\epsilon$ sets will be assumed to be
uniformly bounded with respect to $p$ (to ensure that local neighbourhoods are indeed local) for the remainder of this work. 

By convention, we will also use the hat symbol to denote a family of objects (e.g. a family of sets $\hat{A}$, a family of parameters $\hat{\delta}$ or a system $\hat{\mathcal{N}}_{\hat{\delta}, \hat{A}}$) that has been collectively grouped over  $P$.

\subsection{Attraction}

The differences between forward and pullback attraction lie in the
distinction that forward attraction requires the attraction to
only occur 'eventually', with no information as to the attracting
characteristics of the system at any fixed time. Alternatively,
pullback attraction guarantees attraction to an element at a
distinct point in time. The state of the attraction thereafter is
not an issue. These concepts (introduced in Example \ref{introeg})
are formalised below and followed by illustrative examples
thereafter.

\begin{defn}[Forward Attraction]
A family of uniformly bounded compact sets $\hat{A} =
\{A(p):p \in P \}$ is said to {\bf forward attract} another family of
sets $\hat{B} = \{B(p):p \in P\}$ from $p \in P$ if
\[ \lim_{t \to \infty} H^{*} \left(\Phi_{(t,p)}
                     (B(p),A(\theta_t p)) \right) = 0. \]
\end{defn}

Also note that if $\hat{A}$ forward attracts $\hat{B}$, and
$\hat{C} \subset \hat{B}$ (that is, $C(p) \subset B(p)$ for all $p \in P$),
then $\hat{A}$ forward attracts $\hat{C}$.

Forward attraction of a single set, or family $\hat{B}$, to $\hat{A}$ is
also referred to as {\em forward convergence} of $\hat{B}$ to $\hat{A}$
from $p \in P$.

\begin{defn}[Pullback Attraction]
A family of uniformly bounded compact sets $\hat{A} =
\{A(p):p \in P \}$ is said to {\bf pullback attract} another family of
uniformly bounded sets $\hat{B} = \{B(p):p \in P\}$ at $p \in P$ if
\[ \lim_{t \to \infty} H^{*} \left(\Phi_{(t,\theta_{-t}(p))}
                     (B(\theta_{-t}(p)),A(p)) \right) = 0. \]
\end{defn}

Similarly, this is also referred to as {\em pullback convergence} of
$\hat{B}$ to $\hat{A}$ at $p$.

\begin{defn}[Complete Attraction]
If $\hat{A}$ pullback attracts $\hat{B}$ at some value of $p \in P$ and
also forward attracts $\hat{B}$ from $p$ then $\hat{A}$ is said to {\bf
completely attract} $\hat{B}$ at $p \in P$.
\end{defn}

Often we are concerned only with the attraction of single sets or of a
single set $B$ to a family $\hat{A}$, for some $p \in P$.  In either case,
the corresponding family can be thought of as a family of identical
sets.

The definitions are also consistent with classical ideas in autonomous
systems. The definition of pullback attraction (and also complete
attraction) then become equivalent to the usual definition for forward
attraction as the initial time is not relevant.

\begin{example}[Attraction in Example \ref{introeg}]
Consider again the perturbed autonomous system in Example
\ref{introeg}, and consider possible attraction of the state to
the origin. In order to analyse the behaviour at the origin we
define $\hat{A}$ by $A(\ta) = 0$, for all $\ta$, and consider both
forward and pullback attraction to $\hat{A}$.

{\em i) Forward Attraction:} For any single, bounded set $B$, we
have
\begin{align*}
  \lim_{t \to \infty} H^{*} &\left(\Phi_{(t,t_0)}(B),A(\theta_t (t_0)))
        \right) \\
  &= \lim_{t \to \infty} \max_{x_0 \in B} \left|
        \frac{1}{2} [ \sin (t+t_0) - \cos (t+t_0) ] + [
        x_0 - \frac{1}{2} ( \sin t_0 - \cos t_0 )] \,
        e^{-t} \right|, \\
  &= \lim_{t \to \infty} \left| \frac{1}{2} [\sin (t+t_0) -
        \cos (t+t_0) ] \right|,  \qquad \forall
        t_0 \in \mathbb{R}. \\
\end{align*}
For this there exists no limiting value. Hence $\hat{A}$ (the
origin) does not forward attract any bounded set $B$ for any $t_0
\in \mathbb{R}$.

{\em ii) Pullback Attraction:} For each bounded set $B$, and
initial time $t_0 \in \mathbb{R}$,
\begin{align*}
  \lim_{t \to \infty} H^{*} &\left(\Phi_{(t,\theta_{-t}(t_0))}
        (B),A(t_0)) \right) \\
  &= \lim_{t \to \infty} \max_{x_0 \in B} \left|
        \frac{1}{2} [\sin (t_0) - \cos (t_0)] \right. \\
  & \hspace{1.5cm} \left. + [ x_0 -
        \frac{1}{2} (\sin (t_0-t) - \cos (t_0-t) )] \,
        e^{-t} \right| , \\
  &= \left| \frac{1}{2} [\sin (t_0) - \cos (t_0)] \right|, \\
  &= 0 \hspace{4cm} \forall t_0 = \frac{\pi}{4} + n \pi \quad (n =
        1,2,...).\\
\end{align*}
Hence the system pullback attracts solutions to the origin only at
specific times as determined by the set of discrete values given
by $\{t_0; t_0 = \pi/4 + n\pi \}$ for $n = 1, 2, ..$.
\end{example}

\subsection{Stability}

The following definitions are an extension of those used in
classical stability analysis for non-autonomous systems (refer to
Section \ref{NDSsec}) with further application to families of sets
and with consideration of both forward and pullback analysis.

\begin{defn}[Forward Stability]\label{FSdef}
   A family $\hat{A} = \{A(p);p \in P\}$ of uniformly bounded compact
   subsets of $E$, is said to be {\bf forward stable} with respect to the
   cocycle $\{\Phi_{(t,p)}; t \in \mathbb{R}^{+},p \in P\}$ on $E$ if for
   any $\epsilon > 0$, there exists a $\hat{\delta} = \{\delta_p \in
   \mathbb{R}^+; p\in P\}$ such that for any $p \in P$,
   \begin{equation}\label{FSeq}
   H^*(\Phi_{(t,p)}(\mathcal{N}_{\delta_p}
            (A(p)),A(\theta_t p)) < \epsilon, \qquad \forall t \geq 0.
   \end{equation}
\end{defn}

\begin{defn}[Pullback Stability]\label{PSdef}
   A family $\hat{A} = \{A(p);p \in P\}$ of uniformly bounded compact
   subsets of $E$, is said to be {\bf pullback stable} with respect to the
   cocycle $\{\Phi_{(t,p)}; t \in \mathbb{R}^{+},p \in P\}$ on $E$ if for
   any $\epsilon > 0$ there
   exists a $\hat{\delta} = \{\delta_p \in \mathbb{R}^+; p\in P\}$ so that
   for any $p \in P$,
   \begin{equation}\label{PSeq}
   H^*(\Phi_{(t,\theta_{-t}p)}(\mathcal{N}_{\delta_p}
            (A(\theta_{-t}p)),A(p)) < \epsilon, \qquad \forall t \geq 0.
   \end{equation}
\end{defn}

\begin{defn}[Complete Stability]\label{CSdef}
  If $\hat{A}$ is both {\em pullback} and {\em forward stable}, then it is
  said to be {\bf completely stable}.
\end{defn}

If the $\hat{\delta}$ in any of the above definitions are
independent of the parameter $p$ (that is, the
$\delta_p = \delta$ for some $\delta>0$ and all $p \in P$), then
the respective stability of $\hat{A}$ is said to be {\bf uniform}.

We say $\hat{A}$ is \textbf{positively invariant} under $\Phi$ if for each $p
\in P$, and all $t \geq 0$,
\[ \Phi_{(t,p)}(A(p)) \subseteq A(\theta_{t}p). \]

The property of positive invariance for stable sets in autonomous dynamical
systems is also valid for both pullback and forward stable families in a
non-autonomous dynamical system.

\begin{therm}\label{thmstabtopi}
If $\hat{A}$ is pullback/forward/completely stable, then it is {\bf positively
invariant} under $\Phi$.
\end{therm}
\begin{prf}
Assume that $\hat{A}$ is either pullback or forward stable (or completely) but
is not positively invariant.  Then for some $p \in P$, there exists a $t > 0$
such that
\[ \Phi_{(t, p)}(A(p)) \nsubseteq A(\theta_{t}p). \]
Hence there exists an $a \in A(p)$ and an $\e > 0$ such that
\begin{equation}\label{eqnpiass}
 \dist( \Phi_{(t, p)}(a), A(\theta_{t}p)) > \e.
\end{equation}
\textit{i) $\hat{A}$ is pullback stable} - Since $\hat{A}$ is pullback stable,
there exists a $\delta_{\theta_t p}(\e) > 0$ that ensures pullback stability
at $\theta_{t}p$. Since $a \in \mathcal{N}_{\delta_{\theta_t p}}(A(p))$, then by
pullback stability
\[ \dist( \Phi_{(t, p)}(a), A(\theta_t p)) \leq H^*(\Phi_{(t,
          p)}(\mathcal{N}_{\delta_{\theta_t p}} (A(p)), A(\theta_t p)) < \e, \]
which contradicts the initial assumption (\ref{eqnpiass}). Hence $\hat{A}$ must
be positively invariant.

\textit{ii) $\hat{A}$ is forward stable} - Since $\hat{A}$ is forward stable,
there exists a $\delta_p > 0$ that ensures forward stability from $p$. Since $a
\in \mathcal{N}_{\delta_p}(A(p))$, then by forward stability
\[ \dist( \Phi_{(t, p)}(a), A(\theta_t p)) \leq H^*(\Phi_{(t,
          p)}(\mathcal{N}_{\delta_p} (A(p)), A(\theta_t p)) < \e, \]
which similarly contradicts (\ref{eqnpiass}). Hence $\hat{A}$ is positively
invariant.

The case for complete stability follows immediately from either of the above
arguments for pullback or forward stability.
\end{prf}

\begin{eg} \label{cseg} [ {\em Complete Stability} ] \hfill \\
  Consider the differential equation
  \begin{equation}
  \dot{x} = \frac{\cos(\ta)}{(2+sin(\ta))} \left[ -x + arctan(\ta) \right] +
  \frac{1}{(1 + \ta^2)},
  \end{equation}
  which has solutions given by
  \[ \Phi_{(t,t_0)}(x_0) = arctan(t + t_0) + \frac{(2 + sin(t_0))}{(2 +
                         sin(t+t_0))}(x_0 - arctan(t_0)). \]
  Analysing stability of solutions with initial state $x_0$ to
  the family $\hat{A} = \{ A(\ta) ; \ta \in \mathbb{R}  \}$ where $A(\ta) =
  \arctan(\ta)$, it is easy to see that $\hat{A}$ is uniformly forward
  stable due to the (uniformly) bounded factor in the second term.

  Alternatively, if we consider pullback analysis of solutions to a fixed and
  arbitrary choice of $t_0$,
  \[ \Phi_{(t,t_0-t)}(x_0) = A(t_0) + \frac{(2 + sin(t_0-t))}{(2 +
                         sin(t_0))}(x_0 - A(t_0-t)). \]
  Given any $\epsilon > 0$, we choose $\delta_{t_0} = (2 +
  sin(t_0)) \e / 3$ so that for all $x_0 \in
  \mathcal{N}_{\delta_{t_0},\hat{A}}$,
  \[ H^*(\Phi_{(t, t_0-t)}(x_0), A(t_0)) \leq \epsilon. \]
  for all $t>0$.
  As a result, $\hat{A}$ is pullback stable. In fact, we may choose $\delta =
  \epsilon/3$ independent of $t_0$ and thus the pullback stability
  is uniform. As it is both pullback and forward stable, $\hat{A}$ is
  completely stable. Note, however that the stability is not
  asymptotic as the second term never vanishes completely  in either
  a forward or a pullback sense. Solutions with initial values  
  $(-1,-30)$ and $(-2,-30)$ were simulated and graphed. Refer to Figure \ref{csegpic}.
\end{eg}

\begin{figure}[htb]
\begin{center}
%\framebox[6.0cm][c]{
\leavevmode
\hbox{
\epsfxsize=10.0cm
\epsffile{eps/cseg.eps}  }%}
\protect\caption{Complete Stability of $\hat{A}$ for Example
\ref{cseg}}
        \protect\label{csegpic}
\end{center}
\end{figure}

\subsection{Asymptotic Stability}

The concepts of classical asymptotic stability (presented in
Section \ref{NDSsec}) may also be used in a similar manner as a
basis for extending the fundamentals of asymptotic theory to
families of sets and an encompassing theory inclusive of pullback
analysis. This is formalised below.


\begin{defn}[Forward Asymptotic Stability]\label{FASdef} \hfill \\
   A family $\hat{A} = \{A(p) ; p \in P\}$ of uniformly bounded compact
   subsets of $E$, is said to be
   {\bf forward asymptotically stable} with respect to the cocycle
   $\{\Phi_{(t,p)}; t \in \mathbb{R}^{+},p \in P\}$ on $E$ if it is {\em
   forward stable} and if there exists a $\hat{\delta} = \{\delta_p \in
   \mathbb{R}^+; p \in P\}$ so that for any $p \in P$, $x_0 \in
   \mathcal{N}_{\delta_p}(A(p))$,
   \begin{equation}\label{FASeq}
      \lim_{t \rightarrow \infty} H^{*}(\Phi_{(t,p)}(x_0),A(\theta_t p)) = 0.
   \end{equation}
\end{defn}

Alternatively we may write that $\hat{A}$ is {\em forward
asymptotically stable} if there exists a $\hat{\delta} =
\{\delta_p \in \mathbb{R}^+; p \in P\}$ so that for each $\e > 0$,
and $x_0 \in \mathcal{N}_{\delta_p}(A(p))$, there is a $T(x_0, p,
\epsilon)$ such that
\[ \dist(\Phi_{(t,p)}(x_0),A(\theta_{t}p)) < \epsilon, \qquad \forall t > T. \]

The case for {\em pullback asymptotic stability} however is a
little more complicated. If the family $\hat{A}$ is varying with
$p$, then the initial state $x_0$ may not always remain within the
boundaries of $\mathcal{N}_{\delta, \hat{A}}$ as it is pulled
back.

For example, consider the introductory Example \ref{introeg}, where
$\hat{A}$ was a sinusoidal attractor family, and ignore for the moment that the
attraction is global. A fixed \textit{local} $\delta$-neighbourhood of $\hat{A}$
will vary sinusoidally also, so an initial state may not always remain within
the neighbourhood for all time values.

As a result we shall consider attraction of arbitrary
sequences of initial states $\hat{x}_p = \{x(\theta_{-t}p) \in
\mathcal{N}_{\delta_p} (A(\theta_{-t}p)); t \geq 0 \}$ rather than a single fixed
initial state $x_0$. For any $p \in P$, every element of the sequence is
entirely contained within the $\delta_p$-neighbourhood of $\hat{A}$. We
thus define {\em pullback asymptotic stability} as follows.

\begin{defn}[Pullback Asymptotic Stability]
   \label{PASdef} \hfill \\
   A family $\hat{A} = \{A(p) ; p \in P\}$ of uniformly bounded compact
   subsets of $E$, is said to be {\bf pullback asymptotically stable} with
   respect to the cocycle $\{\Phi_{(t,p)}; t \in \mathbb{R}^{+},p \in P\}$
   on $E$ if it is {\em pullback stable} and if there exists a
   $\hat{\delta} = \{\delta_p \in \mathbb{R}^+; p \in P\}$ so that for any
   $p \in P$, $\hat{x}_p \in \mathcal{N}_{\delta_p, \hat{A}},
   x(\theta_{-t}p) \in \hat{x}_p$,
   \begin{equation}\label{PASeq}
      \lim_{t \rightarrow \infty} H^{*}(\Phi_{(t,\theta_{-t}(p))}
      (x(\theta_{-t}p)),A(p)) = 0.
   \end{equation}
\end{defn}

Again, it may be equivalently said that $\hat{A}$ is {\em pullback
asymptotically stable} if there exists
a $\hat{\delta} = \{\delta_p \in \mathbb{R}^+; p \in P\}$ so that for each $\e
> 0$, and $\hat{x}_p \in \mathcal{N}_{\delta_p, \hat{A}}$, there is a
$T(\hat{x}_p, p, \epsilon)$ such that
\[ \dist(\Phi_{(t,p)}(x(\theta_{-t}p)),A(p)) < \epsilon, \qquad \forall t > T. \]

{\bf Remark 1}
If a single $\delta$-neighbourhood of $\hat{A}$ (that is, $\hat{\delta} =
\delta$), and $T=T(\epsilon)$ can be chosen in the above definitions for
forward(pullback) asymptotic stability, then
$\hat{A}$ is said to be {\em uniformly forward asymptotically stable}
(respectively pullback).

{\bf Remark 2} If in the above definitions $T$ can be chosen so that $T=T(p,
\e)$ only, then $\hat{A}$ is said to be {\em forward(pullback)
equi-asymptotically stable}.

\begin{defn}[Complete Asymptotic Stability]\label{CASdef}
  If $\hat{A}$ is both forward and pullback asymptotically stable, then
  $\hat{A}$ is said to be {\bf completely asymptotically stable}.
\end{defn}

The following three examples illustrate the differences between
the three modes of asymptotic behaviour. Forward attraction is
illustrated in Figure \ref{fasegpic}, pullback attraction in
Figure \ref{pasegpic}, and complete attraction in Figure
\ref{casegpic}.

\begin{eg} \label{faseg}[\textit{Forward Asymptotic Stability without
                   Pullback Attraction}] \hfill \\
  \vspace{2mm}
  Consider the differential equation
  \begin{equation}\label{de2eq}
  \dot{x} = \frac{2\ta}{(1+\ta^2)} \left[ -x + \left( \tanh (\ta/2)
         \right) \right] + \frac{2 e^{-\ta}} {(1 + e^{-\ta})^2}.
  \end{equation}
  Defining $A(\ta) = \tanh (\ta/2)$ (a bipolar sigmoidal
  function) as the family of sets  $\hat{A}$, then the solution to the ODE
  above with initial time $t_0$ and   initial state $x_0$, is
  \[ \Phi_{(t,t_0)}(x_0) = A(t + t_0) + \frac{(1 + (t_0)^2)}{(1 +
                         (t+t_0)^2)}(x_0 - A(t_0)). \]
  Considering pullback attraction of solutions from initial state $x_0$
  to $\hat{A}$ at a fixed element $A(t_0) \in \hat{A}$ we have
  \[ \Phi_{(t,t_0-t)}(x_0) = A(t_0) + \frac{(1 + (t_0-t)^2)}{(1 +
                         (t_0)^2)}(x_0 - A(t_0-t)). \]
  Taking the limit as $t \rightarrow \infty$ in both cases it is obvious
  that $\hat{A}$ forward attracts solutions, but fails to pullback attract
  solutions. It is also forward stable, and hence forward asymptotically
  stable. Several solutions for the differential equation were simulated
  and plotted in Figure \ref{fasegpic}, to display the system's
  pullback instability.

  Note that forward convergence implies `eventual' attraction. It cannot
guarantee attraction at any other time. For example, it does not characterise
 the lack of attraction within this system to $A(0)$.

  \begin{figure}[htb]
  \begin{center}
  %\framebox[6.0cm][c]{
  \leavevmode
  \hbox{
  \epsfxsize=9.5cm
  \epsffile{eps/faseg.eps}  }%}
  \protect\caption{Forward Asymptotic Stability without Pullback
                Convergence}
         \protect\label{fasegpic}
  \end{center}
  \end{figure}
\end{eg}

\begin{eg} \label{paseg} [\textit{Pullback Asymptotic Stability without
                   Forward Convergence}] \hfill \\
  This example is a case of a pullback asymptotically
  stable set which fails to exhibit any forward asymptotic
  behaviour. Given the differential equation
  \begin{equation}\label{de1eq}
  \dot{x} = \frac{-2\ta}{(1+\ta^2)} \left[ -x + \left( \tanh (\ta/2)
         \right) \right] + \frac{2 e^{-\ta}} {(1 + e^{-\ta})^2}.
  \end{equation}
  If we denote $A(\ta) = \tanh (\ta/2)$, then the solution to the ODE above with initial time $t_0$,
  and initial state $x_0$, is
  \[ \Phi_{(t,t_0)}(x_0) = A(t + t_0) + \frac{(1 + (t+t_0)^2)}{(1 +
                         t_0^2)}(x_0 - A(t_0)). \]
  From this it is easy to see that the family of singleton sets $\hat{A} =
  \{ A(\ta) ; \ta \in \mathbb{R} \}$ fails to exhibit either of the usual
  characteristics of forward stability or asymptotic stability.
  Let us now   consider pullback attraction of solutions from initial state
  $x_0$, to   $\hat{A}$ at a fixed element $A(t_0) \in \hat{A}$
  \[ \Phi_{(t,t_0-t)}(x_0) = A(t_0) + \frac{(1 + t_0^2)}{(1 +
                         (t_0-t)^2)}(x_0 - A(t_0-t)). \]
  Taking the limit for some arbitrary $t_0, x_0 \in \mathbb{R}$ we find
  the second term vanishes, and hence
  \[ \lim_{t \to \infty} H^*(\Phi_{(t, t_0-t)}(x_0), A(t_0)) = 0. \]
  Thus $\hat{A}$ is pullback attracting. It is also pullback stable. To
  see this, given $\epsilon > 0$, then there exists a $\delta_{t_0} =
  \epsilon / (1 + t_0^2)$ so that for all $x_0 \in
  \mathcal{N}_{\delta_{t_0},\hat{A}}$, and for all $t \geq 0$,
  \[ H^*(\Phi_{(t, t_0-t)}(x_0), A(t_0)) \leq
                \epsilon \]

  \begin{figure}[htb]
  \begin{center}
  %\framebox[6.0cm][c]{
  \leavevmode
  \hbox{
  \epsfxsize=9.5cm
  \epsffile{eps/paseg.eps}  }%}
  \protect\caption{Pullback Asymptotic Stability without Forward
                Convergence}
        \protect\label{pasegpic}
  \end{center}
  \end{figure}

  Evolution of solutions lying in the interval $[-2,0]$ to $t_0 = 10$ from
  initial times progressively further back have been plotted, see Figure
  \ref{pasegpic}. It is easy to see that solutions become
  characteristically unstable once they have traversed past the switching
  region in the sigmoidal function (the pullback asymptotically stable
  set). However the interesting behaviour for this system is the fast
  attraction that occurs in and just preceding the switching region. It is
  here that pullback analysis describes the attractive behaviour to a
  fixed point within this region in detail and much more easily than with
  conventional techniques.
\end{eg}


\begin{eg} \label{caseg} [ {\em Complete Asymptotic Stability} ] \hfill \\
  We take a slightly different form of the differential equation
  used in the preceding Example (\ref{de1eq}), and (\ref{de2eq}),
  \begin{equation}
  \dot{x} = \left[ -x + \left( \tanh(\ta/2) \right) \right]
        + \frac{2e^{-\ta}} {(1 + e^{-\ta})^2}
  \end{equation}
  Again we have $A(\ta) = \tanh(\ta/2)$ as our family of sets under
  investigation, and the solution to the ODE above is given by
  \[ \Phi_{(t,t_0)}(x_0) = A(t + t_0) + e^{(-t)}(x_0 - A(t_0)). \]
  Pulling back solutions $x_0$ to $t_0$ we have
  \[ \Phi_{(t,t_0-t)}(x_0) = A(t_0) + e^{-t}(x_0 - A(t_0-t)). \]
  Hence, convergence in both the forward and pullback sense occurs
  at an exponential rate. Together with forward and
  pullback stability of $\hat{A}$ (easily shown), $\hat{A}$ is thus shown to
  be {\em completely asymptotically stable}. See Figure \ref{casegpic} for
  a representative simulation of the system's behaviour.
  \begin{figure}[htb]
  \begin{center}
  %\framebox[6.0cm][c]{
  \leavevmode
  \hbox{
  \epsfxsize=9.5cm
  \epsffile{eps/caseg.eps}  }%}
  \protect\caption{Complete Asymptotic Stability}
        \protect\label{casegpic}
  \end{center}
  \end{figure}
\end{eg}

\subsection{Uniformity of Stability}
\label{unissec}

A sufficient condition for stability or asymptotic stability of a
family of sets $\hat{A} = \{A(p) ; p \in P \}$ to be completely
stable, or completely asymptotically stable respectively, is that
of uniformity. This is shown through the following lemmas.

\begin{lemma}
\label{upscslem}
If $\hat{A} = \{A(p) ; p \in P \}$ is uniformly pullback stable then it is
completely stable.
\end{lemma}
\begin{prf}
We show by contradiction that $\hat{A}$ is uniformly forward stable, and
hence by definition completely stable.

Assume $\hat{A}$ is not uniformly forwards stable, but uniformly pullback
stable. Then for each $\e > 0$, there exists a $\delta = \delta( \e )$ as
defined for uniform pullback stability. Now, since $\hat{A}$ is not uniformly
forwards stable, there exists at least one initial state within the
neighbourhood system $\hat{\mathcal{N}}_{\delta, \hat{A}}$ that will escape an
$\epsilon$-neighbourhood of $\hat{A}$ at some later time. That is, there exists
a $p \in P$, $t^* \in \mathbb{R}^+$ and at least one initial state $x_0 \in
\mathcal{N}_{\delta}(A(p))$ such that
\[ \dist( \Phi_{(t^*, p)}(x_0), A(\theta_{t^*}p) ) = \e. \]
However by uniform pullback stability at $\theta_{t^*}p$, it must necessarily
be that
\[ \dist( \Phi_{(t^*, p)}(x_0), A(\theta_{t^*}p) ) < \e. \]
This is a contradiction. Hence $\hat{A}$ must be uniformly forwards stable,
and thus completely stable.
\end{prf}

\begin{lemma}
\label{ufscslem}
If $\hat{A} = \{A(p) ; p \in P \}$ is uniformly forwards stable
then it is completely stable.
\end{lemma}
\begin{prf}
Similarly, we will show by contradiction that $\hat{A}$ is uniformly pullback
stable, and hence by definition completely stable.

Assume $\hat{A}$ is not uniformly pullback stable, but uniformly forwards
stable. Then for each $\e > 0$, there exists a $\delta = \delta( \e )$ as
defined for uniform forward stability.

Now, since $\hat{A}$ is not pullback stable, there exists at least one initial
sequence within the neighbourhood system $\hat{\mathcal{N}}_{\delta, \hat{A}}$
that will escape an $\epsilon$ neighbourhood of $\hat{A}$ after being pulled
back from some previous time. That is, there exists a $p \in P$, $t^* \in
\mathbb{R}^+$ and at least one initial sequence $\hat{x}_p$ with some initial
value $x(\theta_{-t^*}p) \in \hat{x}_p$ such that
\[ \dist( \Phi_{(t^*, \theta_{-t^*}p)}(x(\theta_{-t^*}p)), A(p) ) = \e. \]
However by uniform forward stability from $\theta_{-t^*}p$,
\[ \dist( \Phi_{(t^*, \theta_{-t^*}p)}(x(\theta_{-t^*}p)), A(p) ) < \e. \]
Hence a contradiction. Consequently $\hat{A}$ must be uniformly pullback stable,
and thus completely stable.
\end{prf}

\begin{lemma}
\label{upascaslem}
If $\hat{A} = \{A(p) ; p \in P \}$ is uniformly pullback asymptotically stable
then it is completely asymptotically stable.
\end{lemma}
\begin{prf}
$\hat{A}$ is uniformly pullback asymptotically stable. Thus, there exists a
$\delta > 0$ so that for each $\e > 0$ there exists a $T \in \mathbb{R}^+$ with
$T = T(\e)$ such that for all $p \in P$, $\hat{x}_p \in
\hat{\mathcal{N}}_{\delta, \hat{A}}$ and $t > T$, we have for each $x(\theta_{-t}p) \in \hat{x}_p$,
\[ \dist( \Phi_{(t, \theta_{-t}p)}(x(\theta_{-t}p)), A(p) ) < \e. \]

Let $\delta$ be chosen as for uniform pullback asymptotic stability (above).
We show that $\hat{A}$ is uniformly forward asymptotically stable by
contradiction.

Assume that it is not uniformly forward asymptotically stable. That is for any
$\e > 0$, there exists some $p \in P$ and a sequence of values $t_j, x_j$ with
$t_j \to \infty$ as $j \to \infty$ and $x_j \in \mathcal{N}_{\delta}(A(p))$ such
that
\[ \dist( \Phi_{(t_j, p)}(x_j), A(\theta_{t_j}p)) > \e. \]
However $\hat{A}$ is uniformly pullback asymptotically stable. Hence for all
$t_j > T( \e )$ ($T(\e)$ as defined above),
\[ \dist(\Phi_{(t_j, \theta_{-t_j}(\theta_{t_j}p))}(x_j), A(\theta_{t_j}p)) <
\e, \]
which is the required contradiction. Thus $\hat{A}$ is uniformly forwards
asymptotically stable and hence completely asymptotically stable.
\end{prf}

\begin{lemma}
\label{ufascaslem}
If $\hat{A} = \{A(p) ; p \in P \}$ is uniformly forwards asymptotically stable
then it is completely asymptotically stable.
\end{lemma}
\begin{prf}
Let $\delta > 0$ be as defined for uniform forwards asymptotic stability, and
assume $\hat{A}$ is not uniformly pullback asymptotically stable.
Then for any $\e > 0$, there exists some $p \in P$ and sequences $t_j, x_j$,
with $t_j \to \infty$ as $j \to \infty$ and $x_j \in
\mathcal{N}_{\delta}(A(\theta_{-t_j}p))$ such that

\[ \dist( \Phi_{(t_j, \theta_{-t_j}p)}(x_j), A(p) ) > \e. \]
This leads to a contradiction since we require for uniform forwards asymptotic
stability at each $\theta_{-t_j}p$ that
\[ \dist(\Phi_{(t_j, \theta_{-t_j}p)}(x_j), A(p) ) < \e, \]
for all $t_j > T$, where $T = T( \e )$ as defined for uniform forwards
asymptotic stability. Hence $\hat{A}$ is uniformly pullback asymptotically
stable and thus completely stable.
\end{prf}

It is often easier to determine the forwards nature of a dynamical
system than its pullback characteristics. As a result, given the
above lemmas, the uniform stability of $\hat{A}$ in a forwards
sense is all that is required to verify complete stability of
$\hat{A}$. This will be looked at in further detail upon
consideration of Lyapunov functions in the following chapters.

Uniformity however, is not a necessary condition associated with
complete stability or complete asymptotic stability. To see this consider
Example \ref{egflows} below.

\begin{eg} \label{egflows}
Consider the flows generated by the dynamical system defined by
\[ \dot{x} = \left\{ \begin{array}{ll} -x & |x| \leq e^{-\ta}, \\
                                           -x + 2(x - sgn(x)e^{-\ta}) & |x| > e^{-\ta}. \\
                                        \end{array} \right. \]
See Figure \ref{csnuspic}. Here the parameter set $P = \mathbb{R}$ and the flows
are distinctly different in the regions separated by $\Omega$ (defined by $|x| =
e^{-\ta}$). In the inner region, exponential attraction to the origin occurs,
whilst in the outer region the dynamics may be represented in the form,
\[ \dfrac{d}{d\ta}(x - e^{-\ta}) = (x - e^{-\ta}). \]

\begin{figure}[htb]
\begin{center}
%\framebox[6.0cm][c]{
\leavevmode
\hbox{
%\epsfxsize=9.5cm
\epsffile{eps/ns1.eps}  }%}
\protect\caption{Complete Stability without Uniformity}
      \protect\label{csnuspic}
\end{center}
\end{figure}

Thus solutions diverge from the boundary at $\Omega$ ensuring instability
in the outer region.

Solutions here are obviously completely stable with respect to the
origin, that is they are both pullback and forward stable. However neither is
uniform, since by increasing $t$ the $\delta$-neighbourhood must be chosen
vanishingly smaller ensuring solutions do not begin in the region of
instability beyond $\Omega$.
\end{eg}

In this example the non-uniformity (with respect to $p$) in the neighbourhood of
pullback and forward stability occurs for $\theta_{t}p$ as $t \to \infty$ (that
is, where the neighbourhood of stability vanishes). It is of interest to observe
in the general case for a completely stable family $\hat{A}$, that
non-uniformity of the stable neighbourhood with respect to $p$
cannot take place for $\theta_{-t}p$ as $t \to \infty$
(that is, the neighbourhood of stability may not vanish for $\theta_{-t}p$ as $t
\to \infty$).

To show that this is indeed the case, suppose $\hat{A}$ is completely stable,
with a neighbourhood system of stability defined by $\hat{\delta}$ that vanishes
for $\theta_{-t}p$ as $t \to \infty$ (that is, $\delta_{\theta_{-t}p} \to 0$, as
$t \to \infty$).

However, assuming such a neighbourhood is neccesary contradicts
pullback stability at some $p$ as no constant $\delta > 0$ could be chosen to
assure that forward evolution of solutions from $\theta_{-t}p$ remain within an
$\e$-neighbourhood of $A(p)$ at $p$.

An additional lemma concerning pullback asymptotic stability concerns uniformity
of the local neighbourhood of attraction for all $p^* > p$ given knowledge of
the neighbourhood of attraction at $p$.

\begin{lemma}
If $\hat{A} = \{ A(p) ; p \in P \}$ is pullback asymptotically stable with a
local neighbourhood of attraction at any $p \in P$ given by
$\mathcal{N}_{\delta_p}(A(p))$ for some $\delta_p > 0$, then $\delta_{\theta_tp}
= \delta_p$ for all $t > 0$.
\end{lemma}
\begin{prf}
Let $p, p^* \in P$ and set $p^* > p$ with $t^* > 0$ such that $\theta_{t^*}p =
p^*$. Since $\hat{A}$ is pullback asymptotically stable, $A(p)$ pullback
attracts solutions within the $\delta_p$ - neighbourhood $\hat{\mathcal{N}}_{\delta_p,
\hat{A}}$.

We proceed to show that $A(p^*)$ also pullback attracts solutions within the
$\delta_p$-neighbourhood $\hat{\mathcal{N}}_{\delta_p, \hat{A}}$ by contradiction.

Assume $A(p^*)$ does not pullback attract all solutions within
$\hat{\mathcal{N}}_{\delta_p, \hat{A}}$. Then there exists an initial sequence
$\hat{x}_{p^*} \in \hat{\mathcal{N}}_{\delta_p, \hat{A}}$ such that for any $\e > 0$
small enough, there exists a sequence of values $\{ t_n \}$ with $t_n \to
\infty$ as $n \to \infty$ such that for each $t_n, x(\theta_{-t_n}p^*) \in
\hat{x}_{p^*},$
\begin{equation}\label{egnotatt}
\dist(\Phi_{(t_n, \theta_{-t_n}p^*)}(x(\theta_{-t_n}p^*)),
                       A(p^*)) > \e.
\end{equation}
$\hat{A}$ is also pullback stable, hence for $\e > 0$ as given above, there
exists a $\delta_{p^*} > 0$ as defined for pullback stability at $p^*$.

$A(p)$ pullback attracts $\hat{x}_{p^*}$ and hence there exists
$T=T(\hat{x}_{p^*}, \delta_{p^*})$ so that for all $t_n>T+t^*$,
$x(\theta_{-(t_n)}p^*) \in \hat{x}_{p^*}$,
\[ \Phi_{(t_n-t^*, \theta_{-t_n}p^*)}(x(\theta_{-t_n}p^*)) \in
              \mathcal{N}_{\delta_{p^*}}(A(p)). \]
Hence, by pullback stability of $p^*$,
\[ \dist(\Phi_{(t^*, p)}(\Phi_{(t_n-t^*,
                   \theta_{-t_n}p^*)}(x(\theta_{-t_n}p^*))), A(p^*)) < \e. \]
But this contradicts (\ref{egnotatt}). Hence given $\delta_p > 0$ for some $p
\in P$, we may indeed choose $\delta_{\theta_tp} = \delta_p$ for all $t > 0$.
\end{prf}

Note that the converse is not necessarily true. Refer to Example
\ref{pavnhoodeg} in Section \ref{ANSsec}.

\subsection{Conclusions - Non-Autonomous Attractors}

Examples \ref{paseg}, \ref{faseg}, \ref{caseg}, all examined
attraction to a family of sets $\hat{A} = \{A(t); A(\ta) = \tanh(\ta/2)
\}$. In each case, the family of sets $\hat{A}$ is
$\Phi$-Invariant. In addition, solutions converge to $\hat{A}$
although in each case the definition of attraction varies.

Invariance and attraction represent properties analogous to that
of the semi-group attractor for autonomous systems. Thus in
seeking an appropriate extension for this concept in
non-autonomous systems, each of these examples possess uniquely
differing properties that may be exhibited by a generic form of an
attractor within a non-autonomous environment.

{\bf $\Phi$-Invariance + Pullback Convergence} Pullback
convergence guarantees the attraction of solutions to an element
of $\hat{A}$, although as seen in Figure \ref{pasegpic}, there is
no guarantee that solutions will stay close to the family
$\hat{A}$ afterwards. However this type of structure is useful in
situations where only `capture' of solutions at a particular time
is required. Refer to \cite{Sk91}. It also has the additional advantage in that
certain limit set results for classical autonomous theory are extendable to
these objects (Arnold \cite{Ar98}, Schmalfuss \cite{Sc99}). These results will
be covered in more detail later in the chapter.

{\bf $\Phi$-Invariance + Forward Convergence} Solutions can only be
guaranteed to be close to the attractor for large values in time. That is,
it is characteristic of systems with stable `eventual' characteristics. The
example illustrated in Figure \ref{fasegpic} shows that there is no
attraction to the elements of $\hat{A}$ near $A(0)$. Hence, a good reason
to not consider the whole family $\hat{A}$ as a valid attractor. However,
many systems are only concerned with long time dynamics, and in these cases
such analysis is useful. Note that the usual limit set theorems do not hold for
time-varying structures that possess forward attraction.

{\bf $\Phi$-Invariance + Complete Convergence} Obviously complete
convergence guarantees convergence to the `attracting' family $\hat{A}$ in
every way. Solutions are guaranteed to attract to each element of the
family as well as ensuring they stay close and even converge to $\hat{A}$
thereafter.

The latter structure provides the ideal extension for the
semi-group attractor to non-autonomous systems although the others
are still equally valid and useful concepts.

The following section develops the notion of an attractor for
non-autonomous systems in detail.

\endinput
