\section{Attractors for Non-Autonomous Systems}
\label{ANSsec}

The preceding examples characterise two essential features
required of an attractor, that of $\Phi$-Invariance and
attraction. They also highlight clearly that the
notion of attraction in non-autonomous systems is distinctly
characterised in both a forward and pullback sense.

Hence the construction of non-autonomous attractors used throughout the thesis
will be developed with relevance to forward/pullback/complete attraction, in
both a local and global context, and with application to families of sets.

Before proceeding, a mention of the \textit{cocycle
attractors} developed by P.Kloeden, J.Lorenz, and B.Schmalfuss \cite{KlLo86,
KlSc96,PkSt97,Sc92} (later denoted  \textit{pullback
attractors} by P.Kloeden, D.Cheban, B.Schmalfuss and V.Kozyakin
\cite{ChKlSc98,ChKlSc00,Kl98,KlKo99,KlKo00, KlKo01}) is necessary to explain
the divergence of the definitions for pullback attraction used in
this thesis in comparison with the original concepts used by the authors
mentioned above.

\subsection{Pullback Attractors}

The form of the pullback attractor utilised by Kloeden et. al. was developed
from early ideas originating within random dynamical systems using cocycle
theory and as a result was initially referred to as a cocycle attractor. A
change in the terminology by Kloeden et. al.  to that of a pullback attractor
was made after the investigations in Section \ref{stabatt} by Stonier were made
known.

Initial theory for cocycle attractors was developed in the area of
global attraction for bounded compact subsets of $\mathbb{R}^d$,
although a more general formulation for local and parametric
dependent regions of attraction was devised. Throughout the papers
above \cite{ChKlSc98,ChKlSc00,Kl98,KlKo99,KlKo00, KlKo01} the
following construction is used:

\begin{defn}[Pullback Attractor : Kloeden, Cheban, Schmalfuss]
\label{defnKPA}
A $\Phi$-invariant family of nonempty compact subsets $\hat{A} = \{A(p); p \in
P\}$  will be called a \textit{pullback attractor} with respect to a basin of
attraction system $\mathcal{D}_{att}$ if it satisfies the pullback attraction
property
\[ \lim_{t \to \infty} H^*(\Phi_{(t, \theta_{-t}p)}(D_{\theta_{-t}p}), A(p)) =
                                 0, \]
for all $p \in P$ and all $\hat{D} = \{ D_p ; p \in P \}$ belonging to a
\textit{basin of attraction system} $\mathcal{D}_{att}$.

That is, a collection of families of sets $\hat{D} = \{ D_p ; p \in P \}$ where
$D_p$ is compact in $\mathbb{R}^d$ for each $p \in P$ with the property that
$\hat{D}^1 = \{ D_p^{(1)} ; p \in P \} \in \mathcal{D}_{att}$ if $\hat{D}^2 = \{
D_p^{(2)} ; p \in P \} \in \mathcal{D}_{att}$ and $D_p^{(1)} \subseteq
D_p^{(2)}$ for all $p \in P$.

Obviously $\hat{A} \in \mathcal{D}_{att}$.

In fact, $A(p) \subset int \mathcal{D}_{att}(p)$, where $\mathcal{D}_{att}(p) :=
\bigcup_{\hat{D} = \{ D_p ; p \in P \} \in \mathcal{D}_{att}} D_p$, for each $p
\in P$.
\end{defn}

Although utilising such a basin of attraction determined by
$\mathcal{D}_{att}$ is perfectly feasible for the consideration of forward
attractors, there arise several difficulties when defining pullback attractors
as in Definition  \ref{defnKPA}.

Without any loss in generality, we will consider these difficulties
with an analysis of pullback attraction to a \textit{constant} set $A$ (for ease
of illustration).

%\textbf{i)} Pullback attraction of individual elements $x_0$ may indeed
%converge to $A$ for some $p \in P$, yet if the basin of attraction system is
%not constant, then initial values determined by
%$(\theta_{-t}p, x_0$ may fall in and out of the basin of attraction as the
%initial state is pulled back. It should be expected that for all values of $t >
%0$,  $(\theta_{-t}p, x_0)$ always lies within the basin of attraction if $x_0$
%pullback converges to $A$ at $p$.

\textbf{i)} For a basin of attraction system $\mathcal{D}_{att}$ as described
above, it is assumed that the boundaries of attraction are identical when
investigating pullback attraction to the attractor for differing values of $p
\in P$. In essence, the basin of attraction utilised in this way is uniform
with respect to $p \in P$. For some pullback attractors this is not the case,
see Example \ref{pavnhoodeg}.

The concept of a uniform basin of attraction also contradicts
existing definitions for asymptotic stability \cite{BhSz67,Yo66},
where the attracting neighbourhood is well known to be dependent
on $t_0$.

\textbf{ii)} It is possible to construct objects that are pullback attractors
by Definition \ref{defnKPA} yet exhibit distinctly divergent behaviour.
Refer to Example \ref{improperpaeg}.

\begin{example}\label{pavnhoodeg}
Consider the one dimensional dynamical system generated by
\begin{equation*}
  \dot{x} =
     \begin{cases}
       - e^{\ta}, & \text{if $x > e^{3\ta}$;} \\
       - x^{1/3}, & \text{if $-e^{3\ta} \leq x \leq e^{3\ta}$;} \\
       e^{\ta}, & \text{if $x < -e^{3\ta}$.} \\
     \end{cases}
\end{equation*}
Here the parameter set is simply represented by the set of all initial time
values, that is, $P = R$. Smoothness of the derivative across the boundaries
$s^+(\ta) = e^{3\ta}$ and $s^-(\ta) = -e^{3\ta}$ guarantees existence and uniqueness of
solutions except at the origin which finite attracts solutions in a local
neighbourhood.

Due to the finite convergence caused by the dynamics given by $\dot{x} =
-(3/2)x^{1/3}$ in a close neighbourhood of the origin, it is clear that the
origin has attractive properties and is in fact a global forward attractor
($A$ in Figure \ref{pavnhoodfig}).

It will also be shown that $A$ pullback attracts solutions in a
local neighbourhood of the origin. The size of the neighbourhood
however, is dependent on the point in time from which the
solutions are pulled back.

\begin{figure}[htb]
\begin{center}
\input{eps/patt.pstex_t} \caption{Variable Regions of Pullback
Attraction} \protect\label{pavnhoodfig}
\end{center}
\end{figure}

To investigate properties of pullback attraction, we first consider pullback attraction 
to $A$ at $t_0=0$, and $t_0 = -0.5$. Solutions in this region behave as depicted in Figure 
\ref{pavnhoodfig}.

Reverse integration of the positive solution from (0,0) (highlighted),
indicates the trajectory approaches a limit at approximately $x_0 = 0.94$ as
shown. This value determines the maximum upper
boundary for the neighbourhood of pullback attraction to $A$ at $t_0 = 0$ since pulling back values $x > x_0$ means solutions do not traverse $s^+(\ta)$ early enough to completely finite attract to the origin (regardless of how far they have been pulled back).

Now consider pullback attraction of $x_0$ to $A$
at $t_0 = -0.5$ as marked in Figure \ref{pavnhoodfig}.  Due to the nature
of the system's dynamics, the trajectory originating from $x_0$ will never cross $s^+(\ta)$. Specifically, this implies that
\[ \Phi_{(t, -0.5-t)}(x_0) > s^+(-0.5), \hspace{2cm} \forall t > 0. \]
Consequently $A$ does not pullback attract $x_0$ to $t_0 = -0.5$.

Similarly, it can be shown that for any $x_0$ chosen, we can determine a time $t_0$ (far enough back)
such that $A$ cannot completely pullback attract $x_0$ regardless of how far the initial state 
is pulled back. Thus finding a suitable neighbourhood of pullback attraction to $t_0$ is strictly dependant on 
$t_0$ itself. 

In conclusion, to properly define a local neighbourhood of
pullback attraction for this example, we must choose a neighbourhood that
shrinks as $t_0$ decreases. However, this involves constructing a neighbourhood
that is not uniform with respect to $t_0$.

This then provides a counter-example of a dynamical system which clearly satisfies standard criteria for an attractor 
(in a pullback sense), yet does not have a fixed basin of attraction as required in Definition \ref{defnKPA} (Kloeden et. al ).

\end{example}

\begin{example}
\label{improperpaeg}
Consider the simple dynamics of $\dot{x} = x$. Solutions
\[ S_t(x_0) = x_0 e^{t}, \]
clearly diverge exponentially away from the origin, both in a forward and
pullback sense, yet here we will show there exists a basin of attraction system
that satisfies Definition \ref{defnKPA}, and as such defines the origin as a
pullback attractor.

Let $P = \mathbb{R}$ be the parameter set for this example and propose $A =
\{0\}$ as our pullback attractor.

Consider the basin of attraction system defined by $\mathcal{D}_{att}$ for
which every family of sets $\hat{D} = \{D_{\tau}; \tau \in \mathbb{R}, D_{\tau}
\subseteq [-e^{2\tau},e^{2\tau}] \}$. Then for any $t_0$, and $\hat{D} \in
\mathcal{D}_{att}$,
\[ \lim_{t \to \infty} H^*(\Phi_{(t, t_0 - t)}(D_{t_0 - t}),A) = 0. \]
Thus by Definition \ref{defnKPA}, $A$ must be a pullback attractor.

\begin{figure}[htb]
\begin{center}
\input{eps/improperpa.pstex_t}
\caption{Improper Pullback Attractor}
\protect\label{improperpafig}
\end{center}
\end{figure}

Interpreting $A$ as a pullback attractor however, is obviously in contradiction
with the basic nature of the dynamical system which exhibits no attractive
properties whatsoever. Refer to Figure \ref{improperpafig}.
\end{example}

\subsubsection{Conclusions}

Definition \ref{defnKPA} is valid in a global  context, and it clearly
attributes its construction to the extension of ideas developed for global
cocycle attractors. However, through the examples given it is clear that it
inadequately defines the conditions required by local pullback attractors, and
attractors with parametric dependent regions of attraction.

It was in part the problems with this construction that motivated
the investigations of Section \ref{stabatt} and the redevelopment of several of
the Theorems to follow throughout the remainder of this chapter.

The remainder of the thesis diverges from Kloeden's work, as it
incorporates the redefined concepts of pullback attraction, and also makes use
of the additional definitions and theorems developed for pullback stability and
pullback asymptotic stability.

\subsection{Defining the Non-Autonomous Attractor}

The attractor definitions below are a modification of the original
cocycle attractor concept and a natural extension of the ideas in
Section \ref{stabatt}. They are \textit{valid and applicable in both a local or
global context}  and also combine the concepts of forward/pullback and
complete attraction together in a comprehensive manner that identifies the
uniqueness of each property and retain the underlying axioms that form the basis
of classical asymptotic theory (refer to \cite{BhSz67,Yo66}).

\begin{defn}[Forward Attractor] \label{FAdef}
   A family $\hat{A} = \{A(p);p \in P\}$ of uniformly bounded compact
   subsets of $E$, is called a {\bf forward attractor} for the cocycle
   $\{\Phi_{(t,p)}; t \in \mathbb{R}^{+},p \in P\}$ on $E$ if there exists
   a $\delta$-neighbourhood system $\hat{\mathcal{N}}_{\hat{\delta}
   ,\hat{A}} = \{\hat{\mathcal{N}}_{\delta_p, \hat{A}}; \delta_p \in
   \hat{\delta}, p \in P \}$ defined by a delta set $\hat{\delta} =
   \{\delta_p \in \mathbb{R}^+; p\in P\}$, so that for each $p \in P$ the
   forward attractor $\hat{A}$ satisfies two properties,
   \begin{align} \label{FAInv}
      & \Phi_{(t,p)} \left(A(p)\right)=A(\theta_t(p)) \qquad \text{for each}
          \quad t \in \mathbb{R}^{+} & & \text{($\Phi$-Invariance)} \\
      \label{FAFC}
      & \lim_{t \to \infty} H^{*} \left(\Phi_{(t,p)}
           (\mathcal{N}_{\delta_p}A(p),A(\theta_t p)
           \right) = 0 & & \text{(Forwards Convergence)}
   \end{align}
\end{defn}

\begin{defn}[Pullback Attractor] \label{PAdef}
   A family $\hat{A} = \{A(p);p \in P\}$ of uniformly bounded compact
   subsets of $E$, is called a {\bf pullback attractor} for the
   cocycle $\{\Phi_{(t,p)}; t \in \mathbb{R}^{+},p \in P\}$ on $E$ if there
   exists a $\delta$-neighbourhood system
   $\hat{\mathcal{N}}_{\hat{\delta} ,\hat{A}} =
   \{\hat{\mathcal{N}}_{\delta_p, \hat{A}}; \delta_p \in
   \hat{\delta}, p \in P \}$ defined by a delta set $\hat{\delta} =
   \{\delta_p \in \mathbb{R}^+; p\in P\}$, so that for each $p \in P$ the
   pullback attractor $\hat{A}$ satisfies two properties,
   \begin{align} \label{PAInv}
      & \Phi_{(t,p)} \left(A(p)\right)=A(\theta_t(p)) \qquad \text{for each}
          \quad t \in \mathbb{R}^{+} & & \text{($\Phi$-Invariance)} \\
      \label{PAPC}
      & \lim_{t \to \infty} H^{*} \left(\Phi_{(t,\theta_{-t}(p))}
           (\mathcal{N}_{\delta_p}A(\theta_{-t}(p)),A(p)
           \right) = 0 & & \text{(Pullback Convergence)}
   \end{align}
\end{defn}

\begin{defn}[Complete Attractor] \label{CAdef}
  A family $\hat{A} = \{A(p);p \in P\}$ is called a {\bf complete
  attractor} if it satisfies the properties for both a forward and pullback
  attractor.
\end{defn}

{\bf Remark 1:} The $\Phi$-Invariance property is equivalent in
both a forwards and pullback context and is a generalisation of
the invariance property for autonomous semi-group attractors.

{\bf Remark 2:} Both the forward and pullback convergence
properties (\ref{FAFC}, \ref{PAPC}), are simply the requirement
that a neighbourhood system $\hat{N}_{\hat{\delta},\hat{A}}$
exists that converges to the attractor $\hat{A}$ in either the
forward, or pullback sense respectively. The rate of
attraction is determined by $T = T(p, \e)$ only, that is  it is
independent of the initialising state or sequence. In this
respect the asymptotic attraction must necessarily be
equi-asymptotic. It is this property that also assures the
stability of a non-autonomous attractor.

The definition for all three variations of the attractor reduce to
that of the definition introduced earlier (Definition \ref{att})
for a semi-group attractor when $P$ is a singleton set (as is the
case when the dynamical system is autonomous). Then the cocycle is
in fact a semi-group, and the attractor $\hat{A}$, coincides with
the semi-group attractor $A_0$ for each $p$. It is important to
see here that the defining characteristics of these non-autonomous
attractors implicitly retain all the properties and
characteristics of semi-groups and their attractors when applied
to autonomous systems.

The forward attractor also reduces to a form that retains all the classical
characteristics of asymptotic behaviour presented in Section \ref{NDSsec}
when it is a constant set. That is $A(p) = A_0$, for all $p \in P$, or
more briefly $\hat{A} = A_0$.

Also, if the forward attractor $\hat{A}$ attracts all bounded subsets $D \subset E$, then it is defined to be a {\bf global forward attractor}. A similar definition can be made for \textbf{global pullback attractors} and \textbf{global complete attractors}.

\subsection{Examples}

The three examples in the previous section illustrating asymptotic
behaviour, Examples \ref{faseg}, \ref{paseg}, and \ref{caseg}, all possess
attractors. In each example $\hat{A} = \{A(t) ; A(\ta) = \tanh(\ta/2) \}$
was found to be forward (pullback/completely respectively)
asymptotically stable. However, since $\hat{A}$ is also $\Phi$-Invariant in
each case, $\hat{A}$ is in an attractor too.

The following example constructs a complete attractor within a generalised
dynamical system. It ensures tracking of solutions to a time
dependent path through the state space.

\begin{eg}
Consider the linearised differential equation with input
control $u$,
\begin{equation}\label{poscontroleg}
  \mvec{\dot{x}} = B(\ta) \mvec{x} + \mvec{u}.
\end{equation}
Suppose we wish the system to converge to a time dependent path described
parametrically by the vector $\mvec{c}(\ta)$, with $\ta \in \mathbb{R}$. Given
$\mvec{c} \in C^1$ we define a control by
\[ \mvec{u} = k \mvec{c}(\ta) + \mvec{\dot{c}}(\ta) -
                        (kI + B(\ta)) \mvec{x}, \]
where $k > 0$ is some constant. Solutions to (\ref{poscontroleg}) with this
control are
\[ \Phi_{(t,t_0)}(\svec{x}{0}) = \mvec{c}(t) + e^{-kt} [ \svec{x}{0}
                - \mvec{c}(t_0)], \]
which are similar to those obtained previously in Example
\ref{caseg}. Following a similar forward and pullback analysis of
the problem we can see that here there exists a complete attractor
$\hat{A} = \{A(\ta);\ta\in \mathbb{R} \}$ defined by
\[ A(\ta) = \mvec{c}(\ta). \]
In addition, tracking onto the desired path will occur at an exponential
rate.

The tracking problem was simulated for a periodic path in two dimensional
space expressed in polar co-ordinates by
\begin{align*}
  r(\ta) &= 1 + 0.2 \cos (8\ta), \\
  \theta(\ta) &= \ta. \\
\end{align*}
The system was initialised at the origin, and the exponential
convergence to the path can be seen to occur within
a relatively short time, see Figure \ref{paegpic}.

\begin{figure}[htb]
\begin{center}
%\framebox[6.0cm][c]{
\leavevmode
\hbox{
\epsfxsize=9.5cm
\epsffile{eps/paeg.eps}  }%}
\protect\caption{Complete Attractor in Tracking Control}
       \protect\label{paegpic}
\end{center}
\end{figure}
\end{eg}

\subsection{Attractors and Asymptotic Stability}

The following theorem and its lemma provide an equivalent
counterpart in non-autonomous dynamical systems for the
relationship between semi-group attractors and asymptotically
stable sets in autonomous systems given in Theorem
\ref{attassthm}.

\begin{therm}\label{papasthm}
A pullback attractor is pullback equi-asymptotically stable.
\end{therm}
\begin{prf}
  A pullback attractor by definition automatically satisfies the
  pullback attraction requirement for pullback equi-asymptotic stability, hence
  it is only required to show that it is pullback stable.

  Assume that it is not pullback stable and consider some arbitrary value of
  $\epsilon > 0$, and any  $p \in P$. Then there exists
  sequences $\delta_j \rightarrow 0$ and $t_j \rightarrow \infty$ as $j
  \rightarrow \infty$ such that for each $j$
  \[ \Phi_{(t,\theta_{-t}(p))}
       (\mathcal{N}_{\delta_j}(A(\theta_{-t}(p)),A(p) < \epsilon
       \qquad \forall t < t_j, \]
  but with
  \begin{equation} \label{attstbprf}
  \Phi_{(t_j,\theta_{-t_j}(p))}
       (\mathcal{N}_{\delta_j} (A(\theta_{-t_j}(p)),A(p) =
       \epsilon
  \end{equation}
  However, $\hat{A}$ by definition pullback attracts an open
  neighbourhood system of itself, $\hat{\mathcal{N}}_{\hat{\delta},
  \hat{A}}$. Thus there exists a time $T = T(\epsilon, p)$ such that
  \[ \Phi_{(t,\theta_{-t}(p))}
       (\mathcal{N}_{\delta_p}(A(\theta_{-t}(p)),A(p)) < \epsilon
       \qquad \forall t > T. \]
  Now for $j$ large enough so that both $\delta_j < \delta_p$ and $t_j > T$
  are satisfied, we have
  \[ \Phi_{(t_j,\theta_{-t_j}(p))}
       (\mathcal{N}_{\delta_j}(A(\theta_{-t_j}(p)),A(p))  < \epsilon \]
  This provides the required contradiction with (\ref{attstbprf}),
  and hence $\hat{A}$ must be pullback stable and consequently pullback
  asymptotically stable.
\end{prf}

\begin{lemma}
A forward (complete) attractor is forward (completely) equi-asymptotically
stable.
\end{lemma}

The proof for forward asymptotic stability follows along the same lines as
that for the pullback result, and complete asymptotic stability (for a
complete attractor) is simply a combination of the two results for
the forward and pullback attractors respectively.

Note that a forward (pullback or complete) asymptotically stable set is not
necessarily an attractor.

\subsection{Comments}
Some comment should be made here regarding the terminology used for the
definitions of both asymptotic stability of sets and attractors. Much of the
recent research into stability makes use of the terminology as presented above
\cite{ChKlSc98,PkSt97,Sc92,St94}.

However this is in contradiction with many older publications (most notably
\cite{BhSz67}). In these articles the situation is reversed. An attractor
is defined as a set which simply attracts solutions and is not necessarily
invariant and an asymptotically stable set is defined as an invariant
set to which neighbouring solutions literally asymptote.

This latter terminology I believe is more apt as it characterises the very
definitions of each. Nevertheless, this thesis will continue using the former to conform with that used in current publications.

\endinput
