
\documentclass[11pt]{book}

\usepackage{dlayout}
\pagestyle{empty}
\begin{document}

This thesis investigates the properties of stability within both continuous
non - autonomous dynamical systems  and discretisations of such systems.
Utilising the concept of pullback attraction as used by Kloedon, Schmalfuss,
Arnold and others, it approaches the general question of stability within
non-autonomous systems and recognizes that stability may be of a forward or
pullback nature. Both forward and pullback aspects reduce to the uniform
characteristics of stability/asymptotic stability in autonomous systems.
Observed in this manner, the concepts of stability appear to 'bifurcate' as one
moves from autonomous to non-autonomous systems.

These ideas motivated Chapters 1-3 and the resulting concepts are defined and
formalised within. These chapters are intended as a fundamental framework from
which further research in the various fields of non-autonomous systems
may be extended.

A preliminary Lyapunov-like theory that characterises pullback
attraction is developed as a tool for examining non-autonomous behaviour in
Chapter 5. Its usefulness however is at this stage restricted
to the converse theorem of asymptotic stability.

Chapter 7 introduces the theory of Loci Dynamics. A transformation
is made to an alternative dynamical system where forward
asymptotic (classical asymptotic) behaviour characterises pullback
attraction to a particular point in the original dynamical system.
This has the advantage in that certain conventional techniques for
a forward analysis may be applied.

The remainder of the thesis, Chapters 4, 6 and Section 7.3, investigate the
effects of perturbations and discretisations on non-autonomous dynamical systems
known to possess structures that exhibit some form of stability or attraction.
Chapter 4 pertains to autonomous systems with semi-group attractors that have
been non-autonomously perturbed, Chapter 6 observes the effects of
discretisation on non-autonomous dynamical systems that exhibit properties of
forward asymptotic stability and Chapter 7 similarly investigates
discretisations of pullback asymptotically stable
systems. The theory of Loci Dynamics is used to analyse the nature of the
discretisation, however establishment of results directly analogous to those
discovered in Chapter 6 is shown to be unachievable. Instead a case by case
analysis is provided for specific classes of dynamical systems.

The nature of the results regarding discretisation provide a non-autonomous
extension to the work initiated by A. Stuart and J. Humphries
for the numerical approximation of semi-group attractors within autonomous
systems. Of particular importance is the effect on the system's asymptotic
behaviour over non-finite intervals of discretisation.

\end{document}
