\documentclass{article}
\usepackage{dlayout}
\usepackage{color, graphics}
%\usepackage{amsmath, amsfonts, amssymb}


\begin{document}
\section{Attractor Definitions}

Much of the work developed by Kloeden et. al. originated and
remains in an investigation into global attractors. The development of his ideas
carry these concepts of global attraction into a local context.

Alternatively, the approach that I have used in the thesis
investigates the entire subject of stability within non-autonomous
systems, that is, stability, asymptotic stability and attraction
from a local perspective. It addresses the requirements of local
definitions in a different manner than the approach of Kloeden's.

The following analysis is an attempt to clearly illustrate the
problems with Kloeden's constructions, and hence the justification
and outline for the approach I have used in the thesis. This
change in perspective required a need to reconstruct several key
base theorems before the latter results in the thesis were
developed.

There are several considerations involved when uniquely defining the
differences between global and local forward/pullback attractors. We step
through the process of definition beginning with a global analysis and then
addressing the changes needed when shifting to a local context.

\subsection{Global Attractors}

The situation for global attractors is distinctly simplified in as much as
convergence of solutions to a pullback/forward attractor need only consider
attraction of bounded subsets within $\mathcal{R}^d$. Much of  Kloeden's
analysis involves the assumption of global attractors and hence the mechanics
for such attractors remains relatively simple.

\subsection{Local Forward Attractors}

The local region of attraction for local forward attractors has
long been defined in terms of $\delta$ - neighbourhoods that vary
with respect to time. For example, for each $p \in P$ (the
parameter set $P$ is typically represented by the set of initial
times, however the theory leaves it generalised for various
reasons which are not relevant here), there exists a
$\delta_p$-neighbourhood of the attractor $A(p)$ at $p$ so that
the neighbourhood asymptotically converges in the usual forward
sense to the attractor. That is, for each $p \in P$ there exists a
$\delta_p > 0$ such that
\[ \lim_{t \to \infty} H^*( \Phi_{(t, p)}(\mathcal{N}_{\delta_p}(A(p)),
   A(\theta_t p)) = 0. \]
Here $\mathcal{N}_{\delta_p}(A(p)$ represents a $\delta$ -
neighbourhood of the forward attractor $\hat{A}$ at $p$.

Using Kloeden's terminology this can be alternatively stated as possessing a
domain of attraction $\mathcal{D}_{att}$ consisting of all families of bounded
subsets $\hat{D} = \{D_p ; p \in P \}$ with $\hat{D} \in \mathcal{D}_{att}$ such
that each $\hat{D}$ converges to the attractor $\hat{A}$. That is for each
$\hat{D} \in \mathcal{D}_{att}$, and each $p \in P$
\begin{equation}\label{dattdefn}
  \lim_{t \to \infty} H^*( \Phi_{(t, p)}(D_p), A(\theta_t p)) = 0.
\end{equation}
This construction is expressed in a similar fashion to the usual
definitions for global attractors. However it only applies to
forward equi-asymptotically stable families as attractors are
automatically equi-asymptotically stable.

By contrast, using $\delta$ - neighbourhoods as I have adopted in the
thesis, it is a straightforward procedure to also discuss concepts of
stability, asymptotic stability as well as attraction.

\subsection{Local Pullback Attractors}

How to construct the concept of domains of attraction for local pullback
attractors? As mentioned there are two approaches, one I have used within the
thesis, and the other adopted for cocycle attractors by Kloeden et. al.

\subsubsection{Stonier's Construction:}

Since we are interested in a set of state values which converge to
the attractor at some point $p \in P$, it would be intuitive to analyse the
characteristics of convergence to $A(p)$ from some reasonably constant
neighbourhood of the attractor. Hence for each $p \in P$ we determine
\[ \lim_{t \to \infty} H^*( \Phi_{(t,
        \theta_{-t}p)}(\mathcal{N}_{\delta_p}(A(\theta_{-t}p)), A(p)), \]
that is, the convergence of a neighbourhood of constant size as it is pulled
back in time.

It was deemed contextually important to adopt a similar terminology to that used
for forward attraction, to bring the concepts of stability and
attraction within non-autonomous systems together as a cohesive analysis.
The various examples developed in the thesis justified the modifications made,
and refined the notation. Some of the key concerns that illustrate the
differences with Kloeden's construction (presented later)  are given below.

To be analogous to neighbourhoods of attraction for local forward
attractors, it would be expected that the size (determined by
$\delta_p$) of this neighbourhood may in some instances vary with
$p \in P$. This is in fact demonstrated with the example below
which possesses a non-uniform (with respect to $p$) neighbourhood
of pullback attraction.

\begin{example}\label{pavnhoodeg}
Consider the one dimensional dynamical system generated by
\begin{equation*}
  \dot{x} =
     \begin{cases}
       - e^t, & \text{if $x > e^{3t}$;} \\
       - (3/2)x^{1/3}, & \text{if $-e^{3t} \leq x \leq e^{3t}$;} \\
       e^{t}, & \text{if $x < -e^{3t}$.} \\
     \end{cases}
\end{equation*}
Here the parameter set is simply represented by the set of all initial time
values, that is, $P = R$. We have smoothness of the
derivative across the boundaries  $s^+(t) = e^{3t}$ and $s^-(t) =
e^{3t}$, guaranteeing uniqueness and existence
of solutions except at the origin which finite attracts solutions in
a local neighbourhood.

Due to the finite attraction of $\dot{x} = -(3/2)x^{1/3}$ it is clear that the
origin has attractive properties and is in fact a global forward attractor.
\begin{figure}[hb]
\begin{center}
\input{eps/patt.pstex_t}
\caption{Pullback Attractor with Variable Neighbourhood of Attraction}
\protect\label{pavnhoodfig}
\end{center}
\end{figure}
To investigate properties of pullback convergence let us consider without loss
in generality, pullback attraction to the origin at $t_0=0$ (the same argument
can be applied for other values of $t_0$). Solutions in this region behave as
depicted in Figure \ref{pavnhoodfig}. By reverse integrating solutions from
(0,0) along its positive path (this trajectory is highlighted in the diagram),
we find solutions approach $x_0 \approx 0.94$ as $t \to -\infty$. This value in
fact determines the maximum boundary for the neighbourhood of pullback
attraction. If $x > x_0$ pullback convergence to the origin clearly does not
occur while for $x \leq x_0$ pullback convergence occurs.

Now consider pullback convergence of $x_0$ to the origin at $t_0 = t^*$. As a
result of the system's dynamics we find regardless of how far $x_0$ is pulled
back, it will never cross the boundary defined by $s^+(t)$. In fact we have
\[ \lim_{t \to \infty} \Phi_{(t, t^*-t)}(x_0) = s^+(t^*). \]
The boundary defining neighbourhood of pullback attraction for any $t_0 < 0$
can be developed as was done for $t_0 = 0$, resulting in a smaller
neighbourhood of attraction than that developed for $t_0 = 0$.
Consequently, the neighbourhood of pullback attraction can be show to
be shrinking as $t_0$ decreases.
\end{example}

For the construction defined above there exists one unusual and interesting
characteristic defining the structure of these neighbourhoods. Although the
neighbourhood of attraction may shrink as $t_0$ decreases as illustrated in the
example above, it can conversely be held constant as $t_0$ increases. That is,
the neighbourhood of pullback attraction for some $t^*$ is suitable as a
neighbourhood of attraction for all $t_0> t^*$. The lemma below switches back to
the generalised terminology used in the thesis to develop the proof.

\textbf{Lemma:} Given a neighbourhood of attraction at some $p \in P$ defined by
$\delta_p > 0$, a local neighbourhood of attraction for $\theta_{\tau} p$  may
be defined by setting $\delta_{\theta_{\tau} p} =  \delta_{p}$ for all $\tau >
0$.

\textbf{Proof:} Suppose that for some $p \in P$  there exists a neighbourhood
$\hat{\mathcal{N}}_{(\delta, \hat{A})}$ of $\hat{A}$ that
converges in a pullback sense to $A(p)$, and we wish to consider convergence to
$\hat{A}(\theta_{\tau}p)$ for some $\tau>0$. We define
$\mathcal{F}_{\tau} : B(X) to B(X)$ (here $B(X)$ represents the space of all
bounded subsets of $X$) mapping bounded sets of the dynamical system's state
space at $p$ to their image at $\theta_{\tau}p$. $\mathcal{F}_{\tau}$
continuous, etc etc and
\begin{align*}
  \lim_{t \to \infty} & \Phi_{(t, \theta_{-t+\tau}p)}
              (\mathcal{N}_{\delta_p} (A(\theta_{-t + \tau}p)))
              = \lim_{t \to \infty}  \mathcal{F}_{\tau} ( \Phi_{(t,
              \theta_{-t}p)} (\mathcal{N}_{\delta_p}(A(\theta_{-t}p))) , \\
  &=  \mathcal{F}( \lim_{t \to \infty} \Phi_{(t, \theta_{-t}p)}
              (\mathcal{N}_{\delta_p}(A(\theta_{-t}p))) , \\
  &= F(A(p)) = A(\theta_{\tau}p).
\end{align*}

Concluding, given the above construction, the local neighbourhood of attraction
is no longer a visualisable space within $\mathbb{R}^d$ in a straightforward way
unless a neighbourhood can be chosen such that $\delta_p$ is
uniform with respect to $p$. Rather it is a separate neighbourhood of the
attractor across the entire space for each $p \in P$. This is analogous in
concept design to the construction of neighbourhoods of attraction for forward
attractors.

\subsubsection{Kloeden's construction}

Kloeden's definitions are identical to that for forward attractors (Equation
\ref{dattdefn} and repeated here for convenience).

There exists a space labelled the domain of attraction
$\mathcal{D}_{att}$ consisting of all families of bounded subsets
$\hat{D} = \{D_p ; p \in P \}$ with $\hat{D} \in
\mathcal{D}_{att}$ such that each $\hat{D}$ converges (in the
pullback sense) to the attractor $\hat{A}$. That is for each
$\hat{D} \in \mathcal{D}_{att}$, and each $p \in P$
\[ \lim_{t \to \infty} H^*( \Phi_{(t,
                  \theta_{-t} p)}(D_{\theta_{-t}p}),            A(p)) = 0. \]
This is an intuitive formulation when considering global pullback
attractors where $\mathcal{D}_{att} = \mathbb{R}^n$ yet for pullback attractors
with local neighbourhoods of pullback attraction this presents
difficulties.

\textbf{i)} Elements within a region close to the attractor as they are
pulled back may ultimately converge in a pullback sense but may fall in and
out of the boundaries defined by $\mathcal{D}_{att}$. Refer to Figure
\ref{dattfig}.

Take for example the simple case of a singleton attractor when $\hat{A} = \{0
\}$. An analysis of a single state $x_0$ as it is pulled back in time may no
longer be feasible, in as much as it may fall in and out of the neighbourhood of
attraction, yet still ultimately converge in a pullback sense. Any state that is
in the neighbourhood of attraction at some point in time and known to converge
must naturally be a part of the neighbourhood of attraction at all times.

\begin{figure}
\begin{center}
\input{eps/datt.pstex_t}
\caption{Domains of Pullback Attraction}
\protect\label{dattfig}
\end{center}
\end{figure}

Example \ref{pavnhoodeg} also highlights this issue. $x_0$ is in
the neighbourhood of attraction for $t_0 = 0$, thus pullback attracted to the
origin from all previous times. As a result it should be included
within $\mathcal{D}_{att}$ for all $t \leq 0$.  However, it is not pullback
attracting to the origin for any $t_0 < 0$. A single instance of
$\mathcal{D}_{att}$ cannot represent both these cases, as it cannot represent
the inclusion of $x_0$ within the neighbourhood of attraction for the case of
$t_0 = 0$, and the exclusion of $x_0$ for the case $t_0 < 0$ simultaneously.

\textbf{ii)}Second, how are the boundaries of the domain of attraction defined
at any $p \in P$? Kloeden's definitions give no defining criteria
establishing the boundaries of the domain of attraction except for that of
pullback convergence. The only boundaries of importance that characterise
pullback convergence are as $t \to -\infty$.

The definition of boundaries at all other times are
inconsequential regarding the matter of convergence. In fact using
Kloeden's definitions it is possible to generate attractors that
exhibit pullback convergence by his definition of the attractor,
yet in actuality, exhibit pullback divergence. This can be shown
by taking a pullback divergent system, and constructing
$\mathcal{D}_{att}$ by shrinking the domain of attraction as $t\to
-\infty$ at a rate faster than solutions diverge at $p$. The
example below illustrates this.

\begin{example}
\label{improperpaeg}
Consider the simple dynamics of $\dot{x} = x$. Solutions for this clearly
diverge exponentially. Solutions are given by
\[ x(t) = x_0 e^{t-t_0}. \]
Alternatively expressed using cocycle notation to represent a state $x_0$
pulled back to time $t_0 - t$ and evolved to time $t_0$, we have
\[ \Phi_{(t, t_0-t)}(x_0) = x_0 e^t. \]
Now consider pullback convergence of solutions to the origin at $t_0 =
0$ by pulling back a decreasing sequence of initial points at $(e^{-2t}, -t)$
(procedure as per Kloeden's definitions). We then have
\begin{align*}
  \lim_{t \to \infty} \Phi_{(t, t_0-t)}(e^{-2t}) &= \lim_{t \to \infty}
                    e^{-2t}e^t, \\
             &= 0.
\end{align*}
\begin{figure}
\begin{center}
\input{eps/improperpa.pstex_t}
\caption{Improper Pullback Attractor}
\protect\label{improperpafig}
\end{center}
\end{figure}
Thus for a domain of attraction shrinking with the sequence of points chosen,
the origin is by the above definition a pullback attractor. Yet it is clear this
system diverges in both a pullback and forward sense. Refer to Figure
\ref{improperpafig}.
\end{example}

\subsection{Consequences}

A key theorem in the development of neighbourhood systems (Theorem
2 - "A Lyapunov Function for Pullback Attractors of Non-autonomous
Differential Equations) for pullback attractors that is frequently
later used as a foundation result in Kloeden's papers is that of
the existence of a pullback absorbing neighbourhood system
accompanying a known pullback attractor. There are two key
concerns here.

 \textbf{1)} A conclusion made early in the proof does not appear to hold. That
is the generation of an absorbing neighbourhood exists within the domain of
attraction. Although it does not affect the proof of the theorem at all, it does
confuse the necessities required of the domain of attraction
$\mathcal{D}_{att}$.

\textbf{2)} Second, the key property of positive invariance holds only because
the domain of attraction $\mathcal{D}_{att}$ is defined uniformly with respect
to $p$ (as discussed above), thus providing a link between $B(p)$ and
$B(\theta_t p)$ ensuring positive invariance.

For the construction of attractors I have adopted in the thesis (utilising
$\hat{\delta}$ to define the local neighbourhood of attraction) this is not
possible unless the neighbourhood of attraction is effectively
constant with respect to $p$. With the use of the previous lemma however, an
equivalent result may be made, arriving at the result given by Kloeden, with
the definitions within the thesis for a local analysis.

In conclusion, both constructions are valid and equivalent in a global
context, however when reduced to a local definition the latter
(utilising PK's $\mathcal{D}_{att}$) is inappropriate as shown above
and illustrated in Example \ref{pavnhoodeg}.

For cases where the local neighbourhood of attraction of a known attractor can
be found to be uniform with respect to $p$, it works appropriately and such a
domain of attraction may be constructed, The theorems that follow
then proceed to obtain intuitive results. As a general definition however they
allow for attractors by definition which may exhibit no pullback convergence at
all.

The definitions used in the thesis exclude improper possibilities for the
definition of an attractor, and yields the results in Kloeden's work with the
relevant extensions to definitions for neighbourhood systems and attention to
detail when modifying the proofs for the respective theorems.

Much of the early work in the thesis was involved in correcting
these problems, allowing for the introduction of attractors within
a local context (P. Kloeden's appeared to have problems at the
time, and hence the early chapters of the thesis were undertaken
in an effort to resolve this), as well as redefining the concepts
of stability which had not been done before, and modifying several
theorem proofs significantly to arrive at equivalent results. In
addition the properties of forward and pullback attraction were
brought together in a complete context. This was several months
effort in itself.

Previous to this attractors for non-autonomous systems as used by
P.Kloeden et. al. were referred to as cocycle attractors only.
Since the early chapters of the thesis were initially proof-read
by Kloeden two years ago there has been several articles
investigating both pullback and forward attraction by P. Kloeden
developing on the constructions contained within the thesis
without recognition of its origination. I was not informed of any
of this work by Kloeden that proceeded concurrently with the work
in the thesis.

\section{Lyapunov-like Construction}

There are distinct differences in the ideology behind both the
construction for a Lyapunov like theory used in the thesis and P.
Kloeden's function.

The construction developed in the thesis characterises pullback attraction, and
additionally characterises the rate of pullback attraction, a feature which is
lacking in Kloeden's definition.

Alternatively an attraction property is formulated in Kloeden's Lyapunov
function that resembles forward attraction.  Although this property
is utilised in his paper on the discretisation of uniform pullback
attractors, it is this property which simply utilises the forward attraction of
the uniform attractor (uniformity implying both pullback and forward
simultaneously) to guarantee the result.

This however can be done using existing lyapunov theory for non-autonomous
systems which possess forward asymptotic stability (Chapter 7 of the thesis). As
a result the usefulness of a function designed for pullback stability is not
realised, and quite possibly could not be realised in a pullback context as it
does not characterise the rate of pullback attraction.

The Lyapunov-like function in the thesis is hoped to be useful under such
circumstances (work is currently ensuing to make use of this function). The
differences in approach here I hope are clear and distinctly worthwhile.

\section{Discretisation}

As mentioned, the discretisation of uniform attractors is covered in both
Kloeden's and Stonier's work using different methods.

Within the thesis I proceed to tackle the problem's more difficult
questions which are distinctly avoided in the paper by Kloeden.

\textbf{i)} A non-uniform lipschitz bound on the driving function $f(p, x)$
behind the dynamical system is accounted for. This occurs for almost all
non-autonomous dynamical systems and as such is an important consideration that
is neatly avoided in the written article.

\textbf{ii)} Discretisation for non-uniform forward attractors is covered. Very
little has been established on non-uniform structures and this work took a
considerable period of time to cover.

\end{document}
