
\documentclass{article}

\usepackage{dlayout}

\begin{document}

\textbf{Comments}

The following is a reflection on the contents of my thesis chapter
by chapter, relating the differences between Kloeden's work and my
own, and a summary of the motivation behind each.

\textbf{Chapter Two}

Chapter two was essentially returning to basics to begin a
comprehensive investigation into stability and attraction for
non-autonomous systems. Much earlier I had tried to work towards
results for a Lyapunov-like theory, but never progressed very far
as the basic definitions and tools I needed at the time were not
sufficient. There was also basic problems with the attractor
definitions that I noted at the time, which prompted a return to
scratch to formalise these ideas. See the accompanying document
for a slightly more technical explanation in this regard.

It took several attempts, but eventually a working set of
definitions and ideas that complemented classical analysis well
was formulated. These were sent to Peter approximately two years
ago to read, comments came back with typing corrections and
little feedback.

The attractor constructions, lemmas on uniformity and asymptotic
stability of attractors followed of my own accord. It is now noted
by Peter that these concepts were used and the lemmas shown
concurrently and published in a paper (D.Cheban, P.Kloeden,
B.Schmalfuss, "The relationship between pullback, forwards, and
global attractors of non-autonomous dynamical systems", Nonlinear
Dynamics and Systems Theory). I had not seen this paper and Peter
has not sent me a copy of the paper until requested this month.
Peter is my supervisor and I would have expected him to inform me
of any results he had published for he knew the direction my
research was heading.

The results within this paper build on the preliminary concepts
within my thesis initially sent to Peter for proofreading, and it
is disappointing not to have been encouraged to work in this
direction after having sent Peter the preliminaries leading to
these results. The impression from Peter's comments regarding the
latter theorems in this chapter direct me to refer to this paper,
yet to do so would be indicating this is not work of my own. I
arrived at these results independently and this should be
recognised.

In summary, chapter two although not deeply mathematically
technical, was needed to obtain a better understanding and lay
down a more comprehensive formalisation of the methods needed in
stability analysis of non-autonomous systems. This had not been
previously done, and much of the work (both Peter's and mine), has
noticeably progressed from working simply with cocycle attractors,
to include these concepts as well.

 \textbf{Chapter 5}

Chapter 5 relates to generating lyapunov functions for
non-autonomous problems as a tool to analyse the effects of
discretisation on a dynamical system. Some comments of a slightly
more technical nature on this are discussed in the accompanying
article.

I had worked initially on this in Germany, with little ground made
- needed the tools developed in chapter two that came later. Peter
eventually published a lyapunov function result which he comments
on in the remarks sent back. I returned to the lyapunov problem in
2000, and worked on generating my own as Peter's turned out to be
not useful at all in terms of application, rather as a theoretical
curiosity. His lyapunov function did not have a decrescence
property, which is essential for application in dynamical systems.
Rather it has some appealing properties of theoretical interest
only, even though used in publishing a paper later on. I come to
this conclusion as the same result can be developed using forward
theory already available - it is the nature of his function that
makes use of the same theoretical properties, and hence is not
applicable then to systems which are strictly pullback in nature.
It took six months of work to find a correct formulation for a
useful Lyapunov function, and I kept returning to the construction
given now in the thesis. This Lyapunov function uses elementary
ideas in Yoshizawa's fundamental text in 1966 on Lyapunov
stability, but these concepts seem to favor a natural construction
towards stability analysis, even when utilised for pullback
analysis as covered in the thesis. It is fundamentally different
from Peter's derivation. The thesis derivation yields a
decrescence property applicable for strictly pullback systems and
also flexible Lipschitz results, not found in Peter's formulation.
To show its applicability I need to extend chapter 6 as
justification for its construction and include some partly
finished ideas. I am working on this currently.

\textbf{Chapter 6}

Chapter 6 also includes one major result that has been realised as
having been done concurrently with that of Kloeden. Namely, that
of discretisation of uniform attractors (published P.Kloeden, V.
Kozyakin, "Uniform non-autonomous attractors under
discretisation." to appear), and is summarized in Chapter 6.1 of
the thesis. The method used in the proof of each is distinctly
different and is outlined technically in the accompanying
document. Chapter 6.2 is also new and I have submitted its results
for acceptance to the Journal of Nonlinear Science. Peter
published on the uniform case, which is eminently much simpler
than my result in 6.2, yet it is considered by him,
straightforward. Any work in the non-uniform, non-automonous
domain is difficult - there is very little work published in this
area for this reason. Oddly I considered the results of 6.3 an
afterthought, having spent months generating the results of 6.2.
Chapter 6.3 is remarked as being similar to work already appearing
in a journal. ********* I have not seen this paper yet. Peter has
not informed me of any of his research until now.

I would hope it is clear that my research has been undertaken
independent of this current work mentioned by Peter, some of which
is still to be published.  I have been informed by CQU library
that the unpublished articles will not be available for a number
of months.

I understand the biggest problem  is lack of communication. I
undertook my own direction approximately two years ago in the
development of Chapter Two, for various reasons at the time, and
there has been no communication since the early  chapters were
sent to Peter. As my supervisor, I am disappointed that Peter has
not communicated to me any of the results obtained in the papers
or the current research pursued.

The work that I have produced has resulted from my own personal
research, and overlaps with three or four publications of Peter's,
please refer to the accompanying document. There are further
results in the thesis that I have obtained that also add to this
research.

I would  like to complete results of the pullback discrete
analysis, with an emphasis on the use of the work in Chapter 5 as
a justification for the effort spent on developing the Lyapunov
function.

I would appreciate acknowledgement by  my supervisor that my
research has been done independently of any knowledge of his
progressed research, and is not just a copy of work of others that
should be referenced in the thesis.

I feel strongly that this work contributes towards a PhD in its
originality of independent research.


\end{document}
