
\section{Discretised Autonomous Systems}\label{Dpertautsec}

In the following we consider the same problem under circumstances
where the solution has been numerically approximated using a one-step
numerical scheme (refer to Section \ref{diffeqns}). The results here are
supplementary to those presented in \cite{PkSt97}.

\subsection{The Discretised Perturbed System}

Again, we will consider the perturbed system
\begin{equation}
  \dot{x} = f(x) + \epsilon g(\ta,x),
\end{equation}
where $\epsilon > 0$ is a small perturbation parameter. Solutions
generated by a numerical scheme acting upon the above equation are
discrete cocycle mappings for which we will use the notation (as in
Definition \ref{dcrep})
\[ x_{n + n_0} = \Phi^{\e, h}_{(n,n_0)}(x_0). \]
where $h$ is the step size ($h$ for constant step sizes, and $\textbf{h}$ for
variable step sequences) for the numerical scheme. Also, as mentioned
previously, the autonomous system has a local semi-group attractor, $A_0$.

In Section \ref{Cpertautsec}, it was shown that the perturbed system
generates a corresponding continuous cocycle attractor, $A^{\e}(t_{0})$
which has components close to $A_0$ (for small $\epsilon$), and is in fact,
upper semi-continuous with respect to $A_0$.

We will proceed to show that the numerical scheme for
(\ref{pertnodeeq}) also generates a discretised cocycle attractor
which is upper semi-continuous with respect to the original
autonomous semi-group attractor.

\subsection{Main Result}

\begin{therm}\label{dispertthm}
  Suppose \textbf{P1-P3} hold for the non-autonomously perturbed
  dynamical system (\ref{pertnodeeq}).

  Then a numerical scheme applied to the perturbed non-autonomous system
  (\ref{pertnodeeq}), generates a discrete cocycle $\{
  \Phi^{\e,h}_{(n,t_0)}, n \in \mathbb{Z}^+, t_0 \in \mathbb{R} \}$. The
  discretised perturbed system possesses a discrete pullback attractor
  $\hat{A}^{\e,h} = \{A^{\e,h}(t); t \in \mathbb{R} \}$ (where $h$
is   the step size for the numerical scheme) such that
  \[ \lim_{\epsilon, h \rightarrow 0^{+}}
        H^{*}(A^{\e,h}(t_0),A_0)=0. \]
\end{therm}
\begin{prf}
  {\em 1. Existence: }
  As in Section \ref{Cpertautsec} we will consider the Lyapunov function
  $V$ associated with $A_0$ in the autonomous system, and its nature on
  a neighbourhood of the semi-group attractor defined by $\mathcal{N}(A_0)
  = \{ x \in \mathcal{N}_R(A_0); V(x) < a(R) \}$ within the context of the
  perturbed discretised system.

  Recalling the inequality (\ref{lineq}), for the upper left Dini Derivative
  of the Lyapunov function $V$ for an arbitrary solution of the perturbed
  non-autonomous equation, it was found that
  \begin{align*}
  D^{+}_{(\ref{pertnodeeq})}V(x) & \leq L K \epsilon - c V(x), \\
  & \leq - LK \e,
  \end{align*}
  for all $x \not \in B^{\e}$, where $B^{\e}$ was defined as
  \[ B^{\epsilon} = \{x \in \mathcal{N}_R(A_0) : V(x) \leq 2L K \epsilon
  /c \}. \]
  $B^{\epsilon}$ was shown to be a pullback absorbing neighbourhood for
  solutions of the perturbed continuous system. Also recall from above, for
  all $x_0 \in \mathcal{N}(A_0) \backslash B^{\e}$
  \[  V(\Phi_{(\tau,t_0-t)}(x_0)) \leq V(x_{0}) - LK\epsilon \tau, \]
  where $0 \leq \tau \leq t$ such that $\Phi_{(\tau,t_0 - t)}(x_0) \not
  \in B^{\epsilon}$.

  To construct a similar pullback absorbing neighbourhood for numerical
  solutions, we first note the behaviour of the Lyapunov function for a
  single step of the numerical scheme. Utilising the Lipschitz constant $L$
  associated with the Lyapunov function, and the local truncation error
$C_ph^{p+1}$ for a single   step in the numerical scheme, we obtain
  \begin{align*}
    V(x_{n+1}) & \leq |V(x_{n+1}) -
             V(\Phi_{(h,t_n)}(x_n))| + |V(\Phi_{(h,t_n)}(x_n))| \\
    & \leq LC_{p}h^{p+1} +  \left\{ \begin{array}{c}
             V(x_{n}) - LK \epsilon h ......(1) \\
             2LK\epsilon /c .................(2) \\
             \end{array} \right.  \\
  \end{align*}
  (1) if $\Phi_{(\tau,t_n)}(x_n) \not\in B^{\epsilon} \hspace{3mm} \forall
  \tau \leq h$.

  (2) if $\Phi_{(\tau,t_n)}(x_n) \in B^{\epsilon}$ for some $\tau
  \leq h$. Recall that $B^{\epsilon}$ is a pullback absorbing
  neighbourhood for the continuous system and hence $\Phi_{(t,t_{n})}(x_{n})
  \in B^{\epsilon} \hspace{3mm} \forall t$ with $\tau \leq t \leq h$.

  Now we propose as a pullback absorbing neighbourhood for the numerical
  scheme, namely:
  \[ B^{\epsilon,h} = \{x \in \Re^{d} : V(x) \leq 2LK \epsilon /c +
                    LC_{p}h^{p+1} \}. \]
  If $\epsilon$ and $h$ are small enough so that
  \begin{equation}\label{ehlimitseq}
    \epsilon < c a(R) / 4LK \qquad \text{and} \qquad  h^{p+1}  <  a(R) / 2LC_p,
  \end{equation}
  then $B^{\epsilon,h}$ is strictly a subset of the neighbourhood
  $\mathcal{N}(A_0)$.

  To show that the set $B^{\epsilon,h}$ indeed forms a pullback absorbing
  neighbourhood for the numerical scheme, we need to show that
  it pullback absorbs $\mathcal{N}(A_0)$ in finite time.

  Consider the pullback evolution of any point $x_{0} \in \mathcal{N}(A_0)$
  for the discretised system,
  \[ \Phi^{\e,h}_{(n,\theta_{-n}t_0)}(x_0). \]

  Let us firstly take $x_{0} \not \in B^{\epsilon,h}$. Then
  \[ V(\Phi^{\e,h}_{(1,\theta_{-n}t_0)}(x_0)) \leq LC_{p}h^{p+1} +  V(x_0) - LK
              \epsilon h. \]

  Now if we choose a $h_{1}$ small enough so that $\forall h \leq h_1$,
  we have
  \begin{equation}\label{hlimiteq}
  h^{p} < K\epsilon/2C_{p}.
  \end{equation}
  Then,
  \[ V(\Phi^{\e,h}_{(1,\theta_{-n}t_0)}(x_0)) \leq V(x_0) - LK \epsilon h / 2.
\]   Correspondingly,
  \[ V(\Phi^{\e,h}_{(\eta,\theta_{-n}t_0)}(x_0)) \leq V(x_0) - \eta
                LK \epsilon h/2, \]
  for all $\eta < n^*_{x_0}$, and some $n^*_{x_0} < n$ such that
  $\Phi^{\e,h}_{(\eta,\theta_{-n}t_0)}(x_0) \not \in B^{\e,h}$, and
  \begin{equation}\label{dispulleq}
    \Phi^{\e,h}_{(n^*_{x_0},\theta{-n}t_0)}(x_0) \in B^{\e,h}.
  \end{equation}

  \textbf{$B^{\e,h}$ is positively invariant}. To see this note that
  either   $x_{n} \in B^{\e}$ with
  \[ V(x_{n+1}) \leq LC_{p}h^{p+1} +  2LK \e /c, \]
  and by definition $x_{n+1}$ is automatically an element of $B^{\e,h}$, or
  we have $x_n \in B^{\e,h} \backslash B^{\e}$, in which case
  \[ V(x_{n+1}) \leq V(x_{n}) + LC_{p}h^{p+1} - LK \epsilon h. \]
  Then if $h \in [0,h_{1}]$, and since $x_n \in B^{\e, h}$, we have $V(x_n) \leq
  2LK\e/c + LC_ph^{p+1}$, the following inequalities hold
  \begin{tabbing}
    \hspace{2cm} \= $V(x_{n+1})$ \= $\leq$ \= $V(x_{n}) + LC_{p}h^{p+1} -
                    LK \epsilon h$ \\
    \> \> $\leq$ \> $2LC_{p}h^{p+1} + 2LK\epsilon /c -
                    LK \epsilon h$ \\
    \> \> $\leq$ \> $LC_{p}h^{p+1} + 2LK\epsilon /c - LK
                    \epsilon h /2$ \\
    \> \> $\leq$ \> $LC_{p}h^{p+1} + 2LK\epsilon /c$. \\
  \end{tabbing}
  Hence, $x_{n+1} \in B^{\epsilon,h}$, and $B^{\epsilon,h}$ is positively
  invariant.

  Returning to (\ref{dispulleq}), and since $B^{\e,h}$ is
  positively invariant,
  \[ \Phi^{\e,h}_{(n,\theta_{-n}t_0)}(x_0) \in B^{\e,h}, \]
  for all $n > n^*_{x_0}$ and $h < h_1$.

  Finally, an upper bound $n^*$ for $n^*_{x_0}$ exists for all $x_0 \in
  \mathcal{N}(A_0)$ since the Lyapunov function is bounded by $a(R)$ on
  $\mathcal{N}(A_0)$. Hence
  \[ \Phi^{\e,h}_{(n,\theta{-n}t_0)}(\mathcal{N}(A_0)) \subset B^{\e,h}, \]
  for all $n > n^*$ and $h < h_1$.

  Thus our proposed set $B^{\e,h}$ is indeed a pullback absorbing
  neighbourhood for solutions of the numerical scheme. We can then apply
  Theorem \ref{dabspatthm} to this to verify the existence of a discretised
  pullback attractor, $\hat{A}^{\e, h}$, within $B^{\epsilon,h}$.

  {\em 2. Approximation:}
  For $x$ $\in$ $B^{\epsilon,h}$ we have
  \[ a({\rm dist}(x,A_0)) \leq \ V(x) \leq 2LK \epsilon /c +
                                               LC_{p}h^{p+1}, \]
  so ${\rm dist}(x,A_0)$ $\leq$ $a^{-1}\left(2L K \epsilon /c +
  LC_{p}h^{p+1} \right)$. Hence
  \[ H^{*}(B^{\epsilon,h},A_0) \leq a^{-1}\left(2LK \epsilon /c +
                                       LC_{p}h^{p+1} \right). \]
  Since $A^h_{\epsilon}(t_0)$ $\subset$ $B^{\epsilon,h}$, we have
  \begin{align*}
    H^{*}(A^{\e, h}(t_0), A_0) &\leq \alpha^{-1}(2L K \epsilon
            /c + LC_{p}h^{p+1}), \\
    \intertext{and}
    H^{*}(A^{\e, h}(t_0), A_0) & \to 0^+ \qquad \text{as}  \quad
\epsilon,h             \to 0^+, \\
  \end{align*}
  for arbitrary $t_0$.

  This completes the proof of Theorem \ref{dispertthm}.

  \end{prf}

  {\bf Remark 1.} As in the result for the perturbed continuous system, the
  discrete pullback attractor is only guaranteed to exist under restricted
  values for both the size of the perturbation and step size, given by
  (\ref{ehlimitseq}) and (\ref{hlimiteq}). If the original semi-group
  attractor $A_0$ is in fact a global attractor, then only the step size
  need be restricted by Equation \ref{hlimiteq}.

\subsection{Corollary: Upper Semi-Continuity}

Theorems \ref{npertthm} and \ref{dispertthm} derive the existence
of a pullback attractor within the perturbed system (continuous
and discretised) under certain conditions and make comparisons for
them with the original semi-group attractor. Comparisons can also
be made between the continuous and discretised pullback
attractors, as outlined in the corollary below. It establishes
that in the limit as $h \rightarrow 0^+$, every point on the
discretised pullback attractor is arbitrarily close to a point on
the continuous pullback attractor. This is known as { \em upper
semi-continuity}.

A similar and more general result is achieved
for semi-group attractors in \textit{autonomous systems} by A.Stuart,
\cite{St94}, where it is noted that lower semi-continuity is
impossible to achieve without placing strong conditions on the
dynamical system.

The corresponding upper semi-continuity result for non-autonomously perturbed
autonomous systems is provided below.

\begin{cor}
  $\hat{A}^{\e, h}$, is upper semi-continuous with respect to
  $\hat{A}^{\e}$. That is for each $n \in \mathbb{Z}$, and corresponding
  $t_n \in \mathbb{R}$ where $t_n = t_0 + nh$,
  \begin{equation}
  \lim_{h \to 0^+} H^*(A^{\e, h}(t_n),A^{\e}(t_n)) = 0.
  \end{equation}
\end{cor}
\begin{prf}
  Suppose the above statement is false. Given some arbitrary
  $n_0$ and corresponding $t_0$ (for which the result may be generalised for
  any value of $n$ and $t_n$), there exists a sequence $\{ h_j \}$ with $h_j
  \to 0$ as $j \to \infty$, and an $\epsilon_0 > 0$ such that
  \[ H^*(A^{\e, h_j}(t_0),A^{\e}(t_0)) \geq \epsilon_0, \]
  for all $j$. Hence there exists a sequence $\{ a_j \}$ with $a_j \in
  A^{\e, h_j}(t_0)$ such that for each $j$
  \begin{equation}
  \label{propositioneq}
  \dist (a_j,A^{\e}(t_0)) \geq \epsilon_0.
  \end{equation}
  As the discrete pullback attractor is invariant, for each $j$ there exists a
  corresponding sequence of values $\{ b_{j_n} ; b_{j_n} \in
  A^{\e, h_j}(\theta_{-n}t_0) \}$ such that
  \[ \Phi^{\e,h_j}_{(n,\theta_{-n}t_0)}(b_{j_n}) = a_j. \]
  Define $c_{j_n}$, the continuous image of each $b_{j_n}$ at $t_0$ by
\[ c_{j_n}   = \Phi^{\e}_{(t_n,t_0 - t_n)}(b_{j_n}), \]
  where $t_n =   nh$. Now $b_{j_n} \in   \mathcal{N}(A_0)$ for each $n$, and
  since   $\hat{A}^{\e}$ pullback attracts   $\mathcal{N}(A_0)$ there exists a
  $T(\e_0) > 0$, and corresponding $N > 0$   (defined by $Nh > T$, and $(N-1)h
  < T$) such that
\begin{align}
  \dist(c_{j_N},A^{\e}(t_0) &\leq H^*(\Phi^{\e}_{(t_N,t_0 -
                t_N)}(\mathcal{N}(A_0)),A^{\e}(t_0)), \nonumber \\
  &< \e_0 / 2. \label{43atteq} \\ \nonumber
  \end{align}
  Also, given the cumulative numerical error arising between the continuous and
  numerical solutions, there exists a $J(\e_0) > 0$ such that for all
  $j > J$,
  \begin{align}
    \dist(a_{j},c_{j_N}) &\leq N C_p h^{p+1}_j, \nonumber \\
    & < (T + h_j)C_p h^p_j, \nonumber \\
    & < \e_0 / 2. \label{43erreq} 
  \end{align}
  Combining both equations (\ref{43atteq}) and (\ref{43erreq}),
  \begin{align*}
  \dist(a_{j}, A^{\e}(t_0)) &\leq \dist(a_{j},c_{j_N}) +
                \dist(c_{j_N},A^{\e}(t_0)), \\
  & < \e_0 / 2 + \e_0 / 2, \\
  & < \e_0, \\
  \end{align*}
  for all $j > J$ as defined earlier. This contradicts the proposition
(\ref{propositioneq}), and hence the original statement is true.
\end{prf}

\subsection{Corollary: Variable Time-Step Discretisation}

A similar numerical process may be applied to the continuous perturbed
system (\ref{pertnodeeq}) using a variable time-step discretisation as outlined
in Subsection \ref{ssecvarstep}. The variable time-step is represented as a
bi-infinite real sequence ${\bf h} = \{ h_n \}_{n \in \mathbb{Z}}$ bounded by
some fixed constant $\rho > 0$ such that $\frac{1}{2} \rho \leq h_n \leq \rho$
for all $n$.

\begin{cor}\label{corvardispert}
Consider a variable time-step numerical scheme applied to the perturbed
system under the conditions existing for Theorem \ref{dispertthm}. The
discretised system generates a cocycle $\{ \Phi^{{\bf h}}_{(n,(t_0,{\bf h}))}; n
\in \mathbb{Z}, (t_0,{\bf h})) \in \mathbb{R} \times H^{\rho} \}$ over the
parameter set $P = \mathbb{R} \times H^{\rho}$ with shift mapping $\theta_n
(t_0, {\bf h}) = (t_n, \psi_n {\bf h})$.
Then the variable time-step discretisation possesses a discrete pullback
attractor $\hat{A}^{\bf h}_{\e}$ such that
\[ \lim_{\epsilon, \rho \rightarrow 0^{+}}
        H^{*}(A^{{\bf h}}_{\epsilon}(t_0, {\bf h}),A_0)=0 \]
\end{cor}
\begin{prf}
The analysis follows similarly to the proof for Theorem \ref{dispertthm},
utilising the bound $\rho$ on the variable time-step sequence.

{\em 1. Existence: }
Recalling the details concerning the Lyapunov function associated with the
semi-group attractor, and using the definition for the pullback absorbing
neighbourhood set $B^{\e}$ of the continuous perturbed system we have

\begin{align*}
  V(x_{n+1}) & \leq |V(x_{n+1}) -
           V(\Phi_{(h_n,t_n)}(x_n))| + |V(\Phi_{(h_n,t_n)}(x_n))| \\
  & \leq LC_{p}h_n^{p+1} + |V(\Phi_{(h_n,t_n)}(x_n))| \\
  & \leq LC_{p}\rho^{p+1} +  \left\{ \begin{array}{c}
           V(x_{n}) - LK \epsilon h_n ......(1) \\
           2LK\epsilon /c .................(2) \\
           \end{array} \right.  \\
\end{align*}
(1) if $\Phi_{(\tau,t_n)}(x_n) \not\in B^{\epsilon} \hspace{3mm} \forall
\tau \leq h_n$.

(2) if $\Phi_{(\tau,t_n)}(x_n) \in B^{\epsilon}$ for some $\tau
\leq h_n$. Recall that $B^{\epsilon}$ is a pullback absorbing
neighbourhood for the continuous system and hence $\Phi_{(t,t_{n})}(x_{n})
\in B^{\epsilon} \hspace{3mm} \forall t$ with $\tau \leq t \leq h_n$.

We propose as a discrete pullback absorbing neighbourhood for this
system the set defined by

  \[ B^{\epsilon,{\bf h}} = \{x \in \mathbb{R}^{d} : V(x) \leq 2LK \epsilon /c +
                    LC_{p}\rho^{p+1} \}. \]

  First, if $\epsilon$ and $\rho$ are chosen small enough so that
  \begin{equation}\label{eqerlimits}
    \e < c a(R) / 4LK \qquad \text{and} \qquad  \rho^{p+1} <  a(R) / 2LC_p,
  \end{equation}
  then $B^{\epsilon,{\bf h}}$ is strictly a subset of the neighbourhood
  $\mathcal{N}(A_0)$.

To show that it is pullback absorbing we consider pullback evolution of any
point $x_0 \in \mathcal{N}(A_0)$ for the discretised system, with the
restriction on the variable time-step bound $\rho$ such that $\rho \leq \rho_1$
where
\[ \rho_1^p = K \e / 2C_p. \]
The remainder of the proof follows automatically along the same lines as in
Theorem \ref{dispertthm} by showing that the set  $B^{\epsilon,{\bf h}}$
is positively invariant, and then that it pullback absorbs the neighbourhood
$x_0 \in \mathcal{N}(A_0)$. That is, there exists an $n^* > 0$ such that
  \[ \Phi^{\e,{\bf h}}_{(n,\theta_{-n}(t_0, {\bf h})}(\mathcal{N}(A_0)) \subset
  B^{\e,{\bf h}}, \]
for all $n > n^*$ and $\rho < \rho_1$.

Finally, applying Theorem \ref{dabspatthm} we can verify the existence of the
discretised pullback attractor $\hat{A}^{{\bf h }}_{\e}$.

  {\em 2. Approximation:}
  For $x$ $\in$ $B^{\epsilon,{\bf h}}$ we have
  \[ a({\rm dist}(x,A_0)) \leq \ V(x) \leq 2LK \epsilon /c +
                                               LC_{p}\rho^{p+1}, \]
  so ${\rm dist}(x,A_0)$ $\leq$ $a^{-1}\left(2L K \epsilon /c +
  LC_{p}\rho^{p+1} \right)$. Hence
  \[  H^{*}(B^{\epsilon,{\bf h}},A_0) \leq a^{-1}\left(2LK \epsilon /c +
                                       LC_{p}\rho^{p+1} \right). \]
  Since $A^{\bf h}_{\epsilon}((t_0, {\bf h})$ $\subset$ $B^{\epsilon,{\bf
  h}}$, we have

  \begin{align*}
    H^{*}(A^{{\bf h}}_{\epsilon}(t_0, {\bf h}), A_0) &\leq \alpha^{-1}(2L
            K \epsilon /c + LC_{p}\rho^{p+1}), \\
  \intertext{and consequently,}
    H^{*}(A^{{\bf h}}_{\epsilon}(t_0, {\bf h}), A_0) &\to 0^+ \qquad \text{as}
\quad \epsilon,\rho \to 0^+,\\
\end{align*}
  for arbitrary $t_0$.
\end{prf}

\endinput
