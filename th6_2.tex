\section{Forward Equi-Asymptotic Stability}
\label{disfassec}

This section focuses on the effects of discretisation on a forward
equi - asymptotically stable family of sets,  $\hat{A} = \{A(p) ; p \in
P\}$ that does not necessarily possesses uniformity of stability. It also
incorporates analysis of the problem for which the differential
dynamics is \textit{unbounded} in nature (mentioned
briefly in the preceding section).

We briefly discuss what effects such unbounded dynamics
has on the numerics before continuing with the
main result.

\subsection{Unbounded Systems and a Numerical Analysis}

A preliminary investigation of the long term numerical analysis is
essential. For autonomous systems, that is for $\dot{x} = f(x)$, a
numerical one-step discretisation method always yields a local
truncation error as given by
\[ ||x_{n+1} - \Phi_{(h,p_n)}(x_n)|| \leq C_{r}h^{r+1}. \]
The truncation constant $C_r$ is dependent on the magnitude of
$f$ and its derivatives, and thus in a compact neighbourhood,
will always be bounded.

For non-autonomous systems $f$ (or its derivatives) may become
unbounded with respect to $p$, and hence a suitable truncation
constant may remain valid only over a finite time interval on a
compact neighbourhood of the object under investigation. As a
result, it becomes necessary to define the truncation constant as a
function of $p$, $C_r(p)$.

The elementary example below illustrates the difficulties encountered
whilst analysing discrete asymptotic behaviour for an unbounded
system.

\begin{eg}
Consider the dynamical system generated by the NDE
\[ \dot{x} = -2\ta x. \]
It can be easily shown for this system that the origin is a
forward attractor. If we discretise this system with an Euler scheme
that employs a constant time step, it is obvious that the change in the state
at each iterative step becomes large and possibly unstable when working in
regions for which $\ta$ is large.

For example, consider the initial value problems defined by $(x_0, t_0) =
(1,0)$, and also $(x_0, t_0) = (1,100)$. Using a step
size of $h = 0.1$, the ensuing iterated solutions after $50$ steps
are given by
\begin{align*}
     (x_0, t_0) = &(1,0) & \Rightarrow \hspace{1cm} &x_{20} = 7.21
                  \times 10^{-3}, \\
     (x_0, t_0) = &(1,100) & \Rightarrow \hspace{1cm} &x_{20} = 4.68
                  \times 10^{25}. \\
\end{align*}
In the latter case, the iterated solution oscillates around the
origin in an unstable fashion. This is due to the fact that the
updating term in the Euler scheme, $h f(t_n, x_n)$, grows
without bound as $\ta$ increases. In fact, calculating $x_1$ for the
latter problem yields $x_1 = 1 - 0.1*(200)$. The error for the
numerical scheme in this case is far greater than the rate of
attraction to the origin.
\end{eg}

There are two approaches that may be used to compensate for this instability in
the numerical method.

\textbf{i)} One may consider the discretisation on a decreasing
neighbourhood of attraction system $\mathcal{N}_{\hat{\delta},
\hat{A}}$ for which $h f(t_n, x_n)$ remains bounded, and then
determine the attraction within the defined system.

The difficulty with this approach is that in order to consider the
effects of discretisation, an explicit knowledge of the
neighbourhoods are required, both for the bounds on $f$, and the
neighbourhood upon which $V$ is defined and bounded.

\textbf{ii)} Alternatively one may use a variable time step scheme
for which the step sizes are restricted by the bounds of $f$ in a
local neighbourhood of the solution.

This ensures the error term remains bounded, and is viable for
general systems where $f$ does not approach zero as
solutions approach the attractor.

A discerning question regarding the latter approach is
reachability. That is, if the step sizes are restricted in such a
fashion, can it be ensured that $t_n \to \infty$ as $n \to
\infty$.

This approach is used when considering the discretisation problem
for forward equi-asymptotically stable families, presented below.

\subsection{Main Result}

The main result of this section concerns the discretisation of a
possibly unbounded (in the sense described above) non-autonomous
dynamical system that possesses a forward equi-asymptotically
stable family $\hat{A}$ with a variable time-step discretisation.
This extends the results of Section \ref{Discsecuas} and those by P.
Kloeden and V. Kozyakin \cite{KlKo01} to cases that do not possess uniformity,
nor boundedness of $f$. This introduces distinct changes in the approach needed,
most notably the fact that a variable time-step scheme is \textit{necessary}.

Again, as the proof of the entire procedure is long, it will be given via
a sequences of Lemmas. It will be shown that if the step sizes are
restricted appropriately, then there exists a discrete forward
equi-asymptotically stable family $\hat{A}^{\bf h}$ in the discretised dynamical
system. Further, $\hat{A}^{\bf h}$ is shown to be upper semi-continuous with respect
to $\hat{A}$.  The final declaration of the main result is established following
the Lemmas.

\subsubsection{Preliminaries}

We consider a continuous non-autonomous dynamical system with
cocycle mapping $\{ \Phi_{(t,p)}; t \in \mathbb{R}^+, p \in P \}$.
It is assumed to possess a forward equi-asymptotically stable
family of sets $\hat{A} = \{ A(p); p \in P \}$.

\subsubsection{Constant Neighbourhood of Attraction}

We assume that $\hat{A}$ possesses an attracting local
neighbourhood which may be chosen uniformly with respect to $p$ for analytical
purposes. That is, there exists a $\delta^* > 0$ generating a
constant local neighbourhood $\mathcal{N}_{\delta^*, \hat{A}}$
that is forward attracting to $\hat{A}$.

Note that this is true of the majority of dynamical systems, however it does not
imply that the basin of attraction is constant with respect to $p$.

This assumption is important in investigating attraction to a
discrete structure approximating $\hat{A}$ and ensuring the effects of
the discretisation keep the state within a non-vanishing local
neighbourhood (if the neighbourhood is varying, this is difficult
to realise unless the boundaries of the neighbourhood are
explicitly known).

%**** Lemma, for asymptotically stable systems, $f(p,x)$ in a $\delta$
%neighbourhood ($\delta = \delta(\e)$ for stability) can only become unbounded
%as $p \to +- \infty$.****

\subsubsection{Variable Step Sequence}

We assume a variable time step sequence ${\bf h} \in
\mathcal{H}^{\rho}$ where $H^{\rho}$ is the compact metric space
of all bi-infinite real sequences and $\rho$ is a predefined upper
bound such that $h_n \leq \rho$ for all $h_n \in$ \textbf{h} as
introduced in Section \ref{vtssec}.

\subsubsection{The Discrete Parameter Set $P_d$}

The parameter space for the discretised system $P_d$ is defined as the cross
product space $P_d: P \times H^{\rho}$. Elements of $P_d$
are represented by the couple $(p, {\bf h})$ and the shift mapping
$\theta^d_n : P_d \to P_d$ is constructed as in Section
\ref{ssecvarstep}. A discrete cocycle representation with variable
time step is consequently defined appropriately.

To simplify the notation, the series' $\{ x_n \}$ and $\{ p_n \}$, are
used to represent the sequences of states and time steps arising from
discretisation of the initial value problem $(p_0, x_0)$,
\begin{align*}
  x_n &= \Phi^{{\bf h}}_{(n,(p_0,{\bf h}))}(x_0), \\
  p_n &= \theta_{h_{n-1}}p_{n-1}. \\
\end{align*}

The actual composition of ${\bf h}$ will be restricted to
guarantee asymptotic behaviour.

\subsubsection{Truncation Error}

Since $f$ and its derivatives may become unbounded, the local
truncation error is defined by
\begin{equation*}
|x_{n+1} - \Phi_{(h_n, p_{n})}(x_n)| \leq C^*_{r}(p_n)(h_n)^{r+1}.
\end{equation*}
The value $C^*_{r}(p_n)$ is defined as the appropriate truncation
constant over the finite interval $p_n \to \theta_{\rho}p_n$. We will limit
ourselves to systems for which this term may become unbounded only as $p \to
\theta_t p$ for $t \to \pm \infty$.

\subsubsection{Lipschitz Property of $V$}

By Theorem \ref{confasthm}, there exists an associated Lyapunov
function $V=V(p,x)$ defined for each $p \in P$, and $x \in
\mathcal{N}_{\delta^*}(A(p))$.

We also define a bound on the local Lipschitzness with respect to
$x$ of $V$ by
\[ L^*(p) = \sup_{0 \leq h \leq \rho} l(\theta_{h}p), \]
where $l(p)$ is the usual Lipschitz bound for the Lyapunov
function $V$.

\subsubsection{Attracting Neighbourhood - $\hat{B}$}

Define $\delta_0 > 0$ by $\delta_0 = \delta^*/3$. and a family of sets
$\hat{B} = \{ B(p) ; p \in P \}$, where
\[ B(p) = \{ x; x \in \mathcal{N}_{\delta^*}(A(p)), V(p, x) < a (\delta_0 )
        \}. \]
From this continuous family of sets, a discrete family of sets
will be constructed that is positively invariant under the
discretisation for restricted step sizes.

To see that $\hat{B}$ is well defined, note that
\[ a( \dist(x, A(p)) ) \leq V( p, x ) < a( \delta_0 ), \]
for each $p \in P$ and $x \in B(p)$.
Consequently $B(p) \subset \mathcal{N}_{\delta_0}(A(p))$. Also, by
construction and since $V$ is continuous, it can be seen that
$B(p)$ is open. Hence $\hat{B}$ is appropriately defined.

We show positive invariance of a discrete family through a
sequence of Lemmas.

Define constants $L(p)$, and $C_r(p)$ so that
\begin{align}
C_r(p) &= \max \{ a(\delta_0), C^*_r(p) \}, \label{eqcrdefn} \\
L(p) &= \max \{ 1, a(\delta_0)/\delta_0, L^*(p). \}
   \label{eqldefn}
\end{align}
These are designed so that certain assertions may be made later.
Note that these changes do not affect the
local truncation error or Lipschitzness of $V$ except to
accommodate increased variation. Without loss in generality, we
also define $\rho < 1$ (the upper bound for the variable-time step
sequence).

\begin{lemma}[B1]
    For each $p \in P$, $x_0 \in B(p)$, if $h(p) \in [0, h_1(p))$, where
  the bound $h_1(p)$ is defined as the largest value satisfying the
  inequality
  \begin{equation}\label{h1boundeq}
    4 L(p)C_r(p)h_1(p)^{r+1}/(1 - e^{-ch_1(p)}) <
      \rho^{r+1} a ( \delta_0 ),
  \end{equation}
  then $x_1 \in \mathcal{N}_{\delta^*}(A(\theta_{h(p)}p)$, and $V(\theta_{h(p)}p, x_1)$ is well defined.
\end{lemma}
\begin{prf}
For any $p \in P$, let $x_0 \in B(p)$. Then
\begin{align}
  a(\dist(\Phi_{(h, p)}(x_0), A( \theta_{h(p)}p) ) &\leq V( \theta_{h(p)}p,
                        \Phi_{(h, p)}(x_0)), \nonumber \\
         & \leq e^{-ch(p)}V(p, x_0), \nonumber \\
         & \leq a( \delta_0 ). \label{lemvnhoodref}
\end{align}
Hence $\dist(\Phi_{(h, p)}(x_0), A( \theta_{h(p)}p) \leq
\delta_0$. Thus, for $x_1$,
\begin{align*}
  \dist(x_1,A(\theta_{h(p)}p)) &\leq \dist(\Phi_{(h, p)}(x_0),A(\theta_{h(p)}p))
                          + | x_1 - \Phi_{(h, p)}(x_0)|, \\
           & \leq \delta_0 + C_r(p)h(p)^{r + 1}, \\
           & \leq \delta_0 + \frac{\rho^{r+1} a(\delta_0) (1 - e^{-ch(p)})}
                   {4L(p)}, \\
           & \leq \delta_0 + \delta_0, \\
           & < \delta^*.
\end{align*}
Here we have used the fact that $L(p) \geq a(\delta_0)/\delta_0$.

Since $x_1 \in \mathcal{N}_{\delta^*} (A(\theta_{h(p)}p)$, $V(\theta_{h(p)}p,
x_1)$ is well defined.

Also $h(p) \leq \rho$ for any $h(p) \in [0, h_1(p))$ as required
for the variable time-step scheme. This can be seen from
(\ref{eqcrdefn}) and (\ref{eqldefn}) where $1/C_r(p) \leq 1 /
a(\delta_0)$, and $1/L(p) \leq 1$. Noting this, and substituting
into (\ref{h1boundeq}) we have
\begin{align*}
  h_1(p)^{r+1} &< (1 - e^{-ch_1(p)})\rho^{r+1}a(\delta_0) /
            4L(p)C_r(p), \\
  &< \rho^{r+1}. \\
\end{align*}
\end{prf}

The preceding Lemma ensures that we may now investigate the rate
of change of the Lyapunov function for steps in the discrete
sequence.

\begin{lemma}[B2]
  If the variable step sequence {\bf h} satisfies (\ref{h1boundeq}), then
$\hat{B}$ is positively invariant under the discretisation.
\end{lemma}
\begin{prf}
  Let $p_n \in P$ and choose any $x_n \in B(p_n)$.
  Since \\
  $x_{n+1} \in \mathcal{N}_{\delta^*}(A(\theta_{h(p_n)}p_n)$,
  $V$ is defined for $x_{n+1}$ and
  we may use the properties of Theorem \ref{confasthm} to show
\begin{align*}
  V( \theta_{h(p_n)}p_n, x_{n+1} ) &\leq V( \theta_{h(p)}p_n, \Phi_{(h(p_n),
            p_n)}(x_n) ) + \\
  & \hspace{2cm} L(p_n)|x_{n+1} - \Phi_{(h(p_n), p_n)}(x_n)|, \\
  &\leq e^{-ch(p_n)} V( p_n, x_n ) + L(p_n)C_r(p_n)h(p_n)^{r+1}, \\
  &\leq e^{-ch(p_n)} a( \delta_0 ) + L(p_n)C_r(p_n)h(p_n)^{r+1}. \\
\end{align*}
If the step size at $p_n$ is restricted so that $h(p_n) \in [0,
h_1(p_n))$,
\begin{align*}
  V( \theta_{h(p_n)}p_n, x_{n+1} ) &\leq e^{-ch(p_n)} a( \delta_0 ) +
            \frac{1}{4}(1 - e^{-ch(p_n)}) a( \delta_0), \\
 &\leq a( \delta_0). \\
\end{align*}
Hence $x_{n+1} \in B(\theta_{h(p_n)}p_n)$.
\end{prf}

By restricting the step size at each $p \in P$ in the preceding
Lemma it is ensured that any increase in the Lyapunov value
over one step due to the numerical approximation (characterised by the local
truncation error) is negated by the rate of attraction (characterised by the
exponential decrescence of the Lyapunov function).

Note also that the representation for $h(p)$ is defined continuously over $p$.
This is necessary so that the required variable time step sequence
may be generated for \textit{any} initial value problem.

\subsubsection{Discrete Attracting Neighbourhood - $\hat{B}^{{\bf
        h}}$}

The discretisation of the initial value problem $(p_0, x_0)$ is
now stated as follows.

Let ${\bf h}$ be any variable time step sequence so that $h_n =
h(p_n)$ satisfies (\ref{h1boundeq}) for each $n \in \mathbb{Z}$.
Then solutions to the discrete initial value problem $(x_0, (p_0, {\bf h}))$ are
expressed by
\begin{equation}
\label{divpdef}
 x_n = \Phi^{{\bf h}}_{(n,(p_0, {\bf h})}(x_0).
\end{equation}

From $\hat{B}$ we construct a discrete family of sets that is
positively invariant. Recall that $\psi_n$ is the shift operator acting on the variable time-step sequence (refer to Subsection \ref{ssecvarstep}).

\begin{lemma}[B3]
Define the discrete family $\hat{B}^{{\bf h}} = \{ B^{\bf h}(p_n, \psi_n{\bf
h}); n \in \mathbb{Z} \}$, by
\[ B^{\bf h}(p_n, \psi_n{\bf h})) = B(p_n). \]
Then $\hat{B}^{{\bf h}}$ is a discrete, open and positively invariant family
under the discretisation determined by the choice of {\bf h}.
\end{lemma}
\begin{prf}
  The proof follows by construction and from the results of the
  Lemmas \textbf{B1}, and \textbf{B2}.
\end{prf}

The following Lemmas construct a discrete
family that forward attracts $\hat{B}^{{\bf h}}$ in a finite
number of steps.

\subsubsection{Discrete Forward Equi-Asymptotically Stable
Family - $\hat{A}^{{\bf h}}$}

We propose as our discrete forward equi-asymptotically stable
family:

\begin{defn}[A1]
Define $\hat{A}^{{\bf h}} = \{ A^{\bf h}(p_n, \psi_n{\bf
h}); n \in \mathbb{Z} \}$ by
\[ A^{\bf h}(p_n, \psi_n{\bf h}) = \{x;V(p_n, x) \leq \frac{1}{2} \rho^{r+1}
                      a(\delta_0)  \}. \]
\end{defn}

Obviously the family $\hat{A}^{{\bf h}}$ is a strict subset of $\hat{B}^{{\bf h}}$.
It is now shown via a sequence of Lemmas that $\hat{A}^{{\bf h}}$ is a discrete
forward equi-asymptotically stable family of sets.

\begin{lemma}[A2 - Boundedness and Compactness]
  $\hat{A}^{{\bf h}}$ is uniformly bounded and each element is compact.
\end{lemma}
\begin{prf}
  \hfill \\
  The discrete family $\hat{A}^{{\bf h}}$ is uniformly bounded since $\hat{A}$ is
  uniformly bounded and
  \[ A^{{\bf h}}(p_n,\psi_n{\bf h}) \subset B(p_n) \subset
             \mathcal{N}_{\delta_0}(A(p_n)), \]
  for each $n$. Compactness follows from the continuity of $V$.
\end{prf}

\begin{lemma}[A3 - Positive Invariance]
  $\hat{A}^{{\bf h}}$ is a positively invariant family.
\end{lemma}
\begin{prf}
Given some $n \in \mathbb{Z}$, and any $x_n \in A^{{\bf h}}(p_n, \psi_n{\bf
h})$, and consider one step of the discretisation with $h_{n} =
h(p_n) \in [0, h_1(p_n))$. It follows that
\begin{align*}
  V( p_{n+1}, x_{n+1}) &\leq V( p_{n+1}, \Phi_{(h_n, p_n)}(x_n) ) +
            L(p_n)C_r(p_n)h_n^{r+1}, \\
  &\leq e^{-ch_n} V( p_n, x_n) + L(p_n)C_r(p_n)h_n^{r+1}, \\
  &\leq e^{-ch_n} \frac{1}{2} \rho^{r+1} a(\delta_0) + \frac{1}{4}
            \rho^{r+1} a(\delta_0) (1 - e^{-ch_n}), \\
  &\leq \frac{1}{2} \rho^{r+1} a(\delta_0). \\
\end{align*}
Hence $x_{n+1} \in A^{{\bf h}}(p_{n+1},\psi_{n+1}{\bf h})$. As the choice of
$n$ was arbitrary, it can be concluded that $\hat{A}^{{\bf h}}$ is positively
invariant.
\end{prf}

The following Lemma confirms that $\hat{A}^{{\bf h}}$
forward attracts $\hat{B}^{{\bf h}}$ under further restrictions on the
sequence step bounds. We also need to guarantee that for the
initial value problem $(p_0, x_0)$ any future time is reachable by
the variable-time step sequence {\bf h}. The restrictions required
for these two assertions are as follows and will become relevant
through the process of the following lemmas.

For each $p \in P$, define
\begin{description}
\item[i)] $h_2(p)$ by
\begin{equation*}
h_2(p) = \min \{ h_1(p), \phi \},
\end{equation*}
where $\phi$ satisfies the equation $(1 + e^{-c\phi}) =
2e^{-c\phi/4}$,
\item[ii)]
$h_{min1}(p)$ as the smallest value satisfying
\begin{equation}\label{eqhmin1}
\frac{1}{2}\rho^{r+1} a(\delta_0) < 4 L(p)C_r(p) h_{min1}(p)^{r+1} / (1 -
       e^{-ch_{min1}(p)}),
\end{equation}
\item[iii)] and $h_{min2}(p)$ by
\[ h_{min2}(p) = \min \{ h_{min1}(p), \phi/2 \}. \]
\end{description}

We now restrict the variable-time step sequence {\bf h} for the
discretisation of the initial value problem $(p_0, x_0)$ so that for each $n
\in \mathbb{Z}$, $h_n = h(p_n)$ is chosen so that it satisfies
\begin{equation}
\label{h2boundeq}
h_n \in (h_{min2}(p_n), h_2(p_n)).
\end{equation}
Note that this is non-empty due to the manner of construction of
the bounds given above.

\begin{lemma}[A4 - Reachability]
If the elements of {\bf h} satisfy the bounds (\ref{h2boundeq}),
then
\[ \lim_{N \to \infty} \sum_{j=0}^{N} h(p_j) \to \infty. \]
\end{lemma}
\begin{prf}
Assume otherwise, that is, there exists a $T>0$ such that
\[ \lim_{N \to \infty} \sum_{j=0}^{N} h_j < T. \]
Consider any $p_0 \in P$, then $L(p), C_r(p)$ are both bounded on
the finite interval $p_0 \to \theta_T p_0$. If we denote these
bounds by $L, C$, then for any $p \in [p_0, \theta_T p_0]$
by (\ref{eqhmin1}),
\[ \frac{1}{8LC}\rho^{r+1} a(\delta_0) < h_{min1}(p)^{r+1} / (1 -
      e^{-ch_{min1}(p)}). \]
Hence $h_{min1}(p)$, and consequently $h_{min2}(p)$ is thus
bounded below by some finite quantity $h^* > 0$, for every $p \in
[p_0, \theta_T p_0]$. Let $N = N(T)$ be the first integer such
that $Nh^* > T$. Then
\[ T < Nh^* \leq \sum_{j=1}^{N}h_{min2}(p_j) \leq \sum_{j=1}^{N}h(p_j), \]
providing the required contradiction. Hence the original assertion is true.
\end{prf}

We now verify that $\hat{A}^{{\bf h}}$ forward attracts $\hat{B}^{{\bf h}}$.

\begin{lemma}[A5 - Forward Attracting] \hfill \\
If the elements of {\bf h} satisfy the bounds (\ref{h2boundeq}),
then $\hat{A}^{{\bf h}}$ forward attracts $\hat{B}^{{\bf h}}$.
\end{lemma}
\begin{prf}
Let $p_0 \in P$, and consider any $x_0 \in B^{{\bf h}}(p_0, {\bf h})\backslash
A^{{\bf h}}(p_0,{\bf h})$. Then
\[ \frac{1}{2}\rho^{r+1} a(\delta_0) \leq V(p_0, x_0) \]
 and we have
\begin{align*}
  V(p_1, x_1) &\leq e^{-ch_0} V(p_0, x_0)+\frac{1}{4} \rho^{r+1} a(\delta_0)
            (1 - e^{-ch_0}), \\
  &\leq \frac{1}{2}(1 + e^{-ch_0}) V(p_0, x_0). \\
\end{align*}
Now, for any $h_0 \in (h_{min2}(p_0), h_2(p_0))$,
\[ (1 + e^{-ch_0}) \leq 2e^{-ch_0/4}. \]
This is due to the fact that $h_0
< \phi$ (as defined earlier) for which equality of the above expression occurs.
Below this value the above inequality holds. Now
\[ V(p_1, x_1) \leq e^{-ch_0/4} V(p_0, x_0). \]
Suppose $x_j \in$ $B^{{\bf h}}(p_j,\psi_j{\bf h}))\backslash
A^{{\bf h}}(p_j, \psi_j{\bf h}))$ for $j =$ $1, \dots,$ $n-1$. By
extrapolating the above argument, we have
\begin{align*}
  V(p_n, x_n) &\leq exp\left(-c\sum_{i=0}^{n-1}h_i/4\right) V(p_0, x_0), \\
  &\leq exp\left(-c\sum_{i=0}^{n-1}h_i/4\right) a(\delta_0). \\
\end{align*}
Let $N = N({p_0, \bf h})$ be the first integer satisfying
\[ \sum_{i=0}^{N}h_i \geq \frac{4}{c} \ln \left(
\frac{2}{\rho^{r+1}}\right), \]
then for all $n > N$ (and recalling that $\hat{A}^{{\bf h}}$ is positively
invariant under the discretisation),
\[ V(p_n, x_n) \leq \frac{1}{2} \rho^{r+1} a(\delta_0). \]
Hence $x_n \in A^{{\bf h}}(p_n, \psi_n{\bf h}))$ for all $n > N$.
\end{prf}

Finally, to satisfy the definition for asymptotic attraction, we need to show
$\hat{A}^{{\bf h}}$ forward attracts an appropriately defined neighbourhood
system of itself and also verify its stability.

\begin{lemma}[A6 - Forward Asymptotic Attraction] \hfill \\
\label{lemfaa}
$\hat{A}^{{\bf h}}$ forward attracts the neighbourhood system
$\mathcal{N}_{\hat{\delta}, \hat{A}^{h}}$, where \hfill \\ $\hat{\delta} = \{
\delta_{(p_n, \psi_n{\bf h})}  ; \delta_{(p_n, \psi_n{\bf h})} > 0, n \in
\mathbb{Z} \}$.
\end{lemma}
\begin{prf}
For any $n$,  $B^{{\bf h}}(p_n, \psi_n{\bf h})$ is an open set for which
$V(p_n, x_n) < a(\delta_0)$ for all $x_n \in B^{{\bf h}}(p_n, \psi_n{\bf h})$.
Similarly $A^{{\bf h}}(p_n, \psi_n{\bf h})$ is a compact set whose elements are
constrained by $V(p_n, x_n) \leq \frac{1}{2} \rho^{r+1} a(\delta_0)$.

Since $V$ is a continuous function in $x$, the level curves given
by $V_1 = \frac{1}{2} \rho^{r+1} a(\delta_0)$, and $V_2 =
a(\delta_0)$ cannot meet. Hence there exists a minimum distance
denoted by $\delta_m(p_n, \psi_n{\bf h}) > 0$ separating $V_1$ and $V_2$.
Let $\delta_{(p_n, \psi{\bf h})} = \frac{1}{2} \delta_m(p_n, \psi_n{\bf h})$.
Then $\mathcal{N}_{\hat{\delta}, \hat{A}^{h}}$, where $\hat{\delta} =
\{ \delta_{(p_n, \psi_n{\bf h})} ; n \in \mathbb{Z} \}$ forms an
appropriate neighbourhood system.

This is valid since $\mathcal{N}_{\hat{\delta}, \hat{A}^{h}} \subset
\hat{B}^{{\bf h}}$ and as $\hat{A}^{{\bf h}}$ forward attracts $\hat{B}^{{\bf
h}}$, it subsequently forward attracts $\mathcal{N}_{\hat{\delta},
\hat{A}^{h}}$.
\end{prf}

\begin{lemma}[A7 - Forward Stability]
$\hat{A}^{{\bf h}}$ is forwards stable.
\end{lemma}
\begin{prf}
Given $\e > 0$ small enough, take $\delta^*$ as defined earlier and let
$\delta^* = \e$. Then $\hat{\delta}
= \{ \delta_{(p_n, \psi_n{\bf h})} ; \delta_{(p_n, \psi_n{\bf h})} > 0, n \in
\mathbb{Z} \}$ as defined in Lemma \ref{lemfaa} (note that by progressing
through the proof that $\hat{\delta}$ is ultimately a function of $\delta^*$)
satisfies the requirements for forward stability of $\hat{A}^{{\bf h}}$.

To see this, consider any $x_n \in
\mathcal{N}_{\delta_{(p_n, \psi_n {\bf h})}}(A(p_n, \psi_n {\bf h}))$. Then $x_n
\in B^{{\bf h}}(p_n, \psi_n {\bf h})$. As $\hat{B}^{{\bf h}}$ is positively
invariant, then $x_{\eta} \in B^{\bf h}(p_{\eta}, \psi_{\eta} {\bf h})$ for all
${\eta} > n$. Also
\[ B(p_{\eta}, \psi_{\eta} {\bf h}) \subset
\mathcal{N}_{\delta^*}(A(p_{\eta})) \subset \mathcal{N}_{\delta^*}(A(p_{\eta},
            \psi_{\eta} {\bf h} )). \]
Hence $x_{\eta} \in \mathcal{N}_{\e}(A(p_{\eta}, \psi_{\eta} {\bf
h} ))$ for all ${\eta} > n$. As the initial choice of $n$ was
arbitrary, we may conclude $\hat{A}^{{\bf h}}$ is forwards stable.
\end{prf}

\begin{lemma}[C1 - Upper Semi-Continuity]
$\hat{A}^{\bf h}$ is upper semi - continuous with respect to
$\hat{A}$ in the variable step upper bound $\rho$.
\end{lemma}
\begin{prf}
Note that $\hat{A}$ is contained within $\hat{A}^{\bf h}$, as $V$
is continuous and $\frac{1}{2}\rho^{r+1}a(\delta_0)
>0$. For each $n$ and $x_n \in A(p_n, \psi_n{\bf
h})$,
\begin{align*}
  \dist(x_n, A(p_n)) &\leq a^{-1} (V(p_n, x_n)), \\
  &\leq a^{-1}(\frac{1}{2}\rho^{r+1}a(\delta_0)). \\
\end{align*}
Since $a^{-1}(\frac{1}{2}\rho^{r+1}a(\delta_0)) \to 0$ as $\rho \to 0$, the
required result follows.
\end{prf}

Applying Definitions and Lemmas B1-B3, A1-A7 and C1, we arrive at
the main result.

\begin{therm}[Main Result]
\label{numeasthm1} Let $\{ \Phi_{(t, p)}; t \in \mathbb{R}^+, p
\in P \}$ be a cocycle for $(\ref{NDEeq})$ which contains a
forward equi-asymptotically stable family of sets $\hat{A} = \{
A(p) ; p \in P\}$ for which a constant neighbourhood of attraction
can be defined. Then a variable-time step discretisation (with
bound $\rho > 0$ and restrictions on the individual step sizes)
of the initial value problem $(p_0, x_0)$
with a one-step numerical method generates a
discrete dynamical system possessing a discrete forward
equi-asymptotically stable family $\hat{A}^{{\bf h}} = \{
A^{{\bf h}}(p_n , \psi_n{\bf h}) ; n \in \mathbb{Z} \}$ satisfying
\begin{equation}
\label{con2eq}
 H^*( A(p_n, \psi_n {\bf h}), A(p_n)) \to 0 \hspace{2cm} \text{as} \qquad \rho
\to  0.
\end{equation}
for each $n \in \mathbb{Z}$.
\end{therm}

%The following theorem considers the numerical approximation of a continuous
%non-autonomous system with a forward attractor $\hat{A}$.

%THIS LEMMA IS NOT VALID AS NO LIMIT SET REPRESENTATION EXISTS FOR FORWARD ATTRACTORS

%\begin{therm}
%Let $\{ \Phi_{(t, p)}; t \in \mathbb{R}^+, p \in P \}$ be a
%cocycle for $\ref{NDEeq}$ which possesses a forward attractor
%$\hat{\mathcal{A}} = \{ \mathcal{A}(p) ; p \in P\}$ which
%possesses a constant neighbourhood of attraction (as discussed in
%the preliminaries). Then a variable-time step discretisation (with
%bound $\rho > 0$ and restrictions on the individual step sizes) of
%the system with a one-step numerical method generates a discrete
%dynamical system possessing a discrete forward attractor
%$\hat{\mathcal{A}}^{\bf h} = \{ \mathcal{A}(p, {\bf h}) ; (p, {\bf
%h} \in P_d \}$ satisfying
%\[ H^*( \mathcal{A}(p, {\bf h}), \mathcal{A}(p)) \to 0 \hspace{2cm} \text{as}
%  \qquad \rho \to 0. \]
%for each $(p, {\bf h}) \in P_d$.
%\end{therm}
%\begin{prf}
%From Theorem \ref{papasthm}, and the associated Lemmas, $\hat{\mathcal{A}}$ is
%forward equi-asymptotically stable. Hence the above theorem holds, and thus
%there exists a discrete equi-asymptotically stable family of sets
%$\hat{A}^{{\bf h}}$.
%
%Utilising $\hat{A}^{{\bf h}}$ as  an absorbing neighbourhood, we
%can find a discrete forward attractor as in Theorem
%\ref{dabspatthm},  which we will denote by
%$\hat{\mathcal{A}}^{{\bf h}}$ contained within $\hat{A}^{{\bf
%h}}$. Upper semi-continuity with respect to $\rho$ of discrete and
%continuous forward attractors holds. To see this, note that for
%each $(p, {\bf h}) \in P_d$
%\begin{align*}
%  H^*(\mathcal{A}(p, {\bf h}), \mathcal{A}(p)) &\leq H^*(A(p, {\bf h}),
%               A(p)), \\
%          & \to 0 \hspace{2cm} \text{as} \qquad \rho \to 0. \\
%\end{align*}
%\end{prf}

\subsection{Forward Attractors}

It is not possible to extrapolate the results of Theorem \ref{numeasthm1} for
application to forward attractors as was done in \cite{StHu96} for
semi-groups in ADE's, and in Corollary \ref{coruatt} for uniform
attractors in NDE's. The reason is that no
similar \textit{limit set representation} (compare with Theorem
\ref{abspat}) is available for forward attractors as there is for
semi-group attractors in ADE's, and pullback attractors in NDE's.

\subsection{Non-Constant Neighbourhoods of Attraction}

The results of Theorem \ref{numeasthm1} may also apply to a family
of sets, $\hat{A}$, that are forward equi-asymptotically stable
for which the neighbourhood of attraction varies with $p \in P$.

For this, some knowledge of the $\delta$-neighbourhood system is
essential, as the attraction for the discrete system must occur
faster than the rate at which the $\delta$-neighbourhood changes. The rate of
attraction however can only be estimated by the bound on the decrescence of the
associated Lyapunov function. Due to the manner of construction of the Lyapunov
function, this places an exponential bound on the rate of change for the
neighbourhood of attraction.

Consequently this may be ensured if the associated Lyapunov function
and $\delta$-neighbourhood system are chosen so that for each $p
\in P$ and all $t > 0$, the condition below is satisfied.
\begin{equation}
\label{condnhd}
e^{-ct/2} a (\delta_p) \leq a( \delta_{\theta_t p}).
\end{equation}
The theorem is stated as follows.

\begin{therm}
\label{numeasthm2} Let $\{ \Phi_{(t, p)}; t \in \mathbb{R}^+, p
\in P \}$ be a cocycle for $(\ref{NDEeq})$ which contains a
forward equi-asymptotically stable family of sets $\hat{A} = \{
A(p) ; p \in P\}$. If condition (\ref{condnhd}) is satisfied, then
a variable-time step discretisation of the initial value problem $(p_0, x_0)$
generates a discrete dynamical system
possessing a discrete forward equi-asymptotically stable family
$\hat{A}^{{\bf h}} = \{ A^{{\bf h}}(p_n, \psi_n{\bf h}) ; n \in
\mathbb{Z} \}$ satisfying
\begin{equation}
 H( A^{{\bf h}}(p_n, \psi_n{\bf h}), A(p_n)) \to 0 \hspace{2cm} \text{as} \qquad
\rho \to  0.
\end{equation}
for each $n \in \mathbb{Z}$.
\end{therm}
\begin{prf}
We assume condition (\ref{condnhd}) holds, with a neighbourhood of
attraction defined by $\hat{\delta}^*$.

The proof follows closely the argument presented in Theorem \ref{numeasthm1}
with the following amendments.

Construct $\hat{\delta}^0$ in a similar fashion to the procedure for $\delta_0$
in Theorem \ref{numeasthm1}. This redefines $\hat{B}$ and $\hat{A}$ accordingly.
We also redefine
\begin{align*}
  C_r(p) &= \max \{ \sup_{0 \leq h \leq \rho} a(\delta^0_{\theta_hp}), C^*_r(p)
                 \}, \\
  L(p) &= \max \{ 1, \sup_{0 \leq h \leq \rho} a(\delta^0_{\theta_hp})
                  /\delta^0_{\theta_hp}, L^*(p) \}, \\
\end{align*}
and $h_1(p)$ so that it is the
maximal quantity satisfying the inequality
\[  4 e^{ch_1(p)/2}L(p)C_r(p)h_1(p)^{r+1}/(1-e^{-ch_1(p)/2}) <
      \rho^{r+1} a ( \delta^0_p ). \]
Finally the procedures given in Lemmas B1-B3, A1-A7, C1
of Theorem \ref{numeasthm1} are repeated with the above bounds and applying the
inequality (\ref{condnhd}) where necessary. For example in Lemma B1, inequality
(\ref{condnhd}) must be applied at (\ref{lemvnhoodref}) to ensure the result
remains smaller than $a(\delta^0_{\theta_{h(p)}p})$.

This then verifies the existence of a discrete forward equi - asymptotically
stable family $\hat{A}^{\textbf{h}}$ with the required properties.
\end{prf}

\endinput
