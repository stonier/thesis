\section{Discretisation of $\mathcal{D}_l$}

Here we consider a the discretisation of a globally pullback
equi-asymptotically stable family $\hat{A}$, and observe the effects of the
discretisation from within the loci dynamical system $\mathcal{D}_l$. Since
$\mathcal{D}_l$ characterises pullback dynamical behaviour at some fixed $p \in
P$,  we discuss two approaches that may be used to numerically approximate an
element of the original pullback equi-asymptotically stable family $A(p)$ at $p
\in P$.

\subsection{Discretisation of the Loci Dynamics}
We first examine the discretisation of the loci dynamics. This approach is not
always possible as the loci dynamics (\ref{eqloci}) may only be explicitly
formulated for certain special cases (as for instance in Example
\ref{egsepscld}).

If however the loci dynamics can be generated, then for each $p \in P$ we may
discretise the loci dynamics and apply the results of Theorem \ref{numeasthm1}.
This then verifies the existence of a discrete forward equi-asymptotically
stable set $A_p^{\textbf{h}}$ that serves as an approximation for $A(p)$.

{\bf Remark:} Generating a numerical approximation in this manner,
does not give any indication of the effect of discretisation on
the original system, but merely serves as a technique to approximate the
original pullback behaviour.

A discretisation using this approach is illustrated in the following
example.

\begin{eg}[SDS - Discretisation of $\mathcal{D}_l$]\label{egsepsc}
Consider again the NDE introduced in Example \ref{egsepscld}
\[ \dot{x} = f(\ta)g(x). \]
for which there exists a globally pullback equi-asymptotically stable
set $A$ that is also loci stable.

To analyse pullback attraction to some $t_0 \in \mathbb{R}$, recall that
the loci dynamics are determined by the NDE
\begin{equation} \label{eqsepndeld}
  \frac{d \phi}{dt} = f(t_0 - t) g(\phi),
\end{equation}
and that $A$ was shown to be a global forward equi-asymptotically stable set for
the loci dynamical system $\mathcal{D}_l$.

Discretising (\ref{eqsepndeld}) with a variable time-step scheme and applying
Theorem \ref{numeasthm1} will verify the existence of a discrete forward
equi-asymptotically stable set that approximates $A$ at $t_0$.
\end{eg}

\subsection{Discretisation of the Original Dynamics }

We now examine the case for discretisation of the original dynamics.
With this approach the original pullback dynamics are discretised and the
evolution of the images at $p$, or equivalently, the corresponding numerically
approximated loci in $\mathcal{D}_l$ are observed.

The discretisation problem is formalised as follows:

As in Section \ref{disfassec}, we consider a numerical
scheme (possibly with a variable time-step construction) applied
to a non-autonomous dynamical system defined by $\dot{x} = f(p,
x)$ possessing a global pullback equi-asymptotically stable family
$\hat{A}$. The discretisation is used to approximate pullback
attraction to some $p \in P$.

\begin{figure}[htb]
\begin{center}
\input{eps/pdisc0.pstex_t} \caption{The Discrete Pullback Sequence}
\protect\label{pdisc0}
\end{center}
\end{figure}

The numerical scheme generates a discrete cocycle $\{ \Phi^{{\bf
h}}_{(n,(p, {\bf h}))}, n \in \mathbb{Z}^+, (p, {\bf h}) \in
P_d \}$ as used in Section \ref{disfassec}. To
analyse pullback attraction of an initial sequence $\hat{x}_0 \in
\mathcal{N}_{\delta_p, \hat{A}}$ to $p$, a discrete sequence  of the images
at $p$ is generated, denoted by $\{ x_n \}$ and defined by
\begin{align}
  \label{disseqeq}
  x_n &= \Phi^{{\bf h}}_{(n, \theta_{-n}(p, {\bf h}))}(x_0(\theta_{-n}(p, {\bf
                h}))), \nonumber \\
          &= \Phi^{{\bf h}}_{(n, (p_{-n},\psi_{-n}{\bf
                h}))}(x_0(p_{-n},\psi_{-n}{\bf    h})).
\end{align}
If $\hat{A}$ is not time varying, that is $\hat{A} = A$, then
pullback attraction of a single initial state $x_0$ rather than
initial sequences may be used. For clarity of illustration we have
assumed $\hat{A}$ is not time varying in all the associated
figures.

The {\bf discrete sequence} $\{ x_n \}$ represents a series of approximated
images at $p$ that are each individually generated by an $n$-step
discretisation. This is illustrated in Figure \ref{pdisc0}.

Each element in the discrete sequence is generated from a chain of discrete
points according to the numerical method applied.  The set of $n$ points
associated with the discrete image $x_n$ will be referred to as the {\bf n-th
discrete chain}, and each element in the chain denoted by the notation
$\{ x_n^i \}$ for $i =1, \cdots, n$ (see Figure
\ref{locinum1}). Subsequently, the notation for $x_n$ and $x_n^n$ are
equivalent, and the choice of which is used should hereafter be relevant to the
situation.

\begin{figure}[h]
\begin{center}
\input{eps/locinum1.pstex_t} \caption{Discrete Chains}
\protect\label{locinum1}
\end{center}
\end{figure}

If plotted in $\mathcal{D}_l$, the discrete sequence traces out a discrete
trajectory of points $(t_n, x_n)$ that approximates the continuous loci in
exactly the same way that a numerical discretisation typically approximates a
trajectory in the forward sense. This is shown in Figure \ref{pdisc1}.

\begin{figure}[h]
\begin{center}
\input{eps/pdisc1.pstex_t} \caption{The Discrete Sequence in $\mathcal{D}_l$}
\protect\label{pdisc1}
\end{center}
\end{figure}

\subsection{The Numerical Algorithm in $\mathcal{D}_l$}

To understand the effects of the numerical method applied to the original system
as seen in $\mathcal{D}_l$, it is necessary to derive the \textit{numerical
algorithm} that defines the discrete sequence $\{x_n\}$. If the numerical
algorithm can be derived then an analysis of the discrete pullback dynamics may
be restricted solely to  $\mathcal{D}_l$.

The numerical algorithm defining the discrete sequence $\{x_n\}$ takes the
general form
\begin{equation} \label{eqnumalg}
 x_{n+1} = x_n + h(t_n)F(t_n, x_n, x_0, n) + O(h(t_n)^2), \\
\end{equation}
where $h(t_n) \in {\bf h}$ denotes the variable time-step from
$\theta_{(-t_{n+1})}p$ to $\theta_{(-t_{n})}p$ in the original system (see
Figure \ref{pdisc1}).

For consistency and clarity throughout the remainder of the Chapter,
(\ref{eqnumalg} will be referred to as the \textbf{numerical algorithm} in
$\mathcal{D}_l$ derived from the \textbf{numerical method} applied to the
original dynamical system.

Typically (in an analysis of numerical methods applied in a forward sense) the
dependence of $F$ is limited to the variables $t_n$ and $x_n$. Any knowledge of
the sequence up to that point is unnecessary.

However, except for specific cases, the pullback numerical approximation is
intrinsically more complex and takes the form given by (\ref{eqnumalg}).
Its dependence on $x_0$ is illustrated with the example below.


\begin{eg}
The dynamical system generated by the NDE
\[ \dot{x} = 2 \ta x^3, \]
possesses a constant pullback attractor defined by $A = \{ 0 \}$. We
consider discrete pullback attraction to $t_0 = 0$ using an Euler method,
investigating the dependence of the numerical algorithm (\ref{eqnumalg}) on the
initial state $x_0$.

Let the step size $h = 0.1$ for the first few steps of any
generated discrete sequence, and set $x_1 = 1$ as the second
element in a discrete sequence generated from some unknown $x_0$.

We wish to calculate $x_2$, however this requires more information than simply
the requisite knowledge of the previous element $x_1 = 1$.

Considering the discrete problem from the perspective of the original dynamical
system (see Figure \ref{pdisc5}) generating $x_2$ requires the precise value of
$x_0$. The difficulty however is that the discrete chains are not
reversibly unique.

\begin{figure}[h]
\begin{center}
\input{eps/pdisc5.pstex_t} \caption{$x_0$-Dependence}
\protect\label{pdisc5}
\end{center}
\end{figure}

For instance, given $x_1 = 1$ with an Euler method and step size of $h = 0.1$ we
have
\begin{align*}
  x_1 &= x_0 + 2 h(t_0 - t_1) (x_0)^3, \\
   1    &= x_0 - 0.02 (x_0)^3. \\
\end{align*}
Approximating solutions to the cubic polynomial yields the
approximate solutions $x_0 \approx 1.0213$ or $x_0 \approx
6.5049$. Using these initial states to determine the discrete
image after two steps gives $x_2 \approx 0.96$ or $x_2 \approx
-2.64$ This is illustrated in Figure \ref{pdisc6} as observed from
within $\mathcal{D}_l$.

\begin{figure}[h]
\begin{center}
\input{eps/pdisc6.pstex_t} \caption{$x_0$-Dependence in $\mathcal{D}_l$}
\protect\label{pdisc6}
\end{center}
\end{figure}

As a result, it is concluded that the numerical algorithm (\ref{eqnumalg})
generating the discrete sequences for this problem possesses an
explicit dependence on $x_0$.
\end{eg}

The above example illustrates the effect of an example for which the numerical
method is not uniquely reversible, and outlines the factors influencing a
general case example.

If however, the numerical method is uniquely reversible,
then the numerical algorithm (\ref{eqnumalg}) may be simplified so that it is
independent of $x_0$. Unfortunately, the class of discrete problems for which
this occurs is relatively small.

\begin{lemma}
If the numerical method is uniquely reversible, then the numerical algorithm
(\ref{eqnumalg}) is independent of $x_0$.
\end{lemma}
\begin{prf}
For a numerical method of any order, the $(n+1)$-th element in the discrete
sequence, $x_{n+1}$ may be expressed as a function of $x_0$.

However, $x_n$ is also a function of $x_0$ and since it is uniquely reversible,
$x_0$ may be expressed as a function of $x_n$. Consequently, by substitution,
$x_{n+1}$ may be expressed as a function of $x_n$.
\end{prf}

In addition, the following example illustrates the explicit dependence of
(\ref{eqnumalg}) on the number of steps taken in $\mathcal{D}_l$ to reach some
time $t_n$.

\begin{eg}
The singleton set $A = \{ 0 \}$ is a pullback attractor for the dynamical system
generated by the NDE
\[ \dot{x} = 2 t \left[ sgn(x)x^2\right]. \]
For the dynamical system given, a discrete pullback analysis to $t_0 = 0$ with
an Euler method is made for two uniquely distinct initial points $a_0$ and
$b_0$.

\begin{figure}[h]
\begin{center}
\input{eps/pdisc3.pstex_t} \caption{$n$-Dependence}
\protect\label{pdisc3}
\end{center}
\end{figure}

The first three elements of the sequence for $\{a_n\}$ are calculated using a
constant step size of $0.05$, whereas the sequence for $\{b_n \}$ uses step
sizes $0.1$ and $0.05$ to generate $b_1$ and $b_2$. Additionally, we set $a_2 =
b_1 = 1$. This is shown in Figure \ref{pdisc3}.

In setting $a_2 = b_1 = 1$ the effect of discretisation from a particular point
on a loci in $\mathcal{D}_l$ at a specified time may be observed for two
sequences which differ only in the number of steps taken to arrive there (refer
to Figure \ref{pdisc4}).

\begin{figure}[h]
\begin{center}
\input{eps/pdisc4.pstex_t} \caption{$n$-Dependence in $\mathcal{D}_l$}
\protect\label{pdisc4}
\end{center}
\end{figure}

Since the discrete chains for this example are reversibly unique,
given $a_2 = b_1 = 1$ it is possible to explicitly calculate
$a_0$ and $b_0$, and consequently the progressive steps $a_3$ and $b_2$.

\begin{center}
\begin{tabular}{|c|c|c|c|}
\hline
${\bf a_0}$ & 1.0154 & &  \\ \hline
${\bf a_1}$ & 1.0102 & ${\bf b_0}$ & 1.0208 \\ \hline
${\bf a_2}$ & 1.0000 & ${\bf b_1}$ & 1.0000 \\ \hline
${\bf a_3}$ & 0.9849 & ${\bf b_2}$ & 0.9800 \\ \hline
\end{tabular}
\end{center}

Note that $a_3 \neq b_2$ as expected. As the only variable effecting the
difference between $a_3$ and $b_2$ is the number of steps taken, it is clear
the numerical algorithm that generates the discrete sequences for this example
is explicitly dependent on $n$.
\end{eg}

\begin{defn}
A numerical method is said to be {\em invariant under the loci mapping} if the
numerical algorithm (\ref{eqnumalg}) is equivalent to the application of the
same numerical method directly applied to the loci dynamics.
\end{defn}

It is interesting to observe that for linear non-autonomous dynamical systems,
an r-th order Taylor series numerical method is indeed invariant under the loci
mapping.

\begin{lemma}[LDS 1 - Numerical Algorithm]
\label{lemlde} Given the linear non-autonomous dynamical system
\begin{equation} \label{eqlnde}
  \dot{x} = f(\ta) x,
\end{equation}
where $f$ is $C^r$ continuous, then an r-th order Taylor series numerical method
with variable time steps applied to the original dynamical system is invariant
under the loci mapping.
\end{lemma}
\begin{prf}
Here $P = \mathbb{R}$, and consider pullback attraction to an arbitrarily chosen
$t_0 \in \mathbb{R}$.

The Taylor series method utilises a variable time-step sequence denoted by
${\bf h} = \{ h(t_0), h(t_1), \dots h(t_n), \dots \}$ where each $h(t_i)$
corresponds to the discrete step length between $t_0 - t_{i+1}$ and $t_0 - t_i$,
and each $t_i = \sum_{j=1}^i h(t_{j-1})$.

The discrete sequence $\{x_n\}$ is then generated as follows
\begin{align}
x^n_n &= x^{n-1}_n \left( 1 + \sum_{i=1}^r \frac{h(t_0)^i}{i!} f^{(i-1)} (
            t_0 - t) \right),  \nonumber   \\
&= x_0 \prod_{j=1}^n \left( 1 + \sum_{i=1}^r
            \frac{h(t_{j-1})^i}{i!} f^{(i-1)} ( t_0 - t_{j}) \right).
            \label{eqldeds}
\end{align}
The difference at each stage between elements in the $(n+1)$-th and $n$-th
discrete chains is of the form
\begin{align*}
x^1_{n+1} - x_0 &= x_0 \sum_{i=1}^r \frac{h(t_n)^i}{i!} f^{(i-1)} ( t_0 - t_{n +
             1}), \\
x^2_{n+1} - x^1_n &= (x^1_{n+1} - x_0) - (x^1_{n+1} - x_0) \sum_{i=1}^r
            \frac{h(t_{n-1})^i}{i!} f^{(i-1)} ( t_0 - t_{n}), \\
&= x_0 \left( \sum_{i=1}^r \frac{h(t_n)^i}{i!} f^{(i-1)} ( t_0 - t_{n +
             1}) \right) \\
&\hspace{2cm}  \left( 1 + \sum_{i=1}^r
            \frac{h(t_{n-1})^i}{i!} f^{(i-1)} ( t_0 - t_{n}) \right). \\
\end{align*}
Recursing the procedure, and substituting in (\ref{eqldeds}), the numerical
algorithm (\ref{eqnumalg}) is then formally defined by
\begin{align}
x^{n+1}_{n+1} &= x^n_n + x_0 \left( \sum_{i=1}^r \frac{h(t_n)^i}{i!}
               f^{(i-1)} ( t_0 - t_{n +1}) \right) \nonumber \\
&\hspace{2cm} \prod_{j=1}^n \left( 1 + \sum_{i=1}^r
            \frac{h(t_{j-1})^i}{i!} f^{(i-1)} ( t_0 - t_{j}) \right), \nonumber
           \\
&= x^n_n + x^n_n \left( \sum_{i=1}^r \frac{h(t_n)^i}{i!}
               f^{(i-1)} ( t_0 - t_{n +1}) \right). \label{eqldenum}
\end{align}
Recall the loci dynamics (\ref{scaseeq}) for a separable non-autonomous
dynamical system. Since the linear system is a special case, the loci dynamics
here is of the form
\[ \dot{\phi} = f(t_0 - t) \phi. \]
Comparing this with the form of the numerical algorithm (\ref{eqldenum}), it is
clear that applying the numerical method to the original dynamical system is
equivalent to applying the same r-th order Taylor series numerical method
directly to the loci dynamical system $\mathcal{D}_l$. Hence
the r-th order Taylor series method is invariant under the loci mapping.
\end{prf}

Note that the numerical algorithm for the linear case is independent of both
$x_0$ and $n$. The first is due to the fact that the discrete chains
are uniquely reversible, while the latter is due to the numerical
method for this example possessing properties similar to
`associativity'.  Each element in the $n$-th discrete chain is simply a
progressive application of $n$ scaling factors that can be rearranged or
combined in any alternate fashion. As a result the final image of the discrete
chain associated with $x_0$ that has been pulled back a specified time
is independent of the number of steps taken.

\begin{lemma}[SDS 1 - Numerical Algorithm]
\label{lemsde} Given the separable non-autonomous dynamical system
\begin{equation} \label{eqsnde}
  \dot{x} = f(\ta) g(x),
\end{equation}
where $f$ is $C^r$ continuous, then an Euler numerical method
with variable step sequence {\bf h} applied to the original dynamical system
generates a numerical algorithm of the form
\begin{equation} \label{eqsdsnumalg}
x_{n+1} - x_n = h(t_{n}) (a_0 + e_0) \prod_{j=1}^n (a_j + e_j), \\
\end{equation}
where
\begin{align*}
  F'(a, b) &= (f(b)- f(a))/(b-a), \\
  G'(x, y) &= (g(y) - g(x))/(y - x), \\
  a_0 &= f(t_0 - t_n) g(x_0), \\
  a_i  &= 1 + h(t_{n-i}) f(t_0 - t_{n+1-i}) G'(x_n^i,x_n^{i-1}), \\
  e_0 &=  h(t_{n})F'(t_0 - t_{n+1}, t_0 - t_n) g(x_0), \\
  e_i &= h(t_{n-i}) f(t_0 - t_{n+1-i}) [ G'(x_{n+1}^i,x_n^{i-1}) -
              G'(x_n^i,x_n^{i-1}) ]. \\
\end{align*}
\end{lemma}
\begin{prf}
The substitutions $F', G', a_i, e_i$ are used for clarity of expression and to
present the numerical algorithm in a form that allows calculation of the
truncation error later in the chapter.

Let $P = \mathbb{R}$, and consider discrete pullback attraction to an
arbitrarily chosen $t_0 \in \mathbb{R}$ using an Euler method with variable
step sequence {\bf h} defined as in Lemma \ref{lemlde}.

Generating the difference between elements in the $(n+1)$-th and $n$-th chains
in a similar fashion to that resolved in the linear case, we have
\begin{align*}
x^1_{n+1} - x_0 &= h (t_n)f(t_0 - t_{n+1}) g(x_0), \\
x^2_{n+1} - x^1_n &= (x^1_{n+1} - x_0) + h(t_{n-1})f(t_0 - t_n)(g(x_{n+1}^1) -
                         g(x_0), \\
  &= (x_{n+1}^1 - x_0) \left( 1 + h(t_{n-1})f(t_0 - t_n) \left[ \frac{g(x_{n+1}^1) -
                          g(x_0)}{x_{n+1}^1 - x_0} \right] \right), \\
\end{align*}
Making substitutions for $F', G', a_i, e_i$,
\begin{align*}
x^1_{n+1} - x_0 &= h(t_{n}) \left[ f(t_0 - t_n)g(x_0) \right. \\
   &\hspace{2.5cm} \left. + h(t_{n}) \left(
     (f(t_0-t_{n+1}) - f(t_0 - t_n))/h \right) g(x_0) \right], \\
   &= h(t_{n}) (a_0  + e_0), \\
x^2_{n+1} - x^1_n &= (x^1_{n+1} - x_0) ( 1 + h(t_{n-1}) f(t_0 - t_n) G'(x_{n+1}^1, x_0)),
                 \\
  &= (x^1_{n+1} - x_0) \left[ 1 + h(t_{n-1}) f(t_0 - t_n) G'(x_{n}^1, x_0) + \right.\\
  & \hspace{2cm} \left. h(t_{n-1}) f(t_0 -
                     t_n) ( G'(x_{n+1}^1, x_0) - G'(x_{n}^1, x_0))\right], \\
  &= h(t_{n}) (a_0 + e_0) ( a_1 + e_1). \\
\end{align*}
Consequently
\[ x^{i+1}_{n+1} - x^{i}_n = h(t_{n}) (a_0 + e_0) \prod_{j=1}^i (a_j + e_j). \]
The numerical algorithm for the discrete sequence $\{ x_n \}$ is then of the
form
\[ x_{n+1} - x_n = h(t_{n}) (a_0 + e_0) \prod_{j=1}^n (a_j + e_j), \]
\end{prf}

Once the numerical algorithm (\ref{eqnumalg}) for the loci dynamical system is
determined it is then possible to restrict an analysis to $\mathcal{D}_l$ and
the study of the discrete forward behaviour it possesses. If feasible, this
approach is advantageous as we may then utilise the results of
Chapter \ref{Discsec} where applicable.

The results in Chapter \ref{Discsec} however, rely on the assumption that the
local truncation error is of order $h^{r+1}$. Unfortunately this property
is not necessarily transferred to the discrete sequence in $\mathcal{D}_l$, and
needs to be carefully addressed.

\subsection{Local Truncation Error in $\mathcal{D}_l$}

Consider some initial state $(t_n, x_n)$ in the loci dynamical system
$\mathcal{D}_l$ where $x_n$ is the n-th step in the discrete sequence
originating from some $x_0$. Since $A(p)$ is a global
forward equi-asymptotically stable family, then the point $(t_n, x_n)$ lies on
some continuous loci (not necessarily originating at $x_0$) in $\mathcal{D}_l$.

The error arising from $x_n$ between the continuous evolution of the loci,
that is $\phi_{(h(t_n), t_n)}(x_n)$, and the discretisation,
$x_{n+1}$, over a single step form the local truncation error for forward
analysis on $\mathcal{D}_l$. Refer to Figure \ref{pdisc2}.

\begin{figure}[h]
\begin{center}
\input{eps/pdisc2.pstex_t} \caption{Local Truncation Error in $\mathcal{D}_l$}
\protect\label{pdisc2}
\end{center}
\end{figure}

Recall the loci dynamics (\ref{eqloci}) are governed by an ODE
\begin{equation*}
 \frac{d\phi}{dt} = \int_{t_0 - t}^{t_0}\frac{\partial f^*}{\partial t}(\tau,
   t_0, t, \phi_0) d\tau + f( t_0 - t, \phi_0),
\end{equation*}
where $f^*(\tau, t_0, t, \phi_0) = f( \tau, \Phi_{(\tau
- (t_0 - t),  t_0 - t)}(\phi_0))$.

Expanding $\phi_{(h(t_n, t_n)}(x_n)$ as a Taylor series around
$t_n$, we have
\begin{equation} \label{eqlocits}
  \phi_{(h(t_n), t_n)}(x_n) = x_n + h(t_n)
                      \dfrac{d \phi}{dt}(t_n) + O(h(t_n)^2).
\end{equation}
Combining (\ref{eqnumalg}) and (\ref{eqlocits}), the \textbf{local truncation
error}, $|\phi_{(h(t_n), t_n)}(x_n) - x_{n+1}|$ is then given by
\begin{equation} \label{eqterrordl}
|\phi_{(h(t_n), t_n)}(x_n) - x_{n+1}| = h(t_n) \left| \dfrac{d \phi}{dt}(t_n,
               x_n) - F(t_n, x_n, x_0, n))\right| + O(h(t_n)^2).
\end{equation}

As already mentioned, the intention is to apply Theorem \ref{numeasthm1} in
order to verify the existence of a discrete set that approximates $A(p)$ in
$\mathcal{D}_l$. Application of Theorem \ref{numeasthm1} however, relies on the
assumption that the local truncation error is at least $O(h(t_n)^2)$ and the
truncation bound $C_r$ is a function of $t_n$ only. In general this is not true
as the linear components of $x_{n+1}$ and $\phi_{(h(t_n, t_n)}(x_n)$ do not
equate. Additionally, the numerical algorithm is also dependent on $n$ and
$x_0$.

Nevertheless, the linear case is an exception to the
general rule and the numerical dynamics obey a suitable local truncation error bound that allows application of Theorem \ref{numeasthm1}. The local truncation error for the
linear case is generated below, and its numerical approximation discussed later
in the chapter.

\begin{lemma}[LDS 2 - Truncation Error]\label{lemldete}
\hfill \\
Given the linear non-autonomous dynamical system
\[ \dot{x} = f(\ta)x, \]
the local truncation error in $\mathcal{D}_l$ for a discrete pullback analysis
with an r-th order Taylor series numerical method
is of the form
\begin{equation} \label{eqltelde}
  |\phi_{(h(t_n), t_n)}(x_n) - x_{n+1}| \leq C_r(t_n) h(t_n)^{r+1}.
\end{equation}
\end{lemma}
\begin{prf}
Lemma \ref{lemlde} verifies that the numerical method is invariant under the
loci mapping and hence the numerical algorithm in $\mathcal{D}_l$ is equivalent
to an r-th order Taylor series method applied directly to the loci dynamics.
Since a Taylor series method generates a local truncation error of the order
(\ref{eqltelde}), the result follows.
\end{prf}

What form then does the local truncation error take in the general case? A
literal interpretation of (\ref{eqterrordl}) would imply that it is merely of
$O(h(t_n))$. Nevertheless, the linear components of (\ref{eqterrordl}) are far
from randomly chosen, and in fact, $F(t_n, x_n, x_0, n)$ should provide a rough
approximation to $\dfrac{d\phi}{dt} (t_n, x_n)$.  As a result, it is likely that
the local truncation error is stronger than simply $O(h(t_n)$).

Keeping in mind that the numerical algorithm (\ref{eqnumalg}) possesses an
explicit dependence on $n$ it is reasonable to conclude that
the local truncation error at any point must also reflect a dependence on the
number of steps taken, or equivalently, on $\rho$, the variable time-step upper
bound.


Although the actual formulation of the local truncation error will be
different for each numerical method, the following analysis for separable
dynamical systems highlights the relevance of $\rho$ and provides an
illustrative perspective on what form the local truncation error may take in
general.

\begin{lemma}[SDS 2 - Truncation Error]\label{lemsdete}
Given the separable non-autonomous dynamical system
\[ \dot{x} = f(\ta)g(x), \]
the local truncation error in $\mathcal{D}_l$ for a discrete pullback analysis
with an Euler method using a variable time-step sequence {\bf h}
is of the form
\begin{equation} \label{eqltesde}
  |\phi_{(h(t_n), t_n)}(x_n) - x_{n+1}| \leq C_r(t_n) h(t_n) \gamma(\rho).
\end{equation}
where each time step satisfies $\rho/2 < h(t_i) < \rho$ for some $\rho > 0$,
$C(t_n)$ is the local truncation function, and $\lim_{\rho
\to 0} \gamma(\rho) \to 0$.
\end{lemma}
\begin{prf}
Let $P = \mathbb{R}$, and consider discrete pullback attraction to an
arbitrarily chosen $t_0 \in \mathbb{R}$ using an Euler method with variable time
step sequence ${\bf h} = \{ h(t_0), h(t_1), \dots h(t_n), \dots \}$ where each
$h(t_i)$ corresponds to the discrete step length between $t_0 - t_{i+1}$ and
$t_0 - t_i$, and each $t_i = \sum_{j=1}^i h(t_{j-1})$. The variable time-step
sequence is upper bounded by $\rho > 0$.

Two conditions are postulated here that are essential to the
remainder of the proof.

\textbf{A1} - Given any $x_0$ and $t^* > 0$, then any discrete
chain linking $(x_0, t_0 - t^*)$ and $(x_n^n, t_0)$ remains uniformly bounded
(with respect to $n$) within the state space. This is easily verified.
Note that any continuous solution for the problem is ultimately bounded (this
can be shown by applying the Lemmas and Theorems on ultimate boundedness in
\cite{LaLe67,Yo59}. As a result, any discrete chain possessing a finite number
of steps will obviously remain bounded. If this were otherwise, then the
difference between the discrete chain and the continuous solution from $x_0$
would become larger than any accumulated numerical error which over the finite
interval of length $t^*$ is guaranteed to be bounded.

\textbf{A2} - For any $t^*  > 0$ and n-step discrete chain linking
$(x_0, t_0 - t^*)$ and $(x_n^n, t_0)$,
\[ \lim_{n \to \infty} (x_{n+1}^{i+1} - x_{n}^i ) \to 0, \]
holds for each $0 <  i < n$.

The loci dynamics for a separable differential equation is of the
form
\[ \dot{\phi} = f(t_0 - t) g( \phi ). \]
Given any $(t_n, x_n)$ and expressing $\phi_{(h(t_n), t_n)}(x_n)$ as a Taylor
series expansion around $(t_n, x_n)$,
\begin{equation}\label{eqsdslte1}
  \phi_{(h(t_n), t_n)}(x_n) = x_n + h(t_n) f(t_0 - t_n) g( x_n ) + O(h(t_n)^2).
\end{equation}
Throughout the remainder of the proof it is essential to keep in mind that
$t_n$ is a fixed point on the time axis. This is made clear here as it may
become confused as discrete chains with a varying number of steps will be
considered. Each discrete chain however is the same length (fixed by $t_n$), and
varying the number of steps affects only the step sizes.

Using the notation introduced in Lemma \ref{lemsde},
\begin{align*}
g(x^n_n) &= g(x_n^{n-1}) + (x^n_n - x_n^{n-1})G'(x^n_n, x^{n-1}_n), \\
  &= g(x_n^{n-1})( 1 + h(t_0) f(t_0 - t_1)G'(x^n_n, x^{n-1}_n)), \\
  &= g(x_n^{n-1}) a_n. \\
\end{align*}
Repeating the same process through to $g(x_0)$ we have
\[ g(x^n_n) = g(x_0) \prod_{j=1}^{n} a_j. \]
Substituting back in \ref{eqsdslte1},
\begin{equation}\label{eqsdslte2}
  \phi_{(h(t_n), t_n)}(x_n) = x_n + h(t_n) \left( a_0 \prod_{j=1}^{n} a_j
   \right) +          O(h(t_n)^2).
\end{equation}
Recall that the numerical algorithm (\ref{eqsdsnumalg}) generating the discrete
sequence $\{ x_n \}$ for a separable dynamical system is of the form
\[ x_{n+1} - x_n = h(t_n) (a_0 + e_0) \prod_{j=1}^n (a_j + e_j). \]
The local truncation error for the Euler method is then of the form
\begin{align*}
   \left| x_{n+1} - \phi_{(h(t_n), t_n)}(x_n)\right| &= h(t_n) \left| (a_0 +
e_0)       \prod_{j=1}^n (a_j +      e_j)  -  a_0 \prod_{j=1}^{n} a_j \right| \\
      & \hspace{2cm}+ O(h(t_n)^2). \\
\end{align*}
Given the above expression the local truncation error is of at
least order $h(t_n)$. We proceed to show that it is in fact of order $h(t_n)
\gamma(\rho)$ where $\gamma(\rho)$ is some function such that
\[ \lim_{\rho \to 0} \gamma(\rho) \to 0. \]
To see this, note that $n \to \infty$ as $\rho \to 0$ and consider the limit
\begin{equation}\label{eqsdslte3}
  \lim_{n \to \infty} \left| (a_0 + e_0) \prod_{j=1}^n (a_j +
     e_j)  -  a_0 \prod_{j=1}^{n} a_j \right|.
\end{equation}
Assuming \textbf{A1} holds so that any discrete chain lies within some
bounded compact set $B$, define
\begin{align*}
\bar{g}\,' &= \sup_{x \in B} g'(x).
\intertext{Also let}
\bar{f} &= \sup_{0 \leq t \leq t_n} f(t_0 - t), \\
\bar{a} &= 1 + \rho \bar{f} \bar{g}\,', \\
\Delta G' &= \max_{i=1, \dots, n} \{G'(x^i_{n+1},x^{i-1}_n) - G'(x^i_n,
               x^{i-1}_n) \}, \\
\bar{e} &= \rho \bar{f} \Delta G'. \\
\end{align*}
Note that $|a_i| \leq \bar{a}$ and $|e_i| \leq \bar{e}$ for each $1 \leq i \leq
n$. Returning to (\ref{eqsdslte3}),
\begin{align}
  \lim_{n \to \infty}& \left| (a_0 + e_0) \prod_{j=1}^n (a_j +
       e_j)  -  a_0 \prod_{j=1}^{n} a_j \right|, \notag \\
     & \leq \lim_{n \to \infty} \left[ (| a_0| + |e_0|)  (\bar{a} +
       \bar{e}\,)^n  -  | a_0 |  (\bar{a})^n \right], \notag \\
    & \leq \lim_{n \to \infty} \left[ |a_0| \left( \sum_{j=1}^{n} \left(
      \begin{smallmatrix}
       n \\
       j
       \end{smallmatrix}
       \right) (\bar{a})^{n-j} (\bar{e}\,)^j \right) + |e_0| (\bar{a})^n
            \right] \label{eqsdslte4}.
\end{align}

To resolve the above limit, consider the latter term first. Note that
\begin{align*}
  (\bar{a})^n &=  (1 + \rho \bar{f}\bar{g}\,')^n, \\
         &\leq  (1 + 2 t_n \bar{f}\bar{g}\,' / n)^n. \\
\end{align*}
Taking the limit as $n \to \infty$ and applying L'Hopital's rule,
\[ \lim_{n \to \infty} (\bar{a})^n \leq \exp(2 t_n \bar{f}\bar{g}\,'). \]
Consequently,
\begin{align}
  \lim_{n \to \infty} |e_0| (\bar{a})^n &= \lim_{n \to \infty} h F'(t_0 -
          t_{n+1}, t_0 - t_n) g(x_0) (\bar{a})^n, \notag \\
  &\leq  (0) \left( \exp(2 t_n \bar{f} \bar{g}\,') \right), \notag \\
  &= 0, \label{eqsdslte5}
\end{align}
since $h \to 0$ as $n \to \infty$.

Taking the first term in (\ref{eqsdslte4}),
\begin{align}
  \lim_{n \to \infty}  |a_0| \sum_{j=1}^{n} &\left( \left(
      \begin{smallmatrix}
       n \\
       j
       \end{smallmatrix}
       \right)
       (\bar{a})^{n-j}  (\bar{e}\,)^j \right) \notag \\
   & \leq \lim_{n \to \infty} |a_0| \sum_{j=1}^{\infty} \left( \frac{n^j}{j!}
       (\bar{a})^n (\bar{e})^j \right), \notag \\
  & \leq \lim_{n \to \infty} |a_0|  (\bar{a})^n \sum_{j=1}^{\infty} \left(
      \frac{(n h \bar{f} \Delta G')^j}{j!} \right), \notag \\
  & \leq \lim_{n \to \infty} |a_0|  (\bar{a})^n \sum_{j=1}^{\infty} \left(
      \frac{(t_n \bar{f} \Delta G')^j}{j!} \right), \notag \\
  & \leq |a_0| \exp(2 t_n \bar{f} \bar{g}\,') \lim_{n \to \infty} \left( \exp(
      t_n \bar{f} \Delta G') - 1 \right). \label{eqsdslte6}
\end{align}
As $n \to \infty$, $h \to 0$, and for each $i = 1, \dots, n$,
\begin{align*}
  \lim_{n \to \infty} \Delta G' &= \lim_{n \to \infty} [
           G'(x^i_{n+1},x^{i-1}_n) - G'(x^i_n, x^{i-1}_n) ], \\
  &=  g'(x^{i-1}_n) - g'(x^{i-1}_n), \\
  &= 0, \\
\end{align*}
since $x^i_{n+1} \to x^{i-1}_n$ and $x^i_{n} \to x^{i-1}_n$ as $\rho \to 0$ (due
to both \textbf{A2} and the nature of the Euler method). Consequently
(\ref{eqsdslte6}) $\to 0$ as $\rho \to 0$ (or equivalently $n \to \infty$).
Substituting the evaluation of these limits back into (\ref{eqsdslte4}) we find
\begin{align*}
 \lim_{n \to \infty} &\left| (a_0 + e_0) \prod_{j=1}^n (a_j +
       e_j)  -  a_0 \prod_{j=1}^{n} a_j \right|  \\
  & \hspace{1cm} = \lim_{\rho \to 0} \left| (a_0 + e_0) \prod_{j=1}^n (a_j +
       e_j)  -  a_0 \prod_{j=1}^{n} a_j \right|, \\
  & \hspace{1cm} = 0. \\
\end{align*}
From the above analysis, it may be concluded that the  local
truncation error for a separable dynamical system is bounded by the inequality
\[ \left| x_{n+1} - \phi_{(h(t_n), t_n)}(x_n)\right| \leq C_r(t_n) h(t_n)
  \gamma(\rho), \]
where $\lim_{\rho \to 0} \gamma(h) = 0$ and $C(t_n)$ is some function determined
by the bounds of $f, g$ on the interval of consideration defined by $t_n$.
\end{prf}

\textbf{Remark 1: } The local truncation error at $t_n$ given by
(\ref{eqltesde}) is in fact applicable for any $t < t_n$. Thus for a
discretisation over the finite interval $[0, t_n]$, (\ref{eqltesde}) evaluated
at $t_n$ is applicable for each point in the discrete sequence on that finite
interval.

\textbf{Remark 2:} It is strongly suspected that the local truncation error in
$\mathcal{D}_l$ for the general case is always $O(h(t_n) \rho)$.
However, verification of this, even for individual cases, is extremely
difficult and as yet remains an open problem.

\subsection{The Numerical Approximation}

Here we determine the actual nature of any discrete attracting structures that
characterise the pullback asymptotic stability present in the original
dynamical system. Since the local truncation error in $\mathcal{D}_l$ for the
linear case resolves itself simply, we shall first examine the nature of any
discrete structures that arise in the loci dynamical system $\mathcal{D}_l$
under such conditions. We then proceed to the problem of discretisation for a
broader class of dynamical systems that possess a local truncation error of a
specified form.

\subsubsection{Linear Non-Autonomous Dynamical Systems}

\begin{therm}[LDS 3 - Numerical Approximation]
Assume the origin, $A = \{ 0 \}$, is a globally pullback equi-asymptotically
stable set that is also loci stable for the linear non-autonomous dynamical
system
\[ \dot{x} = f(\ta)x. \]
Then for each $t_0 \in \mathbb{R}$, a variable step discretisation
(with bound $\rho > 0$ and restrictions on the step sizes) of the system with a
Taylor series numerical method generates a discrete dynamical system on
$\mathcal{D}_l$ which possesses a discrete global
forward equi-asymptotically stable family $\hat{A}^{\bf h} = \{A(t_n, \psi_n{\bf
h}); n \in \mathbb{Z} \}$
such that
\[ H(A(t_n, \psi_n {\bf h}), A) \to 0 \hspace{1cm} \text{as} \hspace{1cm} \rho
            \to 0, \]
for each $n$.
\end{therm}
\begin{prf}
Let $t_0 \in \mathbb{R}$ be chosen arbitrarily and consider the dynamics of the
associated loci dynamical system $\mathcal{D}_l$.

By Lemma \ref{lemlde}, the Taylor series numerical method is invariant under the
loci mapping, and thus satisfies the local truncation error condition given in
Lemma \ref{lemldete}.

Consequently, the effects of the discretisation of the original system in
$\mathcal{D}_l$ satisfy the assumptions for Theorem \ref{numeasthm1}, which may
be applied under restrictions on the variable time-step sequence to verify the
existence of a discrete globally forward equi - asymptotically stable family in
$\mathcal{D}_l$ that approximates $A$ at $t_0$ with the required properties.
\end{prf}

\subsubsection{Nonlinear Dynamical Systems}

The numerical approximation for non-linear systems is significantly less trivial
since the local truncation error is not of the form required by Theorem
\ref{numeasthm1} (that is, not $O(h(t_n)^2)$). As a result, a modified approach
is needed.

This is undertaken assuming only that the discretisation fulfills a local
truncation error of the form
\begin{equation} \label{eqltehr}
 \left| x_{n+1} - \phi_{(h(t_n), t_n)}(x_n)\right| \leq C_r(t_n) h(t_n)
  \gamma(\rho),
\end{equation}
where $\gamma(\rho) \to 0$ as $\rho \to 0$. Note that by Lemma \ref{lemsdete},
an Euler discretisation of a separable dynamical system automatically satisfies
(\ref{eqltehr}).  The theorem itself however, is presented in a general context
to allow for analysis of other examples which possess a local truncation error
of the same form.

\subsubsection{Preliminaries}

Let $\hat{A}$ be a pullback equi-asymptotically stable family that is also loci
stable for the non-autonomous dynamical system
\[ \dot{x} = f(p,x), \]
and consider the loci dynamics generated by pullback attraction to some $p \in
P$. By Theorem \ref{thmpastofashatA}, $A(p)$ is a globally forward
equi-asymptotically stable family in $\mathcal{D}_l$. As a consequence of
Theorem \ref{confasthm} there exists a Lyapunov function $V=V(t,x)$ (with $t
\geq 0$) that characterises the forward equi-asymptotic stability of $A(p)$ in
$\mathcal{D}_l$.

Following closely the outline of the proof for Theorem \ref{numeasthm1}, let
$\delta^* > 0$ define a forward attracting neighbourhood in $\mathcal{D}_l$
(since attraction is global any predefined value satisfies the requirements
for $\delta^*$) and set $\delta_0 = \delta^*/3$.

Similarly, we also set
\begin{align*}
  L^*(t) &= \sup_{0 \leq \tau \leq t} l(\tau), \\
  L(t) &= \max \{1, a(\delta_0)/\delta_0, L^*(t) \}, \\
  C^*(t) &= \sup_{0 \leq \tau \leq t} C_r(\tau), \\
  C(t) &= \max \{a(\delta_0), C^*(t) \}, \\
\end{align*}
where $l$ is the Lipschitz function associated with $V$, and $C_r$
is the local truncation error function. Note that these do not
change the local truncation error or Lipschitzness of $V$ except
to accommodate increased variation.

\subsubsection{The Discretisation}

A numerical approximation is made by applying an Euler method with variable
step sequence {\bf h} to the original dynamical system, the bounds of which
will be determined to ensure discrete forward attraction in $\mathcal{D}_l$.

The key difference between the proof for Theorem \ref{numeasthm1} and the proof
detailed here is in the restrictions made to ensure that discrete
attraction occurs.  In Theorem \ref{numeasthm1} this is achieved
by guaranteeing at each step  that $h(t_n)$ remains small enough. Due to the
form of the local truncation error (\ref{eqltehr}) here, it requires restricting
the step sizes used throughout the entire sequence over the interval considered.

With this in mind, we approach the problem by fixing a time $t$, and determining
the required bound $\rho(t)$ on the variable step sizes to ensure discrete
attraction occurs over the finite interval $[0, t]$. As $t$ is increased,
$\rho(t)$ is adjusted and the discrete sequence reconstructed. Finally, we
consider behaviour of the discrete sequences in the limit as $t \to \infty$.
This is formalised as follows.

Let $\lambda$ be some constant such that $0 <
\lambda < 1$ and define $\rho_1(t)$ as the largest value
satisfying

\begin{equation}\label{eqsdsrbnd1}
   4 L(t) C(t) \rho_1(t) \gamma(\rho_1(t)) / (1 - e^{-c\rho_1(t)}) \leq
      \lambda^2 a(\delta_0),
\end{equation}
for each $t \geq 0$. Note that due to the definition of $L(t)$ and $C(t)$,
$\rho(t)$ is a monotonically decreasing function in $t$. We also have
\begin{align*}
  \rho_1(t)^2 &\leq (1 - e^{-c\rho_1(t)}) \lambda^2 a(\delta_0) / 4 L(t) C(t),
             \\
    &\leq \lambda^2 a(\delta_0) / 4 a(\delta_0), \\
    &\leq \lambda^2, \\
\end{align*}
and consequently $\rho_1(t)$ is bounded above by $\lambda$ for all $t \geq 0$.
The variable time step sequence over the interval $[0, t]$ is subsequently
defined so that
 \begin{equation} \label{eqhsteps}
 \rho_1(t)/2 \leq h(t_n) \leq \rho_1(t) \hspace{2cm} \forall t_n \leq t.
\end{equation}
Hence the local truncation error for any single step $x_n$ to $x_{n+1}$
on this interval may be expressed in the form
\begin{align}
  | x_{n+1} - \phi_{(h(t_n), t_n)}(x_n)| &\leq C(t) h(t_n) \gamma(\rho_1(t)),
           \notag \label{eqsdsrbnd2} \\
  &\leq C(t) \rho_1(t) \gamma(\rho_1(t)).
\end{align}

\subsubsection{The Attracting Neighbourhood - $\hat{B}$}

\begin{defn}[B1 - Continuous Attracting Neighbourhood]
Define $\hat{B} = \{B(t); t \geq 0 \}$ in $\mathcal{D}_l$ by
\[ B(t) = \{ x : x \in \mathcal{N}_{\delta^*}(A(p)), V(t, x) \leq a(\delta_0)
           \}. \]
\end{defn}

\begin{lemma}[B2 - Positive Invariance]
If the variable step sequence {\bf h} satisfies the bounds
determined by (\ref{eqhsteps}) over the finite interval $[0,t]$, then $\hat{B}$
is a {\bf positively invariant} family under the discretisation over that
interval.
\end{lemma}
\begin{prf}
Consider any point in the discrete sequence on the interval $[0,t]$ with
$x_n \in B(t_n)$, $t_n \leq t$  and $\rho_1(t)/2 \leq h(t_n) \leq
\rho_1(t)$. Then
\begin{align*}
  V(t_{n+1}, x_{n+1}) &\leq V(t_n, \phi_{(h(t_n), t_n)}(x_n)) + L(t)|x_{n+1} -
           \phi_{(h(t_n), t_n)}(x_n)|, \\
  &\leq e^{-ch(t_n)}V(t_n, x_n) + L(t)C(t)\rho_1(t) \gamma(\rho_1(t)), \\
  &\leq e^{-ch(t_n)} a(\delta_0) + \frac{1}{4}(1 - e^{-c\rho_1(t)}) a(\delta_0),
     \\
  &\leq a(\delta_0). \\
\end{align*}
Consequently, $\hat{B}$ is positively invariant  over the finite interval $[0,
t]$ under the discretisation as constructed above.
\end{prf}

The following definition creates a discrete family that is positively invariant
under the discretisation, the proof for which follows immediately since it is
derived directly from $\hat{B}$.

\begin{lemma}[B3 - Discrete Attracting Neighbourhood]
Define the discrete attracting family $\hat{B}^{{\bf h}} = \{ B^{{\bf
h}}( t_n, \psi_n {\bf h}); n \in \mathbb{Z}^+ \}$ by
\begin{equation} \label{eqsdsdb}
  B^{{\bf h}}(t_n, \psi_n {\bf h}) = B(t_n),
\end{equation}
for all $n \in \mathbb{Z}^+$. If the variable step sequence
{\bf h} satisfies the bounds determined by (\ref{eqhsteps}) over the
finite interval $[0, t]$, then $\hat{B}^{{\bf h}}$ is \textbf{positively
invariant} under the discretisation over that interval.
\end{lemma}

\subsubsection{Discrete Asymptotic Structure $\hat{A}^{{\bf h}}$}

\begin{defn}[A1]
Define $\hat{A}^{{\bf h}} = \{ A^{{\bf h}}( t_n, \psi_n{\bf h}); n \in
\mathbb{Z}^+ \}$ by
\[  A^{{\bf h}}( t_n, \psi_n {\bf h}) = \{ x ; V(t_n, x) \leq \frac{1}{2}
                \lambda^2 a(\delta_0) \}. \]
\end{defn}
We propose $\hat{A}^{{\bf h}}$ as a discrete structure that approximates
$A(p)$ in $\mathcal{D}_l$.

\begin{lemma}[A2 - Boundedness and Compactness]
$\hat{A}^{{\bf h}}$ is uniformly bounded and each element is compact.
\end{lemma}
\begin{prf}
The discrete family $\hat{A}^{{\bf h}}$ is uniformly bounded since $A(p)$ \\ is
bounded and
\[ A^{{\bf h}}(t_n, \psi_n{\bf h}) \subset B(t_n) \subset
         \mathcal{N}_{\delta_0}(A(p)), \]
for each $n$. Compactness follows from the continuity of $V$.
\end{prf}

\begin{lemma}[A3 - Positive Invariance]
If the variable step sequence {\bf h}  satisfies the bounds determined by
(\ref{eqhsteps}) on the finite interval $[0,t ]$,
then $\hat{A}^{{\bf h}}$ is a positively invariant family under the
discretisation on that interval.
\end{lemma}
\begin{prf}
Consider any point in the discrete sequence on the interval $[0,t]$ with
$x_n \in A^{{\bf h}}(t_n, \psi_n{\bf h})$ with $t_n \leq t$  and $\rho_1(t)/2
\leq h(t_n) \leq \rho_1(t)$. Then
\begin{align*}
  V(t_{n+1},  x_{n+1}) &\leq V(t_{n+1},
              \phi_{(h(t_n), t_n)}(x_n)) + L(t)C(t)\rho_1(t) \gamma(\rho_1(t)),
               \\
    &\leq e^{-ch(t_n)} V(t_n, x_n) + L(t)C(t)\rho_1(t) \gamma(\rho_1(t)), \\
    &\leq e^{-ch(t_n)} \frac{1}{2}\lambda^2 a(\delta_0) +
      \frac{1}{4} \lambda^2 a(\delta_0)(1 - e^{-ch(t_n)}), \\
    &\leq \frac{1}{2} \lambda^2 a(\delta_0). \\
\end{align*}
Hence $x_{n + 1} \in A^{{\bf h}}(t_{n+1}, \psi_{n+1}{\bf h})$, and
$\hat{A}^{{\bf h}}$ is positively invariant over the defined interval under the
discretisation specified.
\end{prf}

The following lemma verifies that $\hat{A}^{{\bf h}}$ forward attracts
$\hat{B}^{{\bf h}}$ on the interval $[0, t]$ if the bound on the step sizes is
made correspondingly small enough.

For this we define $\rho(t)$ so that it satisfies
\begin{equation} \label{eqrbnd}
  \rho(t) = \min \{\rho_1(t), \rho_2 \},
\end{equation}
where $\rho_2$ satisfies the equation $(1 + e^{-c\rho_2}) = 2 e^{-c\rho_2 /4}$.
The variable step sequence {\bf h} on $[0,t]$  is then defined so that
\begin{equation} \label{hsteps2}
  \rho(t)/2 \leq h(t_n) \leq \rho(t), \hspace{1.5cm} \forall t_n \leq t.
\end{equation}

\begin{lemma}[A4 - Forward Absorbing]
  If the variable step sequence {\bf h} satisfies the bounds determined by
(\ref{hsteps2}) over the interval $[0,t]$, then in the limit as $t \to
\infty$, $\hat{A}^{{\bf h}}$ forward absorbs $\hat{B}^{{\bf h}}$ in finite time.
\end{lemma}
\begin{prf}
Let $t > 0$ and consider any $x_0 \in B^{{\bf h}}(0,
{\bf h}) \backslash A^{{\bf h}}(0, {\bf h})$. Then
\[ \frac{1}{2} \lambda^2 a(\delta_0) < V( 0, x_0), \]
and we have
\begin{align*}
  V(t_{1}, x_{1}) &\leq e^{-ch(t_0)}V(0, x_0) + \frac{1}{4}\rho(t)
            \gamma(\rho(t))a(\delta_0)(1 - e^{-ch(t_0)}), \\
    &\leq \frac{1}{2}(1 + e^{-ch(t_0)})V(0, x_0). \\
\end{align*}
Now if $h(t_0)$ satisfies (\ref{hsteps2}), then
\[ (1 + e^{-ch(t_0)}) \leq 2 e^{-ch(t_0)/4}, \]
and consequently,
\begin{align*}
  V(t_1, x_1) &\leq e^{-ch(t_0)/4} V(0,x_0), \\
    &\leq e^{-ct_1/4} V(0,x_0), \\
\end{align*}
If we extrapolate the argument presented above for the remainder of the discrete
sequence on $[0, t]$ there are two possible cases that arise.

{\bf i)} The discrete sequence is absorbed into $\hat{A}^{{\bf h}}$ at some
point on the interval $[0,t]$. That is, there exists a $j$ such that \\ $x_{j} \in
B^{{\bf h}}(t_j, \psi_j {\bf h}) \backslash A^{{\bf h}}(t_j, \psi_j {\bf h})$
and $x_{j+1} \in A^{{\bf h}}(t_{j+1}, \psi_{j+1} {\bf h})$ with $t_{j+1} \leq
t$. Since $\hat{A}^{{\bf h}}$ is positively invariant under the discretisation
on $[0,t]$, then $x_i \in A^{{\bf h}}(t_i, \psi_i {\bf h})$ for all $i > j$.

{\bf ii)} The discrete sequence never enters $\hat{A}^{{\bf h}}$ in which case
the Lyapunov value of the final element in the discrete sequence on $[0,t]$,
denoted by $x_n$ with $t_n \leq t$ and $t_{n+1} > t$, is bounded by
\begin{align*}
  V(t_n, x_n) &\leq e^{-ct_n/4} V(0, x_0), \\
    &\leq e^{-c(t- \rho(t))/4} a(\delta_0), \\
    &\leq e^{c/4} e^{-ct/4} a(\delta_0). \\
\end{align*}

Note that if we only consider the problem of discretisation on finite intervals
$[0, t]$ for values of $t$ such that
\begin{equation} \label{eqtbnd}
  t \geq \frac{4}{c} \ln (2/ \lambda^2) + 1,
\end{equation}
then the discrete sequence must at some point be absorbed by $\hat{A}^{{\bf
h}}$, and case i) always occurs for any $x_0 \in B^{{\bf h}}(0,
{\bf h})$. To see this, assume otherwise. That is, it is not absorbed by
$\hat{A}^{{\bf h}}$ for any $t_n < t$ (case ii)). Then
\begin{align*}
 V(t_n, x_n) &\leq e^{c/4} e^{-ct/4} a(\delta_0), \\
   &\leq \frac{1}{2} \lambda^2 a(\delta_0), \\
\end{align*}
where $x_n$ is defined as previously as the last element in the discrete
sequence on $[0, t]$. Consequently $x_n \in A^{{\bf h}}(t_n, \psi_n{\bf h})$
which contradicts ii). Hence i) must occur if $t$ satisfies (\ref{eqtbnd}).

Thus we may conclude that in the limit as $t \to \infty$, $\hat{A}^{{\bf h}}$
forward absorbs $B^{{\bf h}}(0, {\bf h})$ in finite time. That is,
for all $j$ such that $t_j \geq \frac{4}{c} \ln (2/ \lambda^2) + 1$, then
\[ \phi_{ (j, (0, {\bf h}))}(B^{{\bf h}}(0, {\bf h})) \subseteq A^{{\bf h}}(t_j,
   \psi_j{\bf h}). \]
\end{prf}

Due to the fact that the discretisation is only considered on finite intervals
it is not possible to define $\hat{A}^{{\bf h}}$ as a discrete
forward equi-asymptotically family. Nevertheless, it is still possible to obtain
results regarding the attractive properties of the discrete sequences on these
intervals. This is desirable as it translates directly to an understanding of
the discrete pullback analysis of solutions within a neighbourhood of $A(p)$ in
the original dynamical system.

\begin{lemma}[A5 - $\delta$-Neighbourhood of Forward Attraction]
  If the variable step sequence {\bf h} satisfies the bounds determined by
(\ref{hsteps2}) over the interval $[0,t]$, then in the limit as $t \to
\infty$, $\hat{A}^{{\bf h}}$ forward absorbs $\mathcal{N}_{\delta}(A^{{\bf
h}}(0, {\bf h}))$ for some $\delta > 0$.
\end{lemma}
\begin{prf}
Since $V$ is continuous and elements of $\hat{B}^{{\bf h}}$ and $\hat{A}^{{\bf
h}}$ are bounded in the state space by the level Lyapunov curves $V_B =
a(\delta_0)$ and $V_A = \frac{1}{2} \lambda^2 a(\delta_0)$, it must be that
$V_B$ and $V_A$ are separated by some minimum distance $\delta > 0$. This
$\delta$ then forms an appropriate local $\delta$-neighbourhood that is forward
absorbed by $\hat{A}^{{\bf h}}$ in finite time if the appropriate conditions
on the variable time-step sequence are met.
\end{prf}

\begin{lemma}[A6 - Boundedness of the Discrete Sequence]
For any $x_0 \in \mathcal{N}_{\delta}(A^{{\bf h}}(0, {\bf h}))$, the resulting
discrete sequence on $[0, t]$ is bounded so that
\[ x_j \in \mathcal{N}_{\delta^*}(A(t_j, \psi_j{\bf h})), \]
for all $j$ such that $t_j < t$.
\end{lemma}
\begin{prf}
$\hat{B}$ is positively invariant under the discretisation on $[0, t]$, hence
each $x_j \in B(t_j)$ since $x_0 \in B(0)$. Also note that
\[ B(t_j) \subset \mathcal{N}_{\delta^*}(A(p)) \subset
   \mathcal{N}_{\delta^*}(A^{{\bf h}}(t_j, \psi_j{\bf h})), \]
and the desired result follows.
\end{prf}

Finally, we may verify that the numerical approximation converges upper
semi-continuously to $A(p)$ by decreasing the bound on each $\rho(t)$. This is
achieved by letting $\lambda \to 0$ since $\lambda$ provides a suitable bound
for each $\rho(t)$.

\begin{lemma}[C1 - Upper Semi-Continuity] \hfill \\
$\hat{A}^{{\bf h}}$ is upper semi-continuous with respect to $A(p)$ and the
bound $\lambda$.
\end{lemma}
\begin{prf}
Note that $A(p)$ is contained within every element of $\hat{A}^{{\bf h}}$ since
$V$ is continuous and $\frac{1}{2} \lambda^2 a(\delta_0) > 0$. Then for any
variable time step sequence {\bf h} upper bounded by $\rho(t)$, and any $x_j
\in A^{{\bf h}}(t_j, \psi_j{\bf h})$,
\begin{align*}
  \dist( x_j, A(p) ) &\leq a^{-1} (V(t_j, x_j)), \\
    &\leq a^{-1}(\frac{1}{2}\lambda^2 a(\delta_0)). \\
\end{align*}
Since $a^{-1}(\frac{1}{2}\lambda^2 a(\delta_0)) \to 0$ as $\lambda \to 0$, the
required result follows.
\end{prf}

Combining the results of Lemmas and Definitions B1-B3, A1-A6, and C1 we obtain a
meaningful analysis of the effects of discretisation on the original dynamical
system.

\begin{therm}[Main Result]
Let $\hat{A}$ be a pullback equi-asymptotically stable family that is also loci
stable for the dynamical system
\[ \dot{x} = f(p, x). \]
For any $p \in P$, if a discrete pullback analysis is made over a finite
interval $[0,t]$ using a numerical method possessing a local truncation error
in $\mathcal{D}_l$ of the form (\ref{eqltehr}) using suitable restrictions on
the variable time-step sequence as required in Lemmas B1-B6, and A1-A6, then
there exists a discrete family $\hat{A}^{{\bf h}} = \{ A^{{\bf h}}(t_n, \psi_n
{\bf h}); n \in \mathbb{Z}^+ \}$ in the loci dynamical system that approximates
$A(p)$ and possesses the following properties:
\begin{description}
\item[i)] $\exists \delta> 0$, $T >0$, such that on the finite interval $[0,
  t]$ in $\mathcal{D}_l$, $\hat{A}^{{\bf h}}$ forward absorbs all solutions
  originating from within $\mathcal{N}_{\delta}(A^{{\bf h}}(0, {\bf h}))$ within
  finite time.
\item[ii)] The discrete trajectories in $\mathcal{D}_l$
  originating from within $\mathcal{N}_{\delta}(A^{{\bf h}}(0, {\bf h}))$ are
  bounded.
\item[iii)] If $\lambda$ is the uniform (on any finite interval of
  analysis) step size upper bound for the variable time step sequences, then
  for any appropriately defined variable step sequence {\bf h},
\[ \lim_{\lambda \to 0} H^*( A^{{\bf h}}(t_n, \psi_n {\bf h}), A(p)) = 0. \]
\end{description}
\end{therm}

The properties i) - iii) translate to meaningful properties
in the original dynamical system.

\textbf{i)} This property ensures that any discrete sequence originating within
a local neighbourhood of $\hat{A}$ converges to a structure that approximates
$A(p)$ under the appropriate conditions on the step sizes.

\textbf{ii)} The $\delta$-neighbourhood constructed here can be utilised to
generate a neighbourhood of $\hat{A}$ that is pullback stable with respect to
$A(p)$ under the discretisation.

\textbf{iii)} This ensures convergence of the attracting structure created by
numerical method to the continuous family $\hat{A}$.

\subsection{Conclusion}

In summary it may be concluded that the effects of numerical
approximation on systems possessing properties of pullback
asymptotic stability are largely dependent on the nature of the
actual dynamical system and the numerical method used. As a
result, a case by case analysis is needed, due primarily to the
resulting nature of the local truncation error in $\mathcal{D}_l$.
Investigations were made here specifically for linear and
separable dynamical systems, but of more importance, the initial
theory for loci dynamics and a clear interpretation of the local
truncation error is laid down to provide a basis for which further
analysis of individual cases may be undertaken.




\endinput
