\documentstyle{article}

\parskip 3mm
\parindent 0mm

\newlength{\widen} \newlength{\heighten}

\setlength{\widen}{0.55in}
\setlength{\heighten}{0.55in}

\addtolength{\oddsidemargin}{-\widen}
\addtolength{\evensidemargin}{-\widen}
\addtolength{\textwidth}{2\widen}
\addtolength{\topmargin}{-\heighten}
\addtolength{\textheight}{2\heighten}
\addtolength{\oddsidemargin}{+0.0in}
\addtolength{\evensidemargin}{-0.0in}

\title{Perturbed Autonomous Attractor Without Lower Semi-Continuous
            Convergence}
\author{Peter Kloeden, Daniel Stonier}
\date{April 28}

\newtheorem{defn}{Definition}[section]
\newtheorem{eg}{Example}[section]
\newtheorem{therm}{Theorem}[section]
\newtheorem{lemma}{Lemma}[section]


\begin{document}

\maketitle

Referenced books.
\begin{tabbing}
  \hspace{1cm} [1] \= P.Kloeden, D.Stonier \hspace{4mm} \= Cocycle
Attractors
            in Non-Autonomous Perturbed DE's \\ \end{tabbing}

\section{Introduction}

The perturbed autonomous attractor in [1] proved to be upper
semi-continuous in its convergence with respect to the initial
autonomous attractor it is derived from, i.e. the perturbed
attractor is always contained within the autonomous attractor for
the limit as the perturbation goes to zero.

\[ \lim_{\epsilon \rightarrow 0^{+}} H^{*}(A_{\epsilon}(t_{0}),A_{0}) = 0
\]

Lower semi-continuity of the perturbed attractor does not necessarily
hold, however, as the following counter-example shows.

\section{Counter Example}
\subsection{Defining The Perturbed System}

\begin{eg}
  Consider the 2-dimensional autonomous dynamical system

  \begin{equation}
    \dot{x} = y - x(x^{2} + y^{2} - 1)^{2}
    \label{auto}
  \end{equation}
  \[ \dot{y} = -x - y(x^{2} + y^{2} - 1)^{2} \]

  In polar co-ordinates this system is expressed by the equations,

  \begin{equation}
    \dot{r} = - r (r^{2} - 1)^2
    \label{autopolar1}
  \end{equation}
  \[ \dot{\theta} = -1 \]

It consists of a semi-stable limit cycle at $r = 1$, with trajectories
converging to the limit cycle from outside the unit circle, and
trajectories converging to the origin from inside the unit circle. In this
case the global attractor for the system is the disk at the origin with unit
radius.

  Now if we perturb the original system of equations (\ref{auto}) with

  \begin{equation}
    \dot{x} = y - x(x^{2} + y^{2} - 1)^{2} - \epsilon x  |\tanh t|
    \label{pert}
  \end{equation}
  \[ \dot{y} = -x - y(x^{2} + y^{2} - 1)^{2} - \epsilon y  |\tanh t| \]

  The polar equations become:

  \begin{equation}
    \dot{r} = - r (r^{2} - 1)^2 - \epsilon r |\tanh t|
    \label{autopolar2}
  \end{equation}
  \[ \dot{\theta} = -1 - \epsilon |\tanh t| \cos(2\theta) \]
\end{eg}

According to [1], this perturbed system has a cocycle attractor which is
upper semicontinuous with respect to the autonomous attractor (unit disk).
However, it is not lower semicontinuous as will be shown.


\subsection{Pullback Behaviour of the Perturbed System}

Let us consider the forwards asymptotic behaviour of trajectories. The
$\theta$ dynamics remain rotating in the same direction if $\epsilon$ is
small enough, so we need only be concerned with the radial variable.
Let $r$ $=$ $r(s,s_0,r_0)$ where  $s$ $\geq$ $0$ is the time elapsed since
we start at $s_0$. Then

\[ \frac{d}{ds} r^2 = - 2r^2(1-r^2)^2 - 2 \epsilon r^2 |\tanh (s+s_{0})| \]

Consider pull-back convergence of the system, using arbitrary initial values
of $t_{0}$, and $r_{0} > 1$. Substituting the notation for pullback terms we have $s_{0} = t_{0} - t$ and with $0 \leq s \leq t$ we obtain

\[  r = r(s,t_{0}-t,r_{0}) \]

\begin{equation}
  \frac{d}{ds} r^2 = - 2r^2(1-r^2)^2 - 2 \epsilon r^2 |\tanh (s+t_{0}-t)|
\label{deriv}
\end{equation}

Since $r_{0} > 1$ we may choose a $\delta$ small so that $(r_{0}^{2}-1)^{2} \geq \delta^{2}$.  Also, consider the case for $t_{0} = 0$. Then while $r^2(s,t_{0}-t,r_{0}) > 1 + \delta$ we have

\begin{eqnarray*}
  \frac{d}{ds} r^2  & = & - 2r^2(1-r^2)^2 - 2 \epsilon r^2 |\tanh (s-t)| \\
  & \leq & - 2 r^2 \delta^{2}
\end{eqnarray*}

which gives us

\begin{eqnarray*}
r^2(s,-t,r_0)  & \leq & r_0^2 exp(- 2 \delta^{2}s) \\[2ex]
& \leq & 1 + \delta
\end{eqnarray*}

for all $s \geq \frac{1}{2 \delta^{2}} \ln \left( \frac{r_{0}^2}{1+\delta}
\right)$ (provided $t$ is made large enough). For ease of notation, let $s_{1}(r_{0},\delta) = \frac{1}{2 \delta^{2}} \ln \left( \frac{r_{0}^2}{1+\delta}
\right)$. Thus we have

\begin{equation}
  r^2(s,-t,r_{0}) \leq 1 + \delta \hspace{1cm} \forall s_{1} \leq s \leq t
\end{equation}

\subsection{Behaviour Near r = 1}

Consider the evolution of the trajectory accross the boundary of the unit circle now we are within it's neighbourhood. Let $r_{1} = r(s_{1},t_{0}-t,r_{0})$. If $r_{1}^2 \leq 1 - \delta$ we can proceed automatically to the next step, else
we are now interested in the cocycle (note in this case $1 - \delta \leq r_{1}^2 \leq 1 + \delta$).

\begin{equation}
  r(s+s_{1},-t,r_{0})
\end{equation}

where $s_{1}$ is defined as above and $0 \leq s \leq t-s_{1}$.

Using our original equation for the derivative of the system (\ref{deriv}) then we have

\begin{eqnarray*}
  \frac{d}{ds} r^2 & = & - 2r^2(1-r^2)^2 - 2 \epsilon r^2 |\tanh (s+s_{1}-t)| \\
  & \leq & - 2 \epsilon r^2 |\tanh (s+s_{1}-t)|
\end{eqnarray*}

Now if we choose to make $s \leq \frac{1}{2}t - s_1$ and $t$ large enough so that $|\tanh (-\frac{1}{2}t)| \geq \frac{1}{2}$ then

\begin{eqnarray*}
  \frac{d}{ds} r^2 & \leq & - 2 \epsilon r^2 |\tanh (-\frac{t}{2})| \\
  & \leq & - 2 \epsilon r^2 \frac{1}{2} \\
  & = & -  \epsilon r^2
\end{eqnarray*}

Integrating,

\begin{eqnarray*}
  r^{2}(s+s_{1},-t,r_0) & \leq & r_1^{2}\exp\left\{- \epsilon s \right\} \\
  & \leq & (1+\delta) \exp\left\{- \epsilon s \right\} \\
  & \leq & 1 - \delta
\end{eqnarray*}

for all $s \geq \frac{1}{\epsilon}\ln(\frac{1+\delta}{1-\delta})$. For ease of notation denote this value by $s_2$ ie $s_2 = \frac{1}{\epsilon} \ln(\frac{1+\delta}{1-\delta})$. Thus we now have

\begin{eqnarray*}
  r^2(s,-t,r_{0}) & \leq & 1 - \delta \\
  & \forall & s_{1}+s_2 \leq s \leq \frac{t}{2} \\
  & and & t \geq \max \left\{2(s_1+s_2),|\tanh (-t)| \geq \frac{1}{2} \right\}
\end{eqnarray*}

\subsection{Behaviour For r $<$ 1}

Consider the evolution of the trajectory now we are within the unit circle. Let $r_{2} = r(s_1+s_{2},-t,r_{0})$ with $s_1, s_2$ and $t$ defined as above. Here we have $r_2^2 \leq 1-\delta$. Then we are interested in the behaviour of

\begin{equation}
  r(s+s_{1}+s_{2},-t,r_{0}) \hspace{1cm} s+s_1+s_2 \leq t
\end{equation}

Now $ r^2(s+s_{1}+s_{2},-t,r_{0}) \leq r_2^2 \leq 1 - \delta$.
Then from the original derivitive (\ref{deriv})

\begin{eqnarray*}
  \frac{d}{ds} r^2  & = & - 2r^2(1-r^2)^2 - 2 \epsilon r^2 |\tanh(s+s_1+s_2-t)| \\
  & \leq & - 2r^{2}\delta^{2}
\end{eqnarray*}

which gives us

\begin{eqnarray}
r^2(s+s_1+s_2,-t,r_0) & \leq & r^2_2\exp\left\{-2\delta^{2}s                \right\} \nonumber \\
& \leq & (1-\delta)\exp\left\{-2\delta^{2}s \right\}
\end{eqnarray}

and so on taking square roots of positive quantities we have

\begin{equation}
r(s+s_1+s_2,-t,r_0) \leq \sqrt{1-\delta}\exp\left\{- \delta^{2}s \right\}
\end{equation}

Replacing the pullback term (ie substituting $s+s_1+s_2=t$) for $s$

\begin{eqnarray}
r(t,-t,r_0) & \leq & \sqrt{1-\delta} \exp\left\{\delta^2(s_1+s_2) \right\} \exp\left\{- \delta^{2}t \right\} \\
& = & A(r_0,\delta,\epsilon) \exp\left\{- \delta^{2}t \right\}
\end{eqnarray}

where $A$ is a constant replacing and simplying the expression on the previous line. ie

\begin{eqnarray*}
  A(r_0,\delta,\epsilon) & = & \sqrt{1-\delta} \exp\left\{\delta^2(s_1+s_2)             \right\} \\
  s_1 & = & \frac{1}{2 \delta^{2}} \ln \left( \frac{r_{0}^2}{1+\delta}
        \right) \\
  s_2 & = & \frac{1}{\epsilon} \ln(\frac{1+\delta}{1-\delta})
\end{eqnarray*}

Now consider pullback convergence to the point $t_{0}=0$.

\begin{eqnarray*}
  \lim_{t\rightarrow\infty}r(t,-t,r_0) & \leq & \lim_{t                 \rightarrow \infty} A(r_0,\delta,\epsilon) \exp\left\{- \delta^{2}t \right\} \\
  & \rightarrow & 0
\end{eqnarray*}

Therefore the trajectory converges in the pullback sense to the origin and so by definition the cocycle attractor at $t_0=0$ is simply the origin.

\[ A_{\epsilon}(0) = 0 \]

\subsection{The Cocycle Attractor}

This particular nonautonomous example has the special property of having an equilibrium point at the origin (ie $r(s,s_0,0) = 0$ for all $s$ and $s_0$). Combining this fact with the shift property of cocycle attractors (see [1]) which states

\[ S_t(A_{\epsilon}(t_0)) = A_{\epsilon}(t_0+t) \]

or equivalently

\[ r(t,t_{0},A_{\epsilon}(t_{0}) = A(t_{0}+t) \]

we note that the cocycle attractor of this perturbed system for any $\epsilon > 0$ and times forward of 0 must simply be the origin. Replacing $t_0 = 0$ with a negative initial time in the procedure above, will follow through with the same bounds on $s_1$, $s_2$ and lower bound for $t$ and pullback convergence to the origin (the only point where the importance of the initial time comes into play is when considering the behaviour around $r=1$. Since negative initial times don't change the procedure at all, there is no problem here). Hence the entire cocycle attractor is simply the origin.

As a result, lower semicontinuous convergence does not hold.

\[ \lim_{\epsilon \rightarrow 0^{+}} H^{*} (A_{0},A_{\epsilon}(t_0)) = 1 \]

\end{document}
