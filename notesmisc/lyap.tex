\documentstyle{article}

\parskip 3mm
\parindent 0mm

\newlength{\widen} \newlength{\heighten}

\setlength{\widen}{0.45in}
\setlength{\heighten}{0.45in}

\addtolength{\oddsidemargin}{-\widen}
\addtolength{\evensidemargin}{-\widen}
\addtolength{\textwidth}{2\widen}
\addtolength{\topmargin}{-\heighten}
\addtolength{\textheight}{2\heighten}

\addtolength{\oddsidemargin}{+0.5in}
\addtolength{\evensidemargin}{-0.5in}

\title{Lyapunov Functions, Stability, and Boundedness}
\author{Daniel Stonier}
\date{April 7}

\newtheorem{defn}{Definition}[section]
\newtheorem{eg}{Example}[section]
\newtheorem{therm}{Theorem}[section]


\begin{document}

\maketitle

Referenced books.
\begin{tabbing}
  \hspace{1cm} \= [1] \= T.Yoshizawa \hspace{1.7cm} \= Stability Theory by
  Lyapunov's Second Method \\
  \> [2] \> H. Khalil \> Nonlinear Systems - Chapter 4\\
  \> [3] \> P.Kloeden, B.Schmalfus \> Non-Autonomous Systems,  Cocycle
  		Attractors \\
  \> \> \>  and Variable Time-Step Discretization \\
  \> [4] \> R.J.Stonier \> Phd Thesis Ch2 \\
  \> [5] \> T.Yoshizawa \> Lyapunov's Function and Boundedness of Solutions -
  		pp. 97-98\\

\end{tabbing}

\section{Dini Derivitive}

{\bf [1] 'Yoshizawa' pp 3-4}

\begin{defn}[Dini Derivitive]
  {\bf ([1] - page 3)}
  Given the continuous lyapunov function, $V(t,x(t))$, for the system
  $\dot{x} = F(t,x)$, the {\bf dini derivitive} is defined as being
  \[ \dot{V_{d}} = \lim_{h \rightarrow 0} \frac{1}{h} \{
                             V(t+h,x(t)+hF(t,x))) - V(t,x) \} \]
  \end{defn}

\begin{therm}
  If the lyapunov function is lipschitz as well as differentiable, then
  \[ \dot{V_{d}}(t,x(t)) = \dot{V}(t,x(t)) \]
\end{therm}

\begin{footnotesize}
  {\bf  Proof:}
  Begin by proving $\dot{V}(t,x(t)) \leq \dot{V_{d}}(t,x(t))$.

  \begin{tabbing}
    \hspace{1in} \= $V(t+h,x(t+h)) - V(t,x(t))$ \= = \= $V(t+h,x+hF(t,x) +
                   h \epsilon ) - V(t,x)$ \\
    \> \> $\leq$ \> $V(t+h,x+hF(t,x)) + Lh||\epsilon|| - V(x,t)$ \\
  \end{tabbing}
  Using the limit defns of the dini and derivitives, and $\epsilon
  \rightarrow 0$ as $h \rightarrow 0$, we have $\dot{V}(t,x(t)) \leq
  \dot{V_{d}}(t,x(t))$.

  Now, proving $\dot{V_{d}}(t,x(t)) \leq \dot{V}(t,x(t))$.

  \begin{tabbing}
    \hspace{1in} \= $V(t+h,x(t+h)) - V(t,x(t))$ \= = \= $V(t+h,x+hF(t,x) +
                   h \epsilon ) - V(t,x)$ \\
    \> \> $\geq$ \> $V(t+h,x+hF(t,x)) - Lh||\epsilon|| - V(x,t)$ \\
  \end{tabbing}
  And, as before this leads to $\dot{V_{d}}(t,x(t)) \leq \dot{V}(t,x(t))$.
  Combining the two results, we have proven the original statement :
  \[ \dot{V_{d}}(t,x(t)) = \dot{V}(t,x(t)) \]
\end{footnotesize}

\section{Lyapunov Results for Uniform Asymptotic Stability}

{\bf [1] 'Yoshizawa' pp 31-5, [2] 'Kahlil' pp 166-171}

Note the following lyapunov result has a converse, and is applicable only to
\underline{uniformly} aysymptotically stable sets. For the convenience, the
definition for uas is repeated below.

\begin{defn}[Stable Sets]

  {\bf [3] 'Kloeden, Schmalfuss' sec 2}

  A nonempty compact set $A_{0}$, under the semigroup $S(t)$, is {\bf
  uniformly stable} if for any $\epsilon > 0$ there exists a $\delta = \delta
  ( \epsilon) > 0$ (ie dependant only on $\epsilon$, this is what makes it
  uniform w.r.t. time - see example) so that

  \[ dist(u(t;u_{0}),A_{0}) < \epsilon \hspace{1in} \forall t \geq 0 \]

\begin{eg}[Stable, But Not Uniformly Stable Set]
  Consider the dynamical system represented by the equation

\[ x(t) = c \frac{\exp^{(c^{2} - 1)t}}{t} \]

Then the set $x = 0$ is a stable, but not uniformly stable set. To see this
note that the stable neighbourhood is bounded by the solutions for $c=-1,1$
which are $x(t) = \frac{-1}{t} and x(t) = \frac{1}{t}$. See diagram below.

\vspace{10cm}

The 'stableness' of the system decreases for larger initial time. Given any
$\epsilon > 0$, it is impossible to pick a $\delta = \delta(\epsilon)$ only,
since you would eventually get to a point where $\frac{1}{t} < \delta$. The
$\delta$ needed is given by

\[ \delta(\epsilon,t_{0}) = min \{ \epsilon, \frac{1}{t_{0}} \} \]

Hence the system is stable, but is not uniformly stable.

\end{eg}

  In addition it is {\bf uniformly asymptotically stable} if for
  every bounded subset D, and $\epsilon > 0$ there exists a $T_{\epsilon,D} >
  0$ (again uniformity - dependant only on $\epsilon$ and not $t_{0}$) such
  that

  \[ dist(S_{t}(u_{0}),A_{0}) < \epsilon \hspace{1mm} \forall u_{0} \in D, t
   		\geq t_{0} + T_{\epsilon,D} \]
\end{defn}

\begin{therm}[Necessary conditions for uniform asymptotic stability]
  Given a compact uniformly asymptotic stable set $A$, and $f(x)$ is
  uniformly lipschitz on some large neighbourhood of $A$. Then $\exists
  R_{0} > 0$ and a lyapunov function:

  \[ V : \Re^{+} \times \aleph (A, R_{0}) \rightarrow \Re^{+} \]

  which satisfies the following properties.
  \begin{tabbing}
    \hspace{2cm} \= i) \hspace{5mm} \= $V$ is Lipschitz on
                        $\aleph(A,R_{0}) $ \\
    \> ii) \>  $\exists$ cts, strictly increasing functions $\alpha, \beta
      : [0,R_{0}) \rightarrow \Re^{+}$ with $\alpha (0) = \beta(0) = 0$ \\
    \> \> and $\alpha (r) < \beta(r)$ such that  \\
    \> \> \hspace{2cm} $\alpha (dist(x,A)) \leq V(x) \leq \beta(dist(x,A))
                      $ \\
    \> iii) \> $\exists C > 0$ such that \\
    \> \> \hspace{2cm} V(S(t)U) $\leq e^{-Ct} V(U)$, $0 \leq t \leq T$ \\
    \> \> provided $S(t)U \in \aleph (A, R_{0})$, $0 \leq t \leq T$ \\
  \end{tabbing}
  \label{yoshi}
\end{therm}

The sufficiency condition is similar, and can be found in Khalil.

\section{Boundedness}

{\bf [1] 'R.J.Stonier' Ch. 2}

\begin{defn}[Strongly Bounded]
  A Dynamical System is {\em Strongly Bounded} with respect to a point
  $\xi$ if for each $(x_{0},t_{0}) \exists \hspace{4mm} \beta
  (x_{0},t_{0})$ such that
  \[ \rho ( S_{(t,t_{0})}(x_{0}),\xi) < \beta \hspace{1cm} \forall t \geq
					  t_{0} \]
\end{defn}

\begin{defn}[Equi Bounded]
  A Dynamical System is {\em Equi Bounded} with respect to a point
  $\xi$ if for each $\alpha$, and $t_{0} \exists \hspace{4mm}
  \beta (\alpha,t_{0})$ such that

  \[ \rho ( S_{(t,t_{0})}(x_{0}),\xi) < \beta \hspace{1cm} \forall t \geq
  					t_{0} \hspace{0.5cm} x_{0} \in
					\aleph_{\alpha}(\xi) \]
\end{defn}

\begin{defn}[Uniformly Bounded]
  A Dynamical System is {\em Uniformly Bounded} with respect to a point
  $\xi$ if the $\beta$ in the definition above is dependant only on $\alpha$
  and not $t$. ie. $\beta = \beta (\alpha)$.
\end{defn}

{\bf Note:} Equibounded implies strongly bounded but not the other way
around. Consider the example (in polars)

\begin{eg}[Strong, but not Equi]

{\bf [5] 'Yoshizawa' p.97-8}

\[ \dot{r} = r \frac{\dot{g}(t,\theta)}{g(t,\theta)} \]
\[ \dot{\theta} = 0 \]

where the function $g(t,\theta)$ is

\[ g(t,\theta) = \frac{(1+t) \sin^{2}\theta}{sin^{4}\theta +
		 (1-t \sin^{2}\theta)^{2}} + \frac{1}{(1+\sin^{4} \theta)
		 (1+t^{2})} \]

The solution is then simply

\[ r = r_{0} g(t,\theta) \]

Consider solutions for $\theta = 0$. $r = \frac{r_{0}}{1+ t^2}$. And
therefore any point on the $\theta = 0$ line is bounded by $r_{0}$. Every
point approximately close by is also bounded since r has a maximum finite
bound for each point. To see this let $\tau = \frac{1}{\sin^{2}\theta}$

\[ r = r_{0} \frac{(1 + t)}{1 + (t - \tau)^{2}} +
		\frac{\tau^{2}}{1+\tau^{2}} . \frac{1}{1 + t^{2}} \]

The last term is irrevelant as it gets smaller as t increases.
The first term increases until $t = \tau$, after which it decreases until
the whole solution approaches the origin. So any particular individual point
(in particular, for each $\tau$) is bounded by the value set by it's own $t
= \tau$ value.  However, the origin is not equi bounded since in any
neighbourhood of the $\theta = 0$ axis, we can find a sequence of points so
that $\theta \rightarrow 0^{+}$ ($\tau \rightarrow \infty$), and for $t =
\tau$ for all points in the n'hood, we have $r \rightarrow \infty$ when $t
\rightarrow \infty$ for points closing in on the $\theta = 0$ axis. That is
there is no finite bound that works for all solutions within the $\alpha$
n'hood of the origin.
\end{eg}

\begin{defn}[Ultimate Boundedness]
   A dynamical system is {\bf ultimately bounded} in a neighbourhood of the
set $\theta$, if there exists $b > 0$ and for each $(x,t) \in X \times
\Re^{+}$, there exists $T(x,t) > 0$ for which

\[ S_{(t',t)}(x) \in \aleph_{b}(\theta) \hspace{1cm} \forall t' > t + T \]
\end{defn}

\begin{defn}[Quasi Equi-Ultimate Boundedness]
   A dynamical system is {\bf quasi equi-ultimately bounded} in a
neighbourhood of the set $\theta$, if given any $\alpha > 0$ and  $x \in
\aleph_{\alpha}(\theta)$ there exists $T = T(\alpha,t)$ as in the previous
definition for which

\[ S_{(t',t)}(x) \in S_{b}(\theta) \hspace{1cm} \forall t' > t + T \]
\end{defn}

\begin{defn}[Equi-Ultimate Boundedness]
   A dynamical system is {\bf equi-ultimately bounded} in a
neighbourhood of the set $\theta$, if given any $\alpha > 0$ and  $x \in
\aleph_{\alpha}(\theta)$ and any $t_{0} \in \Re^{+}$, there exists $T =
T(\alpha,t)$ as in the previous definition for which

\[ S_{(t',t)}(x) \in S_{b}(\theta) \hspace{1cm} \forall t' > t + T \]
\end{defn}

Note that a system is equi-ultimate if the conditions hold for all t, while
they only have to hold for some values of t to be quasi-equi-ultimate.

\begin{defn}[Uniform-Ultimate Boundedness]
   A dynamical system is {\bf uniformly - ultimately bounded} in a
neighbourhood of the set $\theta$, if in the previous definition the $T$ is
dependant on $\alpha$ only. ie $T = T(\alpha)$.
\end{defn}

\end{document}

