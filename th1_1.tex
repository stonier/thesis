
\chapter{Dynamical Systems and Stability Theory}
\pagenumbering{arabic} \setcounter{page}{1}
\pagestyle{headings}
\section{The Dynamical System}

A dynamical system typically has three defining features. These
are:

\begin{itemize}
  \item \emph{Phase} or \emph{state space} $X$. Elements of this space
        represent possible states of the system at any given time.
  \item {\em Time}, which may be discrete or continuous. Solutions
        must exist for future times, but some systems may also be reversible.
        For discrete systems, the time set is represented by
        $\mathbb{Z}^+/\mathbb{Z}$, and for continuous systems  by
        $\mathbb{R}^+/\mathbb{R}$. 

It is often important to distinguish between the actual time and the time elapsed since the system was initialised. Differential equations typically refers to $t$ as the actual time, whereas dynamical systems theory utilises the same $t$ to represent the elapsed time. Throughout the thesis we will use $\ta$ for the actual time, and $t$ for the elapsed time to avoid confusion with the different conventions.

  \item {\em Evolution of the System}. As a general rule the system behaves
        in a fashion that evolves with time and allows us to uniquely
        determine the state of the system at each moment $t$ from its
        state at any previous moment. If it is a reversible process, then
        we may also determine the state of the system preceding a given
        initial state and moment in time.
\end{itemize}

\subsection{The State Space}

In many dynamical systems the state space, $X$, is represented by
a measure or topological space, or a space possessing the
structure of a smooth manifold. Most of the following work will
involve dynamical systems of ordinary differential equations where
the state is typically an element of topological space, usually
some subset of Euclidean Space. When this is the case we will use
the notation $E$ for the state space, where $E$ is open and $E
\subset \mathbb{R}^d$. Any case which does not follow a similar
approach will be given due attention.

\subsection{Evolution of the System}

The evolution of the state is characterised by a family of
mappings or transformations, which satisfy an evolution property
on the state space. For an {\em autonomous system}, the mapping is
invariant with respect to the initial time, and so depends solely
on the value of the initial state, mapping the state space into
itself. In these systems the evolution property is typically a
group or semi-group property. Mappings for {\em non-autonomous}
dynamical systems, however, generally depend on both initial state
and the initial time, and in general, feature much more
complicated behaviour. In particular a semi-group property no
longer holds, but a similar cocycle property is introduced to
characterise the system's evolution.

\endinput
